% !TEX root = highertopoi.tex
\section{Inner Fibrations}\label{midfibsec}

\setcounter{theorem}{0}

In this section, we will study the theory of {\em inner fibrations} between simplicial sets. The meaning of this notion is somewhat difficult to motivate, because it has no counterpart in classical category theory: Proposition \ref{ruko} implies that {\em every} functor between ordinary categories $\calC \rightarrow \calD$ induces an inner fibration of nerves $\Nerve(\calC) \rightarrow \Nerve(\calD)$.

In the case where $S$ is a point, a map $p: X \rightarrow S$ is an inner fibration if and only if
$X$ is an $\infty$-category. Moreover, the class of inner fibrations is stable under base change: if 
$$\xymatrix{ X' \ar[d]^{p'} \ar[r] & X \ar[d]^{p} \\ 
S' \ar[r] & S }$$ 
is a pullback diagram of simplicial sets and $p$ is an inner fibration, then so is $p'$.
It follows that if $p: X \rightarrow S$ is an arbitrary inner fibration, then each fiber
$X_{s} = X \times_{S} \{s\}$ is an $\infty$-category. We may therefore think of $p$ as encoding a family of $\infty$-categories parametrized by $S$. However, the fibers $X_{s}$ depend functorially on $s$ only in a very weak sense.

\begin{example}
Let $F: \calC \rightarrow \calC'$ be a functor between ordinary categories. Then the map $\Nerve(\calC) \rightarrow \Nerve(\calC')$ is an inner fibration. Yet the fibers
$\Nerve(\calC)_{C} = \Nerve( \calC \times_{\calC'} \{C\} )$ and $\Nerve(\calC)_{D} = \Nerve( \calC \times_{ \calC'} \{D\} )$ over objects $C,D \in \calC'$ can have wildly different properties, even if $C$ and $D$ are isomorphic objects of $\calC'$.
\end{example}

In order to describe how the different fibers of an inner fibration are related to one another, we will introduce the notion of a {\it correspondence} between $\infty$-categories. We review the classical theory of correspondences in \S \ref{corresp}, and explain how to generalize this theory to the $\infty$-categorical setting. 
 
In \S \ref{joyalpr}, we will prove that the class of inner anodyne maps is stable under smash products with arbitrary cofibrations between simplicial sets. As a consequence, we will deduce that the class of inner fibrations (and hence the class of $\infty$-categories) is stable under the formation of mapping spaces.

In \S \ref{minin}, we will study the theory of {\em minimal} inner fibrations, a generalization of Quillen's theory of minimal Kan fibrations. In particular, we will define a class of minimal $\infty$-categories and show that every $\infty$-category $\calC$ is (categorically) equivalent to a minimal $\infty$-category $\calC'$, where $\calC'$ is well-defined up to (noncanonical) isomorphism. We will apply this theory in \S \ref{ncats} to develop a theory of $n$-categories for $n < \infty$.

\subsection{Correspondences}\label{corresp}

Let $\calC$ and $\calC'$ be categories. A {\it correspondence} from $\calC$ to $\calC'$ is a functor
$$ M: \calC^{op} \times \calC' \rightarrow \Set.$$\index{gen}{correspondence!between categories}
If $M$ is a correspondence from $\calC$ to $\calC'$, we can define a new category $\calC \star^{M} \calC'$ as follows. An object of $\calC \star^{M} \calC'$ is either an object of $\calC$ or an object of $\calC'$. For morphisms, we take
$$ \Hom_{\calC \star^{M} \calC'}(X,Y) = \begin{cases} \Hom_{\calC}(X,Y) & \text{if } X,Y \in \calC \\
\Hom_{\calC'}(X,Y) & \text{if } X,Y \in \calC' \\
M(X,Y) & \text{if } X \in \calC, Y \in \calC' \\
\emptyset & \text{if } X \in \calC', Y \in \calC. \end{cases}$$
Composition of morphisms is defined in the obvious way, using the composition laws in $\calC$ and $\calC'$, and the functoriality of $M(X,Y)$ in $X$ and $Y$.

\begin{remark}
In the special case where $F: \calC^{op} \times \calC' \rightarrow \Set$ is the constant functor taking the value $\ast$, the category $\calC \star^{F} \calC'$ coincides with the ordinary join 
$\calC \star \calC'$.
\end{remark}

For any correspondence $M: \calC \rightarrow \calC'$, there is an obvious functor $F: \calC \star^{M} \calC' \rightarrow [1]$ (here $[1]$ denotes the linearly ordered set
$\{0,1\}$, regarded as a category in the obvious way), uniquely determined by the condition that $F^{-1} \{0\} = \calC$ and $F^{-1} \{1\} = \calC'$. Conversely, given any category $\calM$ equipped with a functor $F: \calM \rightarrow [1]$, we can {\em define} $\calC = F^{-1} \{0\}$, $\calC' = F^{-1} \{1\}$, and a correspondence $M: \calC \rightarrow \calC'$
by the formula $M(X,Y) = \Hom_{\calM}(X,Y)$. We may summarize the situation as follows:

\begin{fact}\label{factus}
Giving a pair of categories $\calC$, $\calC'$ and a correspondence between them is equivalent to giving a category $\calM$ equipped with a functor $\calM \rightarrow [1]$.
\end{fact}

Given this reformulation, it is clear how to generalize the notion of a correspondence to the $\infty$-categorical setting.

\begin{definition}\label{quasicorresp}\index{gen}{correspondence!between $\infty$-categories}
Let $\calC$ and $\calC'$ be $\infty$-categories. A {\it correspondence} from $\calC$ to $\calC'$ is a $\infty$-category $\calM$ equipped with a map $F: \calM \rightarrow \Delta^1$ and identifications $\calC \simeq F^{-1} \{0\}$, $\calC' \simeq F^{-1} \{1\}$. 
\end{definition}

\begin{remark}
Let $\calC$ and $\calC'$ be $\infty$-categories. Fact \ref{factus} generalizes to the $\infty$-categorical setting in the following way: there is a canonical bijection between equivalence classes of correspondences from $\calC$ to $\calC'$ and equivalence classes of functors
$\calC^{op} \times \calC' \rightarrow \SSet$, where $\SSet$ denotes the $\infty$-category of spaces.
In fact, it is possible to prove a more precise result (a Quillen equivalence between certain model categories), but we will not need this.
\end{remark}

To understand the relevance of Definition \ref{quasicorresp}, we note the following:

\begin{proposition}
Let $\calC$ be an ordinary category, and let $p: X \rightarrow \Nerve(\calC)$ be a
map of simplicial sets. Then $p$ is an inner fibration if and only if
$X$ is an $\infty$-category.
\end{proposition}

\begin{proof}
This follows from the fact that any map $\Lambda^n_i \rightarrow
\Nerve(\calC)$, $0 < i < n$, admits a {\em unique} extension to
$\Delta^n$.
\end{proof}

It follows readily from the definition that an arbitrary map of simplicial sets $p: X \rightarrow S$
is an inner fibration if and only if the fiber of $p$ over any
simplex of $S$ is an $\infty$-category. In particular, an inner fibration $p$
associates to each vertex $s$ of $S$ an $\infty$-category $X_{s}$, and to each edge $f: s \rightarrow s'$ in $S$ a correspondence
between the $\infty$-categories $X_{s}$ and $X_{s'}$. Higher dimensional simplices give
rise to what may be thought of as compatible ``chains'' of
correspondences.

Roughly speaking, we might think of an inner fibration $p: X \rightarrow
S$ as a functor from $S$ into some kind of
$\infty$-category of $\infty$-categories, where the morphisms are
given by correspondences. However, this description is not quite
accurate, since the correspondences are required to ``compose''
only in a weak sense.
To understand the issue, let us return to the setting of {\em
ordinary} categories. If $\calC$ and $\calC'$ are two categories,
then the correspondences from $\calC$ to $\calC'$ themselves
constitute a category, which we may denote by $M(\calC, \calC')$.
There is a natural ``composition'' defined on correspondences. If
we view an object $F \in M(\calC, \calC')$ as a functor
$\calC^{op} \times \calC' \rightarrow \Set$, and $G \in M(\calC',
\calC'')$, then we can define $(G \circ F)(C,C'')$ to be the coend
$$ \int_{C' \in \calC'} F(C,C') \times G(C',C'').$$\index{gen}{coend}

If we view $F$ as determining a 
category $\calC \star^{F} \calC'$ and $G$ as determining a category
$\calC' \star^{G} \calC''$, then $\calC \star^{G \circ F} \calC''$ is obtained
by forming the pushout
$$ ( \calC \star^{F} \calC') \coprod_{ \calC' } ( \calC' \star^{G} \calC'')$$
and then discarding the objects of $\calC'$.

Now, giving a category equipped with a functor to $[2]$ is equivalent to giving a triple of categories $\calC$, $\calC'$,
$\calC''$, together with correspondences $F \in M(\calC,\calC')$,
$G \in M(\calC', \calC'')$, $H \in M(\calC, \calC'')$ and a map $\alpha: G \circ F \rightarrow H$. But the map $\alpha$ need not be an isomorphism. Consequently, the above data cannot
literally be interpreted as a functor from $[2]$ into a
category (or even a higher category) in which the morphisms are
given by correspondences.

If $\calC$ and $\calC'$ are categories, then a correspondence from $\calC$ to $\calC'$ can be regarded as a kind of generalized functor from $\calC$ to $\calC'$. More specifically, for any functor $f: \calC \rightarrow \calC'$, we can define a correspondence $M_f$ by the formula
$$ M_f(X,Y) = \Hom_{\calC'}(f(X),Y).$$\index{gen}{correspondence!associated to a functor}
This construction gives a fully faithful embedding $\bHom_{\Cat}(\calC, \calC') \rightarrow M(\calC,\calC')$. Similarly, any functor $g: \calC' \rightarrow \calC$ determines a correspondence $M_{g}$ given by the formula $M_{g}(X,Y) = \Hom_{\calC}(X,g(Y))$; we observe that $M_{f} \simeq M_{g}$ if and only if the functors $f$ and $g$ are adjoint to one another.

If an inner fibration $p: X \rightarrow S$ corresponds to a ``functor'' from $S$ to a higher category of $\infty$-categories with morphisms given by correspondences, then some special class of inner fibrations should correspond to functors from $S$ into an $\infty$-category of $\infty$-categories with morphisms given by actual functors. This is indeed the case, and the appropriate notion is that of a {\em (co)Cartesian fibration} which we will study in \S \ref{cartfibsec}.

\subsection{Stability Properties of Inner Fibrations}\label{joyalpr}

Let $\calC$ be an $\infty$-category and $K$ an arbitrary simplicial set. In \S \ref{funcback}, we asserted that $\Fun(K,\calC)$ is an $\infty$-category (Proposition \ref{tyty}). In the course of the proof, we invoked certain stability properties of the class of inner anodyne maps. The goal of this section is to establish the required properties, and deduce some of their consequences.
Our main result is the following analogue of Proposition \ref{usejoyal}:

\begin{proposition}[Joyal \cite{joyalnotpub}]\label{usejoyal2}\index{gen}{inner anodyne}
The following collections all generate the same class of morphisms
of $\sSet$:
\begin{itemize}
\item[$(1)$] The collection $A_1$ of all horn inclusions $\Lambda^n_i
\subseteq \Delta^n$, $0 < i < n$.

\item[$(2)$] The collection $A_2$ of all inclusions $$(\Delta^m \times
\Lambda^2_1) \coprod_{ \bd \Delta^m \times \Lambda^2_1 } (\bd
\Delta^m \times \Delta^2) \subseteq \Delta^m \times \Delta^2.$$

\item[$(3)$] The collection $A_3$ of all inclusions $$(S' \times
\Lambda^2_1) \coprod_{S \times \Lambda^2_1 } (S \times \Delta^2)
\subseteq S' \times \Delta^2,$$ where $S \subseteq S'$.

\end{itemize}
\end{proposition}

\begin{proof}
We will employ the strategy that we used to prove Proposition \ref{usejoyal}, though the details are slightly more complicated. Working cell-by-cell, we conclude that every morphism in $A_3$ belongs to the weakly saturated class of morphisms generated by $A_2$. We next show that every morphism in $A_1$ is a retract of a morphism belonging to $A_3$. More precisely, we will show that for
$0 < i < n$, the inclusion $\Lambda^n_i \subseteq \Delta^n$ is a retract of the inclusion
$$ (\Delta^n \times \Lambda^2_1) \coprod_{ \Lambda^n_i \times \Lambda^2_1 }
( \Lambda^n_i \times \Delta^2) \subseteq \Delta^n \times \Delta^2.$$
To prove this, we embed $\Delta^n$ into $\Delta^n \times \Delta^2$ via
the map of partially ordered sets
$ s: [n] \rightarrow [n] \times [2]$ given by
$$ s(j) = \begin{cases} (j,0) & \text{if } j < i \\
(j,1) & \text{if } j = i \\
(j,2) & \text{if } j > i. \end{cases}$$
and consider the retraction $\Delta^n \times \Delta^2 \rightarrow \Delta^n$ given
by the map
$$ r: [n] \times [2] \rightarrow [n]$$
$$ r(j,k) = \begin{cases} j & \text{if } j < i, k=0 \\
j & \text{if } j > i, k = 2 \\
i & \text{otherwise.} \end{cases}$$

We now show that every morphism in $A_2$ is inner anodyne (that is, it lies in the weakly saturated class of morphisms generated by $A_1$). Choose $m \geq 0$. For each $0 \leq i \leq j < m$, we let
$\sigma_{ij}$ denote the $(m+1)$-simplex of $\Delta^m \times \Delta^2$ corresponding to the map
$$ f_{ij}: [m+1] \rightarrow [m] \times [2]$$
$$f_{ij}(k) = \begin{cases} (k,0) & \text{if } 0 \leq k \leq i \\
(k-1, 1) & \text{if } i+1 \leq k \leq j+1 \\
(k-1, 2) & \text{if } j+2 \leq k \leq m+1. \end{cases}$$
For each $0 \leq i \leq j \leq m$, we let $\tau_{ij}$ denote the $(m+2)$-simplex of $\Delta^m \times \Delta^2$ corresponding to the map
$$ g_{ij}: [m+2] \rightarrow [m] \times [2] $$
$$g_{ij}(k) = \begin{cases} (k,0) & \text{if } 0 \leq k \leq i \\
(k-1, 1) & \text{if } i+1 \leq k \leq j+1 \\
(k-2, 2) & \text{if } j+2 \leq k \leq m+2. \end{cases}$$

Let $X(0) = (\Delta^m \times
\Lambda^2_1) \coprod_{ \bd \Delta^m \times \Lambda^2_1 } (\bd
\Delta^m \times \Delta^2)$. For $0 \leq j < m$, we let
$$ X(j+1) = X(j) \cup \sigma_{0j} \cup \ldots \cup \sigma_{jj}. $$
We have a chain of inclusions
$$ X(j) \subseteq X(j) \cup \sigma_{0j} \subseteq \ldots \subset X(j) \cup \sigma_{0j} \cup
\ldots \cup \sigma_{jj} = X(j+1),$$
each of which is a pushout of a morphism in $A_1$ and therefore inner anodyne. It follows
that each inclusion $X(j) \subseteq X(j+1)$ is inner-anodyne. Set $Y(0) = X(m)$, so that the inclusion $X(0) \subseteq Y(0)$ is inner anodyne. We now set $Y(j+1) = Y(j) \cup \tau_{0j} \cup \ldots \cup \tau_{jj}$
for $0 \leq j \leq m$. As before, we have a chain of inclusions
$$ Y(j) \subseteq Y(j) \cup \tau_{0j} \subseteq \ldots \subseteq Y_{j} \cup \tau_{0j} \cup 
\ldots \cup \tau_{jj} = Y(j+1)$$
each of which is a pushout of a morphism belonging to $A_1$. It follows that
each inclusion $Y(j) \subseteq Y(j+1)$ is inner anodyne. By transitivity, we conclude
that the inclusion $X(0) \subseteq Y(m+2)$ is inner anodyne. We conclude the proof by observing that $Y(m+2) = \Delta^m \times \Delta^2$.
\end{proof}

\begin{corollary}[Joyal \cite{joyalnotpub}]\label{berek}
A simplicial set $\calC$ is an $\infty$-category if and only if the restriction map $$
\Fun(\Delta^2,\calC) \rightarrow \Fun(\Lambda^2_1, \calC)$$ is a trivial fibration.
\end{corollary}

\begin{proof}
By Proposition \ref{usejoyal2}, $\calC \rightarrow \ast$ is an inner fibration if and only if $S$ has the extension property with respect to each of the inclusions in the class $A_2$.
\end{proof}

\begin{remark}
In \S \ref{qqqc}, we asserted that the main function of the weak Kan condition on a simplicial set $\calC$ is that it allows us to compose the edges of $\calC$. We can regard Corollary \ref{berek} as an affirmation of this philosophy: the class of $\infty$-categories $\calC$ is characterized by the requirement that one can compose morphisms in $\calC$, and the composition is well-defined up to a contractible space of choices.
\end{remark}

\begin{corollary}[Joyal \cite{joyalnotpub}]\label{prodprod2}
Let $i: A \rightarrow A'$ be an inner anodyne map of simplicial sets, and let $j: B \rightarrow B'$ be a cofibration. Then
the induced map $$(A \times B') \coprod_{A \times B} (A' \times B)
\rightarrow A' \times B'$$ is inner anodyne.
\end{corollary}

\begin{proof}
This follows immediately from Proposition \ref{usejoyal2}, which characterizes the class of inner anodyne maps as the class generated by $A_3$ (which is stable under smash products with any cofibration).
\end{proof}

\begin{corollary}[Joyal \cite{joyalnotpub}]\index{gen}{inner fibration!and functor categories}
Let $p: X \rightarrow S$ be an inner fibration, and let $i: A \rightarrow B$ be any cofibration of simplicial sets. Then the induced map $q: X^{B} \rightarrow X^A \times_{ S^A } S^B$ is an inner fibration. If $i$ is inner anodyne, then $q$ is a trivial fibration. In particular, if $X$ is a $\infty$-category, then so is $X^B$ for any simplicial set $B$.
\end{corollary}

\subsection{Minimal Fibrations}\label{minin}

One of the aims of homotopy theory is to understand the classification of spaces up to 
homotopy equivalence. In the setting of simplicial sets, this problem admits an attractive formulation in terms of Quillen's theory of {\em minimal} Kan complexes.
Every Kan complex $X$ is homotopy equivalent to a minimal Kan complex, and a map $X \rightarrow Y$ of minimal Kan complexes is a homotopy equivalence if and only if it is an isomorphism. Consequently, the classification of Kan complexes up to homotopy equivalence is equivalent to the classification of {\em minimal} Kan complexes up to isomorphism. Of course, in practical terms, this is not of much use for solving the classification problem. Nevertheless, the theory of minimal Kan complexes (and, more generally, minimal Kan fibrations) is a useful tool in the homotopy theory of simplicial sets. The purpose of this section is to describe a generalization of the theory of minimal models, in which Kan fibrations are replaced by inner fibrations. An exposition of this theory can also be found in \cite{joyalnotpub}.

We begin by introducing a bit of terminology. Suppose given a commutative diagram
$$ \xymatrix{ A \ar[d]^{i} \ar[r]^{u} & X \ar[d]^{p} \\
B \ar[r]^{v} \ar@{-->}[ur] & S }$$
of simplicial sets where $p$ is an inner fibration, and suppose also that we have a pair $f,f': B \rightarrow X$ of candidates for the dotted arrow which render the diagram commutative. We will say that $f$ and $f'$ are {\it homotopic relative to $A$ over $S$}
if they are equivalent when viewed as objects in the $\infty$-category given by
the fiber of the map
$$ X^B \rightarrow X^A \times_{ S^A} S^B.$$
Equivalently, $f$ and $f'$ are homotopic relative to $A$ over $S$ if
there exists a map $F: B \times \Delta^1 \rightarrow X$ such that
$F | B \times \{0\} = f$, $F | B \times \{1\} = f'$, $p \circ F = v \circ \pi_{B}$,
$F \circ (i \times \id_{\Delta^1}) = u \circ \pi_{A}$, and $F| \{b\} \times \Delta^1$
is an equivalence in the $\infty$-category $X_{v(b)}$ for every vertex $b$ of $B$.

\begin{definition}\label{trukea}\index{gen}{minimal!inner fibration}
Let $p: X \rightarrow S$ be an inner fibration of simplicial sets. We will say that $p$ is {\it minimal} if $f=f'$ for every pair of maps $f,f': \Delta^n \rightarrow X$ which are homotopic relative to $\bd \Delta^n$ over $S$.

We will say that an $\infty$-category $\calC$ is {\it minimal} if the associated inner fibration
$\calC \rightarrow \ast$ is minimal.\index{gen}{minimal!$\infty$-category}
\end{definition}

\begin{remark}
In the case where $p$ is a Kan fibration, Definition \ref{trukea} recovers the usual notion of a minimal Kan fibration. We refer the reader to \cite{goerssjardine} for a discussion of minimal fibrations in this more classical setting.
\end{remark}

\begin{remark}
Let $p: X \rightarrow \Delta^n$ be an inner fibration. Then $X$ is an $\infty$-category.
Moreover, $p$ is a minimal inner fibration if and only if $X$ is a minimal $\infty$-category.
This follows from the observation that for any pair of maps $f,f': \Delta^m \rightarrow X$, a homotopy between $f$ and $f'$ is automatically compatible with the projection to $\Delta^n$.
\end{remark}

\begin{remark}
If $p: X \rightarrow S$ is a minimal inner fibration and $T \rightarrow S$ is an arbitrary map of simplicial sets, then the induced map $X_{T} = X \times_{S} T \rightarrow T$ is a minimal inner fibration.
Conversely, if $p: X \rightarrow S$ is an inner fibration and if $X \times_{S} \Delta^n \rightarrow \Delta^n$ is minimal for {\em every} map $\sigma: \Delta^n \rightarrow S$, then $p$ is minimal. Consequently, for many purposes the study of minimal inner fibrations reduces to the study of minimal $\infty$-categories.
\end{remark}

\begin{lemma}\label{ststst}
Let $\calC$ be a minimal $\infty$-category, and let $f: \calC \rightarrow \calC$ be a functor
which is homotopic to the identity. Then $f$ is a monomorphism of simplicial sets.
\end{lemma}

\begin{proof}
Choose a homotopy $h: \Delta^1 \times \calC \rightarrow \calC$ from $\id_{\calC}$ to $f$.
We prove by induction on $n$ that the map $f$ induces an injection from the set of
$n$-simplices of $\calC$ to itself.
Let $\sigma, \sigma': \Delta^n \rightarrow \calC$ be such that
$f \circ \sigma = f \circ \sigma'$. By the inductive hypothesis, we deduce that
$\sigma | \bd \Delta^n = \sigma' | \bd \Delta^n = \sigma_0$. Consider the diagram
$$ \xymatrix{ 
(\Delta^2 \times \bd \Delta^n) \coprod_{ \Lambda^2_2 \times \bd \Delta^n }
( \Lambda^2_2 \times \Delta^n ) \ar[rrr]^-{G_0} \ar@{^{(}->}[d] & & & \calC \\
\Delta^2 \times \Delta^n \ar@{-->}[urrr]^{G} & & & }$$
where $G_0| \Lambda^2_2 \times \Delta^n$ is given by amalgamating
$h \circ (\id_{\Delta^1} \times \sigma)$ with $h \circ (\id_{ \Delta^1} \times \sigma')$, and
$G_0 | \Delta^2 \times \bd \Delta^n$ is given by the composition
$$ \Delta^2 \times \bd \Delta^n \rightarrow \Delta^1 \times \bd \Delta^n
\stackrel{\sigma_0}{\rightarrow} \Delta^1 \times \calC \stackrel{h}{\rightarrow} \calC.$$
Since $h| \Delta^1 \times \{X\}$ is an equivalence for every object
$X \in \calC$, Proposition \ref{goouse} implies the existence of the map $G$ indicated in the diagram. The restriction $G|\Delta^1 \times \Delta^n$ is a homotopy between
$\sigma$ and $\sigma'$ relative to $\bd \Delta^n$. Since $\calC$ is minimal, we deduce that $\sigma = \sigma'$.
\end{proof}

\begin{lemma}\label{stsst}
Let $\calC$ be a minimal $\infty$-category, and let $f: \calC \rightarrow \calC$ be a functor
which is homotopic to the identity. Then $f$ is an isomorphism of simplicial sets.
\end{lemma}

\begin{proof}
Choose a homotopy $h: \Delta^1 \times \calC \rightarrow \calC$ from $\id_{\calC}$ to $f$.
We prove by induction on $n$ that the map $f$ induces a {\em bijection} from the set of
$n$-simplices of $\calC$ to itself. The injectivity follows from Lemma \ref{ststst}, so it will suffice to prove the surjectivity. Choose an $n$-simplex $\sigma: \Delta^n \rightarrow \calC$.
By the inductive hypothesis, we may suppose that
$\sigma| \bd \Delta^n = f \circ \sigma'_0$, for some map $\sigma'_0: \bd \Delta^n \rightarrow \calC$.
Consider the diagram
$$ \xymatrix{ (\Delta^1 \times \bd \Delta^n) \coprod_{ \{1\} \times \bd \Delta^n } (\{1\} \times \Delta^n) \ar[rrr]^-{G_0} \ar@{^{(}->}[d] & & & \calC \\
\Delta^1 \times \Delta^n, \ar@{-->}[urrr]^{G} & &  & }$$
where $G_0| \Delta^1 \times \bd \Delta^n = h \circ (\id_{\Delta^1} \times \sigma'_0)$ and
$G_0 | \{1\} \times \Delta^n = \sigma$. If $n > 0$, then the existence of the map $G$ as indicated in the diagram follows from Proposition \ref{goouse}; if $n=0$ it is obvious. Now let
$\sigma' = G| \{0\} \times \Delta^n$. To complete the proof, it will suffice to show that
$f \circ \sigma' = \sigma$.

Consider now the diagram
$$ \xymatrix{ (\Lambda^2_0 \times \Delta^n) \coprod_{ \Lambda^2_0 \times \bd \Delta^n } ( \Delta^2 \times \bd \Delta^n) \ar[rrr]^-{H_0} \ar[d] & & & \calC \\
\Delta^2 \ar@{-->}[urrr]^{H} & & &  }$$
where $H_0 | \Delta^{ \{0,1\} } \times \Delta^n = h \circ (\id_{\Delta^1} \times \sigma')$, $H_0 | \Delta^{ \{1,2\}} \times \Delta^n = G$, and $H_0| (\Delta^2 \times \bd \Delta^n)$ given by the composition
$$ \Delta^2 \times \bd \Delta^n \rightarrow \Delta^1 \times \bd \Delta^n \stackrel{\sigma'_0}{\rightarrow} \Delta^1 \times \calC \stackrel{h}{\rightarrow} \calC.$$
The existence of the dotted arrow $H$ follows once again from Proposition \ref{goouse}.
The restriction $H| \Delta^{ \{1,2\} } \times \Delta^n$ is a homotopy from $f \circ \sigma'$
to $\sigma$ relative to $\bd \Delta^n$. Since $\calC$ is minimal, we conclude that
$f \circ \sigma' = \sigma$ as desired.
\end{proof}

\begin{proposition}
Let $f: \calC \rightarrow \calD$ be an equivalence of minimal $\infty$-categories. Then
$f$ is an isomorphism.
\end{proposition}

\begin{proof}
Since $f$ is a categorical equivalence, it admits a homotopy inverse $g: \calD \rightarrow \calC$.
Now apply Lemma \ref{stsst} to the compositions $f \circ g$ and $g \circ f$.
\end{proof}

The following result guarantees a good supply of minimal $\infty$-categories:

\begin{proposition}\label{minimod}
Let $p: X \rightarrow S$ be a inner fibration of simplicial sets. Then there exists
a retraction $r: X \rightarrow X$ onto a simplicial subset $X' \subseteq X$ with the following
properties:
\begin{itemize}
\item[$(1)$] The restriction $p|X': X' \rightarrow S$ is a minimal inner fibration.
\item[$(2)$] The retraction $r$ is compatible with the projection $p$, in the sense that
$p \circ r = p$.
\item[$(3)$] The map $r$ is homotopic over $S$ to $\id_{X}$ relative to $X'$.
\item[$(4)$] For every map of simplicial sets $T \rightarrow S$, the induced inclusion
$X' \times_{S} T \subseteq X \times_{S} T$ is a categorical equivalence.
\end{itemize}
\end{proposition}

\begin{proof}
For every $n \geq 0$, we define a relation on the set of $n$-simplices
of $X$: given two simplices $\sigma, \sigma': \Delta^n \rightarrow X$, we will write
$\sigma \sim \sigma'$ if $\sigma$ is homotopic to $\sigma'$ relative to $\bd \Delta^n$. We note that $\sigma \sim \sigma'$ if and only if $\sigma| \bd \Delta^n = \sigma'| \bd \Delta^n$ and
$\sigma$ is equivalent to $\sigma'$ where both are viewed as objects in the $\infty$-category given by a fiber of the map
$$ X^{\Delta^n} \rightarrow X^{ \bd \Delta^n} \times_{ S^{\bd \Delta^n} } S^{\Delta^n}.$$
Consequently, $\sim$ is an equivalence relation.

Suppose that $\sigma$ and $\sigma'$ are both degenerate, and $\sigma \sim \sigma'$.
From the equality $\sigma | \bd \Delta^n = \sigma' | \bd \Delta^n$ we deduce that $\sigma = \sigma'$. Consequently, there is at most one degenerate $n$-simplex of $X$ in each $\sim$-class. Let $Y(n) \subseteq X_n$ denote a set of representatives for the $\sim$-classes of $n$-simplices in $X$, which contains all degenerate simplices. We now define the simplicial subset
$X' \subseteq X$ recursively as follows: an $n$-simplex $\sigma: \Delta^n \rightarrow X$
belongs to $X'$ if $\sigma \in Y(n)$ and $\sigma| \bd \Delta^n$ factors through $X'$.

Let us now prove $(1)$. To show that $p|X'$ is an inner fibration, it suffices to prove that
every lifting problem of the form
$$ \xymatrix{ \Lambda^n_i \ar[r]^{s} \ar@{^{(}->}[d] & X' \ar[d] \\
\Delta^n \ar@{-->}[ur]^{\sigma} \ar[r] & S }$$
with $0 < i < n$ has a solution $f$ in $X'$. Since $p$ is an inner fibration, this lifting problem
has a solution $\sigma': \Delta^n \rightarrow X$ in the original simplicial set $X$. Let
$\sigma'_0 = d_i \sigma: \Delta^{n-1} \rightarrow X$ be the induced map.
Then $\sigma'_0 | \bd \Delta^{n-1}$ factors through $X'$. Consequently, $\sigma'_0$
is homotopic over $S$, relative to $\bd \Delta^{n-1}$ to some map
$\sigma_0: \Delta^{n-1} \rightarrow X'$. Let $g_0: \Delta^1 \times \Delta^{n-1} \rightarrow X$
be a homotopy from $\sigma'_0$ to $\sigma_0$, and let 
$g_1: \Delta^1 \times \bd \Delta^n \rightarrow X$ be the result of amalgamating
$g_0$ with the identity homotopy from $s$ to itself. Let $\sigma_1 = g_1| \{1\} \times \bd \Delta^n$.
Using Proposition \ref{goouse}, we deduce that $g_1$ extends to a homotopy from
$\sigma'$ to some other map $\sigma'': \Delta^{n} \rightarrow X$ with
$\sigma''| \bd \Delta^{n} = \sigma_1$. It follows that $\sigma''$ is homotopic
over $S$ relative to $\bd \Delta^n$ to a map $\sigma: \Delta^n \rightarrow X$ with the desired properties. This proves that $p|X'$ is an inner fibration. It is immediate from the construction that $p|X'$ is minimal.

We now verify $(2)$ and $(3)$ by constructing a map
$h: X \times \Delta^1 \rightarrow X$ such that $h|X \times \{0\}$ is the identity,
$h| X \times \{1\}$ is a retraction $r: X \rightarrow X$ with image $X'$, and $h$ is a homotopy
over $S$ and relative to $X'$. Choose an exhaustion
of $X$ by a transfinite sequence of simplicial subsets 
$$X' = X^0 \subseteq X^1 \subseteq \ldots $$
where each $X^{\alpha}$ is obtained from 
$$X^{< \alpha} = \bigcup_{\beta < \alpha} X^{\beta}$$ by adjoining a single nondegenerate
simplex, if such a simplex exists. We construct $h_{\alpha} = h| X^{\alpha} \times \Delta^1$
by induction on $\alpha$. By the inductive hypothesis, we may suppose that we have already
defined $h_{< \alpha} = h| X^{< \alpha} \times \Delta^1$. If $X = X^{< \alpha}$, then we are done.
Otherwise, we can write $X^{\alpha} = X^{< \alpha} \coprod_{ \bd \Delta^n } \Delta^n$
corresponding to some nondegenerate simplex $\tau: \Delta^n \rightarrow X$, and it
suffices to define $h_{\alpha} | \Delta^n \times \Delta^1$. If $\tau$ factors through $X'$, we define
$h_{\alpha} | \Delta^n \times \Delta^1$ to be the composition
$$ \Delta^n \times \Delta^1 \rightarrow \Delta^n \stackrel{\sigma}{\rightarrow} X.$$
Otherwise, we use Proposition \ref{goouse} to deduce the existence of the dotted arrow $h'$
in the diagram
$$ \xymatrix{ (\Delta^n \times \{0\}) \coprod_{ \bd \Delta^n \times \{0\} } (\bd \Delta^n \times \Delta^1)
\ar[rrrr]^{ (\tau, h_{< \alpha}) } \ar@{^{(}->}[d] & & & & X \ar[d]^{p} \\
\Delta^n \times \Delta^1 \ar[rrrr]^{p \circ \sigma} \ar@{-->}[urrrr]^{h_0} & & & & S. }$$
Let $\tau' = h'| \Delta^n \times \{1\}$. Then $\tau' | \bd \Delta^n$ factors through
$X'$. It follows that there is a homotopy $h'': \Delta^{n} \times \Delta^{ \{1,2\} } \rightarrow
X$ from $\tau'$ to $\tau''$, which is over $S$ and relative to $\bd \Delta^n$, and such that
$\tau''$ factors through $X'$. Now consider the diagram
$$ \xymatrix{ ( \Delta^n \times \Lambda^2_1 ) \coprod_{ \bd \Delta^n \times \Lambda^2_1 }
(\bd \Delta^n \times \Delta^2 ) \ar[rrrr]^{H_0} \ar@{^{(}->}[d] & & & & X \ar[d]^{p} \\
\Delta^n \times \Delta^2 \ar[rrrr] \ar@{-->}[urrrr]^{H} & & & &  S}$$
where $H_0|\Delta^n \times \Delta^{ \{0,1\} } = h'$, $H_0 | \Delta^n \times \Delta^{ \{1,2\} } =
h''$, and $H_0 | \bd \Delta^n \times \Delta^2$ is given by the composition
$$ \bd \Delta^n \times \Delta^2 \rightarrow \bd \Delta^n \times \Delta^1 \stackrel{h_{<\alpha}}{\rightarrow} X.$$
Using the fact that $p$ is an inner fibration, we deduce that there exists a dotted
arrow $H$ rendering the diagram commutative. We may now define
$h_{\alpha}|\Delta^n \times \Delta^1 = H|\Delta^n \times \Delta^{ \{0,2\} }$; it is easy to see that
this extension has all the desired properties.

We now prove $(4)$. Using Proposition \ref{tulky}, we can reduce to the case where $T = \Delta^n$. 
Without loss of generality, we can replace $S$ by $T = \Delta^n$, so that $X$ and $X'$ are $\infty$-categories. The above constructions show that $r: X \rightarrow X'$ is a homotopy inverse of the inclusion $i: X' \rightarrow X$, so that $i$ is an equivalence as desired.
\end{proof}

We conclude by recording a property of minimal $\infty$-categories which makes them very useful for certain applications. 

\begin{proposition}\label{minstrict}
Let $\calC$ be a minimal $\infty$-category, and let $\sigma: \Delta^n \rightarrow \calC$
be an $n$-simplex of $\calC$ such that $\sigma | \Delta^{ \{i, i+1\} } = \id_{C}: C \rightarrow C$
is a degenerate edge. Then $\sigma = s_i \sigma_0$ for some $\sigma_0 : \Delta^{n-1} \rightarrow \calC$.
\end{proposition}

\begin{proof}
We work by induction on $n$. Let $\sigma_0 = d_{i+1} \sigma$ and let $\sigma' = s_i \sigma_0$. 
We will prove that $\sigma = \sigma'$. Our first goal is to prove that $\sigma | \bd \Delta^n = \sigma' | \bd \Delta^n$; in other words, that $d_j \sigma = d_j \sigma'$ for $0 \leq j \leq n$.
If $j= i+1$ this is obvious; if
$j \notin \{ i, i+1\}$ then it follows from the inductive hypothesis. Let us consider the case
$i = j$, and set $\sigma_1 = d^i \sigma$. We need to prove that $\sigma_0 = \sigma_1$. The argument above establishes that $\sigma_0 | \bd \Delta^{n-1} = \sigma_1 | \bd \Delta^{n-1}$.
Since $\calC$ is minimal, it will suffice to show that $\sigma_0$ and $\sigma_1$ are homotopic relative to $\bd \Delta^{n-1}$. We now observe that 
$$( s_{n-1} \sigma_0, s_{n-2} \sigma_0, \ldots, s_{i+1} \sigma_0, \sigma, 
s_{i-1} \sigma_1, \ldots, s_0 \sigma_1)$$
provides the desired homotopy $\Delta^{n-1} \times \Delta^1 \rightarrow \calC$.

Since $\sigma$ and $\sigma'$ coincide on $\bd \Delta^n$, to prove that $\sigma = \sigma'$ it will suffice to prove that $\sigma$ and $\sigma'$ are homotopic relative to $\bd \Delta^{n}$.
We now observe that
$$ ( s_n \sigma', \ldots, s_{i+2} \sigma', s_{i} \sigma', s_{i} \sigma, s_{i-1} \sigma, \ldots, s_0 \sigma)$$
is a homotopy $\Delta^{n} \times \Delta^1 \rightarrow \calC$ with the desired properties.
\end{proof}

We can interpret Proposition \ref{minstrict} as asserting that in a minimal $\infty$-category
$\calC$, composition is ``strictly unital''. For example, in the special case where $n=2$ and $i=1$, Proposition \ref{minstrict} asserts that if $f: X \rightarrow Y$ is a morphism in an $\infty$-category $\calC$, then $f$ is the {\em unique} composition $\id_{Y} \circ f$.

\subsection{$n$-Categories}\label{ncats}

The theory of $\infty$-categories can be regarded as a generalization of classical category theory: 
if $\calC$ is an ordinary category, then its nerve $\Nerve(\calC)$ is an $\infty$-category which determines $\calC$ up to canonical isomorphism. Moreover, Proposition \ref{ruko} provides a precise characterization of those $\infty$-categories which can be obtained from ordinary categories. In this section, we will explain how to specialize the theory of $\infty$-categories to obtain a theory of $n$-categories, for every nonnegative integer $n$. (However, the ideas described here are appropriate for describing only those $n$-categories which have only invertible $k$-morphisms, for every $k \geq 2$.) 

Before we can give the appropriate definition, we need to introduce a bit of terminology.
Let $f,f': K \rightarrow \calC$ be two diagrams in an $\infty$-category $\calC$, and suppose
that $K' \subseteq K$ is a simplicial subset such that $f | K' = f'| K' = f_0$. We will say that
$f$ and $f'$ are {\it homotopic relative to $K'$} if they are equivalent when viewed as objects of the $\infty$-category $\Fun(K,\calC) \times_{ \Fun(K',\calC)} \{ f_0\}$. Equivalently, $f$ and $f'$
are homotopic relative to $K'$ if there exists a homotopy
$$ h: K \times \Delta^1 \rightarrow \calC$$\index{gen}{homotopy!relative to $K' \subseteq K$}
with the following properties:
\begin{itemize}
\item[$(i)$] The restriction $h | K' \times \Delta^1$ coincides with the composition
$$ K' \times \Delta^1 \rightarrow K' \stackrel{ f_0 }{\rightarrow} \calC. $$
\item[$(ii)$] The restriction $h | K \times \{0\}$ coincides with $f$.
\item[$(iii)$] The restriction $h | K \times \{1\}$ coincides with $f'$.
\item[$(iv)$] For every vertex $x$ of $K$, the restriction
$h | \{x\} \times \Delta^1$ is an equivalence in $\calC$.
\end{itemize}

We observe that if $K'$ contains every vertex of $K$, then condition $(iv)$ follows from condition $(i)$.

\begin{definition}\label{ncat}\index{gen}{$n$-category}
Let $\calC$ be a simplicial set and $n \geq -1$ an integer. We will say that
$\calC$ is an {\it $n$-category} if it is an $\infty$-category and the following additional conditions are satisfied:
\begin{itemize}
\item[$(1)$] Given a pair of maps $f, f': \Delta^n \rightarrow \calC$, if
$f$ and $f'$ are homotopic relative to $\bd \Delta^n$, then $f = f'$.

\item[$(2)$] Given $m > n$ and a pair of maps $f,f': \Delta^m \rightarrow \calC$, if
$f | \bd \Delta^m = f' | \bd \Delta^m$, then $f = f'$.

\end{itemize}
\end{definition}

It is sometimes convenient to extend Definition \ref{ncat} to the case where $n = -2$: we will say that
a simplicial set $\calC$ is a {\it $(-2)$-category} if it is a final object of $\sSet$: in other words, if it is isomorphic to $\Delta^0$.

\begin{example}\label{minuscat}
Let $\calC$ be a $(-1)$-category. Using condition $(2)$ of Definition \ref{ncat}, one shows by induction on $m$ that $\calC$ has at most one $m$-simplex. Consequently, we see that up to isomorphism there are precisely two $(-1)$-categories: $\Delta^{-1} \simeq \emptyset$ and $\Delta^0$.
\end{example}

\begin{example}\label{0catdef}
Let $\calC$ be a $0$-category, and let $X= \calC_0$ denote the set of objects of $\calC$.
Let us write $x \leq y$ if there is a morphism $\phi$ from $x$ to $y$ in $\calC$. Since $\calC$ is an $\infty$-category, this relation is reflexive and transitive. Moreover, condition $(2)$ of Definition \ref{ncat} guarantees that the morphism $\phi$ is unique if it exists. If $x \leq y$ and $y \leq x$, it follows that the morphisms relating $x$ and $y$ are mutually inverse equivalences. Condition $(1)$ then implies that $x = y$. 
We deduce that $(X, \leq)$ is a partially ordered set. It follows from Proposition \ref{huka} below that the map $\calC \rightarrow \Nerve(X)$ is an isomorphism. 

Conversely, it is easy to see that the nerve of any partially ordered set $(X, \leq)$ is a $0$-category in the sense of Definition \ref{ncat}. Consequently, the full subcategory of $\sSet$ spanned by the $0$-categories is equivalent to the category of partially ordered sets.
\end{example}

\begin{remark}\label{slurpper}
Let $\calC$ be an $n$-category, and let $m > n+1$. Then the restriction map 
$$\theta: \Hom_{\sSet}( \Delta^m, \calC) \rightarrow \Hom_{\sSet}( \bd \Delta^m, \calC)$$
is bijective.
If $n=-1$, this is clear from Example \ref{minuscat}; let us therefore suppose that $n \geq 0$, so that $m \geq 2$. The injectivity of $\theta$ follows immediately from part $(2)$ of Definition \ref{ncat}. 
To prove the surjectivity, we consider an arbitrary map $f_0: \bd \Delta^m \rightarrow \calC$.
Let $f: \Delta^m \rightarrow \calC$ be an extension of $f_0 | \Lambda^m_1$ (which exists
since $\calC$ is an $\infty$-category, and $0 < 1 < m$). Using condition $(2)$ again, we
deduce that $\theta(f) = f_0$.
\end{remark}

The following result shows that, in the case where $n=1$, Definition \ref{ncat} recovers the usual definition of a category:

\begin{proposition}\label{huka}
Let $S$ be a simplicial set. The following conditions are equivalent:
\begin{itemize}
\item[$(1)$] The unit map $u: S \rightarrow \Nerve(\h{S})$ is an isomorphism of simplicial sets.
\item[$(2)$] There exists a small category $\calC$ and an isomorphism $S \simeq \Nerve(\calC)$ of simplicial sets.
\item[$(3)$] The simplicial set $S$ is a $1$-category.
\end{itemize}
\end{proposition}

\begin{proof}
The implications $(1) \Rightarrow (2) \Rightarrow (3)$ are clear. Let us therefore assume that $(3)$ holds, and show that $f: S \rightarrow \Nerve(\h{S})$ is an isomorphism. We will prove, by induction on $n$, that the map $u$ is bijective on $n$-simplices. 

For $n=0$, this is clear. If $n=1$, the surjectivity of $u$ obvious. To prove the injectivity, we note that if $f(\phi) = f(\psi)$, then the edges $\phi$ and $\psi$ are homotopic in $S$. A simple application of condition $(2)$ of Definition \ref{ncat} then shows that $\phi = \psi$.

Now suppose $n > 1$. The injectivity of $u$ on $n$-simplices follows from condition $(3)$ of Definition \ref{ncat}, and the injectivity of $u$ on $(n-1)$-simplices. To prove the surjectivity, let us suppose given a map $s: \Delta^n \rightarrow \Nerve(\h{S})$. Choose $0 < i < n$. Since
$u$ is bijective on lower-dimensional simplices, the map $s| \Lambda^n_i$ factors uniquely
through $S$. Since $S$ is an $\infty$-category, this factorization extends to a map
$\widetilde{s}: \Delta^n \rightarrow S$. Since $\Nerve(\h{S})$ is the nerve of a category, a pair of maps from $\Delta^n$ into $\Nerve(\h{S})$ which agree on $\Lambda^n_i$ must be the same. We deduce that $u \circ \widetilde{s} = s$, and the proof is complete.
\end{proof}

\begin{remark}
The condition that an $\infty$-category $\calC$ be an $n$-category is not invariant under categorical equivalence. For example, if $\calD$ is a category with several objects, all of which are uniquely isomorphic to one another, then $\Nerve(\calD)$ is categorically equivalent to $\Delta^0$, but is not a $(-1)$-category. Consequently, there can be no intrinsic characterization of the class of $n$-categories itself. Nevertheless, there does exist a convenient description for the class of $\infty$-categories which are {\em equivalent} to $n$-categories; see Proposition \ref{tokerp}.
\end{remark}

Our next goal is to establish that the class of $n$-categories is stable under the formation of functor categories.
In order to do so, we need to reformulate Definition \ref{ncat} in a more invariant manner.
Recall that for any simplicial set $X$, the {\it $n$-skeleton} $\sk^n X$ is defined to be the simplicial subset of $X$ generated by all the simplices of $X$ having dimension $\leq n$.\index{not}{sknX@$\sk^n X$}\index{gen}{skeleton}

\begin{proposition}\label{ncatchar}
Let $\calC$ be an $\infty$-category and $n \geq -1$. The following are equivalent:
\begin{itemize}
\item[$(1)$] The $\infty$-category $\calC$ is an $n$-category.
\item[$(2)$] For every simplicial set $K$ and
every pair of maps $f,f': K \rightarrow \calC$ such that
$f | \sk^{n} K$ and $f'|\sk^{n} K$ are homotopic relative to
$\sk^{n-1} K$, we have $f = f'$.
\end{itemize}
\end{proposition}

\begin{proof}
The implication $(2) \Rightarrow (1)$ is obvious. Suppose that $(1)$ is satisfied and let
$f,f': K \rightarrow \calC$ be as in the statement of $(2)$. To prove that
$f = f'$ it suffices to show that $f$ and $f'$ agree on
every nondegenerate simplex of $K$. We may therefore reduce to the case where
$K = \Delta^m$. We now work by induction on $m$. If $m < n$, there is nothing to prove.
In the case $m=n$, the assumption that $\calC$ is an $n$-category immediately implies that $f=f'$. If $m > n$, the inductive hypothesis implies that $f| \bd \Delta^m = f'| \bd \Delta^m$, so that
$(1)$ implies that $f = f'$.
\end{proof}

\begin{corollary}\label{zooka}\index{gen}{$n$-categories!and functor categories}
Let $\calC$ be an $n$-category and $X$ a simplicial set. Then $\Fun(X,\calC)$ is an $n$-category.
\end{corollary}

\begin{proof}
Proposition \ref{tyty} asserts that $\Fun(X,\calC)$ is an $\infty$-category. 
We will show that $\Fun(X,\calC)$ satisfies condition $(2)$ of Proposition \ref{ncatchar}.
Suppose given a pair of maps $f,f': K \rightarrow \Fun(X,\calC)$ such that
$f| \sk^{n} K$ and $f'| \sk^{n} K$ are homotopic relative to $f| \sk^{n-1} K$. We wish to show
that $f = f'$. We may identify $f$ and $f'$ with maps $F,F': K \times X \rightarrow \calC$.
Since $\calC$ is an $n$-category, to prove that $F = F'$ it suffices to show that
$F | \sk^{n}(K \times X)$ and $F' | \sk^{n}(K \times X)$ are homotopic relative to
$\sk^{n-1}(K \times X)$. This follows at once, since
$\sk^{p}(K \times X) \subseteq (\sk^{p} K) \times X$ for every integer $p$.
\end{proof}

When $n=1$, Proposition \ref{ruko} asserts that the class of $n$-categories can be characterized by the uniqueness of certain horn fillers. We now prove a generalization of this result.

\begin{proposition}\label{ruko2}
Let $n \geq 1$, and let $\calC$ be an $\infty$-category. Then $\calC$ is an $n$-category if and only if it satisfies the following condition:
\begin{itemize}
\item For every $m > n$ and every diagram
$$ \xymatrix{ \Lambda^m_i \ar[r]^{f_0} \ar@{^{(}->}[d] & \calC \\
\Delta^m, \ar@{-->}[ur]^{f} & }$$
where $0 < i < m$, there exists a {\em unique} dotted arrow $f$ as indicated, which renders the diagram commutative.
\end{itemize}
\end{proposition}

\begin{proof}
Suppose first that $\calC$ is an $n$-category. Let $f,f': \Delta^m \rightarrow \calC$
be two maps with $f | \Lambda^m_i = f' | \Lambda^m_i$, where $0 < i < m$ and
$m > n$. We wish to prove that $f = f'$. Since $\Lambda^m_i$ contains the $(n-1)$-skeleton of
$\Delta^m$, it will suffice (by Proposition \ref{ncatchar}) to show that $f$ and $f'$
are homotopic relative to $\Lambda^m_i$. This follows immediately from the fact that the inclusion $\Lambda^m_i \subseteq \Delta^m$ is a categorical equivalence.

Now suppose that every map $f_0: \Lambda^m_i \rightarrow \calC$, where
$0 < i < m$ and $n < m$, extends uniquely to an $m$-simplex of $\calC$.
We will show that $\calC$ satisfies conditions $(1)$ and $(2)$ of Definition \ref{ncat}.
Condition $(2)$ is obvious: if $f,f': \Delta^m \rightarrow \calC$ are two maps which
coincide on $\bd \Delta^m$, then they coincide on $\Lambda^m_1$ and are
therefore equal to one another (here we use the fact that $m > 1$ because of our assumption that $n \geq 1$).
Condition $(1)$ is a bit more subtle. Suppose that $f,f': \Delta^n \rightarrow \calC$ are
homotopic via a homotopy $h: \Delta^n \times \Delta^1 \rightarrow \calC$
which is constant on $\bd \Delta^n \times \Delta^1$. For $0 \leq i \leq n$, let
$\sigma_i$ denote the $(n+1)$-simplex of $\calC$ obtained by composing
$h$ with the map
$$ [n+1] \rightarrow [n] \times [1]$$
$$ j \mapsto \begin{cases} (j,0) & \text{if } j \leq i \\
(j-1,1) & \text{if } j > i. \end{cases}$$
If $i < n$, then we observe that $\sigma_i | \Lambda^{n+1}_{i+1}$ is equivalent
to the restriction $( s_i d_i \sigma_i ) | \Lambda^{n+1}_{i+1}$. Applying our hypothesis, we conclude that $\sigma_i = s_i d_i \sigma_i$, so that $d_i \sigma_i = d_{i+1} \sigma_i$. A dual argument
establishes the same equality for $0 < i$. Since $n > 0$, we conclude that
$d_i \sigma_i = d_{i+1} \sigma_i$ for all $i$. Consequently, we have a chain of equalities
$$ f' = d_0 \sigma_0 = d_1 \sigma_0 = d_1 \sigma_1 = d_2 \sigma_1 = \ldots
= d_{n} \sigma_n = d_{n+1} \sigma_n = f$$
so that $f' = f$, as desired.
\end{proof}

\begin{corollary}\label{ncatsliceee}\index{gen}{$n$-categories!and overcategories}
Let $\calC$ be an $n$-category and let $p: K \rightarrow \calC$ be a diagram.
Then $\calC_{/p}$ is an $n$-category.
\end{corollary}

\begin{proof}
If $n \leq 0$, this follows easily from Examples \ref{minuscat} and \ref{0catdef}. We may therefore suppose that $n \geq 1$. Proposition \ref{gorban3} implies that $\calC_{/p}$ is an $\infty$-category.
According to Proposition \ref{ruko2}, it suffices to show that for every
$m > n$, $0 < i < m$, and every map $f_0: \Lambda^m_i \rightarrow \calC_{/p}$, there
exists a {\em unique} map $f: \Delta^m \rightarrow \calC_{/p}$ extending $f$.
Equivalently, we must show that there is a unique map $g$ rendering the diagram
$$ \xymatrix{ \Lambda^m_i \star K \ar@{^{(}->}[d] \ar[r]^{g_0} & \calC \\
\Delta^m \star K \ar@{-->}[ur]^{g} & }$$
commutative. The existence of $g$ follows from the fact that $\calC_{/p}$ is an $\infty$-category.
Suppose that $g': \Delta^m \star K \rightarrow \calC$ is another map which extends $g_0$.
Proposition \ref{ruko} implies that $g' | \Delta^m = g | \Delta^m$. We conclude that
$g$ and $g'$ coincide on the $n$-skeleton of $\Delta^m \star K$. Since $\calC$ is an $n$-category, we deduce that $g = g'$ as desired.
\end{proof}

We conclude this section by introducing a construction which allows us to pass from an arbitrary $\infty$-category $\calC$ to its ``underlying'' $n$-category, by discarding information about morphisms of order $> n$. In the case where $n=1$, we have already introduced the relevant construction: we simply replace $\calC$ by its homotopy category (or, more precisely, the nerve of its homotopy category).\index{gen}{$n$-category!underlying an $\infty$-category}

\begin{notation}
Let $\calC$ be an $\infty$-category and let $n \geq 1$. 
For every simplicial set $K$, let $[ K, \calC ]_{n} \subseteq \Fun( \sk^{n} K, \calC )$
be the subset consisting of those diagrams $\sk^n K \rightarrow \calC$ which extend
to the $(n+1)$-skeleton of $K$ (in other words, the image of the restriction map
$\Fun( \sk^{n+1} K, \calC) \rightarrow \Fun( \sk^n K, \calC)$). 
We define an equivalence relation $\sim$ on $[K, \calC]_{n}$ as follows: given two 
maps $f,g: \sk^n K \rightarrow \calC$, we write $f \sim g$ if $f$ and $g$ are homotopic
relative to $\sk^{n-1} K$.
\end{notation}

\begin{proposition}\label{undern}\index{not}{hncalC@$\hn{n}{\calC}$}
Let $\calC$ be an $\infty$-category and $n \geq 1$.
\begin{itemize}
\item[$(1)$] There exists a simplicial set $\hn{n}{\calC}$ with the following universal mapping property: $\Fun(K, \hn{n}{ \calC}) = [K, \calC]_{n} / \sim$.
\item[$(2)$] The simplicial set $\hn{n}{\calC}$ is an $n$-category.
\item[$(3)$] If $\calC$ is an $n$-category, then the natural map $\theta: \calC \rightarrow \hn{n}{\calC}$ is an isomorphism.
\item[$(4)$] For every $n$-category $\calD$, composition with $\theta$ induces an isomorphism of simplicial sets
$$ \psi: \Fun( \hn{n}{\calC}, \calD) \rightarrow \Fun( \calC, \calD). $$
\end{itemize}
\end{proposition}

\begin{proof}
To prove $(1)$, we begin by defining $\hn{n}{\calC}([m]) = [ \Delta^m, \calC ]_{n}/ \sim$, so that the desired universal property holds by definition whenever $K$ is a simplex. Unwinding the definitions, to check the universal property for a general simplicial set $K$ we must verify the following fact:
\begin{itemize}
\item[$(\ast)$] Given two maps $f,g: \bd \Delta^{n+1} \rightarrow \calC$ which are homotopic relative to
$\sk^{n-1} \Delta^{n+1}$, if $f$ extends to an $(n+1)$-simplex of $\calC$, then $g$ extends
to an $(n+1)$-simplex of $\calC$. 
\end{itemize} 
This follows easily from Proposition \ref{princex}.

We next show that $\hn{n}{\calC}$ is an $\infty$-category. Let
$\eta_0: \Lambda^m_i \rightarrow \hn{n}{\calC}$ be a morphism, where $0 < i < m$. We wish to show that $\eta_0$ extends to an $m$-simplex $\eta: \Delta^m \rightarrow \calC$. If $m \leq n+2$, then
$\Lambda^m_i = \sk^{n+1} \Lambda^m_i$, so that $\eta_0$ can be written as a composition
$$ \Lambda^m_i \rightarrow \calC \stackrel{\theta}{\rightarrow} \hn{n}{\calC}.$$
The existence of $\eta$ now follows from our assumption that $\calC$ is an $\infty$-category.
If $m > n+2$, then $\Hom_{\sSet}( \Lambda^m_i, \hn{n}{ \calC}) \simeq \Hom_{\sSet}( \Delta^m, \hn{n}{ \calC})$ by construction, so there is nothing to prove.

We next prove that $\hn{n}{ \calC}$ is an $n$-category. It is clear from the construction that for $m > n$, any two $m$-simplices of $\hn{n}{\calC}$ with the same boundary must coincide. Suppose next that we are given two maps $f,f': \Delta^n \rightarrow \hn{n}{ \calC}$ which are homotopic relative to
$\bd \Delta^n$. Let $F: \Delta^n \times \Delta^1 \rightarrow \hn{n}{\calC}$ be a homotopy from
$f$ to $f'$. Using $(\ast)$, we deduce that $F$ is the image under $\theta$ of a map $\widetilde{F}: \Delta^n \times \Delta^1 \rightarrow \hn{n}{\calC}$, where $\widetilde{F}| \bd \Delta^n \times \Delta^1$ factors through the projection $\bd \Delta^n \times \Delta^1 \rightarrow \bd \Delta^n$. Since $n > 0$, we conclude that $\widetilde{F}$ is a homotopy from $\widetilde{F} | \Delta^n \times \{0\}$ to $\widetilde{F} | \Delta^n \times \{1\}$, so that $f = f'$. This completes the proof of $(2)$.

To prove $(3)$, let us suppose that $\calC$ is an $n$-category; we prove by induction on $m$ that the map $\calC \rightarrow \hn{n}{\calC}$ is bijective on $m$-simplices. For $m < n$, this is clear. When $m = n$ it follows from part $(1)$ of Definition \ref{ncat}. When $m = n+1$, surjectivity follows from the construction of $h_n \calC$, and injectivity from part $(2)$ of Definition \ref{ncat}.
For $m > n+1$, we have a commutative diagram
$$ \xymatrix{ \Hom_{\sSet}( \Delta^m, \calC ) \ar[r] \ar[d] & \Hom_{\sSet}( \Delta^m, \hn{n}{ \calC}) \ar[d] \\
\Hom_{\sSet}( \bd \Delta^m, \calC ) \ar[r] & \Hom_{\sSet}( \bd \Delta^m, \hn{n}{\calC}) }$$
where the bottom horizontal map is an isomorphism by the inductive hypothesis, the
left vertical map is an isomorphism by construction, and the right vertical map is an isomorphism by Remark \ref{slurpper}; it follows that the upper horizontal map is an isomorphism as well.

To prove $(4)$, we observe that if $\calD$ is an $n$-category, then the composition
$$ \Fun( \calC, \calD) \rightarrow \Fun( \hn{n}{ \calC}, \hn{n}{ \calD})
\simeq \Fun( \hn{n}{ \calC}, \calD )$$
is an inverse to $\phi$, where the second isomorphism is given by $(3)$.
\end{proof}

\begin{remark}
The construction of Proposition \ref{undern} does not quite work if $n \leq 0$, since there
may exist equivalences in $h_n \calC$ which do not arise from equivalences in $\calC$. 
However, it is a simple matter to give an alternative construction in these cases which
satisfies conditions $(2)$, $(3)$, and $(4)$; we leave the details to the reader.
\end{remark}

\begin{remark}
In the case $n=1$, the $\infty$-category $\hn{1}{\calC}$ constructed in Proposition \ref{undern}
is isomorphic to the nerve of the homotopy category $\h{\calC}$.
\end{remark}

We now apply the theory of minimal $\infty$-categories (\S \ref{minin}) to obtain a characterization of the class of $\infty$-categories which are {\em equivalent} to $n$-categories. First, we need a definition from classical homotopy theory.

\begin{definition}\label{trunckan}\index{gen}{truncated!space}
Let $k \geq -1$ be an integer. A Kan complex $X$ is {\it $k$-truncated} if, for every $i > k$ and every point $x \in X$, we have $$ \pi_{i}(X,x) \simeq \ast.$$
By convention, we will also say that $X$ is {\it $(-2)$-truncated} if $X$ is contractible.
\end{definition}

\begin{remark}
If $X$ and $Y$ are homotopy equivalent Kan complexes, then $X$ is $k$-truncated if and only if $Y$ is $k$-truncated. In other words, we may view $k$-truncatedness as a condition on objects in the homotopy category $\calH$ of spaces.
\end{remark}

\begin{example}
A Kan complex $X$ is $(-1)$-truncated if it is either empty or contractible. It is $0$-truncated if the
natural map $X \rightarrow \pi_0 X$ is a homotopy equivalence (equivalently, $X$
is $0$-truncated if it is homotopy equivalent to a discrete space).
\end{example}

\begin{proposition}\label{tokerp}\index{gen}{$n$-category}
Let $\calC$ be an $\infty$-category and $n \geq -1$. The following conditions are equivalent:
\begin{itemize}
\item[$(1)$] There exists a minimal model
$\calC' \subseteq \calC$ such that $\calC'$ is an $n$-category.
\item[$(2)$] There exists a categorical equivalence $\calD \rightarrow \calC$, where
$\calD$ is an $n$-category.
\item[$(3)$] For every pair of objects $X,Y \in \calC$, the mapping space
$\bHom_{\calC}(X,Y) \in \calH$ is $(n-1)$-truncated.
\end{itemize}
\end{proposition}

\begin{proof}
It is clear that $(1)$ implies $(2)$. Suppose next that $(2)$ is satisfied; we will prove $(3)$. Without loss of generality, we may replace $\calC$ by $\calD$ and thereby assume that $\calC$ is an $n$-category. If $n=-1$, the desired result follows immediately from Example \ref{minuscat}.
Choose $m \geq n$ and an element $\eta \in \pi_m( \bHom_{\calC}(X,Y), f)$.
We can represent $\eta$ by a commutative diagram of simplicial sets
$$ \xymatrix{ \bd \Delta^m \ar@{^{(}->}[d] \ar[r] & \{f\} \ar[d] \\
\Delta^m \ar[r]^-{s} & \Hom^{\rght}_{\calC}(X,Y). }$$
We can identify $s$ with a map $\Delta^{m+1} \rightarrow \calC$ whose restriction to
$\bd \Delta^{m+1}$ is specified. Since $\calC$ is an $n$-category, the inequality
$m+1 > n$ shows that $s$ is uniquely determined. This proves that $\pi_m( \bHom_{\calC}(X,Y), f) \simeq \ast$, so that $(3)$ is satisfied.

To prove that $(3)$ implies $(1)$, it suffices to show that if $\calC$ is a {\em minimal} $\infty$-category which satisfies $(3)$, then $\calC$ is an $n$-category. We must show that the conditions of Definition \ref{ncat} are satisfied. The first of these conditions follows immediately from the assumption that $\calC$ is minimal. For the second, we must show that if $m > n$ and $f,f': \bd \Delta^m \rightarrow \calC$ are such that $f| \bd \Delta^m = f'| \bd \Delta^m$, then $f=f'$.
Since $\calC$ is minimal, it suffices to show that $f$ and $f'$ are homotopic relative
to $\bd \Delta^m$. We will prove that there is a map $g: \Delta^{m+1} \rightarrow \calC$
such that $d_{m+1} g = f$, $d_m g = f'$, and $d_i g = d_i s_m f = d_i s_m f'$ for $0 \leq i < m$.
Then the sequence $(s_0 f, s_1 f, \ldots, s_{m-1} f, g)$ determines a map
$\Delta^m \times \Delta^1 \rightarrow \calC$ which gives the desired homotopy between
$f$ and $f'$ (relative to $\bd \Delta^m$).

To produce the map $g$, it suffices to solve the lifting problem depicted in the diagram
$$ \xymatrix{ \bd \Delta^{m+1} \ar[r]^{ g } \ar@{^{(}->}[d] & \calC \\
\Delta^{m+1} \ar@{-->}[ur]. & }$$
Choose a fibrant simplicial category $\calD$ and an equivalence
of $\infty$-categories $\calC \rightarrow \Nerve(\calD)$.
According to Proposition \ref{princex}, it will suffice to prove that we can solve 
the associated lifting problem
$$ \xymatrix{ \sCoNerve[ \bd \Delta^{m+1}] \ar[r]^{G_0} \ar@{^{(}->}[d] & \calD \\
\sCoNerve[ \Delta^{m+1}] \ar@{-->}[ur]^{G}. }$$ 
Let $X$ denote the initial vertex of $\Delta^{m+1}$, considered as an object of
$\sCoNerve[\bd \Delta^{m+1}]$, and $Y$ the final vertex. Note that $G_0$
determines a map
$$ e_0: \bd (\Delta^1)^m \simeq \bHom_{ \sCoNerve[ \bd \Delta^{m+1}]}(X,Y) \rightarrow
\bHom_{\calD}(G_0(X), G_0(Y))$$
and that giving the desired extension $G$ is equivalent to extending $e_0$ to a map
$$ e: (\Delta^1)^m \simeq \bHom_{ \sCoNerve[ \Delta^{m+1}]}(X,Y) \rightarrow
\bHom_{ \calD}(G_0(X), G_0(Y)).$$
The obstruction to constructing $e$ lies in 
$\pi_{m-1}(\bHom_{\calD}(G_0(X), G_0(Y)), p)$ for an appropriately chosen base point
$p$. Since $(m-1) > (n-1)$, condition $(3)$ implies that this homotopy set is trivial, so that
the desired extension can be found.
\end{proof}

\begin{corollary}\index{gen}{truncated!space}
Let $X$ be a Kan complex. Then $X$ is $($categorically$)$ equivalent to an $n$-category if and only if it is $n$-truncated.
\end{corollary}

\begin{proof}
For $n=-2$ this is obvious. If $n \geq -1$, this follows from characterization $(3)$ of Proposition \ref{tokerp} and the following observation: a Kan complex $X$ is $n$-truncated if and only if, for every pair of vertices $x,y \in X_0$, the Kan complex
$$\{x\} \times_{X}  X^{\Delta^1} \times_{X} \{y\}$$
of paths from $x$ to $y$ is $(n-1)$-truncated.
\end{proof}

\begin{corollary}\label{zook}
Let $\calC$ be an $\infty$-category and $K$ a simplicial set. Suppose that, for every pair of objects
$C,D \in \calC$, the space $\bHom_{\calC}(C,D)$ is $n$-truncated. Then the
$\infty$-category $\Fun(K,\calC)$ has the same property.
\end{corollary}

\begin{proof}
This follows immediately from Proposition \ref{tokerp} and Corollary \ref{zooka}, since the functor
$$ \calC \mapsto \Fun(K,\calC)$$ preserves categorical equivalences between $\infty$-categories.
\end{proof}
