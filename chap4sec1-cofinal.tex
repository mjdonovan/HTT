% !TEX root = highertopoi.tex
 \section{Cofinality}\label{chap3cofinal}
 
 
\setcounter{theorem}{0}


Let $\calC$ be an $\infty$-category, and let $p: K \rightarrow \calC$ be a diagram in $\calC$ indexed by a simplicial set $K$. In \S \ref{limitcolimit} we introduced the definition of a {\em colimit} $\injlim(p)$ for the diagram $p$. In practice, it is often possible to replace $p$ by a simpler diagram without changing the colimit $\injlim(p)$. In this section, we will introduce a general formalism which will allow us to make replacements of this sort: the theory of {\em cofinal} maps between simplicial sets. We begin in \S \ref{cofinal} with a definition of the class of cofinal maps, and show (Proposition \ref{gute}) that if a map $q: K' \rightarrow K$ is cofinal, then there is an equivalence
$\injlim(p) \simeq \injlim(p \circ q)$ (provided that either colimit exists). In \S \ref{smoothness} we will reformulate the definition of cofinality, using the formalism of contravariant model categories (\S \ref{contrasec}). We conclude in \S \ref{quillA} by establishing an important recognition criterion for cofinal maps, in the special case where $K$ is an $\infty$-category. This result can be regarded as a refinement of Quillen's ``Theorem A''. 
 
\subsection{Cofinal Maps}\label{cofinal}

The goal of this section is to introduce the definition of a cofinal map $p: S \rightarrow T$ of simplicial sets, and study the basic properties of this notion. Our main result is Proposition \ref{gute}, which characterizes cofinality in terms of the behavior of $T$-indexed colimits.

\begin{definition}[Joyal \cite{joyalnotpub}]\index{gen}{cofinal}
Let $p: S \rightarrow T$ be a map of simplicial sets. We shall
say that $p$ is {\it cofinal} if, for any right fibration $X
\rightarrow T$, the induced map of of simplicial sets
$$ \bHom_{T}(T,X) \rightarrow \bHom_{T}(S, X)$$
is a homotopy equivalence.
\end{definition}

\begin{remark}
The simplicial set $\bHom_{T}(S,X)$ parametrizes sections of the right fibration $X \rightarrow T$.
It may be described as the fiber of the induced map $X^{S} \rightarrow T^{S}$ over the vertex of $T^S$ corresponding to the map $p$. Since $X^{S} \rightarrow T^{S}$ is a right fibration, the fiber $\bHom_{T}(S,X)$ is a Kan complex. Similarly, $\bHom_{T}(T,X)$ is a Kan complex.
\end{remark}

We begin by recording a few simple observations about the class of cofinal maps:

\begin{proposition}\label{cofbasic}
\begin{itemize}
\item[$(1)$] Any isomorphism of simplicial sets is cofinal.

\item[$(2)$] Let $f: K \rightarrow K'$ and $g: K' \rightarrow K''$ be
maps of simplicial sets. Suppose that $f$ is cofinal. Then $g$ is cofinal if and only if $g \circ f$ is cofinal.

\item[$(3)$] If $f: K \rightarrow K'$ is a cofinal map between simplicial
sets, then $f$ is a weak homotopy equivalence.

\item[$(4)$] An inclusion $i: K \subseteq K'$ of simplicial sets is
cofinal if and only if it is right anodyne.
\end{itemize}
\end{proposition}

\begin{proof}
Assertions $(1)$ and $(2)$ are obvious. We prove $(3)$. Let $S$ be a Kan complex.
Since $f$ is cofinal, the composition
$$ \bHom_{\sSet}(K',S) = \bHom_{K}(K', S \times K) \rightarrow \bHom_{K}(K,S \times
K) = \bHom_{\sSet}(K,S)$$ is a homotopy equivalence. Passing to connected components, we deduce that $K$ and $K'$ co-represent the same functor in the homotopy category $\calH$ of spaces. It follows that $f$ is a weak homotopy equivalence, as desired.

We now prove $(4)$. Suppose first that $i$ is right-anodyne. Let
$X \rightarrow K'$ be a right fibration. Then the induced map $\Hom_{K'}(K',X)
\rightarrow \Hom_{K'}(K,X)$ is a trivial fibration, and in
particular a homotopy equivalence.

Conversely, suppose that $i$ is a cofinal inclusion of simplicial sets. 
We wish to show that $i$ has the left
lifting property with respect to any right fibration. In other
words, we must show that given any diagram of solid arrows
$$ \xymatrix{ K \ar@{^{(}->}[d] \ar[r]^{s} & X \ar[d] \\
K' \ar@{=}[r] \ar@{-->}[ur] & K', }$$
for which the right-vertical map is a right fibration, there exists a dotted arrow as indicated, rendering the diagram commutative. Since $i$ is cofinal, the map $s$ is homotopic
to a map which extends over $K'$. In other words, there exists a map
$$ s': (K \times \Delta^1) \coprod_{ K \times \{1\} } (K'
\times \{1\}) \rightarrow X,$$
compatible with the projection to $K'$, such that $s'| K \times \{0\}$ coincides with $s$.
Since the inclusion $$ (K \times \Delta^1) \coprod_{K \times \{1\} } (K' \times \{1\}) \subseteq
K' \times \Delta^1$$ is right-anodyne, there exists a map $s'': K' \times \Delta^1 \rightarrow
X$ which extends $s'$, and is compatible with the projection to $K'$. The map
$s'' | K \times \{0\}$ has the desired properties.
\end{proof}

\begin{warning}
The class of cofinal maps does {\em not} satisfy the
``two-out-of-three'' property. If $f: K \rightarrow K'$ and $g: K'
\rightarrow K''$ are such that $g \circ f$ and $g$ are cofinal,
then $f$ need not be cofinal.
\end{warning}

Our next goal is to establish an alternative characterization of cofinality, in terms of the behavior of colimits (Proposition \ref{gute}). First, we need a lemma.

\begin{lemma}\label{cogh}
Let $\calC$ be an $\infty$-category, and let $p: K \rightarrow \calC$, $q: K'
\rightarrow \calC$ be diagrams. Define simplicial sets $M$ and $N$ by the
formulas
$$ \Hom(X,M) = \{ f: (X \times K) \star K' \rightarrow \calC :
f|(X \times K) = p \circ \pi_{K}, f|K' = q \}$$
$$ \Hom(X,N) = \{ g: K \star (X \times K') \rightarrow \calC :
f|K = p, f|(X \times K') = q \circ \pi_{K'} \}. $$ Here $\pi_K$
and $\pi_{K'}$ denote the projection from a product to the factor
indicated by the subscript.

Then $M$ and $N$ are Kan complexes, which are (naturally) homotopy
equivalent to one another.
\end{lemma}

\begin{proof}
We define a simplicial set $\calD$ as follows. For every finite, nonempty, linearly ordered set $J$, to give a map $\Delta^{J} \rightarrow \calD$ is to supply the following data:

\begin{itemize}
\item A map $\Delta^{J} \rightarrow \Delta^1$, corresponding to a decomposition of
$J$ as a disjoint union $J_{-} \coprod J_{+}$, where $J_{-} \subseteq J$ is closed downwards and
$J_{+} \subseteq J$ is closed upwards.

\item A map $e: (K \times \Delta^{J_{-}}) \star (K' \times
\Delta^{J_{+}}) \rightarrow \calC$ such that $e| K \times
\Delta^{J_{-}} = p \circ \pi_K$ and $e| K' \times \Delta^{J_{+}} =
q \circ \pi_{K'}$.
\end{itemize}

We first claim that $\calD$ is an $\infty$-category. Fix a finite linearly ordered set $J$ as above, and let $j \in J$ be neither the largest nor the smallest element of $J$. Let 
$f_0: \Lambda^J_{j} \rightarrow \calD$ be any map; we wish to show that there exists a map
$f: \Delta^J \rightarrow \calD$ which extends $f_0$. We first observe that the induced projection
$\Lambda^J_{j} \rightarrow \Delta^1$ extends {\em uniquely} to $\Delta^J$ (since $\Delta^1$ is isomorphic to the nerve of a category). Let $J = J_{-} \coprod J_{+}$ be the induced decomposition of $J$. Without loss of generality, we may suppose that $j \in J_{-}$. In this case, we may identify
$f_0$ with a map
$$ (( K \times \Lambda^{J_{-}}_j ) \star (K' \times \Delta^{J_+} ))
\coprod_{ (K \times \Lambda^{J_-}_j ) \star (K' \times \bd \Delta^{J_+}) }
(( K \times \Delta^{J_{-}} ) \star (K' \times \bd \Delta^{J_+})) \rightarrow 
\calC$$
and our goal is to find an extension 
$$f: ( K \times \Delta^{J_{-}} ) \star (K' \times \Delta^{J_+} ) \rightarrow \calC.$$
Since $\calC$ is an $\infty$-category, it will suffice to show that the inclusion
$$ (( K \times \Lambda^{J_{-}}_j ) \star (K' \times \Delta^{J_+} ))
\coprod_{ (K \times \Lambda^{J_-}_j ) \star (K' \times \bd \Delta^{J_+}) }
(( K \times \Delta^{J_{-}} ) \star (K' \times \bd \Delta^{J_+})) \subseteq 
 ( K \times \Delta^{J_{-}} ) \star (K' \times \Delta^{J_+} )$$ is inner anodyne.
According to Lemma \ref{precough}, it suffices to check that the inclusion $K \times \Lambda^{J_{-}}_j \subseteq K \times \Delta^{J_{-}}$ is 
right anodyne. This follows from Corollary \ref{prodprod1}, since $\Lambda^{J_{-}}_j \subseteq \Delta^{J_{-}}$ is right anodyne.

The $\infty$-category $\calD$ has just two objects, which we will denote by
$x$ and $y$. We observe that $M = \Hom^{\rght}_{\calD}(x,y)$ and $N = \Hom^{\lft}_{\calD}(x,y)$.
Proposition \ref{gura} implies that $M$ and $N$ are Kan complexes.
Propositions \ref{babyy} and \ref{wiretrack} imply each these Kan complexes is weakly homotopy equivalent to $\bHom_{ \sCoNerve[\calD]}(x,y)$, so that $M$ and $N$ are homotopy equivalent to one another as desired.
\end{proof}

\begin{remark}\label{coughi}
In the situation of Lemma \ref{cogh}, the homotopy equivalence
between $M$ and $N$ is furnished by the composition of a chain of weak homotopy
equivalences
$$ M \leftarrow |M|_{Q^{\bigdot}} \rightarrow
\Hom_{\sCoNerve[\calD]}(x,y) \leftarrow |N|_{Q^{\bigdot}} \rightarrow
N,$$ which is functorial in the triple $(\calC,p: K \rightarrow \calC,q: K' \rightarrow \calC)$.
\end{remark}

\begin{proposition}\label{coughing}
Let $v: K' \rightarrow K$ be a cofinal map and $p: K \rightarrow \calC$
a diagram in an $\infty$-category $\calC$. Then the map $\phi: \calC_{p/}
\rightarrow \calC_{pv/}$ is an equivalence of left fibrations
over $\calC$: in other words, it induces a homotopy equivalence of Kan
complexes after passing to the fiber over every object $x$ of $\calC$.
\end{proposition}

\begin{proof}
We wish to prove that the map
$$ \calC_{p/} \times_{\calC} \{x\} \rightarrow \calC_{pv/} \times_{\calC} \{x\}$$
is a homotopy equivalence of Kan complexes. Lemma \ref{cogh} implies that the left
hand side is homotopy equivalent $\bHom_{\calC}(K, \calC_{/x})$. Similarly, the right hand
side can be identified with $\bHom_{\calC}(K', \calC_{/x})$. Using the functoriality implicit in the proof of Lemma \ref{cogh} (see Remark \ref{coughi}), it suffices to show that the restriction map
$$ \bHom_{\calC}(K, \calC_{/x}) \rightarrow \bHom_{\calC}(K', \calC_{/x})$$ is a homotopy equivalence. Since $v$ is cofinal, this follows immediately from the fact that the projection
$\calC_{/x} \rightarrow \calC$ is a right fibration.
\end{proof}

\begin{proposition}\label{gute}
Let $v: K' \rightarrow K$ be a map of (small) simplicial sets. The following conditions are equivalent:
\begin{itemize}
\item[$(1)$] The map $v$ is cofinal.
\item[$(2)$] Given any $\infty$-category $\calC$ and any diagram $p: K \rightarrow \calC$, the induced map $\calC_{p/} \rightarrow \calC_{p'/}$ is an equivalence of $\infty$-categories, where $p' = p \circ v$.
\item[$(3)$] For every $\infty$-category $\calC$ and every diagram $\overline{p}: K^{\triangleright} \rightarrow \calC$ which is a colimit of $p = \overline{p}|K$, the induced map $\overline{p}': {K'}^{\triangleright} \rightarrow \calC$
is a colimit of $p' = \overline{p}'|K'$.
\end{itemize}
\end{proposition}

\begin{proof}
Suppose first that $(1)$ is satisfied. Let $p: K \rightarrow \calC$ be as in $(2)$. Proposition \ref{coughing} implies that the induced map $\calC_{p/} \rightarrow \calC_{p'/}$ induces a homotopy equivalence of Kan complexes, after passing to the fiber over any object of $\calC$. Since
both $\calC_{p/}$ and $\calC_{p'/}$ are left-fibered over $\calC$, Corollary \ref{usefir} implies that
$\calC_{p/} \rightarrow \calC_{p'/}$ is a categorical equivalence. This proves that $(1) \Rightarrow (2)$.

Now suppose that $(2)$ is satisfied, and let $\overline{p}: K^{\triangleright} \rightarrow \calC$
be as in $(3)$. Then we may identify $\overline{p}$ with an initial object of the $\infty$-category
$\calC_{p/}$. The induced map $\calC_{p/} \rightarrow \calC_{p'/}$ is an equivalence, and therefore carries the initial object $\overline{p}$ to an initial object $\overline{p}'$ of $\calC_{p'/}$; thus
$\overline{p}'$ is a colimit of $p'$. This proves that $(2) \Rightarrow (3)$.

It remains to prove that $(3) \Rightarrow (1)$. For this, we make use of the theory of 
classifying right fibrations (\S \ref{universalfib}). Let $X \rightarrow K$ be a right fibration. We wish to show that composition with $v$ induces a homotopy equivalence $\bHom_{K}(K,X) \rightarrow \bHom_{K}(K',X)$. It will suffice to prove this result after replacing $X$ by any equivalent right fibration. Let $\SSet$ denote the $\infty$-category of spaces. According to Corollary \ref{unipull}, there is a classifying map $p: K \rightarrow \SSet^{op}$ and an equivalence of right fibrations between $X$ and $(\SSet_{\ast/})^{op} \times_{\SSet^{op}} K$, where $\ast$ denotes a final object
of $\SSet$.

The $\infty$-category $\SSet$ admits small limits (Corollary \ref{limitsinmodel}). It follows that there exists a map
$\overline{p}: K^{\triangleright} \rightarrow \SSet^{op}$ which is a colimit of $p = \overline{p}|K$. Let $x$
denote the image in $\SSet$ of the cone point of $K^{\triangleright}$. Let $\overline{p}': {K'}^{\triangleright} \rightarrow \SSet^{op}$ be the induced map. Then, by hypothesis, $\overline{p}'$ is a colimit of 
$p' = \overline{p}'|K'$. According to Lemma \ref{cogh}, there is a (natural) chain of weak homotopy equivalences relating $\bHom_{K}(K,X)$ with $(\SSet^{op})_{p/} \times_{ \SSet^{op}} \{y\}$. 
Similarly, there is a chain of weak homotopy equivalences connecting $\bHom_{K}(K',X)$ with
$(\SSet^{op})_{p'/} \times _{\SSet^{op}} \{y\}$. Consequently, we are reduced to proving that the left vertical map in the diagram
$$ \xymatrix{ (\SSet^{op})_{p/} \times_{\SSet^{op}} \{y\} \ar[d] & (\SSet^{op})_{\overline{p}/} \times_{\SSet^{op}} \{y\} \ar[l] \ar[r] \ar[d] & (\SSet^{op})_{x/} \times_{ \SSet^{op} } \{y\} \ar[d] \\
(\SSet^{op})_{p'/} \times_{\SSet^{op}} \{y\} & (\SSet^{op})_{\overline{p}'/} \times_{\SSet^{op}} \{y\} \ar[l] \ar[r]  & (\SSet^{op})_{x/} \times_{ \SSet^{op} } \{y\} } $$
is a homotopy equivalence. Since $\overline{p}$ and $\overline{q}$ are colimits of $p$ and $q$, the left horizontal maps are trivial fibrations. Since the inclusions of the cone points into $K^{\triangleright}$ and
${K'}^{\triangleright}$ are right anodyne, the right horizontal maps are also trivial fibrations.
It therefore suffices to prove that the right vertical map is a homotopy equivalence. But this map is an isomorphism of simplicial sets.
\end{proof}

\begin{corollary}\label{stoog}
Let $p: K \rightarrow K'$ be a map of simplicial sets, and
$q: K' \rightarrow K''$ a categorical equivalence. Then $p$
is cofinal if and only if $q \circ p$ is cofinal. In particular, $($taking $p = \id_{S'}${}$)$
$q$ itself is cofinal.
\end{corollary}

\begin{proof}
Let $\calC$ be an $\infty$-category, $r'': K'' \rightarrow \calC$ a diagram, and
set $r' = r'' \circ q$, $r = r' \circ p$. Since $q$ is a categorical equivalence, $\calC_{r''/}
\rightarrow \calC_{r'/}$ is a categorical equivalence. It follows that $\calC_{r/} \rightarrow \calC_{r''/}$ is a categorical equivalence if and only if $\calC_{r/} \rightarrow \calC_{r'/}$ is a categorical equivalence. We now apply the characterization $(2)$ of Proposition \ref{gute}.
\end{proof}

\begin{corollary}\label{cofinv}
The property of cofinality is homotopy invariant. In other words,
if two maps $f,g: K \rightarrow K'$ have the same image in the
homotopy category of $\sSet$ obtained by inverting all categorical
equivalences, then $f$ is cofinal if and only if $g$ is cofinal.
\end{corollary}

\begin{proof}
Choose a categorical equivalence $K' \rightarrow \calC$, where $\calC$ is an $\infty$-category.
In view of Corollary \ref{stoog}, we may replace $K'$ by $\calC$ and thereby assume that
$K'$ is itself an $\infty$-category. Since $f$ and $g$ are homotopic, there exists a cylinder
object $S$ equipped with a trivial fibration $p: S
\rightarrow K$, a map $q: S \rightarrow \calC$, and two sections
$s,s': K \rightarrow S$ of $p$, such that $f = q \circ s$, $g = q
\circ s'$. Since $p$ is a categorical equivalence, so is every
section of $p$. Consequently, $s$ and $s'$ are cofinal. We now
apply Proposition \ref{cofbasic} to deduce that $f$ is cofinal if
and only if $q$ is cofinal. Similarly, $g$ is cofinal if and only
if $q$ is cofinal.
\end{proof}

\begin{corollary}\label{twork}
Let $p: X \rightarrow S$ be a map of simplicial sets. The following are equivalent:
\begin{itemize}
\item[$(1)$] The map $p$ is a cofinal right fibration.
\item[$(2)$] The map $p$ is a trivial fibration.
\end{itemize}
\end{corollary}

\begin{proof}
Clearly any trivial fibration is a right fibration. Furthermore, any trivial fibration is
a categorical equivalence, hence cofinal by Corollary \ref{stoog}. Thus $(2)$ implies $(1)$. Conversely, suppose that $p$ is a cofinal right fibration. Since $p$ is cofinal, the natural map
$\bHom_{S}(S,X) \rightarrow \bHom_{S}(X,X)$ is a homotopy equivalence of Kan complexes.
In particular, there exists a section $f: S \rightarrow X$ of $p$ such that
$f \circ p$ is (fiberwise) homotopic to the identity map of $X$. Consequently, for each
vertex $s$ of $S$, the fiber $X_{s} = X \times_{S} \{s\}$ is a contractible Kan complex
(since the identity map $X_{s} \rightarrow X_{s}$ is homotopic to the constant map with value $f(s)$). The dual of Lemma \ref{toothie} now shows that $p$ is a trivial fibration.
\end{proof}

\begin{corollary}\label{stufe}
A map $X \rightarrow Z$ of simplicial sets is cofinal if and only
if it admits a factorization $$X \stackrel{f}{\rightarrow} Y
\stackrel{g}{\rightarrow} Z,$$ where $X \rightarrow Y$ is
right-anodyne and $Y \rightarrow Z$ is a trivial fibration.
\end{corollary}

\begin{proof}
The ``if'' direction is clear: if such a factorization exists,
then $f$ is cofinal (since it is right anodyne), $g$ is cofinal
(since it is a categorical equivalence), and consequently $g \circ
f$ is cofinal (since it is a composition of cofinal maps).

For the ``only if'' direction, let us suppose that $X \rightarrow
Z$ is a cofinal map. By the small object argument (Proposition \ref{quillobj}), there is a
factorization $$X \stackrel{f}{\rightarrow} Y
\stackrel{g}{\rightarrow} Z$$ where $f$ is right-anodyne and $g$
is a right fibration. The map $g$ is cofinal by Proposition \ref{cofbasic}, and therefore a trivial fibration by Corollary \ref{twork}.
\end{proof}

\begin{corollary}\label{prodcofinal}
Let $p: S \rightarrow S'$ be a cofinal map, and $K$ any simplicial
set. Then the induced map $K \times S \rightarrow K \times S'$ is
cofinal.
\end{corollary}

\begin{proof}
Using Corollary \ref{stufe}, we may suppose that $p$ is either
right anodyne or a trivial fibration. Then the induced map $K \times S \rightarrow K \times S'$ has the same property.
\end{proof}

\subsection{Smoothness and Right Anodyne Maps}\label{smoothness}

In this section, we explain how to characterize the classes of right anodyne and cofinal morphisms in terms of the contravariant model structures studied in \S \ref{contrasec}. We also introduce a third class of maps between simplicial sets, which we call {\it smooth}.

We begin with the following characterization of right anodyne maps:

\begin{proposition}\label{hunef}\index{gen}{right anodyne}
Let $i: A \rightarrow B$ be a map of simplicial sets. The following conditions are equivalent:
\begin{itemize}
\item[$(1)$] The map $i$ is right anodyne.
\item[$(2)$] For any map of simplicial sets $j: B \rightarrow C$, the map $i$ is a trivial cofibration with respect
to the contravariant model structure on $(\sSet)_{/C}$.
\item[$(3)$] The map $i$ is a trivial cofibration with respect to the contravariant model structure on
$(\sSet)_{/B}$.
\end{itemize}
\end{proposition}

\begin{proof}
The implication $(1) \Rightarrow (2)$ follows immediately from Proposition \ref{onehalf}, and the implication $(2) \Rightarrow (3)$ is obvious. Suppose that $(3)$ holds. To prove $(1)$, it suffices to show that given any diagram
$$ \xymatrix{ A \ar@{^{(}->}[d]^{i} \ar[r] & X \ar[d]^p \\
B \ar[r] \ar@{-->}[ur]^{f} & Y }$$
such that $p$ is a right fibration, one can supply the dotted arrow $f$ as indicated. Replacing
$p: X \rightarrow Y$ by the pullback $X \times_{Y} B \rightarrow B$, we may reduce to the case where
$Y = B$. Corollary \ref{usewhere1} implies that $X$ is a fibrant object of $(\sSet)_{/B}$ (with respect to contravariant model structure) so that the desired map $f$ can be found.
\end{proof}

\begin{corollary}\label{nonobcomp}
Suppose given maps $A \stackrel{i}{\rightarrow} B \stackrel{j}{\rightarrow} C$ of simplicial
sets. If $i$ and $j \circ i$ are right anodyne, and $j$ is a cofibration, then $j$ is right-anodyne.
\end{corollary}

\begin{proof}
By Proposition \ref{hunef}, $i$ and $j \circ i$ are contravariant equivalences in $(\sSet)_{/C}$. It follows that $j$ is a trivial cofibration in $(\sSet)_{/C}$, so that $j$ is right anodyne (by Proposition \ref{hunef} again).
\end{proof}

\begin{corollary}\label{anothernonob}
Let $$ \xymatrix{ A' \ar[d]^{f'} & A \ar[r] \ar[l]^{u} \ar[d]^{f} & A'' \ar[d]^{f''} \\
B' & B \ar[r] \ar[l]^{v} & B'' }$$
be a diagram of simplicial sets. Suppose that $u$ and $v$ are monomorphisms,
and that $f, f'$, and $f''$ are right anodyne. Then the induced map
$$ A' \coprod_{A} A'' \rightarrow B' \coprod_{B} B''$$
is right anodyne.
\end{corollary}

\begin{proof}
According to Proposition \ref{hunef}, each of the maps $f$, $f'$, and $f''$ is a contravariant
equivalence in $(\sSet)_{/B' \coprod_{B} B''}$. The assumption on $u$ and $v$ guarantees that
$f' \coprod_{f} f''$ is also a contravariant equivalence in $(\sSet)_{/B' \coprod_{B} B''}$, so that
$f' \coprod_{f} f''$ is right anodyne by Proposition \ref{hunef} again.
\end{proof}

\begin{corollary}\label{filtanodyne}
The collection of right anodyne maps of simplicial sets is stable under filtered colimits.
\end{corollary}

\begin{proof}
Let $f: A \rightarrow B$ be a filtered colimit of right anodyne morphisms $f_{\alpha}: A_{\alpha} \rightarrow B_{\alpha}$. According to Proposition \ref{hunef}, each $f_{\alpha}$ is a contravariant equivalence in $(\sSet)_{/B}$. Since contravariant equivalences are stable under filtered colimits, we conclude that $f$ is a contravariant equivalence in $(\sSet)_{/B}$ so that $f$ is right anodyne by Proposition \ref{hunef}.
\end{proof}

Proposition \ref{hunef} has an analogue for cofinal maps:

\begin{proposition}\label{huneff}
Let $i: A \rightarrow B$ be a map of simplicial sets. The following conditions are equivalent:
\begin{itemize}
\item[$(1)$] The map $i$ cofinal.
\item[$(2)$] For any map $j: B \rightarrow C$, the inclusion $i$ is a contravariant
equivalence in $(\sSet)_{/C}$.
\item[$(3)$] The map $i$ is a contravariant equivalence in
$(\sSet)_{/B}$.
\end{itemize}
\end{proposition}

\begin{proof}
Suppose $(1)$ is satisfied. By Corollary \ref{stufe}, $i$ admits a factorization as a right anodyne map followed by a trivial fibration. Invoking Proposition \ref{hunef}, we conclude that $(2)$ holds. 
The implication $(2) \Rightarrow (3)$ is obvious. If $(3)$ holds, then we can choose a factorization
$$ A \stackrel{i'}{\rightarrow} A' \stackrel{i''}{\rightarrow} B$$ of $i$, where $i'$ is right anodyne and $i''$ is a right fibration. Then $i''$ is a contravariant fibration (in $\sSet_{/B}$) and a contravariant weak equivalence, and is therefore a trivial fibration of simplicial sets. 
We now apply Corollary \ref{stufe} to conclude that $i$ is cofinal.
\end{proof}

\begin{corollary}\label{weakcont}
Let $p: X \rightarrow S$ be a map of simplicial sets, where $S$ is a Kan complex. Then
$p$ is cofinal if and only if it is a weak homotopy equivalence.
\end{corollary}

\begin{proof}
By Proposition \ref{huneff}, $p$ is cofinal if and only if it is a contravariant equivalence
in $(\sSet)_{/S}$. If $S$ is a Kan complex, then Proposition \ref{strstr} asserts that
the contravariant equivalences are precisely the weak homotopy equivalences.
\end{proof}

Let $p: X \rightarrow Y$ be an arbitrary map of simplicial sets. In \S \ref{contrasec} we showed that $p$ induces a Quillen adjunction $(p_{!}, p^{\ast})$ between the contravariant model categories
$(\sSet)_{/X}$ and $(\sSet)_{/Y}$. The functor $p^{\ast}$ itself has a right adjoint, which we will denote by $p_{\ast}$; it is given by
$$ p_{\ast}(M) =  \bHom_{Y}(X,M).$$
The adjoint functors $(p^{\ast}, p_{\ast})$ are not Quillen adjoints in general. Instead we have:

\begin{proposition}\label{smoothdef}
Let $p: X \rightarrow Y$ be a map of simplicial sets. The following conditions are equivalent:

\begin{itemize}
\item[$(1)$] For any right-anodyne map $i: A \rightarrow B$ in $(\sSet)_{/Y}$, the induced map
$A \times_Y X \rightarrow B \times_{Y} X$ is right-anodyne.

\item[$(2)$] For every Cartesian diagram
$$\xymatrix{ X' \ar[r] \ar[d]^{p'} & X \ar[d]^{p} \\
Y' \ar[r] & Y, },$$ the functor ${p'}^{\ast}: (\sSet)_{/Y'} \rightarrow (\sSet)_{/X'}$ preserves contravariant equivalences. 

\item[$(3)$] For every Cartesian diagram
$$\xymatrix{ X' \ar[r] \ar[d]^{p'} & X \ar[d]^{p} \\
Y' \ar[r] & Y, },$$ the adjoint functors $( {p'}^{\ast}, p'_{\ast})$ give rise to a Quillen adjunction between the contravariant model categories $(\sSet)_{/Y'}$ and $(\sSet)_{/X'}$. 
\end{itemize}

\end{proposition}

\begin{proof}
Suppose that $(1)$ is satisfied; let us prove $(2)$. Since property $(1)$ is clearly stable under base change, we may suppose that $p' = p$. Let $u: M \rightarrow N$ be a contravariant equivalence in
$(\sSet)_{/Y}$. If $M$ and $N$ are fibrant, then $u$ is a homotopy equivalence, so that $p^{\ast}(u): p^{\ast} M \rightarrow p^{\ast} N$ is also a homotopy equivalence. In the general case, we may select a diagram
$$ \xymatrix{ M \ar[r]^i \ar[d]^{u} & M' \ar[d] \ar[dr]^{v} \\
N \ar[r]^-{i'} & N \coprod_{M} M' \ar[r]^-{j} &  N' } $$
where $M'$ and $N'$ are fibrant, and the maps $i$ and $j$ are right anodyne (and therefore $i'$ is also right anodyne). Then $p^{\ast}(v)$ is a contravariant equivalence, while the maps
$p^{\ast}(i)$, $p^{\ast}(j)$, and $p^{\ast}(i')$ are all right anodyne; by Proposition \ref{hunef} they are contravariant equivalences as well. It follows that $p^{\ast}(u)$ is a contravariant equivalence.

To prove $(3)$, it suffices to show that ${p'}^{\ast}$ preserves cofibrations and trivial cofibrations. The first statement is obvious, and the second follows immediately from $(2)$. Conversely the existence of a Quillen adjunction $({p'}^{\ast}, p_{\ast})$ implies that ${p'}^{\ast}$ preserves contravariant equivalences between cofibrant objects. Since every object of $(\sSet)_{/Y'}$ is cofibrant, we deduce that $(3)$ implies $(2)$. 

Now suppose that $(2)$ is satisfied, and let $i: A \rightarrow B$ be a right-anodyne map in $(\sSet)_{/Y}$ as in $(1)$. Then $i$ is a contravariant equivalence in $(\sSet)_{/B}$. Let $p': X \times_{Y} B \rightarrow B$ be base change of $p$; then $(2)$ implies that the induced map
$i': {p'}^{\ast} A \rightarrow {p'}^{\ast} B$ is a contravariant equivalence in $(\sSet)_{/B \times_{Y} X}$. By Proposition \ref{hunef}, the map $i'$ is right anodyne. Now we simply note that $i'$ may be identified with the map $A \times_Y X \rightarrow B \times_{Y} X$ in the statement of $(1)$.
\end{proof}

\begin{definition}\index{gen}{smooth}
We will say that a map $p: X \rightarrow Y$ of simplicial sets is {\em smooth} if it satisfies the (equivalent) conditions of Proposition \ref{smoothdef}.
\end{definition}

\begin{remark}\label{gonau}
Let 
$$ \xymatrix{ X' \ar[d] \ar[r]^{f'} & X \ar[d]^{p} \\
S' \ar[r]^{f} & S }$$
be a pullback diagram of simplicial sets. Suppose that $p$ is smooth and that $f$ is cofinal. Then $f'$ is cofinal: this follows immediately from characterization $(2)$ of Proposition \ref{smoothdef} and characterization $(3)$ of Proposition \ref{huneff}.
\end{remark}

We next give an alternative characterization of smoothness. Let
$$ \xymatrix{ X' \ar[d]^{p'} \ar[r]^{q'} & X \ar[d]^{p} \\
Y' \ar[r]^{q} & Y }$$
be a Cartesian diagram of simplicial sets. Then we obtain an isomorphism
$R {p'}^{\ast} R q^{\ast} \simeq R {q'}^{\ast} R p^{\ast}$ of right-derived functors, which induces
a natural transformation
$$ \psi_{p,q}: L q'_{!} R {p'}^{\ast} \rightarrow R p^{\ast} L q_{!}.$$

\begin{proposition}\label{smoothbase}
Let $p: X \rightarrow Y$ be a map of simplicial sets. The following conditions are equivalent:
\begin{itemize}
\item[$(1)$] The map $p$ is smooth.
\item[$(2)$] For every Cartesian rectangle
$$ \xymatrix{ X'' \ar[d]^{p''} \ar[r]^{q'} & X' \ar[d]^{p'} \ar[r] & X \ar[d]^{p} \\
Y'' \ar[r]^{q} & Y' \ar[r] & Y, }$$ the natural transformation
$\psi_{p',q}$ is an isomorphism of functors from the homotopy category of
$(\sSet)_{/Y''}$ to the homotopy category of $(\sSet)_{/X'}$ (here all categories are
endowed with the contravariant model structure).
\end{itemize}
\end{proposition}

\begin{proof}
Suppose that $(1)$ is satisfied, and consider any Cartesian rectangle as in $(2)$. Since $p$ is smooth, $p'$ and $p''$ are also smooth. It follows that ${p'}^{\ast}$ and ${p''}^{\ast}$ preserve weak equivalences, so they may be identified with their right derived functors. Similarly, $q_{!}$ and $q'_{!}$ preserve weak equivalences, so they may be identified with their left derived functors. Consequently, the natural transformation $\psi_{p',q}$ is simply obtained by passage to the homotopy category from the natural transformation
$$ q'_{!} {p''}^{\ast} \rightarrow {p'}^{\ast} q_{!}.$$
But this is an isomorphism of functors before passage to the homotopy categories.

Now suppose that $(2)$ is satisfied. Let $q: Y'' \rightarrow Y'$ be a right-anodyne map
in $(\sSet)_{/Y}$, and form the Cartesian square as in $(2)$. Let us compute the value of the functors $L q'_{!} R {p''}^{\ast}$ and $R {p'}^{\ast} L q_{!}$ on the object $Y''$ of $(\sSet)_{/Y''}$. The composite $L q'_{!} R {p''}^{\ast}$ is easy: because $Y''$ is fibrant and
$X'' = {p''}^{\ast} Y''$ is cofibrant, the result is $X''$, regarded as an object of $(\sSet)_{/X'}$. The other composition is slightly trickier: $Y''$ is cofibrant, but $q_{!} Y''$ is not fibrant when viewed as an object of $(\sSet)_{/Y'}$. However, in view of the assumption that $q$ is right anodyne, Proposition \ref{hunef} ensures that $Y'$ is a fibrant replacement for $q_{!} Y'$; thus we may identify  $R {p'}^{\ast} L q_{!}$ with the object ${p'}^{\ast} Y' = X'$ of $(\sSet)_{/ X'}$. Condition $(2)$ now implies that the natural map $X'' \rightarrow X'$ is a contravariant equivalence in $(\sSet)_{/X'}$. Invoking Proposition \ref{hunef}, we deduce that $q'$ is right anodyne, as desired.
\end{proof}

\begin{remark}
The terminology ``smooth'' is suggested by the analogy of Proposition \ref{smoothbase} with the {\em smooth base change theorem} in the theory of \'{e}tale cohomology (see, for example, \cite{freitag}). 
\end{remark}

\begin{proposition}\label{usenonob}
Suppose given a commutative diagram
$$ \xymatrix{ X \ar[r]^{i} \ar[dr]^{p} \ar[d] & X' \ar[d]^{p'} \\
X'' \ar[r]^{p''} & S }$$
of simplicial sets. Assume that $i$ is a cofibration, and that $p,p'$, and $p''$ are smooth. Then
the induced map $X' \coprod_{X} X'' \rightarrow S$ is smooth.
\end{proposition}

\begin{proof}
This follows immediately from Corollary \ref{anothernonob} and characterization $(1)$ of Proposition \ref{smoothdef}.
\end{proof}

\begin{proposition}\label{usefiltanodyne}
The collection of smooth maps $p: X \rightarrow S$ is stable under filtered colimits in
$(\sSet)_{/S}$. 
\end{proposition}

\begin{proof}
Combine Corollary \ref{filtanodyne} with characterization $(1)$ of Proposition \ref{smoothdef}.
\end{proof}


\begin{proposition}\label{strokhop}\index{gen}{coCartesian fibration!and smoothness}
Let $p: X \rightarrow S$ be a coCartesian fibration of simplicial sets. Then $p$ is smooth.
\end{proposition}

\begin{proof}
Let $i: B' \rightarrow B$ be a right anodyne map in $(\sSet)_{/S}$; we wish to show that the induced map $B' \times_{S} X \rightarrow B \times_{S} X$ is right anodyne. By general nonsense, we may reduce ourselves to the case where $i$ is an inclusion $\Lambda^n_i \subseteq \Delta^n$ where
$0 <  i \leq n$. Making a base change, we may suppose that $S = B$. By Proposition \ref{simplexplay}, there exists a composable sequence of maps
$$ \phi: A^0 \rightarrow \ldots \rightarrow A^n $$ and a quasi-equivalence
$M^{op}(\phi) \rightarrow X$. Consider the diagram
$$ \xymatrix{ M^{op}(\phi) \times_{\Delta^n} \Lambda^n_i \ar@{^{(}->}[d] \ar[r] \ar[dr]^{f} & 
X \times_{\Delta^n} \Lambda^n_i \ar@{^{(}->}[d]^{h} \\
M^{op}(\phi) \ar[r]^{g} & X }$$
The left vertical map is right-anodyne, since it is a pushout of the inclusion
 $A^0 \times \Lambda^n_i \subseteq A^0 \times \Delta^n$. It follows that $f$ is cofinal, being a composition of a right-anodyne map and a categorical equivalence. Since $g$ is cofinal (being a categorical equivalence) we deduce from Proposition \ref{cofbasic} that $h$ is cofinal. Since
$h$ is a monomorphism of simplicial sets, it is right-anodyne by Proposition \ref{cofbasic}.
\end{proof}

\begin{proposition}\label{longwait5}\index{gen}{bifibration!and smoothness}
Let $p: X \rightarrow S \times T$ be a bifibration. Then the composite map
$\pi_{S} \circ p: X \rightarrow S$ is smooth.
\end{proposition}

\begin{proof}
For every map $T' \rightarrow T$, let $X_{T'} = X \times_{T} T'$. We note that
$X$ is a filtered colimit of $X_{T'}$, as $T'$ ranges over the finite simplicial subsets
of $T$. Using Proposition \ref{usefiltanodyne}, we can reduce to the case where $T$ is finite.
Working by induction on the dimension and the number of nondegenerate simplices of
$T$, we may suppose that $T = T' \coprod_{ \bd \Delta^n } \Delta^n$, where the result is known
for $T'$ and for $\bd \Delta^n$. Applying Proposition \ref{usenonob}, we can reduce to the case $T = \Delta^n$. We now apply Lemma \ref{gork} to deduce that $p$ is a coCartesian fibration, and therefore smooth by Proposition \ref{strokhop}.
\end{proof}

\begin{lemma}\label{covg}
Let $\calC$ be an $\infty$-category containing an object $C$, and let
$f: X \rightarrow Y$ be a covariant equivalence in $(\sSet)_{/\calC}$. The induced map
$$ X \times_{\calC} \calC^{/C} \rightarrow Y \times_{\calC} \calC^{/C}$$ is also a covariant equivalence in $\calC^{/C}$.
\end{lemma}

\begin{proof}
It will suffice to prove that for every object $Z \rightarrow \calC$ of $(\sSet)_{/\calC}$, the fiber
product $Z \times_{\calC} \calC^{/C}$ is a homotopy product of $Z$ with
$\calC^{/C}$ in $(\sSet)_{/\calC}$ (with respect to the covariant model structure). Choose a factorization
$$ Z \stackrel{i}{\rightarrow} Z' \stackrel{j}{\rightarrow} \calC,$$
where $i$ is left anodyne and $j$ is a left fibration. According to Corollary \ref{usewhere1}, we may regard $Z'$ as a fibrant replacement for $Z$ in $(\sSet)_{/\calC}$. It therefore suffices to prove
that the map $i': Z \times_{\calC} \calC^{/C} \rightarrow Z' \times_{\calC} \calC^{/C}$ is a covariant equivalence. According to Proposition \ref{huneff}, it will suffice to prove that $i'$ is left anodyne.
The map $i'$ is a base change of $i$ by the projection $p: \calC^{/C} \rightarrow \calC$; it therefore suffices to prove that $p^{op}$ is smooth. This follows from Proposition \ref{strokhop}, since
$p$ is a right fibration of simplicial sets.
\end{proof}

\begin{proposition}\label{longwait44}
Let $\calC$ be an $\infty$-category, and
$$\xymatrix{ X \ar[rr]^{f} \ar[dr]^{p} & & Y \ar[dl]_{q} \\
& \calC & }$$ be a commutative diagram of simplicial sets. Suppose that
$p$ and $q$ are smooth. The following conditions are equivalent:
\begin{itemize}
\item[$(1)$] The map $f$ is a covariant equivalence in $(\sSet)_{/\calC}$.
\item[$(2)$] For each object $C \in \calC$, the induced map of fibers
$X_{C} \rightarrow Y_{C}$ is a weak homotopy equivalence.
\end{itemize} 
\end{proposition}

\begin{proof}
Suppose that $(1)$ is satisfied, and let $C$ be an object of $\calC$.
We have a commutative diagram of simplicial sets
$$ \xymatrix{ X_{C} \ar[r] \ar[d] & Y_{C} \ar[d] \\
X \times_{\calC} \calC^{/C} \ar[r] & Y \times_{\calC} \calC^{/C}.} $$
Lemma \ref{covg} implies that the bottom horizontal map is a covariant equivalence. The vertical maps are both pullbacks of the right anodyne inclusion 
$ \{ C\} \subseteq \calC^{/C}$ along smooth maps, and are therefore right anodyne. In particular, the vertical arrows and the bottom horizontal arrow are all weak homotopy equivalences; it follows that the map $X_{C} \rightarrow Y_{C}$ is a weak homotopy equivalence as well.

Now suppose that $(2)$ is satisfied. Choose a commutative diagram
$$ \xymatrix{ X \ar[rr]^{f} \ar[d] & & Y \ar[d] \\
X' \ar[rr]^{f'} \ar[dr]^{p'}  & & Y' \ar[dl]^{q'} \\
& \calC & }$$
in $(\sSet)_{/\calC}$, where the vertical arrows are left anodyne and the maps
$p'$ and $q'$ are left fibrations. Using Proposition \ref{strokhop}, we conclude that
$p'$ and $q'$ are smooth. Applying $(1)$, we deduce that for each object $C \in \calC$,
the maps $X_{C} \rightarrow X'_{C}$ and $Y_{C} \rightarrow Y'_{C}$ are weak homotopy equivalences. It follows that each fiber $f'_{C}: X'_{C} \rightarrow Y'_{C}$ is a homotopy equivalence of Kan complexes, so that $f'$ is an equivalence of left fibrations and therefore
a covariant equivalence. Inspecting the above diagram, we deduce that $f$ is also a covariant equivalence, as desired.
\end{proof}

\subsection{Quillen's Theorem A for $\infty$-Categories}\label{quillA}

Suppose that $f: \calC \rightarrow \calD$ is a functor between $\infty$-categories, and that we wish to determine whether or not $f$ is cofinal. According to Proposition \ref{gute}, the cofinality of $f$ is equivalent to the assertion that for any diagram $p: \calD \rightarrow \calE$, $f$ induces an equivalence
$$ \injlim(p) \simeq \injlim(p \circ f).$$
One can always define a morphism
$$ \phi: \injlim(p \circ f) \rightarrow \injlim(p)$$
(provided that both sides are defined); the question is whether or not we can define an inverse $\psi = \phi^{-1}$. Roughly speaking, this involves defining a compatible family of maps
$\psi_{D}: p(D) \rightarrow \injlim(p \circ f)$, indexed by $D \in \calD$. The only reasonable
candidate for $\psi_{D}$ is a composition
$$ p(D) \rightarrow (p \circ f)(C) \rightarrow \injlim(p \circ f),$$
where the first map arises from a morphism $D \rightarrow f(C)$ in $\calC$. Of course, the existence of $C$ is not automatic. Moreover, even if $C$ exists, it may is usually not unique. The collection of candidates for $C$ is parametrized by the $\infty$-category $\calC^{D/} = \calC \times_{\calD} \calD^{D/}$. In order to make the above construction work, we need the $\infty$-category
$\calC^{D/}$ to be weakly contractible. More precisely, we will prove the following result:

\begin{theorem}[Joyal \cite{joyalnotpub}]\label{hollowtt}\index{gen}{Quillen's theorem A!for $\infty$-categories}
Let $f: \calC \rightarrow \calD$ be a map of simplicial sets, where $\calD$ is an $\infty$-category. The following
conditions are equivalent:
\begin{itemize}
\item[$(1)$] The functor $f$ is cofinal.
\item[$(2)$] For every object $D \in \calD$, the simplicial set
$\calC \times_{ \calD } \calD_{D/}$ is weakly contractible.
\end{itemize}
\end{theorem}

We first need to establish the following lemma:

\begin{lemma}\label{trull6prime}
Let $p: U \rightarrow S$ be a Cartesian fibration of simplicial sets. Suppose
that for every vertex $s$ of $S$, the fiber $X_{s} = p^{-1} \{s\}$ is weakly contractible. Then $p$ is cofinal.
\end{lemma}

\begin{proof}
Let $q: N \rightarrow S$ be a right fibration. For every map of simplicial sets $T \rightarrow S$,
let $X_{T} = \bHom_{S}(T,N)$ and $Y_{T} = \bHom_{S}(T \times_{S} U, N)$. Our goal is to prove that the natural map $X_{S} \rightarrow Y_{S}$ is a homotopy equivalence of Kan complexes.
We will prove, more generally, that for any map $T \rightarrow S$, the map
$\phi_{T}: Y_{T} \rightarrow Z_{T}$ is a homotopy equivalence. The proof goes by induction on the
(possibly infinite) dimension of $T$. Choose a transfinite sequence of simplicial
subsets $T(\alpha) \subseteq T$, where each $T(\alpha)$ is obtained from
$T(< \alpha) = \bigcup_{\beta < \alpha} T(\beta)$ by adjoining a single nondegenerate simplex
of $T$ (if such a simplex exists). We prove that $\phi_{T(\alpha)}$ is a homotopy equivalence
by induction on $\alpha$. Assuming that $\phi_{T(\beta)}$ is a homotopy equivalence for every
$\beta < \alpha$, we deduce that $\phi_{T(< \alpha)}$ is the homotopy inverse limit of a tower of equivalences, and therefore a homotopy equivalence. If $T(\alpha) = T(< \alpha)$, we are done. Otherwise, we may write $T(\alpha) = T(< \alpha) \coprod_{ \bd \Delta^n} \Delta^n$. Then
$\phi_{T(\alpha)}$ can be written as a homotopy pullback of $\phi_{T(< \alpha)}$ with
$\phi_{\Delta^n}$ over $\phi_{ \bd \Delta^n}$. The third map is a homotopy equivalence
by the inductive hypothesis. Thus, it suffices to prove that $\phi_{\Delta^n}$ is an equivalence.
In other words, we may reduce to the case $T = \Delta^n$.

By Proposition \ref{simplexplay}, there exists a composable sequence of maps
$$ \theta: A^0 \leftarrow \ldots \leftarrow A^n$$
and a quasi-equivalence $f: M(\theta) \rightarrow X \times_{S} T$, where
$M(\theta)$ denotes the mapping simplex of the sequence $\theta$. 
Given a map $T' \rightarrow T$, we let $Z_{T'} = \bHom_{S}(M(\theta) \times_{T} T', N)$. 
Proposition \ref{funkyfibcatfib} implies that $q$ is a categorical fibration. It follows that, for
any map $T' \rightarrow T$, the categorical equivalence 
$M(\theta) \times_{T} T' \rightarrow U \times_{S} T'$ induces another categorical equivalence
$\psi_{T'} = Y_{T'} \rightarrow Z_{T'}$. Since $Y_{T'}$ and $Z_{T'}$ are Kan complexes, the map $\psi_{T'}$ is a homotopy equivalence. Consequently, to prove that $\phi_{T}$ is an equivalence, it suffices to show that the composite map
$$ X_{T} \rightarrow Y_{T} \rightarrow Z_{T}$$ is an equivalence.

Consider the composition
$$ u: X_{ \Delta^{n-1} } \stackrel{u'}{\rightarrow} Z_{ \Delta^{n-1} } \stackrel{u''}{\rightarrow} \bHom_{S}( \Delta^{n-1} \times A^n, N) \stackrel{u'''}{\rightarrow} \bHom_{S}( \{n-1\} \times A^n, N)$$
Using the fact that $q$ is a right fibration and that $A^n$ is weakly contractible, we deduce that $u$ and $u'''$ are homotopy equivalences. The inductive hypothesis implies that $u'$ is a homotopy equivalence. Consequently, $u''$ is also a homotopy equivalence. 
The space $Z_{T}$ fits into a homotopy Cartesian diagram
$$ \xymatrix{ Z_{T} \ar[r] \ar[d]^{v''} & Z_{\Delta^{n-1}} \ar[d]^{u''} \\
\bHom_{S}( \Delta^n \times A^n,N ) \ar[r] & \bHom_{S}(\Delta^{n-1} \times A^n, N).}$$
It follows that $v''$ is a homotopy equivalence. Now consider the composition
$$ v: X_{\Delta^n} \stackrel{v'}{\rightarrow} Z_{\Delta^n} \stackrel{v''}{\rightarrow}
\bHom_{S}( \Delta^n \times A^n, N) \stackrel{v'''}{\rightarrow} \bHom_{S}( \{n\} \times A^n, N).$$
Again, because $q$ is a right fibration and $A^n$ is weakly contractible, the maps
$v$ and $v'''$ are homotopy equivalences. Since $v''$ is a homotopy equivalence, we deduce
that $v'$ is a homotopy equivalence, as desired.
\end{proof}

\begin{proof}[Proof of Theorem \ref{hollowtt}]
Using the small object argument, we can factor $f$ as a composition
$$ \calC \stackrel{f'}{\rightarrow} \calC' \stackrel{f''}{\rightarrow} \calD$$
where $f'$ is a categorical equivalence and $f''$ is an inner fibration. Then $f''$ is cofinal if and only if $f$ is cofinal (Corollary \ref{cofinv}). For every $D \in \calD$, the map
$\calD_{D/} \rightarrow \calD$ is a left fibration, so the induced map
$\calC_{D/} \rightarrow {\calC'}_{D/}$ is a categorical equivalence (Proposition \ref{basechangefunky}). Consequently, it will suffice to prove that $(1) \Leftrightarrow (2)$ for the morphism $f'': \calC' \rightarrow \calD$. In other words, we may assume that the simplicial set $\calC$ is an $\infty$-category.

Suppose first that $(1)$ is satisfied, and choose $D \in \calD$. The projection
$\calD_{D/} \rightarrow \calD$ is a left fibration, and therefore smooth (Proposition \ref{strokhop}). Applying Remark \ref{gonau}, we deduce that the projection
$\calC \times_{\calD} \calD_{D/} \rightarrow \calD_{D/}$ is cofinal, and therefore a weak homotopy equivalence (Proposition \ref{cofbasic}). Since $\calD_{D/}$ has an initial object, it is weakly contractible. Therefore $\calC \times_{\calD} \calD_{D/}$ is weakly contractible, as desired.

We now prove that $(2) \Rightarrow (1)$.
Let $\calM = \Fun(\Delta^1,\calD) \times_{ \Fun( \{1\}, \calD )} \calC$. Then the map $f$ factors as a composition
$$ \calC \stackrel{f'}{\rightarrow} \calM \stackrel{f''}{\rightarrow} \calD$$
where $f'$ is the obvious map and $f''$ is given by evaluation at the vertex $\{0\} \subseteq
\Delta^1$. Note that there is a natural projection map $\pi: \calM \rightarrow \calC$, that
$f'$ is a section of $\pi$, and that there is a simplicial homotopy
$h: \Delta^1 \times \calM \rightarrow \calM$ from $\id_{\calM}$ to $f' \circ \pi$
which is compatible with the projection to $\calC$. It follows from
Proposition \ref{trull11} that $f'$ is right anodyne.

Corollary \ref{tweezegork} implies that $f''$ is a Cartesian fibration. 
The fiber of $f''$ over an object $D \in \calD$ is isomorphic to $\calC \times_{\calD} \calD^{D/}$, 
which is equivalent to $\calC \times_{\calD} \calD_{D/}$ and therefore weakly contractible (Proposition \ref{certs}). By assumption, the fibers of $f''$ are weakly contractible.
Lemma \ref{trull6prime} asserts that $f''$ is cofinal. It follows that $f$, as a composition of cofinal maps, is also cofinal.
\end{proof}

Using Theorem \ref{hollowtt} we can easily deduce the following classical result of Quillen:

\begin{corollary}[Quillen's Theorem A]\index{gen}{Quillen's theorem A}
Let $f: \calC \rightarrow \calD$ be a functor between ordinary categories. Suppose that, for
every object $D \in \calD$, the fiber product category $\calC \times_{\calD} \calD_{D/}$ has
weakly contractible nerve. Then $f$ induces a weak homotopy equivalence of simplicial sets
$\Nerve(\calC) \rightarrow \Nerve(\calD)$.
\end{corollary}

\begin{proof}
The assumption implies that $\Nerve(f): \Nerve(\calC) \rightarrow \Nerve(\calD)$ satisfies the hypotheses of Theorem \ref{hollowtt}. It follows that $\Nerve(f)$ is a cofinal map of simplicial sets, and therefore a weak homotopy equivalence (Proposition \ref{cofbasic}).
\end{proof}

