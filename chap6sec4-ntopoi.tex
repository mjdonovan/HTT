\section{$n$-Topoi}\label{chap6sec3}

\setcounter{theorem}{0}

Roughly speaking, an ordinary topos is a category which resembles the category of sheaves of {\em sets} on a topological space $X$. In \S \ref{chap6sec1}, we introduced the definition of an $\infty$-topos. In the same rough terms, we can think of an $\infty$-topos as an $\infty$-category which resembles the $\infty$-category of sheaves of $\infty$-groupoids on a topological space $X$. Phrased in this way, it is natural to guess that these two notions have a common generalization. In \S \ref{c631}, we will introduce the notion of an $n$-topos, for every $0 \leq n \leq \infty$. The idea is that an $n$-topos should be an $n$-category which resembles the $n$-category of sheaves of $(n-1)$-groupoids on a topological space $\calX$. Of course, there are many approaches to making this idea precise. Our main result, Theorem \ref{nchar}, asserts that several candidate definitions are equivalent to one another. The proof of Theorem \ref{nchar} will occupy our attention for most of this section. In \S \ref{provengiraud}, we study an axiomatization of the class of $n$-topoi in the spirit of Giraud's theorem, and in \S \ref{provengiraudeasy} we will give a characterization of $n$-topoi based on their descent properties. The case of $n=0$ is somewhat exceptional, and merits special treatement. In \S \ref{0topoi} we will show that a $0$-topos is essentially the same thing as a {\em locale} (a mild generalization of the notion of a topological space). 

Our main motivation for introducing the definition of an $n$-topos is that it allows us to study $\infty$-topoi and topological spaces (or, more generally, $0$-topoi) in the same setting. In \S \ref{nlocalic}, we will introduce constructions which allow us to pass back and forth between $m$-topoi and $n$-topoi, for any $0 \leq m \leq n \leq \infty$. We introduce an $\infty$-category
$\Geo_{n}^{\GeoR}$ of $n$-topoi, for each $n \leq 0$, and show that each
$\Geo_{n}^{\GeoR}$ can be regarded as a {\em localization} of the $\infty$-category
$\RGeom$. In other words, the study of $n$-topoi for $n < \infty$ can be regarded as a special case of the theory of $\infty$-topoi.

\subsection{Characterizations of $n$-Topoi}\label{c631}

In this section, we will introduce the definition of $n$-topos for $0 \leq n < \infty$. In view of Theorem \ref{mainchar}, there are several reasonable approaches to the subject. We will begin with extrinsic approach.

\begin{definition}\label{ntopdefff}\index{gen}{$n$-topos}
Let $0 \leq n < \infty$. An $\infty$-category $\calX$ is an {\it $n$-topos} if there
exists a small $\infty$-category $\calC$ and an (accessible) left exact localization
$$ L: \calP_{\leq n-1}(\calC) \rightarrow \calX,$$
where $\calP_{\leq n-1}(\calC)$ denotes the full subcategory of $\calP(\calC)$ spanned by
the $(n-1)$-truncated objects.\index{not}{Pcalleq@$\calP_{\leq n}(\calC)$}
\end{definition}

\begin{remark}
The accessibility condition on the localization functor $L: \calP_{\leq n-1}(\calC) \rightarrow \calX$ of Definition \ref{ntopdefff} is superfluous: we will show that such left exact localization of $\calP_{\leq n-1}(\calC)$ is automatically accessible (combine Proposition \ref{alltoploc} with Corollary \ref{topaccess}).
\end{remark}

\begin{remark}
An $\infty$-category $\calX$ is a $1$-topos if and only if it is equivalent to the nerve of an ordinary (Grothendieck) topos; this follows immediately from characterization $(B)$ of Proposition \ref{toposdefined}.
\end{remark}

\begin{remark}
Definition \ref{ntopdefff} makes sense also in the case $n=-1$, but is not very interesting. Up to equivalence, there is precisely one $(-1)$-topos: the final $\infty$-category $\ast$.
\end{remark}

Our main goal is to prove the following result:

\begin{theorem}\label{nchar}\index{gen}{Giraud's theorem!for $n$-topoi}
Let $\calX$ be a presentable $\infty$-category and let $0 \leq n < \infty$. The following conditions are equivalent:
\begin{itemize}
\item[$(1)$] There exists a small $n$-category $\calC$ which admits finite limits, a Grothendieck topology on $\calC$, and an equivalence of $\calX$ with the full subcategory of $\Shv_{\leq n-1}(\calC) \subseteq \Shv(\calC)$ consisting of $(n-1)$-truncated objects of $\Shv(\calC)$.

\item[$(2)$] There exists an $\infty$-topos $\calY$ and an equivalence
$\calX \rightarrow \tau_{\leq n-1} \calY$.

\item[$(3)$] The $\infty$-category $\calX$ is an $n$-topos.

\item[$(4)$] Colimits in $\calX$ are universal, $\calX$ is equivalent to an $n$-category, and the class of $(n-2)$-truncated morphisms
in $\calX$ is local $($ see \S \ref{magnet} $)$.

\item[$(5)$] Colimits in $\calX$ are universal, $\calX$ is equivalent to an $n$-category, and for all sufficiently large regular cardinals $\kappa$, there exists an object of $\calX$ which classifies
$(n-2)$-truncated, relatively $\kappa$-compact morphisms in $\calX$.

\item[$(6)$] The $\infty$-category $\calX$ satisfies the following $n$-categorical versions
of Giraud's axioms:

\begin{itemize}\index{gen}{Giraud's axioms!for $n$-topoi}
\item[$(i)$] The $\infty$-category $\calX$ is equivalent to a presentable $n$-category.
\item[$(ii)$] Colimits in $\calX$ are universal.
\item[$(iii)$] If $n > 0$, then coproducts in $\calX$ are disjoint.
\item[$(iv)$] Every $n$-efficient $($see \S \ref{provengiraud}$)$ groupoid object of $\calX$ is effective.
\end{itemize}
\end{itemize}
\end{theorem}

\begin{proof}
The case $n=0$ will be analyzed very explicitly in \S \ref{0topoi}; let us therefore restrict our attention to the case $n > 0$.
The implication $(1) \Rightarrow (2)$ is obvious (take $\calY = \Shv(\calC)$). Suppose
that $(2)$ is satisfied. Without loss of generality, we may suppose that $\calY$ is an (accessible) left exact localization of $\calP(\calC)$ for some small $\infty$-category $\calC$. Then
$\calX$ is a left-exact localization of $\calP_{\leq n-1}(\calC)$, which proves $(3)$.

We next prove the converse $(3) \Rightarrow (2)$. We first observe that
$\calP_{\leq n-1}(\calC) = \Fun( \calC^{op}, \tau_{\leq n-1} \SSet)$. Let $h_n \calC$ be the underlying $n$-category of $\calC$, as in Proposition \ref{undern}. Since $\tau_{\leq n-1} \SSet$
is equivalent to an $n$-category, we conclude that composition with the projection
$\calC \rightarrow \hn{n}{\calC}$ induces an equivalence $\calP_{\leq n-1}(\hn{n}{\calC}) \rightarrow \calP_{\leq n-1}(\calC)$. Consequently, we may assume without loss of generality (replacing $\calC$ by $\hn{n}{\calC}$ if necessary) that there is an accessible left exact localization $L: \calP_{\leq n-1}(\calC) \rightarrow \calX$, where $\calC$ is an $n$-category. Let $S$ be the collection of all morphisms $u$ in $\calP_{\leq n-1}(\calC)$ such that $Lu$ is an equivalence, so that $S$ is of small generation. Let
$\overline{S}$ be the strongly saturated class of morphisms in $\calP(\calC)$ generated by $S$.
We observe that $\tau_{\leq n-1}^{-1}(S)$ is a strongly saturated class of morphisms containing
$S$, so that $\overline{S} \subseteq \tau_{\leq n-1}^{-1}(S)$. It follows that $S^{-1} \calP_{\leq n-1}(\calC)$
is contained in $\calY = \overline{S}^{-1} \calP(\calC)$, and may therefore be identified with the collection of $(n-1)$-truncated objects of $\calY$. To complete the proof, it will suffice to show that $\calY$ is an $\infty$-topos. For this, it will suffice to show that $\overline{S}$ is stable under pullbacks. Let $T$ be the collection of all morphisms $f: X \rightarrow Y$ in $\calP(\calC)$ such that
for every pullback diagram
$$ \xymatrix{ X' \ar[r] \ar[d]^{f'} & X \ar[d]^{f} \\
Y' \ar[r] & Y }$$
the morphism $f'$ belongs to $\overline{S}$. It is easy to see that $T$ is strongly saturated; we wish to show that $T \subseteq \overline{S}$. It will therefore suffice to prove that $S \subseteq T$. Let us therefore fix $f:X \rightarrow Y$ belonging to $S$, and let $\calD$ be the full subcategory of $\calP(\calC)$ spanned by those objects $Y'$ such that for {\em any} pullback diagram
$$ \xymatrix{ X' \ar[r] \ar[d]^{f'} & X \ar[d]^{f} \\
Y' \ar[r] & Y, }$$
$f'$ belongs to $\overline{S}$. Since colimits in $\calP(\calC)$ are universal and $\overline{S}$ is stable under colimits, we conclude that $\calD$ is stable under colimits in $\calP(\calC)$. 
Since $\calP(\calC)$ is generated under colimits by the essential image of the Yoneda embedding
$j: \calC \rightarrow \calP(\calC)$, it will suffice to show that $j(C) \in \calD$ for each $C \in \calC$.
We now observe that $\calP_{\leq n-1}(\calC) \subseteq \calD$ (since $S$ is stable under pullbacks in $\calP_{\leq n-1}(\calC)$), and that $j(C) \in \calP_{\leq n-1}(\calC)$ in virtue of our assumption that $\calC$ is an $n$-category.

The implication $(2) \Rightarrow (4)$ will be established in \S \ref{provengiraudeasy} (Propositions \ref{tigre} and \ref{tigress}).
The proof of Theorem \ref{colimsurt} adapts without change to show that $(4) \Leftrightarrow (5)$.
The implication $(4) \Rightarrow (6)$ will be proven in \S \ref{provengiraudeasy} (Proposition \ref{ncharles}). Finally, the ``difficult'' implication $(6) \Rightarrow (1)$ will be proven in \S \ref{provengiraud} (Proposition \ref{diamondstep}), using an inductive argument quite similar to the proof of Giraud's original result.
\end{proof}

\begin{remark}
Theorem \ref{nchar} is slightly stronger than its $\infty$-categorical analogue, Theorem \ref{mainchar}: it asserts that every $n$-topos arises as an $n$-category of sheaves on some
$n$-category $\calC$ equipped with a Grothendieck topology.
\end{remark}

\begin{remark}
Let $\calX$ be a presentable $\infty$-category in which colimits are universal. Then
there exists a regular cardinal $\kappa$ such that every monomorphism is relatively $\kappa$-compact. In this case, characterization $(5)$ of Theorem \ref{nchar} recovers a classical
description of ordinary topos theory: a category $\calX$ is a topos if and only if it is presentable, colimits in $\calX$ are universal, and $\calX$ has a subobject classifier.
\end{remark}

\subsection{$0$-Topoi and Locales}\label{0topoi}

Our goal in this section is to prove Theorem \ref{nchar} in the special case $n=0$. A byproduct of our proof is a classification result (Corollary \ref{charlocale}), which identifies the theory of $0$-topoi with the classical theory of {\em locales} (Definition \ref{deflocale}).

We begin by observing that when $n=0$, a morphism in an $\infty$-category $\calX$ is
$(n-2)$-truncated if and only if it is an equivalence. Consequently, any final object of $\calX$ is an $(n-2)$-truncated morphism classifier, and the class of $(n-2)$-truncated morphisms is automatically local (in the sense of Definition \ref{localitie}). 
Moreover, if $\calX$ is a $0$-category then every groupoid object in $\calX$ is equivalent to a constant groupoid, and therefore automatically effective. Consequently, characterizations $(4)$ through $(6)$ in Theorem \ref{nchar} all reduce to the same condition on $\calX$, and we may restate the desired result as follows:

\begin{theorem}\label{sumatc}\index{gen}{Giraud's theorem!for $0$-topoi}
Let $\calX$ be a presentable $0$-category. The following conditions are equivalent:
\begin{itemize}
\item[$(1)$] There exists a small $0$-category $\calC$ which admits finite limits, 
a Grothendieck topology on $\calC$, and an equivalence $\calX \rightarrow \Shv_{\leq -1}(\calC)$.
\item[$(2)$] There exists an $\infty$-topos $\calY$ and an equivalence
$\calX \rightarrow \tau_{\leq -1} \calY$.
\item[$(3)$] The $\infty$-category $\calX$ is a $0$-topos.
\item[$(4)$] Colimits in $\calX$ are universal.
\end{itemize}
\end{theorem}

Before giving a proof of Theorem \ref{sumatc}, it is convenient to reformulate
condition $(4)$. Recall that any $0$-category $\calX$ is equivalent to
$\Nerve(\calU)$, where $\calU$ is a partially ordered set which is well-defined up to canonical isomorphism (see Example \ref{0catdef}). The presentability of $\calX$ is equivalent to the assertion that $\calU$ is a {\em complete lattice}: that is, every subset of $\calU$ has a least upper bound in $\calU$ (this condition formally implies the existence of greater lower bounds, as well).\index{gen}{complete lattice}

\begin{remark}
If $n=0$, then every presentable $n$-category is essentially small. This is typically not true for $n > 0$.
\end{remark}

We note that the condition that colimits in $\calX$ be universal can also be formulated in terms of the partially ordered set $\calU$: it is equivalent to the assertion that meets in $\calU$ commute with infinite joins in the following sense:

\begin{definition}\label{deflocale}
Let $\calU$ be a partially ordered set. We will say that $\calU$ is a {\em locale} if the following conditions are satisfied:\index{gen}{locale}
\begin{itemize}
\item[$(1)$] Every subset $\{ U_{\alpha} \}$ of elements of $\calU$ has a least upper bound
$\bigcup_{\alpha} U_{\alpha}$ in $\calU$.

\item[$(2)$] The formation of least upper bounds commutes with meets, in the sense that
$$ \bigcup (U_{\alpha} \cap V) = ( \bigcup U_{\alpha} ) \cap V.$$ (Here $(U \cap V)$ denotes the greatest lower bound of $U$ and $V$, which exists in virtue of assumption $(1)$.)
\end{itemize}
\end{definition}

\begin{example}
For every topological space $X$, the collection $\calU(X)$ of open subsets of $X$ forms a locale. 
Conversely, if $\calU$ is a locale, then there is a natural topology on the collection of prime filters of $\calU$ which allows us to extract from $\calU$ a topological space. These two constructions are adjoint to one another, and in good cases they are actually inverse equivalences. More precisely, the adjunction gives rise to an equivalence between the category of {\em spatial} locales and the category of {\em sober} topological spaces\label{sober}. In general, a locale can be regarded as a sort of generalized topological space, in which one may speak of open sets but one does not generally have a sufficient supply of points. We refer the reader to \cite{johnstone} for details.
\end{example}

We can summarize the above discussion as follows:

\begin{proposition}\label{colim0univ}
Let $\calX$ be a presentable $0$-category. Then colimits in $\calX$ are universal if and only if $\calX$ is equivalent to $\Nerve(\calU)$, where $\calU$ is a locale.
\end{proposition}

We are now ready to give the proof of Theorem \ref{sumatc}.

\begin{proof}
The implications $(1) \Rightarrow (2) \Rightarrow (3)$ are easy. Suppose that $(3)$ is satisfied, so that $\calX$ is a left-exact localization of $\calP_{\leq -1}(\calC)$, for some small $\infty$-category $\calC$. Up to equivalence, there are precisely two $(-1)$-truncated spaces: $\emptyset$ and $\ast$. Consequently, $\tau_{\leq -1} \SSet$ is equivalent to the 
two-object $\infty$-category $\Delta^1$. It follows that $\calP_{\leq -1}(\calC)$ is equivalent
to $\Fun( \calC^{op},\Delta^1)$.

Let $\widetilde{X}$ denote the collection of sieves on $\calC$, ordered by inclusion. Then, identifying a
a functor $f: \calC \rightarrow \Delta^1$
with the sieve $f^{-1} \{0\} \subseteq \calC$, we deduce that
$\Fun(\calC, \Delta^1)$ is isomorphic to the nerve $\Nerve(\widetilde{X})$.

Without loss of generality, we may identify $\calX$ with the essential image of a localization
functor $L: \Nerve(\widetilde{X}) \rightarrow \Nerve(\widetilde{X})$. The map $L$ may be identified with a map
of partially ordered sets from $\widetilde{X}$ to itself. Unwinding the definitions, we find that the condition that $L$ be a left exact localization is equivalent to the following three properties:

\begin{itemize}
\item[$(A)$] The map $L: \widetilde{X} \rightarrow \widetilde{X}$ is idempotent.
\item[$(B)$] For each $U \in \widetilde{X}$, $U \subseteq L(U)$.
\item[$(C)$] The map $L: \widetilde{X} \rightarrow \widetilde{X}$ preserves
finite intersections (since $\calX$ is a {\em left exact} localization of $\Nerve(\widetilde{X})$.)
\end{itemize}

Let $\calU = \{ U \in \widetilde{X}: LU = U \}$.
Then it is easy to see that $\calX$ is equivalent to the nerve $\Nerve(\calU)$, and that the partially ordered set $X$ satisfies conditions $(1)$ and $(2)$ of Definition \ref{deflocale}. Therefore $\calU$ is a locale, so that colimits in $\Nerve( \calU)$ are universal by Proposition \ref{colim0univ}. This proves that $(3) \Rightarrow (4)$. 

Now suppose that $(4)$ is satisfied. Using Proposition \ref{colim0univ}, we may suppose without loss of generality that $\calX = \Nerve(\calU)$, where $\calU$ is a locale. We observe that
$\calX$ is itself small. Let us say that
a sieve $\{ U_{\alpha} \rightarrow U \}$ on an object $U \in \calX$ is {\it covering} if 
$$ U = \bigcup_{\alpha} U_{\alpha} $$ in $\calU$. Using the assumption that $\calU$ is a locale, it is easy to see that the collection of covering sieves determines a Grothendieck topology on
$\calX$. The $\infty$-category $\calP_{\leq -1}(\calX)$ can be identified with the nerve of
the partially ordered set of all downward-closed subsets $\calU_0 \subseteq \calU$. Moreover, an object of $\calP_{\leq -1}(\calX)$ belongs to $\Shv_{\leq -1}(\calX)$ if and only if the corresponding subset $\calU_0 \subseteq \calU$ is stable under joins. Every such subset $\calU_0 \subseteq \calU$ has a largest element $U \in \calU$, and we then have an identification
$\calU_0 = \{ V \in \calU: V \leq U\}$. It follows that $\Shv_{\leq -1}(\calX)$ is equivalent
to the nerve of the partially ordered set $\calU$, which is $\calX$. This proves $(1)$, and concludes the argument.
\end{proof}

We may summarize the results of this section as follows:

\begin{corollary}\label{charlocale}
An $\infty$-category $\calX$ is a $0$-topos if and only if it is equivalent to
$\Nerve(\calU)$, where $\calU$ is a locale.
\end{corollary}

\begin{remark}
Coproducts in a $0$-topos are typically {\em not} disjoint.
\end{remark}

In classical topos theory, there are functorial constructions for passing back and forth between topoi and locales. Given a locale $\calU$ (such as the locale $\calU(X)$ of open subsets of a topological space $X$), one 
may define a topos $\calX$ of {\it sheaves $($ of sets $)$ on $\calU$}. The original locale $\calU$ may then be recovered as the partially ordered set of subobjects of the final object of $\calX$. In fact, for {\em any} topos $\calX$, the partially ordered set $\calU$ of subobjects of the final object forms a locale. In general, $\calX$ cannot be recovered as the category of sheaves on $\calU$; this is true if and only if
$\calX$ is a {\it localic} topos\index{gen}{localic topos}\index{gen}{topos!localic}: that is, if and only if $\calX$ is generated under colimits
by the collection of subobjects of the final object ${1}_{\calX}$. In \S \ref{chap6sec4} we will discuss a generalization of this picture, which will allow us to pass between $m$-topoi and $n$-topoi for any $m \leq n$.

\subsection{Giraud's Axioms for $n$-Topoi}\label{provengiraud}

In \S \ref{axgir}, we sketched an axiomatic approach to the theory of $\infty$-topoi. The axioms
we introduced were closely parallel to Giraud's axioms for ordinary topoi, with one important difference. If $\calX$ is an $\infty$-topos, then {\em every} groupoid object of $\calX$ is effective. If $\calX$ is an ordinary topos, then a groupoid $U_{\bigdot}$ is effective only if the diagram
$$\xymatrix{ U_{1} \ar@<.5ex>[r] \ar@<-.5ex>[r] & U_0 }$$
exhibits $U_{1}$ as an equivalence relation on $U_0$. Our first goal in this section is to formulate an analogue of this condition, which will lead us to an axiomatic description of $n$-topoi for all
$0 \leq n \leq \infty$.

\begin{definition}\index{gen}{groupoid object!$n$-efficient}
Let $\calX$ be an $\infty$-category and $U_{\bigdot}$ a groupoid object of $\calX$. We will say that
$U_{\bigdot}$ is {\it $n$-efficient} if the natural map $$ U_{1} \rightarrow U_0 \times U_0 $$
(which is well-defined up to equivalence) is $(n-2)$-truncated.
\end{definition}

\begin{remark}
By convention, we regard every groupoid object as $\infty$-efficient.
\end{remark}

\begin{example}
If $\calC$ is (the nerve of) an ordinary category, then giving a $1$-efficient groupoid object $U_{\bigdot}$ of $\calC$ is equivalent to giving an object $U_0$ of $\calC$ and an equivalence relation $U_1$ on $U_0$.
\end{example}

\begin{proposition}\label{storage}
An $\infty$-category $\calX$ is equivalent to an $n$-category if and only if every
effective groupoid in $\calX$ is $n$-efficient.
\end{proposition}

\begin{proof}
Suppose first that $\calC$ is equivalent to an $n$-category. Let $U_{\bigdot}$ be an effective groupoid in $\calX$. Then $U_{\bigdot}$ has a colimit $U_{-1}$. The existence of a pullback diagram
$$ \xymatrix{ U_1 \ar[r] \ar[d] & U_0 \ar[d] \\
U_0 \ar[r] & U_{-1} }$$
implies that the map $f': U_{1} \rightarrow U_0 \times U_0$ is a pullback of the diagonal map
$f: U_{-1} \rightarrow U_{-1} \times U_{-1}$. We wish to show that $f'$ is $(n-2)$-truncated.
By Lemma \ref{tunc} it suffices to show that $f$ is $(n-2)$-truncated. By Lemma \ref{trunc}, this is equivalent to the assertion that $U_{-1}$ is $(n-1)$-truncated. Since $\calC$ is equivalent to an $n$-category, every object of $\calC$ is $(n-1)$-truncated.

Now suppose that every effective groupoid in $\calX$ is $n$-efficient. Let $U \in \calX$ be
an object; we wish to show that $U$ is $(n-1)$-truncated.
The constant simplicial object $U_{\bigdot}$ taking the value $U$ is an effective groupoid, and therefore $n$-efficient. It follows that the diagonal map $U \rightarrow U \times U$ is $(n-2)$-truncated. Lemma \ref{trunc} implies that $U$ is $(n-1)$-truncated as desired.
\end{proof}

We are now almost ready to supply the ``hard'' step in the proof of Theorem \ref{nchar} (namely, the implication $(6) \Rightarrow (1)$). We first need a slightly technical lemma, whose proof requires routine cardinality estimates.

\begin{lemma}\label{nocake}
Let $\calX$ be a presentable $\infty$-category in which colimits are universal. There exists a regular cardinal $\tau$ such that $\calX$ is $\tau$-accessible, and the full subcategory of $\calX^{\tau} \subseteq \calX$ spanned by the $\tau$-compact objects is stable under the formation of subobjects and finite limits.
\end{lemma}

\begin{proof}
Choose a regular cardinal $\kappa$ such that $\calX$ is $\kappa$-accessible. We observe that, up to equivalence, there are a bounded number of $\kappa$-compact objects of $\calX$, and therefore a bounded number of {\em subobjects} of $\kappa$-compact objects of $\calX$. Now choose an uncountable regular cardinal $\tau \gg \kappa$ such that:
\begin{itemize}
\item[$(1)$] The $\infty$-category $\calX^{\kappa}$ is essentially $\tau$-small.
\item[$(2)$] For each $X \in \calX^{\kappa}$ and each monomorphism $i: U \rightarrow X$, $U$ is $\tau$-compact.
\end{itemize}
It is clear that $\calX$ is $\tau$-accessible, and $\calX^{\tau}$ is stable under finite limits (in fact, $\kappa$-small limits) by Proposition \ref{tcoherent}. To complete the proof, we must show that $\calX^{\tau}$ is stable under the formation of subobjects. Let $i: U \rightarrow X$ be a monomorphism, where $X$ is $\tau$-compact. Since $\calX$ is $\kappa$-accessible, we can write
$X$ as the colimit of $\kappa$-filtered diagram $p: \calJ \rightarrow \calX^{\kappa}$. Since $X$ is $\tau$-compact, it is a retract of the colimit $X'$ of some $\tau$-small subdiagram $p| \calJ'$.
Since $\tau$ is uncountable, we can use Proposition \ref{autokan} to write
$X$ as the colimit of a $\tau$-small diagram $\Idem \rightarrow \calX$, which carries
the unique object of $\Idem$ to $X'$. Since colimits in $\calX$ are universal, it follows that
$U$ can be written as a $\tau$-small colimit of a diagram $\Idem \rightarrow \calX$ which takes the value $U' = U \times_{X} X'$. It will therefore suffice to prove that $U'$ is $\tau$-compact.
Invoking the universality of colimits once more, we observe that $U'$ is a $\tau$-small colimit of objects of the form $U'' = U' \times_{X'} p(J)$, where $J$ is an object of $\calJ'$. We now observe
that $U''$ is a subobject of $p(J) \in \calX^{\kappa}$, and is therefore $\tau$-compact by assumption $(2)$. It follows that $U'$, being a $\tau$-small colimit of $\tau$-compact objects of $\calX$, is also $\tau$-compact.
\end{proof}

\begin{proposition}\label{diamondstep}
Let $0 < n < \infty$, and let $\calX$ be an $\infty$-category satisfying the following conditions:
\begin{itemize}
\item[$(i)$] The $\infty$-category $\calX$ is presentable.
\item[$(ii)$] Colimits in $\calX$ are universal.
\item[$(iii)$] Coproducts in $\calX$ are disjoint.
\item[$(iv)$] The effective groupoid objects of $\calX$ are precisely the $n$-efficient groupoids.
\end{itemize}
Then there exists a small $n$-category $\calC$ which admits finite limits, a Grothendieck topology on $\calC$, and an equivalence $\calX \rightarrow \Shv_{\leq n-1}(\calC)$.
\end{proposition}

\begin{proof}
Without loss of generality we may suppose that $\calX$ is minimal. Choose a regular cardinal $\kappa$ such that $\calX$ is $\kappa$-accessible, and the full subcategory $\calC \subseteq \calX$ spanned by the $\kappa$-compact objects of $\calX$ is stable under the formation of subobjects and finite limits (Lemma \ref{nocake}).
We endow $\calC$ with the canonical topology induced by the inclusion
$\calC \subseteq \calX$. According to Theorem \ref{charpresheaf}, there is an (essentially unique) colimit preserving functor $F: \calP(\calC) \rightarrow \calX$ such that $F \circ j$ is equivalent
to the inclusion $\calC \subseteq \calX$, where $j: \calC \rightarrow \calP(\calC)$ denotes the Yoneda embedding. The proof of Theorem \ref{pretop} shows that $F$ has a fully faithful right adjoint $G: \calX \rightarrow \calP(\calC)$. We will complete the proof by showing that the essential image of $G$ is precisely $\Shv_{\leq n-1}(\calC)$. 

Since $\calX$ is equivalent to an $n$-category (Proposition \ref{storage}) and $G$ is left exact, we conclude that $G$ factors through $\calP_{\leq n-1}(\calC)$. It follows from Proposition \ref{preciselate} that $G$ factors through $\Shv_{\leq n-1}(\calC)$. Let $\calX' \subseteq \Shv_{\leq n-1}(\calC)$ denote the essential image of $G$. To complete the proof, it will suffice to show that $\calX' = \Shv_{\leq n-1}(\calC)$. Let $\emptyset$ be an initial object of $\calX$. The space $\bHom_{\calX}(X, \emptyset)$ is contractible if $X$ is an initial object of $\calX$, and empty otherwise (Lemma \ref{sumoto}). It follows from Proposition \ref{suture} that $G(\emptyset)$ is an initial object of $\Shv_{\leq n-1}(\calC)$. 

We next claim that $\calX'$ is stable under small coproducts in $\Shv_{\leq n-1}(\calC)$. It will suffice to show that the map $G$ preserves coproducts. Let $\{ U_{\alpha} \}$ be a small collection of objects of $\calX$ and $U$ their coproduct in $\calX$. According to Lemma \ref{sumdescription}, we have a pullback diagram
$$ \xymatrix{ V_{\alpha,\beta} \ar[r]^{\phi} \ar[d]^{\phi'} & U_{\alpha} \ar[d] \\
U_{\beta} \ar[r] & U, }$$
where $V_{\alpha,\beta}$ is an initial object of $\calX$ if $\alpha \neq \beta$, while
$\phi$ and $\phi'$ are equivalences if $\alpha = \beta$.
The functor $G$ preserves all limits, so that the diagram
$$ \xymatrix{ G(V_{\alpha,\beta}) \ar[r] \ar[d] & G(U_{\alpha}) \ar[d] \\
G( U_{\beta} ) \ar[r] & G(U) }$$
is a pullback in $\Shv_{\leq n-1}(\calC)$. Let $U'$ denote a coproduct
of the objects $i(U_{\alpha})$ in $\Shv_{\leq n-1}(\calC)$, and let $g: U' \rightarrow G(U)$ be the induced map. Since colimits in $\calX$ are universal, we obtain a natural identification of
$U' \times_{G(U)} U'$ with the coproduct
$$ \coprod_{\alpha, \beta} ( G(U_{\alpha}) \times_{ G(U) } G(U_{\beta}) )
\simeq \coprod_{\alpha} U_{\alpha} \simeq U',$$
where the second equivalence follows from our observation that $G$ preserves initial objects.
Applying Lemma \ref{trunc}, we deduce that $g$ is a monomorphism.

To prove that $g$ is an equivalence, it will suffice to show that the map 
$$\pi_0 U'(C) \rightarrow \pi_0 G(U)(C) = \pi_0 \bHom_{\calX}(C,U)$$ is surjective for every object $C \in \calC$.
Since colimits in $\calX$ are universal,
every map $h: C \rightarrow U$ can be written as a coproduct of maps $h_{\alpha}: C_{\alpha} \rightarrow U_{\alpha}$. Each $C_{\alpha}$ is a subobject of $C$ (Lemma \ref{corsumoto}) and therefore belongs to $\calC$. Let $h'_{\alpha} \in \pi_0 U'(C_{\alpha})$ denote the homotopy class
of the composition $G(C_{\alpha}) \stackrel{h_{\alpha}}{\rightarrow} G(U_{\alpha}) \rightarrow U'$.
Since the topology on $\calC$ is canonical, Lemma \ref{canonicalcoproducts} implies that
$\pi_0 U'(C) \simeq \prod_{\alpha} \pi_0 U'(C_{\alpha})$ contains an element $h'$ which
restricts to each $h'_{\alpha}$. It is now clear that $h$ is the image of $h'$ under
the map $\pi_0 U'(C) \rightarrow \pi_0 \bHom_{\calX}(C,U)$. 

We will prove the following result by induction on $k$: if there exists a $k$-truncated morphism
$f: X \rightarrow Y$, where $Y \in \calX'$ and $X \in \Shv_{\leq n-1}(\calC)$, then $X \in \calX'$.
Taking $k = n-1$ and $Y$ to be a final object of $\Shv_{\leq n-1}(\calC)$ (which belongs to
$\calX'$ because $\calC$ contains a final object), we conclude that every object of
$\Shv_{\leq n-1}(\calC)$ belongs to $\calX'$, which completes the proof.

If $k = -2$, then $f$ is an equivalence so that $X \in \calX'$ as desired.
Assume now that $k \geq -1$. Since $\calX'$ contains the essential image of the Yoneda embedding and is stable under coproducts, there exists an effective epimorphism $p: U \rightarrow X$ in $\Shv_{\leq n-1}(\calC)$, where $U \in \calX'$. Let $\overline{U}_{\bigdot}$ be a \Cech nerve of $p$ in 
$\Shv_{\leq n-1}(\calC)$, and $U_{\bigdot}$ the associated groupoid object. We claim that
$U_{\bigdot}$ is a groupoid object of $\calX'$. Since $\calX'$ is stable under limits in
$\Shv_{\leq n-1}(\calC)$, it suffices to prove that $U_0 = U$ and $U_1 = U \times_{X} U$ belong to $\calX'$. We now observe that there exists a pullback diagram
$$ \xymatrix{ U \times_{X} U \ar[r]^-{\delta'} \ar[d] & U \times_{Y} U \ar[d] \\
X \ar[r]^-{\delta} & X \times_{Y} X. }$$
Since $f$ is $k$-truncated, $\delta$ is $(k-1)$-truncated (Lemma \ref{trunc}), so 
that $\delta'$ is $(k-1)$-truncated. Since $U \times_{Y} U$ belongs to $\calX'$ (because
$\calX'$ is stable under limits), our inductive hypothesis allows us to conclude that $U \times_{X} U
\in \calX'$, as desired.  

We observe that $U_{\bigdot}$ is an $n$-efficient groupoid object of $\calX'$. Invoking assumption $(iv)$, we conclude that $U_{\bigdot}$ is effective in $\calX'$. Let $X' \in \calX'$ be a colimit of $U_{\bigdot}$ in $\calX'$, so that we have a morphism $u: X \rightarrow X'$ in $\Shv_{\leq n-1}(\calC)_{U_{\bigdot}/}$. To complete the proof that $X \in \calX'$, it will suffice to show that $u$ is an equivalence. Since $u$ induces an equivalence
$$ U \times_{X} U \rightarrow U \times_{X'} U,$$
it is a monomorphism (Lemma \ref{trunc}). It will therefore suffice to show that $u$ is an effective epimorphism. We have a commutative diagram
$$ \xymatrix{ U \ar[dr]^{p} \ar[rr]^{p'} & & X' \\
& X \ar[ur]^{u} & }$$
where $p$ is an effective epimorphism; it therefore suffices to show that $p'$ is an effective epimorphism which follows immediately from Proposition \ref{preciselate}.
\end{proof}

\begin{remark}
Proposition \ref{diamondstep} is valid also for $n=0$, but is almost vacuous: coproducts
in a $0$-topos $\calX$ are never disjoint unless $\calX$ is trivial (equivalent to the final $\infty$-category $\ast$).
\end{remark}

\begin{remark}
In a certain respect, the theory of $\infty$-topoi is {\em simpler than} the theory of ordinary topoi: in an $\infty$-topos, {\em every} groupoid object is effective; it is not necessary to impose any additional conditions like $n$-efficiency. The absense of this condition gives the theory
of $\infty$-topoi a slightly different flavor than ordinary topos theory. In an $\infty$-topos, we are free
to form quotients of objects not only by equivalence relations, but by arbitrary groupoid actions.
In geometry, this extra flexibility allows the construction of useful objects such as orbifolds and algebraic stacks, which are useful in a variety of mathematical situations.

One can imagine weakening the gluing conditions even further, and
considering axioms having the form ``every category object is
effective''. This seems like a natural approach to a theory
of topos-like $(\infty,\infty)$-categories. However, we will not pursue the matter any further here.
\end{remark}

It follows from Proposition \ref{diamondstep} (and arguments to be given in \S \ref{provengiraudeasy}) that every left-exact localization of a presheaf $n$-category
$\calP_{\leq n-1}(\calC)$ can also be obtained as an $n$-category of sheaves. According to the next two results, this is no accident: every left exact localization of $\calP_{\leq n-1}(\calC)$ is topological, and the topological localizations of $\calP_{\leq n-1}(\calC)$ correspond precisely to the Grothendieck topologies on $\calC$ (provided that $\calC$ is an $n$-category).

\begin{proposition}\label{alltoploc}
Let $\calX$ be a presentable $n$-category, $0 \leq n < \infty$, and suppose that colimits in $\calX$ are universal. Let $L: \calX \rightarrow \calY$ be a left exact localization. Then $L$ is a topological localization.
\end{proposition}

\begin{proof}
Let $S$ denote the collection of all monomorphisms $f: U \rightarrow V$ in $\calX$ such that $Lf$ is an equivalence. Since $L$ is left exact, it is clear that $S$ is stable under pullback. Let $\overline{S}$ be the strongly saturated class of morphisms generated by $S$. Proposition \ref{swimmer} implies that $\overline{S}$ is stable under pullback, and therefore topological.
Proposition \ref{toplocsmall} implies that $\overline{S}$ is generated by a (small) set of morphisms. 
Let $\calX' \subseteq \calX$ denote the full subcategory spanned by $\overline{S}$-local objects. According to Proposition \ref{local}, $\calX'$ is an accessible localization of $\calX$; let $L'$ denote the associated localization functor. Since $Lf$ is an equivalence for each $f \in \overline{S}$, the localization $L$ is equivalent to the composition
$$ \calX \stackrel{L'}{\rightarrow} \calX' \stackrel{L|\calX'}{\rightarrow} \calY.$$
We may therefore replace $\calX$ by $\calX'$ and thereby reduce to the case where
$S$ consists precisely of the equivalences in $\calX$; we wish to prove that $L$ is an equivalence.

We now prove the following claim: if $f: X \rightarrow Y$ is a $k$-truncated morphism in $\calC$
such that $Lf$ is an equivalence, then $f$ is an equivalence. The proof goes by induction on 
$k$. If $k = -1$, then $f$ is a monomorphism, and so belongs to $S$, and is therefore an equivalence. Suppose that $k \geq 0$. Let
$\delta: X \rightarrow X \times_{Y} X$ be the diagonal map (which is well-defined up to equivalence). According to Lemma \ref{trunc}, $\delta$ is $(k-1)$-truncated. Since
$L$ is left exact, $L(\delta)$ can be identified with a diagonal map $LX \rightarrow LX \times_{LY} LX$, which is therefore an equivalence. The inductive hypothesis implies that $\delta$ is an equivalence. Applying Lemma \ref{trunc} again, we deduce that $f$ is a monomorphism, so that $f \in S$ and is therefore an equivalence as noted above. 

Since $\calX$ is an $n$-category, every morphism in $\calX$ is $(n-1)$-truncated. We conclude that for {\em every} morphism $f $ in $\calX$, $f$ is an equivalence if and only if $Lf$ is an equivalence. Since $L$ is a localization functor, it must be an equivalence.
\end{proof}

\subsection{$n$-Topoi and Descent}\label{provengiraudeasy}

Let $\calX$ be an $\infty$-category which admits finite limits, and let
$\calO_{\calX}$ denote the functor $\infty$-category $\Fun(\Delta^1, \calX)$ equipped
with the Cartesian fibration $e: \calO_{\calX} \rightarrow \calX$ (given by evaluation
at $\{1\} \subseteq \Delta^1$), as in \S \ref{axgir}. Let $F: \calX^{op} \rightarrow \widehat{\Cat}_{\infty}$ be a functor which classifies $e$; informally, $F$ associates to each object $U \in \calX$ the $\infty$-category $\calX_{/U}$. According to Theorem \ref{charleschar}, 
$\calX$ is an $\infty$-topos if and only if $F$ the functor
$F$ preserves limits, and factors through $\LPres \subseteq \widehat{\Cat}_{\infty}$.
The assumption that $F$ preserves limits can be viewed as a descent condition: it asserts that if $X \rightarrow U$ is a morphism of $\calX$, and $U$ is decomposed into ``pieces'' $U_{\alpha}$, then $X$ can be canonically reconstructed from the ``pieces'' $X \times_{U} U_{\alpha}$.
The goal of this section is to obtain a similar characterization of the class of $n$-topoi, for $0 \leq n < \infty$. 

We begin by considering the case where $\calX$ is the (nerve of) the category of sets. In this case, we can think of $F$ as a contravariant functor from sets to categories,
which carries a set $U$ to the category $\Set_{/U}$. This functor does not preserve pullbacks: given a pushout square $$ \xymatrix{ & X \ar[dl] \ar[dr] &  \\
Y \ar[dr] & & Z \ar[dl] \\
& Y \amalg_{X} Z & }$$
in the category $\Set$, there is an associated functor
$$ \theta: \Set_{/ Y \amalg_{X} Z} \rightarrow \Set_{/Y} \times_{ \Set_{/X} } \Set_{/Z}$$
(here the right hand side indicates a {\em homotopy} fiber product of categories). The functor $\theta$ is generally not an equivalence of categories: for example, $\theta$ fails to be an 
equivalence if $Y = Z = \ast$, provided that $X$ has cardinality at least $2$. However, $\theta$ is always fully faithful. Moreover, we have the following result:

\begin{fact}\label{stiro}
The functor $\theta$ induces an isomorphism of partially ordered sets
$$ \Sub(Y \amalg_{X} Z) \rightarrow \Sub(Y) \times_{ \Sub(X) } \Sub(Z)$$
where $\Sub(M)$ denotes the partially ordered set of subsets of $M$.
\end{fact}

In this section, we will show that an appropriate generalization of Fact \ref{stiro} can be used to characterize the class of $n$-topoi, for all $0 \leq n \leq \infty$. First, we need to introduce some terminology.

\begin{notation}\label{sumahum}
Let $\calX$ be an $\infty$-category which admits pullbacks, and let $0 \leq n \leq \infty$. We let
$\calO_{\calX}^{n}$ denote the full subcategory of $\calO_{\calX}$ spanned by morphisms
$f: U \rightarrow X$ which are $(n-2)$-truncated, and $\calO_{\calX}^{(n)} \subseteq
\calO^{n}_{\calX}$ the subcategory whose objects are $(n-2)$-truncated morphisms
in $\calX$, and whose morphisms are Cartesian transformations (see Notation
\ref{ugaboo}).\index{not}{OcalXn@$\calO_{\calX}^{n}$}\index{not}{OcalX(n)@$\calO_{\calX}^{(n)}$}
\end{notation}

\begin{example}\label{trivexa}
Let $\calX$ be an $\infty$-category which admits pullbacks. Then $\calO^{0}_{\calX}$
is the full subcategory of $\calO_{\calX}$ spanned by the final objects in each fiber of the morphism $p: \calO_{\calX} \rightarrow \calX$. Since $p$ is a coCartesian fibration (Corollary \ref{tweezegork}), Proposition \ref{topaz} asserts that the restriction $p | \calO_{\calX}^{0}$ is a trivial fibration of simplicial sets.
\end{example}

\begin{lemma}\label{nstab}
Let $\calX$ be a presentable $\infty$-category in which colimits are universal and coproducts are disjoint, and let $n \geq -2$. Then the class of $n$-truncated morphisms in $\calX$ is stable under small coproducts.
\end{lemma}

\begin{proof}
The proof is by induction on $n$, where the case $n=-2$ is obvious. Suppose that the
$\{ f_{\alpha}: X_{\alpha} \rightarrow Y_{\alpha} \}$ is a family of $n$-truncated morphisms
in $\calX$ having coproduct $f: X \rightarrow Y$. Since colimits in $\calX$ are universal, we conclude that $X \times_{Y} X$ can be written as a coproduct 
$$\coprod_{ \alpha, \beta} (X_{\alpha} \times_{Y} X_{\beta})
\simeq \coprod_{\alpha,\beta} (X_{\alpha} \times_{Y_{\alpha} } (Y_{\alpha} \times_{Y} Y_{\beta} )
\times_{Y_{\beta}} X_{\beta}). $$
Applying Lemma \ref{sumdescription}, we can rewrite this coproduct as
$$ \coprod_{\alpha} (X_{\alpha} \times_{ Y_{\alpha} } X_{\alpha}). $$
Consequently, the diagonal map $\delta: X \rightarrow X \times_{Y} X$ is a coproduct
of diagonal maps $\{ \delta_{\alpha}: X_{\alpha} \rightarrow X_{\alpha} \times_{Y_{\alpha} } X_{\alpha} \}$. Applying Lemma \ref{trunc}, we deduce that each $\delta_{\alpha}$ is $(n-1)$-truncated, so that $\delta$ is $(n-1)$-truncated by the inductive hypothesis. We now apply Lemma \ref{trunc} again to deduce that $f$ is $n$-truncated, as desired.
\end{proof}

Combining Lemmas \ref{ib1}, \ref{ib2}, \ref{ib3}, and \ref{nstab}, we deduce the following analogue of Theorem \ref{charleschar}.

\begin{theorem}\label{countercon}
Let $\calX$ be a presentable $\infty$-category in which colimits are universal and coproducts are disjoint. The following conditions are equivalent:
\begin{itemize}
\item[$(1)$] For every 
pushout diagram
$$ \xymatrix{ f \ar[r]^{\alpha} \ar[d]^{\beta} & g \ar[d]^{\beta'} \\
f' \ar[r]^{\alpha'} & g' }$$
in $\calO^{n}_{\calX}$, if $\alpha$ and $\beta$ are Cartesian transformations, then
$\alpha'$ and $\beta'$ are also Cartesian transformations.

\item[$(2)$] The class of $(n-2)$-truncated morphisms
in $\calX$ is local.

\item[$(3)$] The Cartesian fibration $\calO^{n}_{\calX} \rightarrow \calX$ is classified
by a limit-preserving functor $\calX^{op} \rightarrow \hat{\Cat}_{\infty}$

\item[$(4)$] The right fibration $\calO^{(n)}_{\calX} \rightarrow \calX$ is classified by a limit-preserving functor $\calX^{op} \rightarrow \hat{\SSet}$.

\item[$(5)$] Let $K$ be a small simplicial set and 
$\overline{\alpha}: \overline{p} \rightarrow \overline{q}$ a natural transformation of colimit diagrams
$\overline{p}, \overline{q}: K^{\triangleright} \rightarrow \calX$. Suppose that
$\alpha = \overline{\alpha} | K$ is a Cartesian transformation, and that $\alpha(x)$ is $(n-2)$-truncated for every vertex $x \in K$. Then $\overline{\alpha}$ is a Cartesian transformation, and
$\overline{\alpha}(\infty)$ is $(n-2)$-truncated, where $\infty$ denotes the cone point of
$K^{\triangleright}$.
\end{itemize}
\end{theorem}

Our next goal is to establish the implication $(2) \Rightarrow (4)$ of Theorem \ref{nchar}.
We will deduce this from the equivalence $(2) \Leftrightarrow (3)$ (which we have already established) together with Propositions \ref{tigre} and \ref{tigress} below.

\begin{proposition}\label{tigre}
Let $\calX$ be an $n$-topos, $0 \leq n \leq \infty$. Then colimits in $\calX$ are universal.
\end{proposition}

\begin{proof}
Using Lemma \ref{tryme}, we may reduce to the case $\calX = \calP_{\leq n-1}(\calC)$ for some small $\infty$-category $\calC$. Using Proposition \ref{limiteval}, we may further reduce to the case where $\calX = \tau_{\leq n-1} \SSet$. 

Let $f: X \rightarrow Y$ be a map of $(n-1)$-truncated spaces, and let $f^{\ast}: \SSet^{/Y} \rightarrow \SSet^{/X}$ be a pullback functor.
Since $\calX$ is stable under limits in $\SSet$, $f^{\ast}$ restricts to give a functor $\calX^{/Y} \rightarrow \calX^{/X}$; we wish to prove that this restricted functor commutes with colimits. 
We observe
that $\calX^{/X}$ and $\calX^{/Y}$ can be identified with the full subcategories of
$\SSet^{/X}$ and $\SSet^{/Y}$ spanned by the $(n-1)$-truncated objects, by Lemma \ref{trunccomp}. Let $\tau_{X}: \SSet^{/X} \rightarrow \calX^{/X}$ and $\tau_{Y}: \SSet^{/Y} \rightarrow \calX^{/Y}$ denote left adjoints to the inclusions. The functor $f^{\ast}$ preserves all colimits (Lemma \ref{sugartime}) and all limits (since
$f^{\ast}$ has a left adjoint). Consequently, Proposition \ref{compattrunc} implies that
$\tau_{X} \circ f^{\ast} \simeq f^{\ast} \circ \tau_{Y}$.

Let $p: K^{\triangleright} \rightarrow \calX^{/Y}$ be a colimit diagram. We wish to show that
$f^{\ast} \circ p$ is a colimit diagram. According to Remark \ref{localcolim}, we may assume that $p = \tau_{Y} \circ p'$, for some colimit diagram $p': K^{\triangleright} \rightarrow \SSet^{/Y}$. Since colimits in $\SSet$ are universal (Lemma \ref{sugartime}), the composition
$f^{\ast} \circ p': K^{\triangleright} \rightarrow \SSet^{/X}$ is a colimit diagram. Since $\tau_{X}$ preserves colimits, we conclude that $\tau_{X} \circ f^{\ast} \circ p': K^{\triangleright} \rightarrow \calX^{/X}$ is a colimit diagram, so that $f^{\ast} \circ \tau_{Y} \circ p' = f^{\ast} \circ p$ is also a colimit diagram, as desired.
\end{proof}

\begin{proposition}\label{tigress}
Let $\calY$ be an $\infty$-topos and let $\calX = \tau_{\leq n} \calY$, $0 \leq n \leq \infty$. Then the class of $(n-2)$-truncated morphisms in $\calX$ is local.
\end{proposition}

\begin{proof}
Combining Propositions \ref{hintdescent0}, \ref{torque}, and Lemma \ref{nstab}, we conclude
that the class of $(n-2)$-truncated morphisms in $\calY$ is local. Consequently, the Cartesian fibration
$\calO_{\calY}^{n} \rightarrow \calY$ is classified by a colimit preserving functor
$ F: \calY \rightarrow \hat{\Cat}_{\infty}^{op}$. It follows that $\calO_{\calX}^{(n)} \rightarrow \calX$
is classified by $F | \calX$. To prove that $F | \calX$ is colimit-preserving, it will suffice to show that 
$F$ is equivalent to $F \circ \tau_{\leq n}$; in other words, that $F$ carries
each $n$-truncation $Y \rightarrow \tau_{\leq n} Y$ to an equivalence in $\hat{\Cat}_{\infty}^{op}$.
Replacing $\calY$ by $\calY_{/ \tau_{\leq n} Y}$, we reduce to Lemma \ref{nicelemma}.
\end{proof}

We conclude this section by proving the following generalization of Proposition \ref{lemonade2}, which also establishes the implication $(4) \Rightarrow (6)$ of Theorem \ref{nchar}. We will assume $n > 0$; the case $n = 0$ was analyzed in \S \ref{0topoi}.

\begin{lemma}\label{corsumoto}
Let $\calX$ be a presentable $\infty$-category in which colimits are universal, and let
$f: \emptyset \rightarrow X$ be a morphism in $\calX$, where $\emptyset$ is an initial object of $\calX$. Then $f$ is a monomorphism.
\end{lemma}

\begin{proof}
Let $Y$ be an arbitrary object of $\calX$, we wish to show that composition with $f$
induces a $(-1)$-truncated map
$$ \bHom_{\calX}( Y, \emptyset) \rightarrow \bHom_{\calX}(Y, X).$$
If $Y$ is an initial object of $\calX$, then both sides are contractible; otherwise
the left side is empty (Lemma \ref{sumoto}).
\end{proof}

\begin{proposition}\label{ncharles}
Let $1 \leq n \leq \infty$, and let $\calX$ be a presentable $n$-category. Suppose that
colimits in $\calX$ are universal, and that the class of $(n-2)$-truncated morphisms in $\calX$ is local. Then $\calX$ satisfies the $n$-categorical Giraud axioms:
\begin{itemize}
\item[$(i)$] The $\infty$-category $\calX$ is equivalent to a presentable $n$-category.
\item[$(ii)$] Colimits in $\calX$ are universal.
\item[$(iii)$] Coproducts in $\calX$ are disjoint.
\item[$(iv)$] Every $n$-efficient groupoid object of $\calX$ is effective.
\end{itemize}
\end{proposition}

\begin{proof}
Axioms $(i)$ and $(ii)$ hold by assumption.
To show that coproducts in $\calX$ are disjoint, let us consider an arbitrary pair of objects
$X, Y \in \calX$, and let $\emptyset$ denote an initial object of $\calX$. Let $f: \emptyset \rightarrow X$ be a morphism (unique up to homotopy, since $\emptyset$ is initial). Since
colimits in $\calX$ are universal, $f$ is a monomorphism (Lemma \ref{corsumoto}) and therefore belongs to
$\calO_{\calX}^{n}$, since $n \geq 1$. We observe that
$\id_{\emptyset}$ is an initial object of $\calO_{\calX}$, so we can form a pushout diagram
$$ \xymatrix{ \id_{\emptyset} \ar[r]^{\alpha} \ar[d]^{\beta} & \id_{Y} \ar[d]^{\beta'}  \\
f \ar[r]^{\alpha'} & g }$$
in $\calO^{n}_{\calX}$. It is clear that $\alpha$ is a Cartesian transformation, and Lemma \ref{sumoto} implies that $\beta$ is Cartesian as well. Invoking Theorem \ref{countercon}, we deduce that $\alpha'$ is
a Cartesian transformation. But $\alpha'$ can be identified with a pushout diagram
$$ \xymatrix{ \emptyset \ar[r] \ar[d] & Y \ar[d] \\
X \ar[r] & X \amalg Y. }$$
This proves $(iii)$. 

Now suppose that $U_{\bigdot}$ is an $n$-efficient groupoid object of $\calX$; we wish to prove that $U_{\bigdot}$ is effective. Let $\overline{U}_{\bigdot}  : \Nerve(\cDelta_{+})^{op} \rightarrow \calX$ be a colimit of $U_{\bigdot}$. Let $U'_{\bigdot}: \Nerve(\cDelta_{+})^{op} \rightarrow \calX$
be the result of composing $\overline{U}_{\bigdot}$ with the shift functor
$$ \cDelta_{+} \rightarrow \cDelta_{+}$$
$$ J \mapsto J \amalg \{ \infty \}.$$
(In other words, $U'_{\bigdot}$ is the shifted simplicial object given by
$U'_{n} = U_{n+1}$.)
Lemma \ref{bclock} asserts that $U'_{\bigdot}$ is a colimit diagram in $\calX$.
We have a transformation $\overline{\alpha}: U'_{\bigdot} \rightarrow \overline{U}_{\bigdot}$.
Let $\overline{V}_{\bigdot}$ denote the constant augmented simplicial object of $\calX$ taking the value $U_0$, so that we have a natural transformation $\overline{\beta}: U'_{\bigdot} \rightarrow \overline{V}_{\bigdot}$. Let $\overline{W}_{\bigdot}$ denote a product of $\overline{U}_{\bigdot}$ and $\overline{V}_{\bigdot}$ in the $\infty$-category $\calX_{\Delta_{+}}$ of augmented simplicial objects, and let $\overline{\gamma}: U'_{\bigdot} \rightarrow \overline{W}_{\bigdot}$ be
the induced map. We observe that for each $n \geq 0$, the map 
$\overline{\gamma}(\Delta^n): U_{n+1} \rightarrow \overline{W}_n$ is a pullback of
$U_1 \rightarrow U_0 \times U_0$, and therefore $(n-2)$-truncated (since $U_{\bigdot}$ is
assumed to be $n$-efficient). Since $U_{\bigdot}$ is a groupoid, we conclude that
$\gamma = \overline{\gamma} | \Nerve(\cDelta)^{op}$ is a Cartesian transformation. Invoking Theorem \ref{countercon}, we deduce that $\overline{\gamma}$ is also a Cartesian transformation, so that the diagram
$$ \xymatrix{ U_1 \ar[r] \ar[d] &  U_0 \ar[d] \\
\overline{W}_0 \ar[r] & U_0 \times \overline{W}_{-1} }$$
is Cartesian. Combining this with the Cartesian diagram
$$ \xymatrix{ \overline{W}_0 \ar[r] \ar[d] & \overline{W}_{-1} \ar[d] \\
U_0 \ar[r] & \overline{U}_{-1} }$$
we deduce that $U$ is effective, as desired.
\end{proof}

\subsection{Localic $\infty$-Topoi}\label{nlocalic}

The standard example of an ordinary topos is the category $\Shv(X; \Set)$ of sheaves (of sets) on a topological space $X$. Of course, not every topos is of this form: the category $\Shv(X; \Set)$ 
is generated under colimits by subobjects of its final object (which can be identified with open subsets of $X$). A topos $\calX$ with this property is said to be {\it localic}, and is determined up to equivalence by the locale $\Sub(1_{\calX})$, which we may view as a $0$-topos. The objective of this section is to obtain an $\infty$-categorical analogue of this picture, which will allow us to relate the theory of $n$-topoi to that of $m$-topoi, for all $0 \leq m \leq n \leq \infty$.

\begin{definition}\label{geomorphn}\index{gen}{geometric morphism!of $n$-topoi}
Let $\calX$ and $\calY$ be $n$-topoi, $0 \leq n \leq \infty$
A {\it geometric morphism} from $\calX$ to $\calY$ is a functor $f_{\ast}: \calX \rightarrow \calY$ which admits a left exact left adjoint (which we will typically denote by $f^{\ast}$). 

We let $\Fun_{\ast}(\calX, \calY)$ denote the full subcategory
of the $\infty$-category $\Fun(\calX,\calY)$ spanned by the geometric morphisms, and
$\Geo^{\GeoR}_{n}$ denote the subcategory of $\widehat{\Cat}_{\infty}$ whose objects are $n$-topoi and whose morphisms are geometric morphisms.\index{not}{FunLast@$\Fun_{\ast}(\calX, \calY)$}\index{not}{TopRn@$\Geo^{\GeoR}_{n}$}
\end{definition}

\begin{remark}
In the case where $n=1$, the $\infty$-category of geometric morphisms $\Fun_{\ast}(\calX, \calY)$ between two $1$-topoi is equivalent to (the nerve of) the category of geometric morphisms
between the ordinary topoi $\h{\calX}$ and $\h{\calY}$.
\end{remark}

\begin{remark}
In the case where $n=0$, the $\infty$-category of geometric morphisms $\Fun_{\ast}(\calX, \calY)$
between two $0$-topoi is equivalent to the nerve of the partially ordered set of homomorphisms
from the underlying locale of $\calY$ to the underlying locale of $\calX$. (A {\it homomorphism} between locales is a map of partially ordered sets which preserve finite meets and arbitrary joins.) In the case where $\calX$ and $\calY$ are associated to (sober) topological spaces $X$ and $Y$, this is simply the set of continuous maps from $X$ to $Y$, partially ordered by specialization.
\end{remark}

If $m \leq n$, then the $\infty$-categories 
$\Geo^{\GeoR}_{m}$ and $\Geo^{\GeoR}_{n}$ are related by the following observation:

\begin{proposition}\label{goto1topos}
Let $\calX$ be an $n$-topos, and let $0 \leq m \leq n$. Then the full subcategory
$\tau_{\leq m-1} \calX$ spanned by the $(m-1)$-truncated objects is an $m$-topos.
\end{proposition}

\begin{proof}
If $m = n = \infty$, the result is obvious. Otherwise, it follows immediately from
$(2)$ of Theorem \ref{nchar}.
\end{proof}

\begin{lemma}\label{eoi1}
Let $\calC$ be a small $n$-category which admits finite limits, and let $\calY$ be an $\infty$-topos.
Then the restriction map
$$ \Fun_{\ast}(\calY, \calP(\calC) ) \rightarrow \Fun_{\ast}( \tau_{\leq n-1} \calY, \calP_{\leq n-1}(\calC) )$$
is an equivalence of $\infty$-categories.
\end{lemma}

\begin{proof}
Let $\calM \subseteq \Fun(\calP(\calC), \calY)$ and
$\calM' \subseteq \Fun(\calP_{\leq n-1}(\calC) , \tau_{\leq n-1} \calY)$
denote the full subcategories spanned by left exact, colimit preserving functors. 
In view of Proposition \ref{switcheroo}, it will suffice to prove that the restriction map
$\theta: \calM \rightarrow \calM'$ is an equivalence of $\infty$-categories.

Let $\calM''$ denote full subcategory of $\Fun(\calP(\calC), \tau_{\leq n-1} \calY)$ spanned by
colimit preserving functors whose restriction to $\calP_{\leq n-1}(\calC)$ is left exact. 
Corollary \ref{truncprop} implies that the restriction map $\theta': \calM'' \rightarrow \calM'$
is an equivalence of $\infty$-categories.

Let $j: \calC \rightarrow \calP_{\leq n-1}(\calC) \subseteq \calP(\calC)$ denote the Yoneda embedding. Composition with $j$ yields a commutative diagram
$$ \xymatrix{ \calM \ar[r]^{\theta} \ar[d]^{\psi} & \calM' \ar[d]^{\psi'} \\
\Fun(\calC,\tau_{\leq n-1} \calY) \ar@{=}[r] & \Fun(\calC,\tau_{\leq n-1} \calY). }$$
Theorem \ref{charpresheaf} implies that $\psi$ and $\psi' \circ \theta'$ are fully faithful. Since $\theta'$ is an equivalence of $\infty$-categories, we deduce that $\psi'$ is fully faithful.
Thus $\theta$ is fully faithful; to complete the proof, we must show that $\psi$ and $\psi'$ have the same essential image. Suppose that $f: \calC \rightarrow \tau_{\leq n-1} \calY$ belongs to the essential image of $\psi'$. Without loss of generality, we may suppose that $f$ is a composition
$$ \calC \stackrel{j}{\rightarrow} \calP_{\leq n-1}(\calC) \stackrel{g^{\ast}}{\rightarrow} \tau_{\leq n-1} \calY. $$
As a composition of left exact functors, $f$ is left exact. We may now invoke Proposition \ref{natash} to deduce that $f$ belongs to the essential image of $\psi$.
\end{proof}

\begin{lemma}\label{eoi2}
Let $\calC$ be a small $n$-category which admits finite limits and is equipped with a Grothendieck topology, and let $\calY$ be an $\infty$-topos.
Then the restriction map
$$ \theta: \Fun_{\ast}(\calY, \Shv(\calC) ) \rightarrow \Fun_{\ast}( \tau_{\leq n-1} \calY, \Shv_{\leq n-1}(\calC) )$$
is an equivalence of $\infty$-categories.
\end{lemma}

\begin{proof}
We have a commutative diagram
$$ \xymatrix{ \Fun_{\ast}( \calY, \Shv(\calC) ) \ar[r]^-{\theta} \ar[d] & \Fun_{\ast}( \tau_{\leq n-1} \calY, \Shv_{\leq n-1}(\calC) ) \ar[d] \\
\Fun_{\ast}(\calY, \calP(\calC) ) \ar[r]^-{\theta'} & \Fun_{\ast}( \tau_{\leq n-1} \calY, \calP_{\leq n-1}(\calC))}$$
where the vertical arrows are inclusions of full subcategories, and $\theta'$ is an equivalence of $\infty$-categories (Lemma \ref{eoi1}). To complete the proof, it will suffice to show that if
$f_{\ast}: \calY \rightarrow \calP(\calC)$ is a geometric morphism such that $f_{\ast} | \tau_{\leq n-1} \calY$ factors through $\Shv_{\leq n-1}(\calC)$, then $f_{\ast}$ factors through $\Shv(\calC)$. 

Let $f^{\ast}$ be a left adjoint to $f_{\ast}$, and let $\overline{S}$ denote the collection of all
morphisms in $\calP(\calC)$ which localize to equivalences in $\Shv(\calC)$. We must show that $f^{\ast}\overline{S} $ consists of equivalences in $\calY$. Let $S \subseteq \overline{S}$ be the collection of monomorphisms which belong to $\overline{S}$. Since $\Shv(\calC)$ is a topological localization of $\calP(\calC)$, it will suffice to show that $f^{\ast} S$ consists of equivalences in $\calY$. Let $g: X \rightarrow Y$ belong to $S$. Since $\calP(\calC)$ is generated under colimits by the essential image of the Yoneda embedding, we can write $Y$ as a colimit of a diagram
$K \rightarrow \calP_{\leq n-1}(\calC)$. Since colimits in $\calP(\calC)$ are universal, we obtain a corresponding expression of $g$ as a colimit of morphisms $\{ g_{\alpha}: X_{\alpha} \rightarrow Y_{\alpha} \}$ which are pullbacks of $g$, where $Y_{\alpha} \in \calP_{\leq n-1}(\calC)$. In this case, $g_{\alpha}$ is again a monomorphism, so that $X_{\alpha}$ is also $(n-1)$-truncated.
Since $f^{\ast}$ commutes with colimits, it will suffice to show that each $f^{\ast}(g_{\alpha})$ is
an equivalence. But this follows immediately from our assumption that $f_{\ast} | \tau_{\leq n-1} \calY$ factors through $\Shv_{\leq n-1}(\calY)$.
\end{proof}

\begin{proposition}\label{swmad}
Let $0 \leq m \leq n \leq \infty$, and let $\calY$ be an $m$-topos. There exists an $n$-topos
$\calX$ and a categorical equivalence 
$f_{\ast}: \tau_{\leq m-1} \calX \rightarrow \calY$ with the following universal property: for any $n$-topos $\calZ$, composition with $f_{\ast}$ induces
an equivalence of $\infty$-categories
$$ \theta: \Fun_{\ast}( \calZ, \calX ) \rightarrow \Fun_{\ast}( \tau_{\leq m-1} \calZ, \calY).$$
\end{proposition}

\begin{proof}
If $m = \infty$, then also $n= \infty$ and we may take $\calX = \calY$. Otherwise, we may apply
Theorem \ref{nchar} to reduce to the case where $\calY = \Shv_{\leq m-1}(\calC)$, where $\calC$ is a small $m$-category which admits finite limits and is equipped with a Grothendieck topology.
In this case, we
let $\calX = \Shv_{\leq n-1}(\calC)$ and define $f_{\ast}$ to be the identity. Let $\calZ$ be an arbitrary $n$-topos. According to Theorem \ref{nchar}, we may assume without loss of generality that $\calZ = \tau_{\leq n-1} \calZ'$, where $\calZ'$ is an $\infty$-topos. We have a commutative diagram
$$ \xymatrix{ & \Fun_{\ast}( \calZ, \calX) \ar[dr]^{\theta} & \\
\Fun_{\ast}(\calZ', \Shv(\calC)) \ar[ur]^{\theta'} \ar[rr]^{\theta''} & & \Fun_{\ast}(\tau_{\leq m-1} \calZ, \calY). }$$
Lemma \ref{eoi2} implies that $\theta'$ and $\theta''$ are equivalences of $\infty$-categories,
so that $\theta$ is also an equivalence of $\infty$-categories.
\end{proof}

\begin{definition}\index{gen}{localic!$n$-topos}\label{stuffera}
Let $0 \leq m \leq n \leq \infty$, and let $\calX$ be an $n$-topos. We will say that
$\calX$ is {\it $m$-localic} if, for any $n$-topos $\calY$, the natural map
$$ \Fun_{\ast}( \calY, \calX) \rightarrow \Fun_{\ast}( \tau_{\leq m-1} \calY, \tau_{\leq m-1} \calX )$$
is an equivalence of $\infty$-categories.
\end{definition}

According to Proposition \ref{swmad}, every
$m$-topos $\calX$ is equivalent to the subcategory of $(m-1)$-truncated objects in an $m$-localic $n$-topos $\calX'$, and $\calX'$ is determined up to equivalence. More precisely, the truncation functor
$$ \Geo^{\GeoR}_{n}  \stackrel{\tau_{\leq m-1}}{\rightarrow} \Geo^{\GeoR}_{m}$$
induces an equivalence $ \calC \rightarrow \Geo^{\GeoR}_{m}$, where
$\calC \subseteq \Geo^{\GeoR}_{n}$ is the full subcategory spanned by the $m$-localic $n$-topoi.
In other words, we may view the $\infty$-category of
$m$-topoi as a {\it localization} of the $\infty$-category of $n$-topoi. In particular, the theory of
$m$-topoi for $m < \infty$ can be regarded as a special case of the theory of $\infty$-topoi. For this reason, we will focus our attention on the case $n = \infty$ for most of the remainder of this book.

\begin{proposition}\label{useiron}
Let $\calX$ be an $n$-localic $\infty$-topos. Then any topological localization of $\calX$ is also $n$-localic.
\end{proposition}

\begin{proof}
The proof of Proposition \ref{swmad} shows that $\calX$ is $n$-localic if and only if there
exists a small $n$-category $\calC$ which admits finite limits, a Grothendieck topology on $\calC$, and an equivalence $\calX \rightarrow \Shv(\calC)$. In other words, $\calX$ is $n$-localic if and only if it is equivalent to a topological localization of $\calP(\calC)$, where $\calC$ is a small $n$-category which admits finite limits. It is clear that any topological localization of $\calX$ has the same property.
\end{proof}

%\begin{remark} *I'm not sure this is true
%If $0 \leq m \leq n < \infty$, then an $n$-topos $\calX$ is $m$-localic if and only if it is 
%generated under colimits by $(m-1)$-truncated objects. If $n = \infty$, then this statement is no longer true, since not every localization of a presheaf $\infty$-category $\calP(\calC)$ is topological.
%\end{remark}

Let $\calX$ be an $\infty$-topos. One should think of the $\infty$-categories $\tau_{\leq n-1} \calX$ as  ``Postnikov sections'' of $\calX$. The classical $1$-truncation $\tau_{\leq 1} X$ of a
homotopy type $X$ remembers only the fundamental groupoid of $X$.
It therefore knows all about local systems of sets on $X$, but
nothing about fibrations over $X$ with non-discrete fibers. The
relationship between $\calX$ and $\tau_{\leq 0} \calX$ is analogous:
$\tau_{\leq 0} \calX$ knows about the sheaves of sets on $\calX$, but
has forgotten about sheaves with nondiscrete stalks.

\begin{remark}
In view of the above discussion, the notation $\tau_{\leq 0} \calX$ is unfortunate
because the analogous notation for the $1$-truncation of a
homotopy type $X$ is $\tau_{\leq 1} X$. We caution the reader not to
regard $\tau_{\leq 0} \calX$ as the result of applying an operation
$\tau_{\leq 0}$ to $\calX$; it instead denotes the essential image of the truncation
functor  $\tau_{\leq 0}: \calX \rightarrow \calX$.
\end{remark}


