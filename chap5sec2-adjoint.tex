% !TEX root = highertopoi.tex
\section{Adjoint Functors}\label{c5s2}

\setcounter{theorem}{0}

Let $\calC$ and $\calD$ be (ordinary) categories. Two functors
$$ \Adjoint{F}{ \calC }{\calD}{G}$$
are said to be {\it adjoint} to one another if there is a functorial bijection
$$ \Hom_{\calD}(F(C), D) \simeq \Hom_{\calC}( C, G(D))$$
defined for $C \in \calC$, $D \in \calD$. Our goal in this section is
to extend the theory of adjoint functors to the $\infty$-categorical setting.\index{gen}{adjoint functor!between ordinary categories}

By definition, a pair of functors $F$ and $G$ (as above) are adjoint if and only if they
determine the same correspondence $$ \calC^{op} \times \calD \rightarrow \Set.$$ 
In \S \ref{corresp}, we introduced an $\infty$-categorical generalization of the notion of a correspondence. In certain cases, a correspondence $\calM$ from an $\infty$-category
$\calC$ to an $\infty$-category $\calD$ determines a functor $F: \calC \rightarrow \calD$, which we say is a functor {\it associated} to $\calM$. We will study these associated functors in \S \ref{afunc1}. The notion of a correspondence is self-dual, so it is possible that the correspondence $\calM$ also determines an associated functor $G: \calD \rightarrow \calC$. In this case, we will say that
$F$ and $G$ are adjoint. We will study the basic properties of adjoint functors in \S \ref{afunc2}.

One of the most important features of adjoint functors is their behavior with respect to limits and colimits: left adjoints preserve colimits, while right adjoints preserve limits. We will prove an $\infty$-categorical analogue of this statement in \S \ref{afunc3}. In certain situations, the {\it adjoint functor theorem} provides a converse to this statement: see \S \ref{aftt}.

The theory of model categories provides a host of examples of adjoint functors between $\infty$-categories. In \S \ref{afunc4}, we will show that a simplicial Quillen adjunction
between a pair of model categories $(\bfA,\bfA')$ determines an adjunction between the associated $\infty$-categories $(\sNerve(\bfA^{\degree}), \sNerve({\bfA'}^{\degree}))$. We will also consider some other examples of situations which give rise to adjoint functors.

In \S \ref{afunc4half}, we study the behavior of adjoint functors when restricted to overcategories.
Our main result (Proposition \ref{curpse}) can summarized as follows: suppose that
$F: \calC \rightarrow \calD$ is a functor between $\infty$-categories which admits a right adjoint
$G$. Assume further that the $\infty$-category $\calC$ admits pullbacks. Then for every object
$C$, the induced functor $\calC_{/C} \rightarrow \calD_{/FC}$ admits a right adjoint, given by
the formula 
$$(D \rightarrow FC) \mapsto (GD \times_{GFC} C \rightarrow C).$$

If a functor $F: \calC \rightarrow \calD$ has a right adjoint $G$, then $G$ is uniquely determined up to equivalence. In \S \ref{afunc5}, we will prove a strong version of this statement, phrased as an (anti)equivalence of functor categories.

In \S \ref{locfunc}, we will restrict the theory of adjoint functors to the special case in which one of the functors is the inclusion of a full subcategory. In this case, we obtain the theory of {\em localizations} of $\infty$-categories. This theory will play a central role in our study of presentable $\infty$-categories (\S \ref{c5s6}), and later in the study of $\infty$-topoi (\S \ref{chap6}). It is also useful in the study of {\it factorization systems} on $\infty$-categories, which we will discuss in
\S \ref{factgen1}.

Finally, in \S \ref{cataut}, we will apply some of the ideas of this and earlier sections to analyze the $\infty$-category of equivalences from $\Cat_{\infty}$ to itself. The main result (Theorem \ref{cabbi}) is that there is essentially only one nontrivial self-equivalence of $\Cat_{\infty}$: namely, the operation which carries each $\infty$-category $\calC$ to its opposite $\calC^{op}$.

\subsection{Correspondences and Associated Functors}\label{afunc1}

Let $p: X \rightarrow S$ be a Cartesian fibration of simplicial sets. In \S \ref{universalfib}, we
saw that $p$ is classified by a functor $S^{op} \rightarrow \Cat_{\infty}$. In particular, if
$S = \Delta^1$, then $p$ determines a diagram
$$ G: \calD \rightarrow \calC$$
in the $\infty$-category $\Cat_{\infty}$, which is well-defined up to equivalence.
We can obtain this diagram by applying the straightening
functor $\St_{S}^{+}$ to the marked simplicial set $X^{\natural}$, and then taking a fibrant replacement. In general, this construction is rather complicated. However, in the special case where
$S = \Delta^1$, it is possible to give a direct construction of $G$; that is our goal in this section.

\begin{definition}\label{fibas}\index{gen}{functor!associated to a correspondence}\index{gen}{correspondence!associated functor}
Let $p: \calM \rightarrow \Delta^1$ be a Cartesian fibration, and suppose given
equivalences of $\infty$-categories $h_0: \calC \rightarrow p^{-1} \{0\}$ and $h_1: \calD \rightarrow p^{-1} \{1\}$. We will say that a functor $g: \calD \rightarrow \calC$ is {\it associated to $\calM$} if
there is a commutative diagram
$$ \xymatrix{ \calD \times \Delta^1 \ar[dr] \ar[rr]^{s} & & \calM \ar[dl] \\
& \Delta^1 & }$$
such that $s| \calD \times \{1\} = h_1$, $s| \calD \times \{0\} = h_0 \circ g$, and
$s| \{x\} \times \Delta^1$ is a $p$-Cartesian edge of $\calM$ for every object $x$ of $\calD$.
\end{definition}

\begin{remark}
The terminology of Definition \ref{fibas} is slightly abusive: it would be more accurate to say that $g$ is associated to the triple $(p: \calM \rightarrow \Delta^1, h_0: \calC \rightarrow p^{-1} \{0\},
h_1: \calD \rightarrow p^{-1} \{1\} )$. 
\end{remark}

\begin{proposition}\label{candi}
Let $\calC$ and $\calD$ be $\infty$-categories, and let $g: \calD \rightarrow \calC$
be a functor. 
\begin{itemize}
\item[$(1)$] There exists a diagram
$$ \xymatrix{ \calC \ar[r] \ar[d] & \calM \ar[d]^{p} & \calD \ar[l] \ar[d] \\
\{0\} \ar[r] & \Delta^1 & \{1\} \ar[l] }$$
where $p$ is a Cartesian fibration, the associated maps $\calC \rightarrow p^{-1} \{0\}$ and $\calD \rightarrow p^{-1} \{1\}$ are isomorphisms, and $g$ is associated to $\calM$.

\item[$(2)$] Suppose given a commutative diagram
$$ \xymatrix{ \calC \ar[r] \ar[dd] & \calM' \ar[d]^{s} & \calD \ar[l] \ar[dd] \\
& \calM \ar[d]^{p} & \\
\{0\} \ar[r] & \Delta^1 & \{1\} \ar[l] }$$ 
where $s$ is a categorical equivalence, $p$ and $p' = p \circ s$ are Cartesian fibrations,
and the maps $\calC \rightarrow p^{-1} \{0\}$, $\calD \rightarrow p^{-1} \{1\}$ are categorical equivalences. The functor $g$ is associated to $\calM$ if and only if it is associated to $\calM'$.

\item[$(3)$] Suppose given diagrams 
$$ \xymatrix{ \calC \ar[r] \ar[d] & \calM' \ar[d]^{p'} & \calD \ar[l] \ar[d] \\
\{0\} \ar[r] & \Delta^1 & \{1\} \ar[l] }$$
$$ \xymatrix{ \calC \ar[r] \ar[d] & \calM'' \ar[d]^{p''} & \calD \ar[l] \ar[d] \\
\{0\} \ar[r] & \Delta^1 & \{1\} \ar[l] }$$
as above, such that $g$ is associated to both $\calM'$ and $\calM''$. Then there
exists a third such diagram
$$ \xymatrix{ \calC \ar[r] \ar[d] & \calM \ar[d]^{p} & \calD \ar[l] \ar[d] \\
\{0\} \ar[r] & \Delta^1 & \{1\} \ar[l] }$$
and a diagram 
$$ \calM' \leftarrow \calM \rightarrow \calM''$$
of categorical equivalences in $(\sSet)_{\calC \coprod \calD/ \,/\Delta^1}$.
\end{itemize}
\end{proposition}

\begin{proof}
We begin with $(1)$. Let $\calC^{\natural}$ and $\calD^{\natural}$ denote
the simplicial sets $\calC$ and $\calD$ considered as marked simplicial sets, where the marked edges are precisely the equivalences. We set
$$N = ( \calD^{\natural} \times (\Delta^{1})^{\sharp}) \coprod_{ \calD^{\natural} \times \{0\}^{\sharp} } \calC^{\natural}.$$ 
The small object argument implies the existence of a factorization
$$ N \rightarrow N(\infty) \rightarrow (\Delta^1)^{\sharp},$$
where the left map is marked anodyne and the right map has the right lifting property with respect to all marked anodyne morphisms. We remark
that we can obtain $N(\infty)$ as the colimit of a transfinite sequence of simplicial sets $N(\alpha)$, where $N(0) = N$, $N(\alpha)$ is the colimit of the sequence $\{ N(\beta) \}_{\beta < \alpha}$ when
$\alpha$ is a limit ordinal, and each $N(\alpha+1)$ fits into a pushout diagram
$$ \xymatrix{ A \ar@{^{(}->}[d] \ar[r] & N(\alpha) \ar[d] \ar[dr] &  \\
B \ar[r] \ar@{-->}[ur] & N(\alpha+1) \ar[r] & (\Delta^1)^{\sharp} }$$
where the left vertical map is one of the generators for the class of marked anodyne maps
given in Definition \ref{markanod}. We may furthermore assume that there does {\em not} exist a dotted arrow as indicated in the diagram. It follows by induction on $\alpha$ that
$N(\alpha) \times_{\Delta^1} \{0\} \simeq \calC^{\natural}$ and $N(\alpha) \times_{ \Delta^1} \{1\} \simeq \calD^{\natural}$. According to Proposition \ref{dubudu}, $N(\infty) \simeq \calM^{\natural}$ for
some Cartesian fibration $\calM \rightarrow \Delta^1$. It follows immediately that
$\calC \simeq \calM_{\{0\}}$, $\calD \simeq \calM_{\{1\}}$, and that $g$ is associated to $\calM$.

We now prove $(2)$. The ``if'' direction is immediate from the definition. Conversely, suppose that $g$ is associated to $\calM$. To show that $g$ is associated to $\calM'$, we need to produce the dotted arrow indicated in the diagram
$$ \xymatrix{ \calD \times \bd \Delta^1 \ar[r] \ar@{^{(}->}[d] & \calM' \\
\calD \times \Delta^1. \ar@{-->}[ur] & }$$
According to Proposition \ref{princex}, we may replace $\calM'$ by the equivalent $\infty$-category
$\calM$; the desired result then follows form the assumption that $g$ is associated to $\calM$.

To prove $(3)$, we take $\calM$ to be the correspondence constructed in the course of proving $(1)$. It will suffice to construct an appropriate categorical equivalence $\calM \rightarrow \calM'$; the same argument will construct the desired map $\calM \rightarrow \calM''$. Consider the diagram
$$ \xymatrix{ N \ar@{^{(}->}[d]^{s'} \ar[r]^{s} & \calM' \ar[d] \\
\calM \ar[r] \ar@{-->}[ur]^{s''} & \Delta^1. }$$
(Here we identify $N$ with its underlying simplicial set by forgetting the class of marked edges, and
the top horizontal map exhibits $g$ as associated to $\calM'$.) In the terminology of \S \ref{funkystructure}, the maps $s$ and $s'$ are both quasi-equivalences. By Proposition
\ref{qequiv}, they are categorical equivalences. The projection $\calM' \rightarrow \Delta^1$ is
a categorical fibration and $s'$ is a trivial cofibration, which ensures the existence of the arrow $s''$. 
The factorization $s = s'' \circ s'$ shows that $s''$ is a categorical equivalence, and completes the proof.
\end{proof}

Proposition \ref{candi} may be informally summarized by saying that every functor
$g: \calD \rightarrow \calC$ is associated to some Cartesian fibration $p: \calM \rightarrow \Delta^1$, and that $\calM$ is determined up to equivalence. Conversely, the Cartesian fibration also determines $g$:

\begin{proposition}\label{funcas}
Let $p: \calM \rightarrow \Delta^1$ be a Cartesian fibration, and let
$h_0: \calC \rightarrow p^{-1} \{0\}$ and $h_1: \calD \rightarrow p^{-1} \{1\}$ be
categorical equivalences. There exists a functor $g: \calD \rightarrow \calC$ associated to $\calM$.
Any other functor $g': \calC \rightarrow \calD$ is associated to $p$ if and only if $g$ is equivalent to $g'$ as objects of the $\infty$-category $\calC^{\calD}$.
\end{proposition}

\begin{proof}
Consider the diagram
$$ \xymatrix{ \calD^{\flat} \times \{1\} \ar[d] \ar[r] & \calM^{\natural} \ar[d] \\
\calD^{\flat} \times (\Delta^1)^{\sharp} \ar[r] \ar@{-->}^{s}[ur] & (\Delta^1)^{\sharp}.}$$
By Proposition \ref{markanodprod}, the left vertical map is marked anodyne, so the
dotted arrow exists. Consider the map $s_0: s| \calD \times \{0\}: \calD \rightarrow p^{-1} \{0\}$. 
Since $h_0$ is a categorical equivalence, there exists a map $g: \calD \rightarrow \calC$
such that the functions $h_0 \circ g$ and $s_0$ are equivalent. Let $e: \calD \times \Delta^1 \rightarrow \calM$ be an equivalence from $h_0 \circ g$ to $s_0$. Let $e': \calD \times \Lambda^2_1 \rightarrow \calM$ be the result of amalgamating $e$ with $s$. Then we have a commutative diagram of marked simplicial sets
$$ \xymatrix{ \calD^{\flat} \times (\Lambda^2_1)^{\sharp} \ar@{^{(}->}[d] \ar[r]^{e'} & \calM^{\natural} \ar[d] \\
\calD^{\flat} \times (\Delta^2)^{\sharp} \ar[r] \ar@{-->}[ur]^{e''} & (\Delta^1)^{\sharp}. }$$
Because left vertical map is marked anodyne there exists a morphism $e''$ as indicated, rendering the diagram commutative. The restriction
$e''| \calD \times \Delta^{ \{0,2\} }$ exhibits $g$ as associated to $\calM$.

Now suppose that $g'$ is another functor associated to $p$. Then there exists a commutative diagram of marked simplicial sets
$$ \xymatrix{ \calD^{\flat} \times \{1\} \ar@{^{(}->}[d] \ar[r] & \calM^{\natural} \ar[d] \\
\calD^{\flat} \times (\Delta^1)^{\sharp} \ar[r] \ar@{-->}[ur]^{s'} & (\Delta^1)^{\sharp},}$$
with $g' = s'| \calD \times \{0\}$. Let $s''$ be the map obtained by amalgamating
$s$ and $s'$. Consider the diagram
$$ \xymatrix{ \calD^{\flat} \times (\Lambda^2_2)^{\sharp} \ar@{^{(}->}[d] \ar[r]^{s''} & \calM^{\natural} \ar[d] \\
\calD^{\flat} \times (\Delta^2)^{\sharp} \ar[r] \ar@{-->}[ur]^{s'''} & (\Delta^1)^{\sharp}. }$$
Since the left vertical map is marked anodyne, the indicated dotted arrow $s''$ exists.
The restriction $s''| \calD \times \Delta^{ \{0,1\} }$ is an equivalence between $h_0 \circ g$ and $h_0 \circ g'$. Since $h_0$ is a categorical equivalence, $g$ and $g'$ are themselves homotopic.

Conversely, suppose that $f: \calD \times \Delta^1 \rightarrow \calC$ is an equivalence from $g'$ to $g$. The maps $s$ and $h_0 \circ f$ amalgamate to give a map $f': \calD \times \Lambda^2_1 \rightarrow \calC$ which fits into a commutative diagram of marked simplicial sets:
$$ \xymatrix{ \calD^{\flat} \times (\Lambda^2_1)^{\sharp} \ar@{^{(}->}[d] \ar[r]^{f'} & \calM^{\natural} \ar[d] \\
\calD^{\flat} \times (\Delta^2)^{\sharp} \ar[r] \ar@{-->}[ur]^{f''} & (\Delta^1)^{\sharp}. }$$
The left vertical map is marked anodyne, so there exists a dotted arrow $f''$ as indicated; then
the map $f'' | \calD \times \Delta^{ \{0,2\} }$ exhibits that $g'$ is associated to $p$.
\end{proof}

\begin{proposition}\label{compass}
Let $p: \calM \rightarrow \Delta^2$ be a Cartesian fibration, and suppose given equivalences
of $\infty$-categories $\calC \rightarrow p^{-1} \{0\}$, $\calD \rightarrow p^{-1} \{1\}$, and
$\calE \rightarrow p^{-1} \{2\}$. Suppose that $\calM \times_{\Delta^2} \Delta^{ \{0,1\} }$
is associated to a functor $f: \calD \rightarrow \calC$, and that $\calM \times_{ \Delta^2} \Delta^{ \{1,2\} }$ is associated to a functor $g: \calE \rightarrow \calD$. Then $\calM \times_{\Delta^2} \Delta^{ \{0,2\} }$ is associated to the composite functor $f \circ g$.
\end{proposition}

\begin{proof}
Let $X$ be the mapping simplex of the sequence of functors
$$ \calE \stackrel{g}{\rightarrow} \calD \stackrel{f}{\rightarrow} \calC.$$
Since $f$ and $g$ are associated to restrictions of $\calM$, we obtain a commutative diagram
$$ \xymatrix{ X \times_{\Delta^2} \Lambda^2_1 \ar@{^{(}->}[d] \ar[r] & \calM \ar[d] \\
X \ar[r] \ar@{-->}[ur]^{s} & \Delta^2.}$$
The left vertical inclusion is a pushout of $\calE \times \Lambda^2_1 \subseteq \calE \times \Delta^2$, which is inner anodyne. Since $p$ is inner anodyne, there exists a dotted arrow $s$ as indicated in the diagram. The restriction $s| X \times_{\Delta^2} \Delta^{ \{0,2\}}$ exhibits
that the functor $f \circ g$ is associated to the correspondence $\calM \times_{\Delta^2} \Delta^{ \{0,2\}}$.
\end{proof}

\begin{remark}
Taken together, Propositions \ref{candi} and \ref{funcas} assert that there is a bijective correspondence between equivalence classes of functors $\calD \rightarrow \calC$
and equivalence classes of Cartesian fibrations $p: \calM \rightarrow \Delta^1$ equipped with
equivalences $\calC \rightarrow p^{-1} \{0\}$, $\calD \rightarrow p^{-1} \{1\}$.
\end{remark}

\subsection{Adjunctions}\label{afunc2}

In \S \ref{afunc1}, we established a dictionary that allows us to pass back and forth between functors
$g: \calD \rightarrow \calC$ and Cartesian fibrations $p: \calM \rightarrow \Delta^1$. The dual argument shows if $p$ is a coCartesian fibration it also determines a functor
$f: \calC \rightarrow \calD$. In this case, we will say that $f$ and $g$ are {\em adjoint} functors.\index{gen}{adjoint functor!between $\infty$-categories}

\begin{definition}\index{gen}{adjunction}\index{gen}{correspondence!adjunction}
Let $\calC$ and $\calD$ be $\infty$-categories. An {\it adjunction} between $\calC$ and $\calD$ is a map $q: \calM \rightarrow \Delta^1$ which is both a Cartesian fibration and a coCartesian fibration, together with equivalences $\calC \rightarrow \calM_{\{0\}}$ and $\calD \rightarrow \calM_{\{1\}}$.

Let $\calM$ be an adjunction between $\calC$ and $\calD$, and let $f: \calC \rightarrow \calD$ and $g: \calD \rightarrow \calC$ be functors associated to $\calM$. In this case, we will say that $f$ is {\it left adjoint} to $g$ and $g$ is {\it right adjoint} to $f$.\index{gen}{right adjoint}\index{gen}{left adjoint}
\end{definition}

\begin{remark}
Propositions \ref{candi} and \ref{funcas} imply that if a functor $f: \calC \rightarrow \calD$ has a right adjoint $g: \calD \rightarrow \calC$, then $g$ is uniquely determined up to homotopy. In fact, we will later see that $g$ is determined up to a contractible ambiguity.
\end{remark}

We now verify a few basic properties of adjunctions:

\begin{lemma}\label{gruft}
Let $p: X \rightarrow S$ be a locally Cartesian fibration of simplicial sets. Let $e: s \rightarrow s'$ be an edge of $S$ with the following property:
\begin{itemize}
\item[$(\ast)$] For every $2$-simplex
$$ \xymatrix{ & x' \ar[dr]^{\overline{e}'} & \\
x \ar[rr]^{\overline{e}''} \ar[ur]^{\overline{e}} & & x''}$$
in $X$ such that $p( \overline{e}) = e$, if $\overline{e}$ and $\overline{e}'$ are locally $p$-Cartesian, then $\overline{e}''$ is locally $p$-Cartesian.
\end{itemize}
Let $\overline{e}: x \rightarrow y$ be a locally $p$-coCartesian edge such that
$p( \overline{e} ) = e$. Then $\overline{e}$ is $p$-coCartesian.
\end{lemma}

\begin{proof}
We must show that for any $n \geq 2$ and any diagram
$$ \xymatrix{ \Lambda^n_0 \ar[r]^{f} \ar@{^{(}->}[d] & X \ar[d] \\
\Delta^n \ar[r] \ar@{-->}[ur] & S }$$
such that $f|\Delta^{ \{0,1\} }=\overline{e}$, there exists a dotted arrow as indicated. Pulling back along the bottom horizontal map, we may reduce to the case $S = \Delta^n$; in particular, $X$ and
$S$ are both $\infty$-categories.

According to (the dual of) Proposition \ref{charCart}, it suffices to show that composition with $\overline{e}$ gives a homotopy Cartesian diagram
$$ \xymatrix{ \bHom_{X}(y,z) \ar[r] \ar[d] & \bHom_{X}(x,z) \ar[d] \\
\bHom_{S}(p(y),p(z)) \ar[r] & \bHom_{S}(p(x),p(z))}.$$
There are two cases to consider: if $\bHom_{S}(p(y), p(z)) = \emptyset$, there is nothing to prove. Otherwise, we must show that composition with $f$ induces a homotopy equivalence $\bHom_{X}(y,z) \rightarrow \bHom_{X}(x,z)$.

In view of the assumption that $S = \Delta^n$, there is a unique morphism $g_0: p(y) \rightarrow p(z)$. Let $g: y' \rightarrow z$ be a locally $p$-Cartesian edge lifting $g_0$. We have a commutative diagram
$$ \xymatrix{ \bHom_{X}(y,y') \ar[r] \ar[d] & \bHom_{X}(x,y') \ar[d] \\
\bHom_{X}(y,z) \ar[r] & \bHom_{X}(x,z). }$$
Since $g$ is locally $p$-Cartesian, the left vertical arrow is a homotopy equivalence. Since
$e$ is locally $p$-coCartesian, the top horizontal arrow is a homotopy equivalence. It will therefore suffice to show that the map $\bHom_{X}(x,y') \rightarrow \bHom_{X}(x,z)$ is a homotopy equivalence.

Choose a locally $p$-Cartesian edge $\overline{e}': x' \rightarrow y'$ in $X$ with $p( \overline{e}')=e$, so that we have another commutative diagram
$$ \xymatrix{ & \bHom_{X}(x,x') \ar[dl] \ar[dr] & \\
\bHom_{X}(x,y') \ar[rr] & & \bHom_{X}(x,z). }$$
Using the two-out-of-three property, we are reduced to proving that both of the diagonal
arrows are homotopy equivalences. For the diagonal arrow on the left, this follows from our assumption that $\overline{e}'$ is locally $p$-Cartesian. For the arrow on the right, it suffices to show that the composition $g \circ \overline{e}'$ is locally $p$-coCartesian, which follows
from assumption $(\ast)$.
\end{proof}

\begin{corollary}\label{grutt1}
Let $p: X \rightarrow S$ be a Cartesian fibration of simplicial sets. An edge $e: x \rightarrow y$ of $X$ is $p$-coCartesian if and only if it is locally $p$-coCartesian (see the discussion preceding Proposition \ref{gotta}).
\end{corollary}

\begin{corollary}\label{getcocart}
Let $p: X \rightarrow S$ be a Cartesian fibration of simplicial sets. The following conditions are equivalent:
\begin{itemize}
\item[$(1)$] The map $p$ is a coCartesian fibration. 
\item[$(2)$] For every edge $f: s \rightarrow s'$ of $S$, the induced functor
$f^{\ast}: X_{s'} \rightarrow X_{s}$ has a left adjoint.
\end{itemize}
\end{corollary}

\begin{proof}
By definition, the functor corresponding to an edge $f: \Delta^1 \rightarrow S$ has a left adjoint if and only if the pullback $X \times_{S} \Delta^1 \rightarrow \Delta^1$ is a coCartesian fibration.
In other words, condition $(2)$ is equivalent to the assertion that for every
edge $f: s \rightarrow s'$ and every vertex $\widetilde{s}$ of $X$ lifting $s$, there
exists a {\em locally} $p$-coCartesian edge $\widetilde{f}: \widetilde{s} \rightarrow \widetilde{s}'$ lifting $f$. Using Corollary \ref{grutt1}, we conclude that $\widetilde{f}$ is automatically
$p$-coCartesian, so that $(2)$ is equivalent to $(1)$.
\end{proof}

\begin{proposition}\label{compadjoint}\index{gen}{adjoint functor!and composition}
Let $f: \calC \rightarrow \calD$ and $f': \calD \rightarrow \calE$ be functors between $\infty$-categories. Suppose that $f$ has a right adjoint $g$ and that $f'$ has a right adjoint $g'$. Then
$g \circ g'$ is right adjoint to $f' \circ f$.
\end{proposition}

\begin{proof}
Let $\phi$ denote the composable sequence of morphisms
$$ \calC \stackrel{g}{\leftarrow} \calD \stackrel{g'}{\leftarrow} \calE.$$
Let $M(\phi)$ denote the mapping simplex, and choose a factorization
$$ M(\phi) \stackrel{s}{\rightarrow} X \stackrel{q}{\rightarrow} \Delta^2$$
where $s$ is a quasi-equivalence and $X \rightarrow \Delta^2$ is a Cartesian fibration (using
Proposition \ref{sharpsimplex}). We first show that $q$ is a coCartesian fibration. In other words, we must show that for every object $\overline{x} \in \calC$ and every morphism $e: q(\overline{x}) \rightarrow y$, there is a $q$-Cartesian edge $\overline{e}: \overline{x} \rightarrow \overline{y}$ lifting $e$. This is clear if $e$ is degenerate. If $e = \Delta^{ \{0,1\} } \subseteq \Delta^2$, then
the existence of a left adjoint to $g$ implies that $e$ has a locally $q$-coCartesian lift $\overline{e}$. Lemma \ref{gruft} implies that $\overline{e}$ is $q$-coCartesian. Similarly,
if $e = \Delta^{ \{1,2\} }$, then we can find a $q$-coCartesian lift of $e$. Finally, if $e$ is the long edge $\Delta^{ \{0,2\} }$, then we may write $e$ as a composite $e' \circ e''$; the existence of a $q$-coCartesian lift of $e$ follows from the existence of $q$-coCartesian lifts of $e'$ and $e''$.
We now apply Proposition \ref{compass} and deduce that the adjunction $X \times_{\Delta^2} \Delta^{ \{0,2\} }$ is associated to both $g \circ g'$ and $f' \circ f$.
\end{proof}

In classical category theory, one can spell out the relationship between a pair of adjoint
functors $f: \calC \rightarrow \calD$ and $g: \calD \rightarrow \calC$ by specifying a {\it unit} transformation $\id_{\calC} \rightarrow g \circ f$ (or, dually, a {\it counit} $f \circ g \rightarrow \id_{\calD}$). This concept generalizes to the $\infty$-categorical setting as follows:

\begin{definition}
Suppose given a pair of functors
$$ \Adjoint{f}{\calC}{\calD}{g}$$
between $\infty$-categories. A {\it unit transformation} for $(f,g)$ is a morphism
$u: \id_{\calC} \rightarrow g \circ f$ in $\Fun(\calC,\calC)$ with the following property:
for every pair of objects $C \in \calC$, $D \in \calD$, the composition
$$ \bHom_{\calD}(f(C), D) \rightarrow \bHom_{\calC}(g(f(C)), g(D))
\stackrel{u(C)}{\rightarrow} \bHom_{\calC}(C, g(D))$$ is an isomorphism in the homotopy category $\calH$.\index{gen}{transformation!unit}\index{gen}{transformation!counit}\index{gen}{unit transformation}\index{gen}{counit transformation}
\end{definition}

\begin{proposition}\label{storut}\index{gen}{adjoint functor!and unit transformations}
Let $f: \calC \rightarrow \calD$ and $g: \calD \rightarrow \calC$ be a pair of functors between $\infty$-categories $\calC$ and $\calD$. The following conditions are equivalent:
\begin{itemize}
\item[$(1)$] The functor $f$ is a left adjoint to $g$.
\item[$(2)$] There exists a unit transformation $u: \id_{\calC} \rightarrow g \circ f$.
\end{itemize}
\end{proposition}

\begin{proof}
Suppose first that $(1)$ is satisfied.
Choose an adjunction $p: M \rightarrow \Delta^1$ which is associated to $f$ and $g$; according to $(1)$ of Proposition \ref{candi} we may identify $M_{ \{0\} }$ with $\calC$ and $M_{ \{1\} }$ with $\calD$. Since $f$ is associated to $M$, there is a map $F: \calC \times \Delta^1 \rightarrow M$
such that $F | \calC \times \{0\} = \id_{\calC}$ and $F| \calC \times \{1\} = f$, with each edge
$F| \{c\} \times \Delta^1$ $p$-coCartesian. Similarly, there is a map $G: \calD \times \Delta^1 \rightarrow M$ with $G | \calD \times \{1\} = \id_{\calD}$, $G| \calD \times \{0\} = g$, and 
$G| \{d\} \times \Delta^1$ is $p$-Cartesian for each object $d \in \calD$. Let 
$F' : \Lambda^2_2 \times \calC \rightarrow M$ be such that $F' | \Delta^{ \{0,2\} } \times \calC = F$ and $F' | \Delta^{ \{1,2\} } \times \calC = G \circ ( f \times \id_{\Delta^1} )$. Consider the diagram
$$ \xymatrix{ \Lambda^2_2 \times \calC \ar@{^{(}->}[d] \ar[r]^{F'} & M \ar[d] \\
\Delta^2 \times \calC \ar[r] \ar@{-->}[ur]^{F''} & \Delta^1. }$$ Using the fact $F' | \{c\} \times \Delta^{ \{1,2\} }$ is $p$-Cartesian for every object $c \in C$, we deduce the existence of the indicated dotted arrow $F''$. We now define $u = F' | \calC \times \Delta^{ \{0,1\} }$. We may regard $u$ as a natural transformation $\id_{\calC} \rightarrow g \circ f$. We claim that $u$ is a unit transformation.
 In other words, we must show that for any objects $C \in \calC$, $D \in \calD$, the composite map
$$ \bHom_{\calD}(fC, D) \rightarrow \bHom_{\calC}(gfC, gD)
\stackrel{u}{\rightarrow} \bHom_{\calC}(C,gD)$$
is an isomorphism in the homotopy category $\calH$ of spaces. This composite map
fits into a commutative diagram
$$ \xymatrix{ \bHom_{\calD}( f(C), D) \ar[r] \ar[d] &  \bHom_{\calD}(g(f(C)), g(D)) \ar[r] &
\bHom_{\calD}(C,g(D)) \ar[d] \\
\bHom_{M}(C,D) \ar[rr] & & \bHom_{M}(C,D). }$$
The left and right vertical arrows in this diagram are given by composition with
a $p$-coCartesian and a $p$-Cartesian morphism in $M$, respectively. Proposition \ref{compspaces} implies that these maps are homotopy equivalences.

We now prove that $(2) \Rightarrow (1)$. Choose a correspondence $p:M \rightarrow \Delta^1$ from
$\calC$ to $\calD$ which is associated to the functor $g$, via a map
$G: \calD \times \Delta^1 \rightarrow M$ as above. We have natural transformations
$$ \id_{\calC} \stackrel{u}{\rightarrow} g \circ f \stackrel{G \circ (f \times \id_{\Delta^1})}{\longrightarrow}
f.$$
Let $F: \calC \times \Delta^1 \rightarrow M$ be a composition of these transformations. We will complete the proof by showing that $F$ exhibits $M$ as a correspondence associated to the functor $f$. It will suffice to show that for each object $C \in \calC$, $F(C): C \rightarrow fC$ is 
$p$-coCartesian. According to Proposition \ref{charCart}, it will suffice to show that for each object $D \in \calD$, composition with $F(C)$ induces a homotopy equivalence
$ \bHom_{\calD}(f(C),D) \rightarrow \bHom_{M}(C,D)$. As above, this map fits into a commutative diagram 
$$ \xymatrix{ \bHom_{\calD}( f(C), D) \ar[r] \ar[d] &  \bHom_{\calD}(g(f(C)), g(D)) \ar[r] &
\bHom_{\calD}(C,g(D)) \ar[d] \\
\bHom_{M}(C,D) \ar[rr] & & \bHom_{M}(C,D) }$$
where the upper horizontal composition is an equivalence (since $u$ is a unit transformation)
and the right vertical arrow is an equivalence (since it is given by composition with a $p$-Cartesian morphism). It follows that the left vertical arrow is also a homotopy equivalence, as desired.
\end{proof}

\begin{proposition}\label{adjhom}
Let $\calC$ and $\calD$ be $\infty$-categories, and let $f: \calC \rightarrow \calD$ and
$g: \calD \rightarrow \calC$ be adjoint functors. Then $f$ and $g$ induce adjoint functors
$\h{f}: \h{\calC} \rightarrow \h{\calD}$ and $\h{g}:\h{\calD} \rightarrow \h{\calC}$ between $(${}$\calH$-enriched$)$ homotopy categories.
\end{proposition}

\begin{proof}
This follows immediately from Proposition \ref{storut}, since a unit transformation
$\id_{\calC} \rightarrow g \circ f$ induces a unit transformation
$\id_{\h{\calC}} \rightarrow (\h{g}) \circ (\h{f})$.
\end{proof}

The converse to Proposition \ref{adjhom} is false. 
If $f: \calC \rightarrow \calD$ and $g: \calD \rightarrow \calC$ are functors such that
$\h{f}$ and $\h{g}$ are adjoint to one another, then $f$ and $g$ are not necessarily adjoint.
Nevertheless, the existence of adjoints can be tested at the level of (enriched) homotopy categories.

\begin{lemma}\label{storkk}
Let $p: \calM \rightarrow \Delta^1$ be an inner fibration of simplicial sets, giving a correspondence
between the $\infty$-categories $\calC = \calM_{ \{0\} }$ and $\calD = \calM_{ \{1\} }$. Let $c$ be an object of $\calC$, $d$ an object of $\calD$, and $f: c \rightarrow d$ a morphism. The following are equivalent:
\begin{itemize}
\item[$(1)$] The morphism $f$ is $p$-Cartesian.
\item[$(2)$] The morphism $f$ gives rise to a Cartesian morphism in the enriched homotopy category $\h{\calM}$; in other words, composition with $p$ induces homotopy equivalences
$$ \bHom_{\calC }(c',c) \rightarrow \bHom_{\calM}(c',d)$$ for every
object $c' \in \calC$.
\end{itemize}
\end{lemma}

\begin{proof}
This follows immediately from Proposition \ref{charCart}.
\end{proof}

\begin{lemma}\label{storkkkk}
Let $p: \calM \rightarrow \Delta^1$ be an inner fibration, so that $\calM$ can be identified with a correspondence from $\calC = p^{-1} \{0\}$ to $\calD = p^{-1} \{1\}$. The following conditions are equivalent:
\begin{itemize}
\item[$(1)$] The map $p$ is a Cartesian fibration.
\item[$(2)$] There exists a $\calH$-enriched functor functor $g: \h{ \calD} \rightarrow \h{\calC}$ and a functorial identification $$\bHom_{\calM}(c,d) \simeq \bHom_{\calC}(c,g(d)).$$
\end{itemize}
\end{lemma}

\begin{proof}
If $p$ is a Cartesian fibration, then there is a functor $\calD \rightarrow \calC$ associated to $\calM$; we can then take $g$ to be the associated functor on enriched homotopy categories. Conversely, suppose that there exists a functor $g$ as above. We wish to show that $p$ is a Cartesian fibration. In other words, we must show that for every object $d \in \calD$, there is an object $c \in \calC$
and a $p$-Cartesian morphism $f: c \rightarrow d$. We take $c=g(d)$; in view of the identification
$\bHom_{\calM}(c,d) \simeq \bHom_{\calC}(c,c)$, there exists a morphism $f: c \rightarrow d$ corresponding to the identity $\id_{c}$. Lemma \ref{storkk} implies that $f$ is $p$-Cartesian, as desired.
\end{proof}

\begin{proposition}\index{gen}{adjoint functor!existence of}\label{sumpytump}
Let $f: \calC \rightarrow \calD$ be a functor between $\infty$-categories. Suppose that
the induced functor of $\calH$-enriched categories $\h{f}: \h{\calC} \rightarrow \h{\calD}$ admits a right adjoint. Then $f$ admits a right adjoint.
\end{proposition}

\begin{proof}
According to $(1)$ of Proposition \ref{candi}, there is a coCartesian fibration
$p: \calM \rightarrow \Delta^1$ associated to $f$. Let $\h{g}$ be the right adjoint of
$\h{f}$. Applying Lemma \ref{storkkkk}, we deduce that $p$ is a Cartesian fibration. Thus
$p$ is an adjunction, so that $f$ has a right adjoint as desired.
\end{proof}

\subsection{Preservation of Limits and Colimits}\label{afunc3}

Let $\calC$ and $\calD$ be ordinary categories, and let $F: \calC \rightarrow \calD$ be a functor.
If $F$ has a right adjoint $G$, then $F$ preserves colimits; we have a chain of natural isomorphisms
\begin{eqnarray*}
\Hom_{\calD}( F (\varinjlim C_{\alpha}), D) & \simeq & \Hom_{\calC}( \varinjlim C_{\alpha}, G(D)) \\
& \simeq & \varprojlim \Hom_{\calC}( C_{\alpha}, G(D)) \\
& \simeq & \varprojlim \Hom_{\calD}(F(C_{\alpha}), D) \\
& \simeq & \Hom_{\calD}( \varinjlim F(C_{\alpha}), D).
\end{eqnarray*}
In fact, this is in some sense the {\em defining} feature of left adjoints: under suitable set-theoretic assumptions, the {\em adjoint functor theorem} asserts that any colimit preserving functor admits a right adjoint. We will prove an $\infty$-categorical version of the adjoint functor theorem in \S \ref{aftt}. Our goal in this section is to lay the groundwork, by showing that left adjoints preserve colimits in the $\infty$-categorical setting. We will first need to establish several lemmas.

\begin{lemma}\label{lotusss}
Suppose given a diagram
$$ \xymatrix{ K \times \Delta^1 \ar[rr]^P \ar[dr] & & \calM \ar[dl]^q \\
& \Delta^1 & }$$
of simplicial sets, where $\calM$ is an $\infty$-category and $P| \{k\} \times \Delta^1$
is $q$-coCartesian for every vertex $k$ of $K$. Let $p = P| K \times \{0\}$. Then the induced map
$$ \psi: \calM_{P/} \rightarrow \calM_{p/}$$ induces a trivial fibration
$$ \psi_1: \calM_{P/} \times_{\Delta^1} \{1\} \rightarrow \calM_{p/} \times_{\Delta^1} \{1\}.$$
\end{lemma}

\begin{proof}
If $K$ is a point, then the assertion of the Lemma reduces immediately to the definition of a coCartesian edge. In the general case, we note that
$\psi$ and $\psi_1$ are both left fibrations between $\infty$-categories. Consequently, it suffices to show that $\psi_1$ is a categorical equivalence. In doing so, we are free to replace
$\psi$ by the equivalent map $\psi': \calM^{P/} \rightarrow \calM^{p/}$. To prove that $\psi'_1:
\calM^{P/} \times_{ \Delta^1} \{1\} \rightarrow \calM^{p/} \times_{\Delta^1} \{1\}$ is a trivial fibration, we must show that for every inclusion $A \subseteq B$ of simplicial sets and any map
$$ k_0: ((K \times \Delta^1) \diamond A) \coprod_{ (K \times \{0\}) \diamond A }
((K \times \{0\}) \diamond B) \rightarrow \calM$$
with $k_0| K \times \Delta^1 = P$ and $k_0(B) \subseteq q^{-1} \{1\}$, there exists an extension
of $k_0$ to a map $k: (K \times \Delta^1) \diamond B \rightarrow \calM$. Let
$$X = (K \times \Delta^1) \coprod_{K \times \Delta^1 \times B \times \{0\}} (K \times \Delta^1 \times B \times \Delta^1)$$ and let $h: X \rightarrow K \diamond B$ be the natural map. 
Let 
$$ X' = h^{-1} ((K \times \Delta^1) \diamond A) \coprod_{ (K \times \{0\}) \diamond A }
((K \times \{0\}) \diamond B) \subseteq X,$$
and let $\widetilde{k}_0: X' \rightarrow \calM$ be the composition $k_0 \circ h$. It suffices to
prove that there exists an extension of $\widetilde{k}_0$ to a map $\widetilde{k}: X \rightarrow \calM$.
Replacing $\calM$ by $\bHom_{\Delta^1}(K,\calM)$, we may reduce to the case where $K$ is a point, which we already treated above.
\end{proof}

\begin{lemma}\label{storka}
Let $q: \calM \rightarrow \Delta^1$ be a correspondence between $\infty$-categories
$\calC = q^{-1} \{0\}$ and $\calD = q^{-1} \{1\}$, and let $p: K \rightarrow \calC$ be a diagram in $\calC$. Let $f: c \rightarrow d$ be a $q$-Cartesian morphism in $\calM$ from $c \in \calC$
to $d \in \calD$. Let $r: \calM_{p/} \rightarrow \calM$ be the projection, and let
$\overline{d}$ be an object of $\calM_{p/}$ with $r(\overline{d})=d$. Then:
\begin{itemize}
\item[$(1)$] There exists a morphism $\overline{f}: \overline{c} \rightarrow \overline{d}$
in $\calM_{p/}$ satisfying $f = r( \overline{f} )$.
\item[$(2)$] Any morphism $\overline{f}: \overline{c} \rightarrow \overline{d}$ which
satisfies $r( \overline{f}) = f$ is $r$-Cartesian.
\end{itemize}
\end{lemma}

\begin{proof}
We may identify $\overline{d}$ with a map $\overline{d}: K \rightarrow \calM_{/d}$. Consider the set of pairs $( L, s )$ where $L \subseteq K$ and $s: L \rightarrow \calM_{/f}$ sits in a commutative diagram
$$ \xymatrix{ L \ar[r] \ar@{^{(}->}[d] & \calM_{/f} \ar[d] \\
K \ar[r] & \calM_{/d}. }$$
We order these pairs by setting $(L,s) \leq (L',s')$ if $L \subseteq L'$ and $s = s' | L$. By Zorn's lemma, there exists a pair $(L,s)$ which is maximal with respect to this ordering. To prove $(1)$, it suffices to show that $L = K$. Otherwise, we may obtain a larger simplicial subset $L' = L \coprod_{ \bd \Delta^n } \Delta^n \subseteq K$ by adjoining a single nondegenerate simplex. By maximality, there is no solution to the associated lifting problem
$$ \xymatrix{ \bd \Delta^n \ar[r] \ar@{^{(}->}[d] & \calM_{/f} \ar[d] \\
\Delta^n \ar[r] \ar@{-->}[ur] & \calM_{/d},}$$
nor to the associated lifting problem
$$ \xymatrix{ \Lambda^{n+2}_{n+2} \ar@{^{(}->}[d] \ar[r]^{s} & \calM \ar[d]^{q} \\
\Delta^{n+2} \ar[r] \ar@{-->}[ur] & \Delta^1, }$$
which contradicts the fact that $s$ carries $\Delta^{ \{n+1, n+2\} }$ to the $q$-Cartesian
morphism $f$ in $\calM$.

Now suppose that $\overline{f}$ is a lift of $f$. To prove that $\overline{f}$ is $r$-Cartesian, it suffices to show that for every $m \geq 2$ and every diagram
$$ \xymatrix{ \Lambda^{m}_m \ar[r]^{g_0} \ar@{^{(}->}[d] & \calM_{p/} \ar[d] \\
\Delta^m \ar[r] \ar@{-->}[ur]^{g} \ar[r] & \calM }$$
such that $g_0 | \Delta^{ \{m-1, m\} } = \widetilde{f}$, there exists a dotted arrow
$g$ as indicated, rendering the diagram commutative. We can identify the diagram with a map
$$t_0: ( K \star \Lambda^m_m) \coprod_{ \Lambda^m_m} \Delta^m \rightarrow \calM.$$
Consider the set of all pairs $(L, t)$, where $L \subseteq K$ and 
$$t: (K \star \Lambda^m_m) \coprod_{ L \star \Lambda^m_m } (L \star \Delta^m) \rightarrow \calM$$
is an extension of $t_0$. As above, we order the set of such pairs by declaring $(L,t) \leq (L',t')$ if $L \subseteq L'$
and $t = t' | L$. Zorn's lemma guarantees the existence of a maximal pair $(L,t)$. If $L = K$, we are done; otherwise let $L'$ be obtained from $L$ by adjoining a single nondegenerate $n$-simplex of $K$. By maximality, the map $t$ does not extend to $L'$; consequently the associated mapping problem
$$ \xymatrix{ (\Delta^n \star \Lambda^m_m) \coprod_{ \bd \Delta^n \star \Lambda^m_m }
(\bd \Delta^n \star \Delta^m) \ar[r] \ar@{^{(}->}[d] & \calM \ar[d] \\
\Delta^n \star \Delta^m \ar[r] \ar@{-->}[ur] & \Delta^1 }$$
has no solution. But this contradicts the assumption that $r(\widetilde{f})=f$ is a $q$-Cartesian edge of $\calM$.
\end{proof}

\begin{lemma}\label{lotusk}
Let $q: \calM \rightarrow \Delta^1$ be a correspondence between the $\infty$-categories
$\calC = q^{-1} \{0\}$ and $\calD = q^{-1} \{1 \}$. Let $f: c \rightarrow d$ be a morphism
in $\calM$ between objects $c \in \calC$, $d \in \calD$. Let $p: K \rightarrow \calC$
be a diagram, and consider an associated map
$$ k: \calM_{p/} \times_{\calM} \{c\} \rightarrow \calM_{p/} \times_{\calM} \{d\}$$
$($ the map $k$ is well-defined up to homotopy, according to Lemma \ref{functy} $)$.
If $f$ is $q$-Cartesian, then $k$ is a homotopy equivalence.
\end{lemma}

\begin{proof}
Let $X = (\calM_{p/})^{\Delta^1} \times_{\calM^{\Delta^1} } \{f\}$, and consider the diagram
$$ \xymatrix{ & X \ar[dr]^{u} \ar[dl]^{v} & \\
\calM_{p/} \times_{\calM} \{c\} & & \calM_{p/} \times_{\calM} \{d\}.}$$
The map $u$ is a homotopy equivalence, and $k$ is defined as the composition
of $v$ with a homotopy inverse to $u$. Consequently, it will suffice to show that
$v$ is a trivial fibration. To prove this, we must show that $v$ has the right lifting
property with respect to $\bd \Delta^n \subseteq \Delta^n$, which is equivalent to solving a lifting problem
$$ \xymatrix{ (\bd \Delta^n \times \Delta^1) \coprod_{ \bd \Delta^n \times \{1\} } (\Delta^n \times \{1\}) \ar[r] \ar@{^{(}->}[d] & \calM_{p/} \ar[d]^{r} \\
\Delta^n \times \Delta^1 \ar[r] \ar@{-->}[ur] & \calM}.$$
If $n=0$, we invoke $(1)$ of Lemma \ref{storka}. If $n > 0$, then Proposition \ref{goouse} implies
that it suffices to show that the upper horizontal map carries $\{n \} \times \Delta^1$
to an $r$-Cartesian edge of $\calM_{p/}$, which also follows from assertion $(2)$ of Lemma \ref{storka}.
\end{proof}

\begin{lemma}\label{lotuss}
Let $q: \calM \rightarrow \Delta^1$ be a Cartesian fibration, and let $\calC = q^{-1} \{0\}$.
The inclusion $\calC \subseteq \calM$ preserves all colimits which exist in $\calC$.
\end{lemma}

\begin{proof}
Let $\overline{p}: K^{\triangleright} \rightarrow \calC$ be a colimit of $p = \overline{p}|K$.
We wish to show that $\calM_{\overline{p}/} \rightarrow \calM_{p/}$ is a trivial fibration. Since
we have a diagram
$$ \calM_{\overline{p}/} \rightarrow \calM_{p/} \rightarrow \calM$$
of left fibrations, it will suffice to show that the induced map
$$ \calM_{\overline{p}/} \times_{\calM} \{d\} \rightarrow \calM_{p/} \times_{\calM} \{d\}$$ is a homotopy
equivalence of Kan complexes, for each object $d$ of $\calM$. If $d$ belongs to
$\calC$, this is obvious. In general, we may choose a $q$-Cartesian morphism
$f: c \rightarrow d$ in $\calM$. Composition with $f$ gives a commutative
diagram
$$ \xymatrix{ [\calM_{\overline{p}/} \times_{\calM} \{c\}] \ar[r] \ar[d] & [\calM_{p/} \times_{\calM} \{c\}] \ar[d] \\
[\calM_{\overline{p}/} \times_{\calM} \{d\}] \ar[r] & [\calM_{p/} \times_{\calM} \{d\}] }$$
in the homotopy category $\calH$ of spaces. The upper horizontal map is a homotopy equivalence since $\overline{p}$ is a colimit of $p$ in $\calC$. The vertical maps are homotopy equivalences by Lemma \ref{lotusk}. Consequently, the bottom horizontal map is also a homotopy equivalence, as desired.
\end{proof}

\begin{proposition}\label{adjointcol}\index{gen}{adjoint functor!and (co)limits}
Let $f: \calC \rightarrow \calD$ be a functor between $\infty$-categories which has a right
adjoint $g: \calD \rightarrow \calC$. Then $f$ preserves all colimits which exist in $\calC$, and $g$ preserves all limits which exist in $\calD$.
\end{proposition}

\begin{proof}
We will show that $f$ preserves colimits; the analogous statement for $g$ follows by a dual argument. Let $\overline{p}: K^{\triangleright} \rightarrow \calC$ be a colimit for
$p = \overline{p}|K$. We must show that $f \circ \overline{p}$ is a colimit of
$f \circ p$.

Let $q: \calM \rightarrow \Delta^1$ be an adjunction between $\calC = \calM_{\{0\}}$ and $\calD = \calM_{ \{1\} }$ which is associated to $f$ and $g$.  We wish to show that
$$\phi_1: \calD_{f \overline{p}/} \rightarrow \calD_{f p/}$$
is a trivial fibration. Since $\phi_1$ is a left fibration, it suffices to show that
$\phi_1$ is a categorical equivalence.

Since $\calM$ is associated to $f$, there is a map
$F: \calC \times \Delta^1 \rightarrow \calM$ with $F| \calC \times \{0\} = \id_{\calC}$, 
$F| \calC \times \{1\} = f$, and $F| \{c\} \times \Delta^1$ a $q$-coCartesian morphism of $\calM$
for every object $c \in \calC$. 
Let $\overline{P} = F \circ (\overline{p} \times \id_{\Delta^1})$
be the induced map $K^{\triangleright} \times \Delta^1 \rightarrow \calM$, and let
$P = \overline{P}|K \times \Delta^1$.

Consider the diagram
$$\xymatrix{ \calM_{\overline{p}/} \ar[r]^{\phi'} & \calM_{p/} \\
\calM_{\overline{P}/} \ar[r] \ar[u]^{\overline{v}} \ar[d]^{\overline{u}} & \calM_{P/} \ar[u]^{v} \ar[d]^{u}\\
\calM_{f \overline{p}/} \ar[r]^{\phi} & \calM_{f p/}. }$$
We note that every object in this diagram is an $\infty$-category with a map to $\Delta^1$; moreover, the map $\phi_{1}$ is obtained from $\phi$ by passage to the fiber over $\{1\} \subseteq \Delta^1$. Consequently, to prove that $\phi_1$ is a categorical equivalence, it suffices to verify three things:

\begin{itemize}
\item[$(1)$] The bottom vertical maps $u$ and $\overline{u}$ are trivial fibrations.
This follows from the fact that $K \times \{1\} \subseteq K \times \Delta^1$ and
$K^{\triangleright} \times \{ 1\} \subseteq K^{\triangleright} \times \Delta^1$ are right anodyne inclusions (Proposition \ref{sharpen2}).

\item[$(2)$] The upper vertical maps $v$ and $\overline{v}$ are trivial fibrations when restricted to
$\calD \subseteq \calM$. This follows from Lemma \ref{lotusss}, since $F$ carries
each $\{c\} \times \Delta^1$ to a $q$-coCartesian edge of $\calM$.

\item[$(3)$] The map $\phi'$ is a trivial fibration, since $\overline{p}$ is a colimit of
$p$ in $\calM$ according to Lemma \ref{lotuss}.

\end{itemize}

\end{proof}

\begin{remark}
Under appropriate set-theoretic hypotheses, one can prove a converse to Proposition \ref{adjointcol}. See Corollary \ref{adjointfunctor}. 
\end{remark}

\subsection{Examples of Adjoint Functors}\label{afunc4}

In this section, we describe a few simple criteria for establishing the existence of adjoint functors.

\begin{lemma}\label{adjfunclemma}
Let $q: \calM \rightarrow \Delta^1$ be a coCartesian fibration associated to a functor
$f: \calC \rightarrow \calD$, where $\calC = q^{-1} \{0\}$ and $\calD = q^{-1} \{1\}$. Let $D$ be an object of $\calD$. The following are equivalent:
\begin{itemize}
\item[$(1)$] There exists a $q$-Cartesian morphism $g: C \rightarrow D$ in $\calM$, where
$C \in \calC$.
\item[$(2)$] The right fibration $\calC \times_{\calD} \calD^{/D} \rightarrow \calC$ is
representable.
\end{itemize}
\end{lemma}

\begin{proof}
Let $F: \calC \times \Delta^1 \rightarrow \calM$ be
a $p$-coCartesian natural transformation from $\id_{\calC}$ to $f$. Define a simplicial
set $X$ so that for every simplicial set $K$, $\Hom_{\sSet}(K,X)$ parametrizes
maps $H: K \times \Delta^2 \rightarrow \calM$ such that
$h=H | K \times \{0\}$ factors through $\calC$, $H | K \times \Delta^{ \{0,1\} } = F \circ 
(h|(K \times \{0\}) \times \id_{\Delta^1})$, and $H | K \times \{2\}$ is the constant map at the vertex $D$. We have restriction maps
$$ \xymatrix{ & X \ar[dr] \ar[dl] & \\
\calC \times_{\calM} \calM^{/D} & & \calC \times_{\calD} \calD^{/D}.}$$
which are both trivial fibrations (the map on the right because $\calM$ is an $\infty$-category, the map on the left because $F$ is a $p$-coCartesian transformation). Consequently, $(2)$ is equivalent to 
the assertion that the $\infty$-category $\calC \times_{\calM} \calM^{/D}$ has a final object.
It now suffices to observe that a final object of $\calC \times_{\calM} \calM^{/D}$ is {\em precisely}
a $q$-Cartesian morphism $C \rightarrow D$, where $C \in \calC$.
\end{proof}

\begin{proposition}\label{adjfuncbaby}
Let $F: \calC \rightarrow \calD$ be a functor between $\infty$-categories.
The following are equivalent:
\begin{itemize}
\item[$(1)$] The functor $F$ has a right adjoint.
\item[$(2)$] For every pullback diagram
$$ \xymatrix{ \overline{\calC} \ar[r] \ar[d]^{p'} & \overline{\calD} \ar[d]^{p} \\
\calC \ar[r]^{F} & \calD, }$$
if $p$ is a representable right fibration, then $p'$ is also a representable right fibration.
\end{itemize}
\end{proposition}

\begin{proof}
Let $\calM$ be a correspondence from $\calC$ to $\calD$ associated to $F$, and apply
Lemma \ref{adjfunclemma} to each object of $D$.
\end{proof}

\begin{proposition}\label{quuquu}
Let $p: \calC \rightarrow \calD$ be a Cartesian fibration of $\infty$-categories, and let
$s: \calD \rightarrow \calC$ be a section of $p$ such that $s(D)$ is an initial object
of $\calC_{D} = \calC \times_{\calD} \{D\}$ for every object $D \in \calD$. Then
$s$ is a left adjoint of $p$.
\end{proposition}

\begin{proof}
Let $\calC^{0} \subseteq \calC$ denote the full subcategory of $\calC$ spanned by those
objects $C \in \calC$ such that $C$ is initial in the $\infty$-category $\calC_{p(C)}$. According to Proposition \ref{topaz}, the restriction $p| \calC^0$ is a trivial fibration from
$\calC^0$ to $\calD$. Consequently, it will suffice to show that the inclusion
$\calC^{0} \subseteq \calC$ is left adjoint to the composition $s \circ p: \calC \rightarrow \calC^{0}$.
Let $\calM \subseteq \calC \times \Delta^1$ be the full subcategory spanned by the vertices
$(C, \{i\})$ where $i = 1$ or $C \in \calC^0$. Let $q: \calM \rightarrow \Delta^1$ be the projection. It is clear that $q$ is a coCartesian fibration which is associated to the inclusion $\calC^0 \subseteq \calC$. To complete the proof, it will suffice to show that $q$ is also a Cartesian fibration
which is associated to $s \circ p$.

We first show that $q$ is a Cartesian fibration. It will suffice to show that for any object
$C \in \calC$, there is a $q$-Cartesian edge $(C',0) \rightarrow (C,1)$ in $\calM$. By assumption,
$C' = (s \circ p)(C)$ is an initial object of $\calC_{p(C)}$. Consequently, there exists
a morphism $f: C' \rightarrow C$ in $\calC_{p(C)}$; we will show that $f \times \id_{\Delta^1}$
is a $q$-Cartesian edge of $\calM$. To prove this, it suffices to show that for every $n \geq 2$ and every diagram
$$ \xymatrix{ \Lambda^n_n \ar@{^{(}->}[d] \ar[r]^{G_0} & \calM \ar[d] \\ 
\Delta^n \ar[r] \ar@{-->}[ur]^{G} & \Delta^1 }$$
such that $F_0 | \Delta^{ \{n-1, n\} } = f \times \id_{\Delta^1}$, there exists a dotted arrow $F: \Delta^n \rightarrow \calM$
as indicated, rendering the diagram commutative. We may identify $G_0$ with a map
$g_0: \Lambda^n_n \rightarrow \calC$. The composite map $p \circ g_0$ carries
$\Delta^{ \{n-1,n\} }$ to a degenerate edge of $\calD$, and therefore admits an extension
$\overline{g}: \Delta^n \rightarrow \calD$. Consider the diagram
$$ \xymatrix{ \Lambda^n_n \ar[r]^{g_0} \ar@{^{(}->}[d] & \calC \ar[d]^{p} \\
\Delta^n \ar[r]^{ \overline{g} } \ar@{-->}[ur]^{g} & \calD. }$$
Since $g_0$ carries the initial vertex $v$ of $\Delta^n$ to an initial object of the fiber
$\calC_{\overline{g}(v)}$, Lemma \ref{sabretooth} implies the existence of the indicated
map $g$ rendering the diagram commutative. This gives rise to a map $G: \Delta^n \rightarrow \calM$
with the desired properties, and completes the proof that $q$ is a Cartesian fibration.

We now wish to show that $s \circ p$ is associated to $q$. To prove this, it suffices to prove the existence of a map $H: \calC \times \Delta^1 \rightarrow \calC$ such that
$p \circ H = p \circ \pi_{\calC}$, $H | \calC \times \{1\} = \id_{\calC}$, and $H| C \times \{0\} = s \circ p$. We construct the map $H$ inductively, working cell-by-cell on $\calC$. Suppose that we have a nondegenerate simplex $\sigma: \Delta^n \rightarrow \calC$ and that $H$ has already been defined on $\sk^{n-1} \calC \times \Delta^1$. To define $H \circ (\sigma \times \id_{\Delta^1})$, we must solve a lifting problem that may be depicted as follows:
$$ \xymatrix{ (\bd \Delta^n \times \Delta^1) \coprod_{ \bd \Delta^n \times \bd \Delta^1} (\Delta^n \times \bd \Delta^1) \ar[rrrr]^{h_0} \ar@{^{(}->}[d] & & & & \calC \ar[d]^{p} \\
\Delta^n \times \Delta^1 \ar[rrrr] \ar@{-->}[urrrr]^{h} & & & & \calD.}$$
We now consider the filtration
$$ X(n+1) \subseteq X(n) \subseteq \ldots \subseteq X(0) = \Delta^n \times \Delta^1$$
defined in the proof of Proposition \ref{usejoyal}. Let $Y(i) = X(i) \coprod_{ \bd \Delta^n \times \{0\} } (\Delta^n \times \{1\})$. For $i > 0$, the inclusion $Y(i+1) \subseteq Y(i)$ is a pushout of the inclusion $X(i+1) \subseteq X(i)$, and therefore inner anodyne. Consequently, we may use the assumption that $p$ is an inner fibration to extend $h_0$ to a map defined on $Y(1)$. The
inclusion $Y(1) \subseteq \Delta^n \times \Delta^1$ is a pushout of $\bd \Delta^{n+1} \subseteq \Delta^{n+1}$; we then obtain the desired extension $h$ by applying Lemma \ref{sabretooth}.
\end{proof}

\begin{proposition}\label{simpex}
Let $\calM$ be a fibrant simplicial category equipped with a functor $p: \calM \rightarrow \Delta^1$ $($here we identify $\Delta^1$ with the two-object category whose nerve is $\Delta^1${}$)$, so that we may view $\calM$ as a correspondence between the simplicial categories $\calC = p^{-1} \{0\}$ and
$\calD = p^{-1} \{1\}$. The following are equivalent
\begin{itemize}
\item[$(1)$] The map $p$ is a Cartesian fibration.
\item[$(2)$] For every object $D \in \calD$, there exists a morphism $f: C \rightarrow D$ in $\calM$
which induces homotopy equivalences
$$ \bHom_{\calC}(C',C) \rightarrow \bHom_{\calM}(C',D)$$
for every $C' \in \calC$.
\end{itemize}
\end{proposition}

\begin{proof}
This follows immediately from Proposition \ref{trainedg}, since nonempty morphism spaces
in $\Delta^1$ are contractible.
\end{proof}

\begin{corollary}\label{sturk}
Let $\calC$ and $\calD$ be fibrant simplicial categories, and let 
$$\Adjoint{F}{\calC}{\calD}{G}$$
be a pair of adjoint functors
$F: \calC \rightarrow \calD$ $($ in the sense of enriched category theory, so that there is a natural isomorphism of simplicial sets $\bHom_{\calC}(F(C),D) \simeq \bHom_{\calD}(C,G(D))$ for $C \in \calC$, $D \in \calD$ $)$. Then the induced functors
$$ \Adjoint{f}{\Nerve(\calC)}{\Nerve(\calD)}{g}$$
are also adjoint to one another.
\end{corollary}

\begin{proof}
Let $\calM$ be the correspondence associated to the adjunction $(F,G)$. In other words, $\calM$ is a simplicial category containing $\calC$ and $\calD$ as full (simplicial) subcategories, with
$$\bHom_{\calM}(C,D) = \bHom_{\calC}(C,G(D)) = \bHom_{\calD}(F(C),D)$$
$$ \bHom_{\calM}(D,C) = \emptyset$$
for every pair of objects $C \in \calC$, $D \in \calD$. Let $M = \sNerve(\calM)$. Then
$M$ is a correspondence between $\sNerve(\calC)$ and $\sNerve(\calD)$. By Proposition
\ref{simpex}, it is an adjunction. It is easy to see that this adjunction is associated to both $f$ and $g$.
\end{proof}

The following variant on the situation of Corollary \ref{sturk} arises very often in practice:

\begin{proposition}\label{quiladj}\index{gen}{adjoint functor!Quillen}\index{gen}{Quillen adjunction}
Let $\bfA$ and $\bfA'$ be simplicial model categories, and let
$$ \Adjoint{F}{\bfA}{\bfA'}{G}$$
be a $($simplicial$)$ Quillen adjunction. Let $\calM$ be the simplicial category defined as in the proof of Corollary $\ref{sturk}$, and let $\calM^{\degree}$ be the full subcategory of $\calM$ consisting of those objects which are fibrant-cofibrant $($either as objects of $\bfA$ or as objects of $\bfA'${}$)$. Then $\sNerve(\calM^{\degree})$ determines an adjunction between $\sNerve( \bfA^{\degree})$ and $\sNerve( {\bfA'}^{\degree})$.
\end{proposition}

\begin{proof}
We need to show that $\sNerve(\calM^{\degree})\rightarrow \Delta^1$ is both a Cartesian fibration and a coCartesian fibration. We will argue the first point; the second follows from a dual argument.
According to Proposition \ref{princex}, it suffices to show that for every fibrant-cofibrant object
$D$ of $\bfA'$, there is a fibrant-cofibrant object $C$ of $\bfA$ and a morphism $f: C \rightarrow D$
in $\calM^{\degree}$ which induces weak homotopy equivalences
$$ \bHom_{ \bfA}(C',C) \rightarrow \bHom_{\calM}(C',D)$$ for every
fibrant-cofibrant object $C' \in \bfA$. We define $C$ to be a cofibrant replacement for $GD$: in other words, we choose a cofibrant object $C$ with a trivial fibration $C \rightarrow G(D)$ in the model category $\bfA$. Then $\bHom_{\bfA}(C',C) \rightarrow \bHom_{\calM}(C',D) = \bHom_{\bfA}(C',G(D))$ is a trivial fibration of simplicial sets, whenever $C'$ is a cofibrant object of $\bfA$.
\end{proof}

\begin{remark}
Suppose that $F: \bfA \rightarrow \bfA'$ and $G: \bfA' \rightarrow \bfA$ are as in Proposition \ref{quiladj}. We may associate to the adjunction $\sNerve(M^{\degree})$ a pair of adjoint functors $f: \sNerve( \bfA^{\degree}) \rightarrow \sNerve( {\bfA'}^{\degree})$ and $g: \sNerve( { \bfA'}^{\degree}) \rightarrow \sNerve( \bfA^{\degree} )$. In this situation, $f$
is often called a (nonabelian) {\it left derived functor} of $F$, and $g$ a (nonabelian) {\it right derived functor} of $G$. On the level of homotopy categories, $f$ and $g$ reduce to the usual derived functors associated to the Quillen adjunction (see \S \ref{quilladj}).\index{gen}{functor!derived}\index{gen}{derived functor}\index{gen}{left derived functor}\index{gen}{right derived functor}
\end{remark}

\subsection{Adjoint Functors and Overcategories}\label{afunc4half}

Our goal in this section is to prove the following result:

\begin{proposition}\label{curpse}
Suppose given an adjunction of $\infty$-categories
$$ \Adjoint{F}{\calC}{\calD}{G}.$$
Assume that the $\infty$-category $\calC$ admits pullbacks, and let $C$ be an object of $\calC$.
Then:
\begin{itemize}
\item[$(1)$] The induced functor $f: \calC^{/C} \rightarrow \calD^{/FC}$ admits a right adjoint $g$.
\item[$(2)$] The functor $g$ is equivalent to the composition
$$ \calD^{/FC} \stackrel{g'}{\rightarrow} \calC^{/GFC} \stackrel{g''}{\rightarrow} \calC^{/C},$$ 
where $g'$ is induced by $G$ and $g''$ is induced by pullback along the unit map
$C \rightarrow GFC$.
\end{itemize}
\end{proposition}

Proposition \ref{curpse} is an immediate consequence of the following more general result, which we will prove at the end of this section:

\begin{lemma}\label{starfi}
Suppose given an adjunction between $\infty$-categories
$$ \Adjoint{F}{\calC}{\calD}{G}.$$
Let $K$ be a simplicial set, and suppose given a pair of diagrams
$p_0: K \rightarrow \calC$, $p_1: K \rightarrow \calD$, and a natural
transformation $h: F \circ p_0 \rightarrow p_1$. 
Assume that $\calC$ admits pullbacks and $K$-indexed limits. Then:
\begin{itemize}
\item[$(1)$] 
Let
$f: \calC^{/p_0} \rightarrow \calD^{/p_1}$ denote the composition
$$ \calC^{/p_0} \rightarrow \calD^{/Fp_0} \stackrel{\circ 
\alpha}{\rightarrow}
\calD^{/p_1}.$$
Then $f$ admits a right adjoint $g$.
\item[$(2)$] The functor $g$ is equivalent the composition
$$ \calD^{/p_1} \stackrel{g'}{\rightarrow}
\calC^{/Gp_1} \stackrel{g''}{\rightarrow} \calC^{/p_0}.$$
Here $g''$ is induced by pullback along the natural transformation
$p_0 \rightarrow Gp_1$ adjoint to $h$ $($see below$)$.
\end{itemize}
\end{lemma}

We begin by recalling a bit of notation which will be needed in the proof. Suppose that $q: X \rightarrow S$ is an inner fibration of simplicial sets and
$p_S: K \rightarrow X$ is an arbitrary map, then we have defined a map of simplicial sets
$X^{/p_S} \rightarrow S$, which is characterized by the following universal property:
for every simplicial set $Y$ equipped with a map to $S$, there is a pullback diagram
$$ \xymatrix{ \Hom_{S}( Y, X^{/p_S}) \ar[r] \ar[d] & \Hom_{S}( Y \diamond_{S} S, X) \ar[d] \\
\{ p \} \ar[r] & \Hom_{S}(S, X). }$$ We refer the reader to \S \ref{consweet} for a more detailed discussion.

%The main result of this section is the following:

%\begin{theorem}
%Let $\calC$ be a symmetric monoidal $\infty$-category. Assume that $\calC$ is stable
%and that the tensor product functor $\otimes: \calC \times \calC \rightarrow \calC$ is exact in each variable. Let $G: \CAlg^{\aug}(\calC) \rightarrow \calC$ be the functor which associates
%to each augmented commutative algebra $u: A \rightarrow 1_{\calC}$ the augmentation
%ideal $\ker(u) \in \calC$. Then $G$ induces an equivalence of $\infty$-categories
%$$ \Stab( \CAlg^{\aug}(\calC) ) \rightarrow \Stab(\calC) \simeq \calC.$$ 
%\end{theorem}

\begin{lemma}\label{curtwell}
Let $q: \calM \rightarrow \Delta^1$ be a coCartesian fibration of simplicial sets, classifying a functor
$F$ from $\calC = \calM \times_{ \Delta^1 } \{0\}$ to $\calD = \calM \times_{ \Delta^1 } \{1\}$.
Let $K$ be a simplicial set, and suppose given a commutative diagram
$$ \xymatrix{ K \times \Delta^1 \ar[dr] \ar[rr]^{g_{\Delta^1} } & & 
\ar[dl] \calM \\
& \Delta^1, & }$$
which restricts to give a pair of diagrams
$$ \calC \stackrel{g_0}{\leftarrow} K \stackrel{g_1}{\rightarrow} \calD.$$
 
Then:
\begin{itemize}
\item[$(1)$] The projection $q': \calM^{/g_{\Delta^1}} \rightarrow 
\Delta^1$ is a coCartesian fibration of simplicial sets, classifying a 
functor $F': \calC^{/g_0} \rightarrow \calD^{/g_1}$. Moreover, an
edge of $\calM^{/ g_{\Delta^1} }$ is $q'$-coCartesian if and only if its image in 
$\calM$ is $q$-coCartesian.

\item[$(2)$] Suppose that for every vertex $k$ in $K$, the map 
$g_{\Delta}$ carries $\{ k\} \times \Delta^1$
to a $q$-coCartesian morphism in $\calM$, so that $g_{\Delta^1}$ 
determines an equivalence $g_1 \simeq F \circ g_0$. Then $F'$ is homotopic 
to the composite functor
$$ \calC^{/g_0} \rightarrow \calD^{/ Fg_0} \simeq \calD^{/ g_1}.$$

\item[$(3)$] Suppose that $\calM = \calD \times \Delta^1$, and that $q$ is 
the projection onto the second factor, so that we can identify $F$ with 
the identity functor from $\calD$ to itself. Let $\overline{g}: K \times 
\Delta^1 \rightarrow \calD$ 
denote the composition $g_{\Delta^1}$ with the projection map
$\calM \rightarrow \calD$, so that we can regard
$\overline{g}$ as a morphism from $g_0$ to $g_1$ in $\Fun(K, \calD)$. Then 
the functor $F': \calD^{/g_0} \rightarrow \calD^{/g_1}$ is induced by 
composition with $\overline{g}$.

%\item[$(4)$] In the general case, choose a $2$-simplex $\sigma$ $$ 
%\xymatrix{ & FC \ar[dr]^{g''} & \\ C \ar[ur]^{g'} \ar[rr]^{g} & & D }$$ 
%in $\calM$, where $g'$ is $q$-coCartesian. Then $F'$ is homotopic to the 
%composition $$ \calC^{/C} \stackrel{F'_0}{\rightarrow} \calD^{/FC} 
%\stackrel{F'_1}{\rightarrow} \calD^{/D}.$$ Here $F'_0$ is the map 
%determined by $F$, and $F'_1$ is given by composition with $g''$.
\end{itemize}
\end{lemma}

\begin{proof}
Assertion $(1)$ follows immediately from Proposition \ref{colimfam}.

We now prove $(2)$. Since $F$ is associated to the correspondence $\calM$, 
there exists a natural transformation $\alpha: \calC \times \Delta^1 
\rightarrow \calM$ from $\id_{\calC}$ to $F$, such that for each $C' \in 
\calC$, the induced map $\alpha_{C}: C' \rightarrow FC'$ is 
$q$-coCartesian. Without loss of generality, we may assume that 
$g_{\Delta^1}$ is given by the composition
$$ K \times \Delta^1 \stackrel{ g_0}{\rightarrow} \calC \times \Delta^1
\stackrel{\alpha}{\rightarrow} \calM.$$
In this case, $\alpha$ induces a map $\alpha': \calC^{/g_0} 
\times \Delta^1 \rightarrow \calM^{/ g_{\Delta^1}}$, which we may identify 
with a natural transformation from $\id_{\calC^{/g_0}}$ to the functor 
$\calC^{/g_0} \rightarrow \calD^{/Fg_0}$ determined by $F$. To show that 
this functor coincides with $F'$, it will suffice to show that $\alpha'$ 
carries each object of $\calC^{/g_0}$ to a $q'$-coCartesian morphism in 
$\calM^{ /g_{\Delta^1}}$. This follows immediately from the description of 
the $q'$-coCartesian edges given in assertion $(1)$.

We next prove $(3)$. Consider the diagram
$$ \calD^{ / g_0} \stackrel{p}{\leftarrow} \calD^{/\overline{g}} 
\stackrel{p'}{\rightarrow} \calD^{/g_1}.$$
By definition, ``composition with $\overline{g}$'' refers to a functor 
from $\calD_{/g_0}$ 
to $\calD^{/g_1}$ obtained by composing $p'$ with a section to the trivial 
fibration $p$.
To prove that this functor is homotopic to $F'$, it will suffice to show that
$F' \circ p$ is homotopic to $p'$. For this, we must produce a map
$\beta: \calD^{/\overline{g}} \times \Delta^1 \rightarrow \calM^{ / 
g_{\Delta^1}}$ from
$p$ to $p'$, such that $\beta$ carries each object of 
$\calD^{/\overline{g}}$ to a 
$q'$-coCartesian edge of $\calM^{/g_{\Delta^1}}$. We observe that
$\calD^{/\overline{g}} \times \Delta^1$ can be identified with 
$\calM^{/h_{\Delta^1}}$, where
$h: \Delta^1 \times \Delta^1 \rightarrow \calM \simeq \calD \times \Delta^1$ is the product
of $\overline{g}$ with the identity map. We now take $\beta$ to be the 
restriction map
$\calM^{ / h_{\Delta^1} } \rightarrow \calM^{ / g_{\Delta^1} }$ induced by the diagonal
inclusion $\Delta^1 \subseteq \Delta^1 \times \Delta^1$. Using $(1)$, we readily deduce that
$\beta$ has the desired properties.

%We now prove $(4)$. Let $X$ denote the fiber product $\calM \times_{ \Delta^1 } \Delta^2$, where
%the map from $\Delta^2$ to $\Delta^1$ is given by $q \circ \sigma$. We may identify
%$\sigma$ with a section to the coCartesian fibration $X \rightarrow \Delta^2$. Applying
%Proposition \ref{colimfam}, we deduce that the induced map
%$X^{/ \sigma_{\Delta^2}} \rightarrow \Delta^2$ is a coCartesian fibration. This coCartesian fibration is associated to a diagram of $\infty$-categories
%$$ \xymatrix{ & \calD^{/ FC} \ar[dr]^{F'_1} & \\
%\calC^{/C} \ar[rr]^{F'} \ar[ur]^{F'_0}& & \calD^{/D}, }$$
%so that we obtain a factorization $F' \simeq F'_1 \circ F'_0$. The identifications
%of $F'_0$ and $F'_1$ follow from assertions $(2)$ and $(3)$, respectively.
\end{proof}

We will also need the following counterpart to Proposition \ref{colimfam}:

\begin{lemma}\label{limfam}
Suppose given a commutative diagram of simplicial sets
$$ \xymatrix{ K \times S \ar[r]^{p_S} \ar[dr] & X \ar[r]^{q} \ar[d] & Y \ar[dl] \\
& S & }$$
where the left diagonal arrow is projection onto the second factor, and
$q$ is an Cartesian fibration. Assume further that:
\begin{itemize}
\item[$(\ast)$] For every vertex $k \in K$, the map $p_S$ carries each
edge of $\{k \} \times S$ to a $q$-Cartsian edge in $X$.
\end{itemize}
Let $p'_S = q \circ p_S$. Then the map $q': X^{/p_S} \rightarrow Y^{/p'_S}$ is a Cartesian fibration. Moreover, an edge of $X^{/p_S}$ is $q'$-Cartesian if and only if its image in $X$ is $q$-Cartesian.
\end{lemma}

\begin{proof} To give the proof, it is convenient to use the language of 
marked simplicial sets (see \S \ref{twuf}). Let $X^{\natural}$ denote 
the marked simplicial set whose underlying simplicial set is $X$, where we 
consider an edge of $X^{\natural}$ to be marked if it is $q$-Cartesian. 
Let $\overline{X}^{\natural}$ denote the marked simplicial set whose 
underlying simplicial set is $X^{/p_S}$, where we consider an edge to be 
marked if and only if its image in $X$ is marked. According to Proposition 
\ref{dubudu}, it will suffice to show that the map 
$\overline{X}^{\natural} \rightarrow ( Y^{/p'_S} )^{\sharp}$ has the right 
lifting property with respect to every marked anodyne map $i: A 
\rightarrow B$. Let $\overline{A}$ and $\overline{B}$ denote the 
simplicial sets underlying $A$ and $B$, respectively. Suppose given a 
diagram of marked simplicial sets $$ \xymatrix{ A \ar@{^{(}->}[d]^{i} \ar[r] & 
\overline{X}^{\natural} \ar[d] \\ B \ar[r] \ar@{-->}[ur] & ( 
Y^{/p'_S})^{\sharp}. }$$ We wish to show that there exists a dotted arrow, 
rendering the diagram commutative. We begin by choosing a solution to the 
associated lifting problem $$ \xymatrix{ A \ar@{^{(}->}[d] \ar[r] & X^{\natural} 
\ar[d] \\ B \ar[r] \ar@{-->}[ur] & Y^{\sharp}, }$$ which is possible in 
view of our assumption that $q$ is a Cartesian fibration. To extend this 
to a solution to the original problem, it suffices to solve another 
lifting problem $$ \xymatrix{ (\overline{A} \times K \times \Delta^1) 
\coprod_{ (\overline{A} \times K \times \bd \Delta^1) } ( \overline{B} 
\times K \times \bd \Delta^1) \ar[r]^-{f} \ar@{^{(}->}[d]^{j} & X 
\ar[d]^{q} \\ 
\overline{B} \times K \times \Delta^1 \ar[r] \ar@{-->}[ur] & Y. }$$ By 
construction, the map $f$ induces a map of marked simplicial sets from $B 
\times K^{\flat} \times \{0\}$ to $X^{\natural}$. Using assumption 
$(\ast)$, we conclude that $f$ also induces a map of marked simplicial 
sets from $B \times K^{\flat} \times \{1\}$ to $X^{\natural}$. Using 
Proposition \ref{dubudu} again (and our assumption that $q$ is a 
Cartesian fibration), we are reduced to proving that the map $j$ induces a 
marked anodyne map $$ (A \times (K \times \Delta^1)^{\flat}) \coprod_{ A 
\times (K \times \bd \Delta^1)^{\flat} } ( B \times (K \times \bd 
\Delta^1)^{\flat}) \rightarrow B \times (K \times \Delta^1)^{\flat}.$$ 
Since $i$ is marked anodyne by assumption, this follows immediately from 
Proposition \ref{markanodprod}. \end{proof}

\begin{lemma}\label{spitfure} Let $q: \calM \rightarrow \Delta^1$ be a 
Cartesian fibration of simplicial sets, associated to a functor $G$ from 
$\calD = \calM \times_{ \Delta^1} \{1\}$ to $\calC = \calM 
\times_{\Delta^1} \{0\}$. Suppose given a simplicial set $K$ 
and a commutative diagram
$$ \xymatrix{ K \times \Delta^1 \ar[rr]^{g_{\Delta^1} } \ar[dr] & & \calM 
\ar[dl] \\
& \Delta^1, & }$$
so that $g_{\Delta^1}$ restricts to a pair of functors
$$ \calC \stackrel{g_0}{\leftarrow} K \stackrel{g_1}{\rightarrow} \calD.$$
Suppose furthermore that, for every vertex $k$ of $K$, the corresponding 
morphism $g_0(k) \rightarrow g_1(k)$ is $q$-Cartesian. Then:
\begin{itemize}
\item[$(1)$] The induced map $q': \calM^{ / f_{\Delta^1} } \rightarrow \Delta^1$ is a Cartesian fibration.
Moreover, an edge of $\calM^{/ f_{\Delta^1}}$ is $q'$-Cartesian if and only if its image in
$\calM$ is $q$-Cartesian.
\item[$(2)$] The associated functor $\calD^{/g_1} \rightarrow 
\calC^{/g_0}$ is 
homotopic to the composition of the functor $G': \calD^{/g_1} \rightarrow 
\calC^{/Gg_1}$ induced by $G$ and the equivalence
$\calC^{/Gg_1} \simeq \calC^{/g_0}$ determined by the map $g_{\Delta^1}$.
\end{itemize}
\end{lemma}

\begin{proof} Assertion $(1)$ follows immediately from Lemma \ref{limfam}. 
We will prove $(2)$. Since the functor $G$ is associated to $q$, there 
exists a map $\alpha: \calD \times \Delta^1 \rightarrow \calM$ which is a 
natural transformation from $G$ to $\id_{\calD}$, such that for every 
object $D \in \calD$ the induced map $\alpha_{D}: \{D \} \times \Delta^1 
\rightarrow \calM$ is a $q$-Cartesian edge of $\calM$. Without loss of 
generality, we may assume that $g$
coincides with the composition
$$ K \times \Delta^1 \stackrel{g_1}{\rightarrow} \calD \times \Delta^1
\stackrel{\alpha}{\rightarrow} \calM.$$
In this case, $\alpha$ 
induces a map $\alpha': \calD^{/g_1} \times \Delta^1 \rightarrow 
\calM^{/f_{\Delta^1}}$, which is a natural transformation from $G'$ to the 
identity. Using $(1)$, we deduce that $\alpha'$ carries each object of 
$\calD^{/g_1}$ to a $q'$-Cartesian edge of $\calM^{/ f_{\Delta^1}}$. It 
follows that $\alpha'$ exhibits $G'$ as the functor associated to the 
Cartesian fibration $q'$, as desired. \end{proof}

\begin{proof}[Proof of Lemma \ref{starfi}]
Let $q: \calM \rightarrow \Delta^1$ be a correspondence from $\calC = 
\calM \times_{\Delta^1} \{0\}$ to
$\calD = \calM \times_{\Delta^1} \{1\}$, which is associated to the pair of adjoint functors $F$ and $G$.
The natural transformation $h$ determines a map
map $\alpha$ determines a map
$\alpha: K \times \Delta^1 \rightarrow \calM$, which is a natural 
transformation from $p_0$ to $p_1$. Using the fact that $q$ is both a 
Cartesian and a coCartesian fibration, we can form a commutative
square $\sigma$:
$$ \xymatrix{ & Gp_1 \ar[dr]^{\phi} & \\
p_0 \ar[ur] \ar[rr]^{\alpha} \ar[dr]^{\psi} & & p_1 \\
& Fp_0 \ar[ur] & }$$
in the $\infty$-category $\Fun(K, \calM)$, where the morphism $\phi$ is
$q$-Cartesian and the morphism $\psi$ is $q$-coCartesian.

Let $\calN = \calM \times \Delta^1$. We can identify $\sigma$ with a map
$\sigma_{\Delta^1 \times \Delta^1}: K \times \Delta^1 \times \Delta^1 
\rightarrow \calM \times \Delta^1$. Let $\calN' = \calN^{/ 
\sigma_{\Delta^1 \times \Delta^1}}$. Proposition \ref{colimfam} 
implies that the projection $\calN' \rightarrow \Delta^1 \times \Delta^1$ 
is a coCartesian fibration, associated to some diagram of 
$\infty$-categories
$$ \xymatrix{ & \calC^{/Gp_1} \ar[dr]^{f'} & \\
\calC^{/p_0} \ar[rr] \ar[dr] \ar[ur]^{f''} & & \calD^{/p_1} \\
& \calD^{/Fp_0} \ar[ur]. & }$$
Lemma \ref{curtwell} allows us to identify the functors in the lower 
triangle, so we see that the horizontal composition is homotopic to the 
functor $f$. To complete the proof of $(1)$, it will suffice to show that
the functors $f'$ and $f''$ admit right adjoints. To prove $(2)$, it 
suffices to show that those right adjoints are given by $g'$ and $g''$, 
respectively. The adjointness of $f'$ and $g'$ follows from Lemma 
\ref{spitfure}.

It follows from Lemma \ref{curtwell} that the functor $f'': \calC^{/p_0}
\rightarrow \calC^{/Gp_1}$ is given by composition with the transformation
$h': p_0 \rightarrow Gp_1$ which is adjoint to $h$.
The pullback functor 
$g''$ 
is right adjoint to $f''$ by definition; the only nontrivial point is to 
establish the existence of $g''$. Here we must use our hypotheses on the 
$\infty$-category $\calC$. Let $\overline{p}_0: K^{\triangleleft} 
\rightarrow \calC$ be a limit of $p_0$, let $\overline{Gp}_1: 
K^{\triangleleft} \rightarrow \calC$ be a limit of $Gp_1$. Let us identify 
$h'$ with a map $K \times \Delta^1 \rightarrow \calC$, and choose an 
extension $\overline{h}': K^{\triangleleft} \times \Delta^1 \rightarrow 
\calC$ which is a natural transformation from $\overline{p}_0$ to
$\overline{Gp}_1$. Let $C \in \calC$ denote the image under
$\overline{p}_0$ of the cone point of $K^{\triangleleft}$, let
$C' \in \calC$ denote the image under $\overline{Gp}_1$ of the cone point
of $K^{\triangleleft}$, and let $j: C \rightarrow C'$ be the morphism
induced by $\overline{h}'$. We have a commutative diagram of 
$\infty$-categories:
$$ \xymatrix{ \calC^{/ p_0} & \calC^{ / h' } \ar[r]^{f''_1} \ar[l]^{f''_0} 
& \calC^{/ 
Gp_1} \\
\calC^{/ \overline{p}_0 } \ar[u] \ar[d] & \calC^{/ \overline{h}'} \ar[l] 
\ar[r] \ar[u] \ar[d] & \calC^{/ \overline{Gp}_1} \ar[u] \ar[d] \\
\calC^{/C} & \calC^{/ j} \ar[r] \ar[l] & \calC^{ /C '}. } $$
In this diagram, the left horizontal arrows are trivial Kan fibrations, as 
are all of the vertical arrows. The functor $f''$ is obtained by composing
$f''_0$ with a section to the trivial Kan fibration $f''_1$. Utilizing
the vertical equivalences, we can identify $f''$ with the functor
$\calC^{/C} \rightarrow \calC^{/C'}$ given by composition with $j$.
But this functor admits a right adjoint, in view of our assumption that
$\calC$ admits pullbacks.
\end{proof}

\subsection{Uniqueness of Adjoint Functors}\label{afunc5}

We have seen that if $f: \calC \rightarrow \calD$ is a functor which admits a right adjoint
$g: \calD \rightarrow \calC$, then $g$ is uniquely determined up to homotopy. Our next result is a slight refinement of this assertion. 

\begin{definition}
Let $\calC$ and $\calD$ be $\infty$-categories. We let $\LFun(\calC, \calD)
\subseteq \Fun(\calC, \calD)$ denote the full subcategory of $\Fun(\calC,\calD)$ spanned by those functors $F: \calC \rightarrow \calD$ which are left adjoints. Similarly, we define
$\RFun(\calC,\calD)$ to be the full subcategory of $\Fun(\calC, \calD)$ spanned by those functors which are right adjoints.\index{not}{FunL@$\LFun(\calC, \calD)$}\index{not}{FunR@$\RFun(\calC, \calD)$}
\end{definition}

\begin{proposition}\label{switcheroo}
Let $\calC$ and $\calD$ be $\infty$-categories. Then the $\infty$-categories
$\LFun(\calC, \calD)$ and $\RFun(\calD, \calC)^{op}$ are $($canonically$)$ equivalent to one another.
\end{proposition}

\begin{proof}
Enlarging the universe if necessary, we may assume without loss of generality that $\calC$ and $\calD$ are small. Let $j: \calD \rightarrow \calP(\calD)$ be the Yoneda embedding.
Composition with $j$ induces a fully faithful embedding
$$ i: \Fun(\calC,\calD) {\rightarrow} \Fun(\calC, \calP(\calD)) \simeq
\Fun( \calC \times \calD^{op}, \SSet).$$
The essential image of $i$ consists of those functors $G: \calC \times \calD^{op} \rightarrow \SSet$
with the property that, for each $C \in \calC$, the induced functor
$G_{C}: \calD^{op} \rightarrow \SSet$ is representable by an object $D \in \calD$. 
The functor $i$ induces a fully faithful embedding
$$ i_0: \RFun(\calC, \calD) \rightarrow \Fun(\calC \times \calD^{op}, \SSet)$$
whose essential image consists of those functors $G$ which belong to the essential image
of $i$, and furthermore satisfy the additional condition that for each $D \in \calD$, the induced
functor $G_{D}: \calC \rightarrow \SSet$ is corepresentable by an object $C \in \calC$ (this
follows from Proposition \ref{adjfuncbaby}). Let $\calE \subseteq \Fun(\calC \times \calD^{op}, \SSet)$
be the full subcategory spanned by those functors which satisfy these two conditions, so that
the Yoneda embedding induces an equivalence
$$ \RFun(\calC,\calD) \rightarrow \calE.$$
We note that the above conditions are self-dual, so that the same reasoning gives an equivalence of $\infty$-categories
$$ \RFun(\calD^{op}, \calC^{op}) \rightarrow \calE.$$ 
We now conclude by observing that there is a natural equivalence of $\infty$-categories
$\RFun(\calD^{op}, \calC^{op}) \simeq \LFun( \calD, \calC)^{op}$.
\end{proof}

We will later need a slight refinement of Proposition \ref{switcheroo}, which exhibits some functoriality in $\calC$. We begin with a few preliminary remarks concerning the construction of presheaf $\infty$-categories.

Let $f: \calC \rightarrow \calC'$ be a functor between small $\infty$-categories. Then composition with $f$ induces a restriction functor $G: \calP(\calC') \rightarrow \calP(\calC)$. However, there
is another slightly less evident functoriality of the construction $\calC \mapsto \calP(\calC)$. Namely, according to Theorem \ref{charpresheaf}, there is a colimit-preserving functor $\calP(f): \calP(\calC) \rightarrow \calP(\calC')$, uniquely determined up to equivalence, such that the diagram
$$ \xymatrix{ \calC \ar[d] \ar[r]^{f} & \calC' \ar[d] \\
\calP(\calC) \ar[r]^{ \calP(f) } & \calP(\calC') } $$\index{not}{Pcalf@$\calP(f)$}
commutes up to homotopy (here the vertical arrows are given by the Yoneda embeddings).

The functor $\calP(f)$ has an alternative characterization in the language of adjoint functors:

\begin{proposition}\label{adjobs}
Let $f: \calC \rightarrow \calC'$ be a functor between small $\infty$-categories, let
$G: \calP(\calC') \rightarrow \calP(\calC)$ be the functor given by composition with $f$.
Then $G$ is right adjoint to $\calP(f)$.
\end{proposition}

\begin{proof}
We first prove that $G$ admits a left adjoint. Let $e: \calP(\calC) \rightarrow \Fun( \calP(\calC)^{op}, \widehat{\SSet})$ denote the Yoneda embedding.
According to Proposition \ref{adjfuncbaby}, it will suffice to show that for each
$M \in \calP(\calC)$, the composite functor $e(M) \circ G$ is corepresentable.
Let $\calD$ denote the full subcategory of $\calP(\calC)$ spanned by those objects $M$ such that $G \circ e_M$ is corepresentable. Since $\calP(\calC)$ admits small colimits, Proposition \ref{yonedaprop}
implies that the collection of corepresentable functors on $\calP(\calC)$ is stable under small colimits. According to Propositions \ref{yonedaprop} and \ref{limiteval}, the functor
$M \mapsto e(M) \circ G$ preserves small colimits. It follows that $\calD$ is stable under small colimits in $\calP(\calC)$. Since $\calP(\calC)$ is generated under small colimits by the Yoneda embedding $j_{\calC'}: \calC \rightarrow \calP(\calC)$ (Corollary \ref{gencolcot}), it will suffice to show that $j_{\calC}(C) \in \calD$ for each $C \in \calC$. According to 
Lemma \ref{repco}, $e(j_{\calC}(C))$ is equivalent to the functor $\calP(\calC) \rightarrow \widehat{\SSet}$ given by evaluation at $C$. Then $e( j_{\calC}(C) ) \circ G$ is equivalent to the functor given by evaluation at $f(C) \in \calC'$, which is corepresentable (Lemma \ref{repco} again).
We conclude that $G$ has a left adjoint $F$.

To complete the proof, we must show that $F$ is equivalent to $\calP(f)$. To prove this, it will suffice to show that $F$ preserves small colimits and that the diagram
$$ \xymatrix{ \calC \ar[r]^{f} \ar[d] & \calC' \ar[d] \\
\calP(\calC) \ar[r]^{F} & \calP(\calC') }$$
commutes up to homotopy. The first point is obvious: since $F$ is a left adjoint, it preserves all colimits which exist in $\calP(\calC)$ (Proposition \ref{adjointcol}). For the second, choose a counit map $v: F \circ G \rightarrow \id_{\calP(\calC')}$. By construction, the functor $f$ induces
a natural transformation $u: j_{\calC} \rightarrow G \circ j_{\calC'} \circ f$. To complete
the proof, it will suffice to show that the composition
$$ \theta: F \circ j_{\calC} \stackrel{u}{\rightarrow} F \circ G \circ j_{\calC'} \circ f
\stackrel{v}{\rightarrow} j_{\calC'} \circ f$$
is an equivalence of functors from $\calC$ to $\calP(\calC')$. Fix objects
$C \in \calC$, $M \in \calP(\calC')$. We have a commutative diagram
$$ \xymatrix{ \bHom_{\calP(\calC')}( j_{\calC'}(f(C)), M) \ar[r] \ar@{=}[d]  
& \bHom_{\calP(\calC)}( G(j_{\calC'}(f(C))), G(M)) \ar[r] \ar[d] & 
\bHom_{\calP(\calC)}( j_{\calC}(C), G(M) ) \ar[d] \\
\bHom_{\calP(\calC')}( j_{\calC'}(f(C)), M) \ar[r] & 
\bHom_{\calP(\calC')}( F(G( j_{\calC'}( f(C) ))), M) \ar[r] & \bHom_{\calP(\calC')}(F(j_{\calC}(C)), M) }$$
in the homotopy category $\calH$ of spaces, where the vertical arrows are isomorphisms. 
Consequently, to prove that the lower horizontal composition is an isomorphism, it suffices to prove that the upper horizontal composition is an isomorphism. Using Lemma \ref{repco}, we reduce to the assertion that $M( f(C)) \rightarrow (G(M))(C)$ is an isomorphism in $\calH$, which follows immediately from the definition of $G$.
\end{proof}

\begin{remark}\label{switcheroo2}
Suppose given a functor $f: \calD \rightarrow \calD'$
which admits a right adjoint $g$. 
Let $\calE \subseteq \Fun(\calC \times \calD^{op}, \SSet)$
and $\calE' \subseteq \Fun(\calC \times (\calD')^{op}, \SSet)$
be defined as in the proof of Proposition \ref{switcheroo}, and consider the diagram
$$ \xymatrix{ \RFun(\calC, \calD) \ar[d]^{\circ g} \ar[r] & \calE \ar[d] & \LFun(\calD, \calC)^{op} \ar[d]^{\circ f} \ar[l] \\
\RFun(\calC, \calD') \ar[r] & \calE' & \LFun(\calD', \calC)^{op}. \ar[l] }$$
Here the middle vertical map is given by composition with $\id_{\calC} \times f$. The square on the right is manifestly commutative, but the square on the left commutes only up to homotopy. To verify the second point, we observe that the square in question is given by applying the functor
$\bHom(\calC, \bigdot)$ to the diagram
$$ \xymatrix{ \calD \ar[r] \ar[d]^{g} & \calP(\calD) \ar[d]^{G} \\
\calD' \ar[r] & \calP(\calD') }$$
where $G$ is given by composition with $f$ and the horizontal arrows are given by the Yoneda embedding. Let $\calP^0(\calD) \subseteq \calP(\calD)$ and $\calP^0(\calD')$ denote the essential images of the Yoneda embeddings. Proposition \ref{adjfuncbaby} asserts that $G$ carries $\calP^{0}(\calD')$ into $\calP^{0}(\calD)$, so that it will suffice to verify that the
diagram
$$ \xymatrix{ \calD \ar[r] \ar[d]^{g} & \calP^0(\calD) \ar[d]^{G^0} \\
\calD' \ar[r] & \calP^0(\calD') }$$
is homotopy commutative. In view of Proposition \ref{compadjoint}, it will suffice to show that
$G^0$ admits a left adjoint $F^0$ and that the diagram
$$ \xymatrix{ \calD \ar[r]  & \calP^0(\calD) \\
\calD' \ar[r] \ar[u] & \calP^0(\calD') \ar[u]^{F_0} }$$
is homotopy commutative. 
According to Proposition \ref{adjobs}, the functor $G$ has a left adjoint $\calP(f)$ which fits into a commutative diagram $$ \xymatrix{ \calD \ar[r] & \calP(\calD) \\
\calD' \ar[r] \ar[u]^{f} & \calP(\calD') \ar[u]^{\calP(f)}. }$$ 
In particular, $\calP(f)$ carries $\calP^0(\calD)$ into $\calP^{0}(\calD')$ and therefore
restricts to give a left adjoint $F^0: \calP^{0}(\calD) \rightarrow \calP^{0}(\calD')$ which verifies the desired commutativity.
\end{remark}

We conclude this section by establishing the following consequence of
Proposition \ref{adjobs}: 

\begin{corollary}\label{coughspaz}
Let $\calC$ be a small $\infty$-category, $\calD$ a locally small $\infty$-category which admits small colimits. Let $F: \calP(\calC) \rightarrow \calD$ be a colimit-preserving functor, let
$f: \calC \rightarrow \calD$ denotes the composition of $F$ with the Yoneda embedding of
$\calC$, and let $G: \calD \rightarrow \calP(\calC)$ be the functor given by the composition
$$ \calD \stackrel{j'}{\rightarrow} \Fun( \calD^{op}, \SSet) \stackrel{\circ f}{\rightarrow} \calP(\calC).$$
Then $G$ is a right adjoint to $F$. Moreover, the map 
$$f = F \circ j \rightarrow (F \circ (G \circ F)) \circ j = (F \circ G) \circ f$$
exhibits $F \circ G$ as a left Kan extension of $f$ along itself.
\end{corollary}

The proof requires a few preliminaries:

\begin{lemma}\label{stimp}
Suppose given a pair of adjoint functors
$$ \Adjoint{f}{\calC}{\calD}{g}$$
between $\infty$-categories. Let $T: \calC \rightarrow \calX$ be any functor. Then
$T \circ g: \calD \rightarrow \calX$ is a left Kan extension of $T$ along $f$.
\end{lemma}

\begin{proof}
Let $p: \calM \rightarrow \Delta^1$ be a correspondence associated to the pair of adjoint functors
$f$ and $g$. Choose a $p$-Cartesian homotopy $h$ from $r$ to $\id_{M}$, where
$r$ is a functor from $\calM$ to $\calC$; thus $r | \calD$ is homotopic to $g$. 
It will therefore suffice to show that the composition
$$ \overline{T}: \calM \stackrel{r}{\rightarrow} \calC \stackrel{T}{\rightarrow} \calX$$
is a left Kan extension of $\overline{T} | \calC \simeq T$. For this, we must show that
for each $D \in \calD$, the functor $\overline{T}$ exhibits $\overline{T}(D)$
as a colimit of the diagram
$$ (\calC \times_{\calM} \calM_{/D}) \rightarrow \calM \stackrel{\overline{T}}{\rightarrow} \calX.$$
We observe that $\calC \times_{\calM} \calM_{/D}$ has a final object, given by any
$p$-Cartesian morphism $e: C \rightarrow D$. It therefore suffices to show that
$\overline{T}(e)$ is an equivalence in $\calX$, which follows immediately from the construction of $\overline{T}$.
\end{proof}

\begin{lemma}\label{kanspaz}
Let $f: \calC \rightarrow \calC'$ be a functor between small $\infty$-categories and
$\calX$ an $\infty$-category which admits small colimits.
Let $H: \calP(\calC) \rightarrow \calX$ be a functor which preserves small colimits, and 
$h: \calC \rightarrow \calX$ the composition of $F$ with the Yoneda embedding $j_{\calC}: \calC \rightarrow \calP(\calC)$. Then the composition 
$$\calC' \stackrel{j_{\calC'}}{\rightarrow} \calP(\calC') \stackrel{ \circ f}{\rightarrow} \calP(\calC)
\stackrel{H}{\rightarrow} \calX$$
is a left Kan extension of $h$ along $f$.
\end{lemma}

\begin{proof}
Let $G: \calP(\calC') \rightarrow \calP(\calC)$ be the functor given by composition with $f$.
In view of Proposition \ref{acekan}, it will suffice to show that $H \circ G$ is a 
left Kan extension of $h$ along $j_{\calC} \circ f$.
 
Theorem \ref{charpresheaf} implies the existence of a functor
$F: \calP(\calC) \rightarrow \calP(\calC')$ which preserves small colimits, such that
$F \circ j_{\calC} \simeq j_{\calC'} \circ f$. Moreover, Lemma \ref{longwait1} ensures that
$F$ is a left Kan extension of $f$ along the fully faithful Yoneda embedding $j_{\calC}$. 
Using Proposition \ref{acekan} again, we are reduced to proving that
$H \circ G$ is a left Kan extension of $H$ along $F$. This follows immediately from Proposition \ref{adjobs} and Lemma \ref{stimp}.
\end{proof}

\begin{proof}[Proof of Corollary \ref{coughspaz}]
The first claim follows from Proposition \ref{adjobs}. To prove the second, we may assume without loss of generality that $\calD$ is minimal, so that $\calD$ is union of small full subcategories
$\{ \calD_{\alpha} \}$. It will suffice to show that, for each index $\alpha$ such that $f$ factors
through $\calD_{\alpha}$, the restricted transformation $f \rightarrow ((F \circ G)| \calD_{\alpha}) \circ f$ exhibits $(F \circ G)| \calD_{\alpha}$ as a left Kan extension of
$f$ along the induced map $\calC \rightarrow \calD_{\alpha}$, which follows from Lemma \ref{kanspaz}.
\end{proof}

\subsection{Localization Functors}\label{locfunc}

Suppose we are given a $\infty$-category $\calC$ and a collection $S$ of morphisms
of $\calC$ which we would like to invert. In other words, we wish to find an $\infty$-category $S^{-1} \calC$ equipped with a functor $\eta: \calC \rightarrow S^{-1} \calC$ which carries each morphism in $S$ to an equivalence, and is in some sense universal with respect to these properties.
One can give a general construction of $S^{-1} \calC$ using the formalism of \S \ref{bicat1}. Without loss of generality, we may suppose that $S$ contains all the identity morphisms in $\calC$. Consequently, the pair $(\calC, S)$ may be regarded as a marked simplicial set, and we can choose a marked anodyne map $( \calC, S) \rightarrow ( S^{-1} \calC, S')$, where $S^{-1} \calC$ is an $\infty$-category and $S'$ is the collection of all equivalences in $S^{-1} \calC$. However, this construction is generally very difficult to analyze, and the properties of $S^{-1} \calC$ are very difficult to control. For example, it might be the case that $\calC$ is locally small and $S^{-1} \calC$ is not.

Under suitable hypotheses on $S$ (see \S \ref{invloc}), there is a drastically simpler approach: we can find the desired $\infty$-category $S^{-1} \calC$ {\em inside} of $\calC$, as the full subcategory of {\em $S$-local} objects of $\calC$.\index{not}{SinvC@$S^{-1} \calC$}

\begin{example}\label{excom}
Let $\calC$ be the (ordinary) category of abelian groups, $p$ a
prime number, and let $S$ denote the collection of morphisms $f$
whose kernel and cokernel consist entirely of $p$-power torsion elements. A
morphism $f$ lies in $S$ if and only if it induces an isomorphism
after inverting the prime number $p$. In this case, we may
identify $S^{-1} \calC$ with the full subcategory of $\calC$ consisting of those abelian groups which
are {\em uniquely $p$-divisible}. The functor $\calC \rightarrow S^{-1} \calC$ is given
by $$ M \mapsto M \otimes_{\Z} \Z[ \frac{1}{p}].$$
\end{example}

In Example \ref{excom}, the functor $\calC \rightarrow
S^{-1} \calC$ is actually left adjoint to an inclusion functor. We will take this as our starting point.

\begin{definition}\label{swagga}
A functor $f: \calC \rightarrow \calD$ between $\infty$-categories is a {\it localization}
if $f$ has a fully faithful right adjoint.\index{gen}{localization}\index{gen}{functor!localization}
\end{definition}

\begin{warning}
Let $f: \calC \rightarrow \calD$ be a localization functor, and let $S$ denote the collection of all morphisms $\alpha$ in $\calC$ such that $f(\alpha)$ is an equivalence. Then, for any
$\infty$-category $\calE$, composition with $f$ induces a fully faithful functor
$$ \Fun( \calD, \calE) \stackrel{ \circ f}{\rightarrow} \Fun(\calC, \calE)$$
whose essential image consists of those functors $p: \calC \rightarrow \calE$ which
carry each $\alpha \in S$ to an equivalence in $\calE$ (Proposition \ref{unlap}). We may describe the situation more
informally by saying that $\calD$ is obtained from $\calC$ by inverting the morphisms of $S$.

Some authors use the term ``localization'' in a more general sense, to describe any
functor $f: \calC \rightarrow \calD$ in which $\calD$ is obtained by inverting some collection $S$ of morphisms in $\calC$. Such a morphism $f$ need not be a localization in the sense of Definition \ref{swagga}; however, it is in many cases (see Proposition \ref{local}).
\end{warning}

If $f: \calC \rightarrow \calD$ is a localization of $\infty$-categories, then we will also say that $\calD$ is a {\it localization} of $\calC$. In this case, a right adjoint $g: \calD \rightarrow \calC$
of $f$ gives an equivalence between $\calD$ and a full subcategory of $\calC$ (the essential image of $g$). We let $L: \calC \rightarrow \calC$ denote the composition $g \circ f$. We will abuse terminology by referring to $L$ as a {\it localization functor} if it arises in this way.

The following result will allow us to recognize localization functors:

\begin{proposition}\label{recloc}
Let $\calC$ be a $\infty$-category, and let $L: \calC \rightarrow \calC$ be a functor with essential
image $L\calC \subseteq \calC$. The following conditions are equivalent:
\begin{itemize}
\item[$(1)$] There exists a functor $f: \calC \rightarrow \calD$ with a fully faithful 
right adjoint $g: \calD \rightarrow \calC$, and an equivalence between $g \circ f$ and $L$.

\item[$(2)$] When regarded as a functor from $\calC$ to $L\calC$, $L$ is a left adjoint of the
inclusion $L\calC \subseteq \calC$.

\item[$(3)$] There exists a natural transformation $\alpha: \calC \times \Delta^1 \rightarrow \calC$ from $\id_{\calC}$ to $L$ such that, for every object $C$ of $\calC$, the morphisms
$L(\alpha(C)), \alpha(LC): LC \rightarrow LLC$ of $\calC$ are equivalences.
% and homotopic to one another.
\end{itemize}
\end{proposition}

\begin{proof}
It is obvious that $(2)$ implies $(1)$ (take $\calD= L\calC$, $f = L$, and $g$ to be the inclusion).
The converse follows from the observation that, since $g$ is fully faithful, we are free to replace $\calD$ by the essential image of $g$ (which is equal to the essential image of $L$).

We next prove that $(2)$ implies $(3)$. Let $\alpha: \id_{\calC} \rightarrow L$ be a unit
for the adjunction. Then, for each pair of objects $C \in \calC$, $D \in L \calC$, composition with $\alpha(C)$ induces a homotopy equivalence
$$ \bHom_{\calC}( LC, D) \rightarrow \bHom_{\calC}( C, D),$$
and in particular a bijection $\Hom_{\h{\calC}}(LC, D) \rightarrow \Hom_{\h{\calC}}(C,D)$.
If $C$ belongs to $L \calC$, then Yoneda's lemma implies that
$\alpha(C)$ is an isomorphism in $\h{\calC}$. This proves that $\alpha(LC)$ is an equivalence for every $C \in \calC$. Since $\alpha$ is a natural transformation, we have obtain a diagram
$$ \xymatrix{ C \ar[r]^{\alpha(C)} \ar[d]^{\alpha(C)} & LC \ar[d]^{L \alpha(C)}  \\
LC \ar[r]^{\alpha(LC)} & LLC.}$$
Since composition with $\alpha(C)$ gives an injective map
from $\Hom_{\h{\calC}}(LC,LLC)$ to $\Hom_{\h{\calC}}(C,LLC)$, we conclude that
$\alpha(LC)$ is homotopic to $L \alpha(C)$; in particular, $\alpha(LC)$ is also an equivalence. This proves $(3)$.

Now suppose that $(3)$ is satisfied; we will prove that $\alpha$ is the unit of an adjunction between $\calC$ and $L \calC$. In other words, we must show that for each $C \in \calC$ and
$D \in \calC$, composition with $\alpha(C)$ induces a homotopy equivalence 
$$ \phi: \bHom_{\calC}( LC, LD) \rightarrow \bHom_{\calC}( C, LD).$$
By Yoneda's lemma, it will suffice to show that for every Kan complex $K$, the induced map
$$ \Hom_{\calH}( K, \bHom_{\calC}(LC, LD)) \rightarrow \Hom_{\calH}(K, \bHom_{\calC}(C,LD))$$
is a bijection of sets, where $\calH$ denotes the homotopy category of spaces. Replacing
$\calC$ by $\Fun(K, \calC)$, we are reduced to proving the following:

\begin{itemize}
\item[$(\ast)$] Suppose that $\alpha: \id_{\calC} \rightarrow L$ satisfies $(3)$. Then, for every
pair of objects $C, D \in \calC$, composition with $\alpha(C)$ induces a bijection of sets
$$ \phi: \Hom_{ \h{\calC}}( LC, LD) \rightarrow \Hom_{ \h{\calC}}(C, LD).$$
\end{itemize}

We first show that $\phi$ is surjective. Let $f$ be a morphism from $C$ to $LD$. We then have a commutative diagram
$$ \xymatrix{ C \ar[r]^{f} \ar[d]^{\alpha(C)} & LD \ar[d]^{\alpha(LD)} \\
LC \ar[r]^{Lf} & LLD, }$$
so that $f$ is homotopic to the composition $(\alpha(LD)^{-1} \circ Lf) \circ \alpha(C)$; this proves that the homotopy class of $f$ lies in the image of $\phi$.

We now show that $\phi$ is injective. Let $g: LC \rightarrow LD$ be an arbitrary morphism. 
We have a commutative diagram
$$ \xymatrix{ LC \ar[r]^{g} \ar[d]^{\alpha(LC)} & LD \ar[d]^{ \alpha(LD) } \\
LLC \ar[r]^{ Lg } & LLD, }$$
so that $g$ is homotopic to the composition
\begin{eqnarray*}
\alpha(LD)^{-1} \circ Lg \circ \alpha(LC) & 
\simeq & \alpha(LD)^{-1} \circ Lg \circ L \alpha(C) \circ (L\alpha(C))^{-1} \circ \alpha(LC) \\
& \simeq & \alpha(LD)^{-1} \circ L( g \circ \alpha(C) ) \circ (L \alpha(C))^{-1} \circ \alpha(LC)
\end{eqnarray*}
In particular, $g$ is determined by $g \circ \alpha(C)$ up to homotopy.
%We will prove that $\psi$ is a an inverse to $\phi$ in the homotopy category 
%$\calH$. We first compute the composition
%$\psi \circ \phi$. Since $L$ is a functor, $\psi' \circ \phi$ is equivalent
%to the composition
%$$ \bHom_{\calC}(LC,LD) \stackrel{L}{\rightarrow} \bHom_{\calC}(LLC,LLD)
%\stackrel{ L \alpha(C) }{\rightarrow} \bHom_{\calC}(LC,LLD).$$ 
%Since $L \alpha(C)$ is equivalent to $\alpha(LC)$ and $\alpha$ is a natural
%transformation of functors, this composition is equivalent to the map
%given by composition with $\alpha(LD)$, which is the inverse of
%$\psi''$. Thus $\psi \circ \phi$ is the identity on $\bHom_{\calC}(LC,LD)$.

%We now compute the composition $\phi \circ \psi = \phi \circ \psi'' \circ \psi'$.
%Clearly $\phi \circ \psi''$ can be rewritten as the composition
%$$ \bHom_{\calC}( LC, LLD) \stackrel{\theta}{\rightarrow}
%\bHom_{\calC}(C, LLD) \stackrel{\theta'}{\rightarrow} \bHom_{\calC}(C,LD)$$
%where $\theta$ is given by composition with $\alpha(C)$ and
%$\theta'$ is the inverse of the map given by composition with $\alpha(LD)$. 
%Since $\alpha$ is a natural transformation of functors, $\theta \circ \psi'$
%is given by composition with $\alpha(LD)$, and is therefore inverse to $\theta'$ as desired.
\end{proof}

\begin{remark}\label{localcolim}\index{gen}{localization!and colimits}
Let $L: \calC \rightarrow \calD$ be a localization functor and $K$ a simplicial set. Suppose
that every diagram $p: K \rightarrow \calC$ admits a colimit in $\calC$. Then the $\infty$-category $\calD$ has the same property. Moreover, we can give an explicit prescription for computing colimits in $\calD$. Let $q: K \rightarrow \calD$ be a diagram, and let
$p: K \rightarrow \calC$ be the composition of $q$ with a right adjoint to $L$. 
Choose a colimit $\overline{p}: K^{\triangleright} \rightarrow \calC$. Since $L$ is a left
adjoint, $L \circ \overline{p}$ is a colimit diagram in $\calD$, and $L \circ p$ is equivalent
to the diagram $q$.
\end{remark}

We conclude this section by introducing a few ideas which will allow us to recognize localization functors, when they exist.

\begin{definition}\label{locaobj}\index{gen}{localization!of an object}
Let $\calC$ be an $\infty$-category and $\calC^{0} \subseteq \calC$ a full subcategory. We will say that a morphism $f: C \rightarrow D$ in $\calC$ {\it exhibits $D$ as a $\calC^0$-localization of $C$} if $D \in \calC^{0}$, and composition with $f$ induces an isomorphism
$$ \bHom_{\calC^{0}}(D,E) \rightarrow \bHom_{\calC}(C,E)$$ in the homotopy category $\calH$, for each object $E \in \calC^{0}$.
\end{definition}

\begin{remark}\label{initrem}
In the situation of Definition \ref{locaobj}, a morphism $f: C \rightarrow D$ exhibits
$D$ as a localization of $C$ if and only if $f$ is an initial object of the $\infty$-category
$\calC^{0}_{C/} = \calC_{C/} \times_{\calC} \calC^{0}$. In particular, $f$ is uniquely determined up to equivalence.
\end{remark}

\begin{proposition}\label{testreflect}
Let $\calC$ be an $\infty$-category and $\calC^0 \subseteq \calC$ a full subcategory.
The following conditions are equivalent:
\begin{itemize}
\item[$(1)$] For every object $C \in \calC$, there exists a localization
$f: C \rightarrow D$ relative to $\calC^{0}$.
\item[$(2)$] The inclusion $\calC^{0} \subseteq \calC$ admits a left adjoint.
\end{itemize}
\end{proposition}

\begin{proof}
Let $\calD$ be the full subcategory of $\calC \times \Delta^1$ spanned by objects
of the form $(C,i)$, where $C \in \calC^0$ if $i=1$. Then the projection $p: \calD \rightarrow \Delta^1$ is a correspondence from $\calC$ to $\calC^0$ which is associated to the inclusion functor $i: \calC^0 \subseteq \calC$. It follows that $i$ admits a left adjoint if and only if
$p$ is a coCartesian fibration. It now suffices to observe that if $C$ is an object of $\calC$, then we may identify $p$-coCartesian edges $f: (C,0) \rightarrow (D,1)$ of $\calD$ with
localizations $C \rightarrow D$ relative to $\calC^{0}$.
\end{proof}

\begin{remark}\label{reflective}\index{gen}{subcategory!reflective}\index{gen}{reflective subcategory}
By analogy with classical category theory, we will say that a full subcategory $\calC^0$ of an $\infty$-category $\calC$ is a {\it reflective subcategory} if the hypotheses of Proposition \ref{testreflect} are satisfied by the inclusion $\calC^0 \subseteq \calC$. 
\end{remark}

\begin{example}\label{swang}
Let $\calC$ be an $\infty$-category which has a final object, and let $\calC^{0}$ be the full subcategory of $\calC$ spanned by the final objects. Then the inclusion $\calC^{0} \subseteq \calC$ admits a left adjoint.
\end{example}

\begin{corollary}\label{sweng}
Let $p: \calC \rightarrow \calD$ be a coCartesian fibrations between $\infty$-categories, let
$\calD^{0} \subseteq \calD$ be a full subcategory, and let $\calC^{0} = \calC \times_{\calD} \calD^{0}$.
If the inclusion $\calD^{0} \subseteq \calD$ admits a left adjoint, then the inclusion
$\calC^{0} \subseteq \calC$ admits a left adjoint. 
\end{corollary}

\begin{proof}
In view of Proposition \ref{testreflect}, it will suffice to show that for every object $C \in \calC$, there
a morphism $f: C \rightarrow C_0$ which is a localization of $C$ relative to $\calC^{0}$. Let
$D = p(C)$, let $\overline{f}: D \rightarrow D_0$ be a localization of $D$ relative to $\calD_0$, and let
$f: C \rightarrow C_0$ be a $p$-coCartesian morphism in $\calC$ lifting $\overline{f}$. We claim that
$f$ has the desired property. Choose any object $C' \in \calC^{0}$, and let $D' = p(C') \in \calD^{0}$.
We obtain a diagram of spaces
$$ \xymatrix{ \bHom_{\calC}( C_0, C') \ar[r]^{\phi} \ar[d] &  \bHom_{\calC}(C, C') \ar[d] \\
\bHom_{\calD}( D_0, D') \ar[r]^{\psi} & \bHom_{\calD}( D, D') }$$
which commutes up to preferred homotopy. By assumption, the map $\psi$ is a homotopy equivalence. Since $f$ is $p$-coCartesian, the map $\phi$ induces a homotopy equivalence after passing to the homotopy fibers over any pair of points $\eta \in \bHom_{\calD}(D_0, D')$, $\psi(\eta)
\in \bHom_{\calD}(D,D')$. Using the long exact sequence of homotopy groups associated to the vertical fibrations, we conclude that $\phi$ is a homotopy equivalence, as desired.
\end{proof}

\begin{proposition}\label{unlap}
Let $\calC$ be an $\infty$-category, let $L: \calC \rightarrow \calC$ be a localization functor with essential image $L\calC$. Let $S$ denote the collection of all morphisms $f$ in $\calC$ such that $Lf$ is an equivalence. Then, for every $\infty$-category $\calD$, composition with
$f$ induces a fully faithful functor
$$ \psi: \Fun( L \calC , \calD) \rightarrow \Fun( \calC, \calD).$$
Moreover, the essential image of $\psi$ consists of those functors $F: \calC \rightarrow \calD$
such that $F(f)$ is an equivalence in $\calD$, for each $f \in S$.
\end{proposition}

\begin{proof}
Let $S_0$ be the collection of all morphisms $C \rightarrow D$ in $\calC$ which exhibit
$D$ as an $L\calC$-localization of $C$. We first claim that, for any functor
$F: \calC \rightarrow \calD$, the following conditions are equivalent:
\begin{itemize}
\item[$(a)$] The functor $F$ is a right Kan extension of $F | L\calC$.
\item[$(b)$] The functor $F$ carries each morphism in $S_0$ to an equivalence in $\calD$.
\item[$(c)$] The functor $F$ carries each morphism in $S$ to an equivalence in $\calD$.
\end{itemize}

The equivalence of $(a)$ and $(b)$ follows immediately from the definitions
(since a morphism $f: C \rightarrow D$ exhibits $D$ as an $L \calC$-localization
of $C$ if and only if $f$ is an initial object of $(L \calC) \times_{\calC} \calC_{C/}$ ), and the
implication $(c) \Rightarrow (b)$ is obvious. To prove that $(b) \Rightarrow (c)$, let us
consider any map $f: C \rightarrow D$ which belongs to $S$. We have a commutative diagram
$$ \xymatrix{ C \ar[r]^{f} \ar[d] & LC \ar[d]^{f} \\
D \ar[r] & LD. }$$
Since $f \in S$, the map $Lf$ is an equivalence in $\calC$. If $F$ satisfies $(b)$, then $F$ carries
each of the horizontal maps to an equivalence in $\calD$. It follows from the two-out-of-three property that $Ff$ is an equivalence in $\calD$ as well, so that $F$ satisfies $(c)$.

Let $\Fun^{0}(\calC, \calD)$ denote the full subcategory of $\Fun(\calC, \calD)$ spanned by those functors which satisfy $(a)$, $(b)$, and $(c)$. Using Proposition \ref{lklk}, we deduce
that the restriction functor $\phi: \Fun^{0}(\calC, \calD) \rightarrow \Fun( L \calC, \calD)$ is fully faithful.
We now observe that $\psi$ is a right homotopy inverse to $\phi$. It follows that
$\phi$ is essentially surjective, and therefore an equivalence. Being right homotopy inverse to an equivalence, the functor $\psi$ must itself be an equivalence.
\end{proof}

\subsection{Factorization Systems}\label{factgen1}

Let $f: X \rightarrow Z$ be a map of sets. Then $f$ can be written as a composition
$$ X \stackrel{f'}{\rightarrow} Y \stackrel{f''}{\rightarrow} Z$$
where $f'$ is surjective and $f''$ is injective. This factorization is uniquely determined up to (unique) isomorphism: the set $Y$ can be characterized either as the image of the map $f$, or as the quotient
of $X$ by the equivalence relation $R = \{ (x,y) \in X^2: f(x) = f(y) \}$. We can describe the situation formally by saying that the collections of surjective and injective maps form a {\em factorization system} on the category $\Set$ of sets (see Definition \ref{spanhun}). In this section, we will describe a theory of factorization systems in the $\infty$-categorical setting. These ideas are due to Joyal and we refer the reader to \cite{joyalnotpub} for further details.

\begin{definition}\label{defperp}\index{gen}{orthogonal}\index{gen}{orthogonal!left}\index{gen}{orthogonal!right}\index{gen}{left orthogonal}\index{gen}{right orthogonal}
Let $f: A \rightarrow B$ and $g: X \rightarrow Y$ be morphisms in an $\infty$-category $\calC$.
We will say that $f$ is {\it left orthogonal} to $g$, or that $g$ is {\it right orthogonal to $f$},
if the following condition is satisfied:
\begin{itemize}
\item[$(\ast)$] For every commutative diagram
$$ \xymatrix{ A \ar[r] \ar[d]^{f} & X \ar[d]^{g} \\
B \ar[r] & Y }$$
in $\calC$, the mapping space $\bHom_{\calC_{A/ \, /Y}}(B,X )$
is contractible. (Here we abuse notation by identifying $B$ and $X$ with the corresponding objects
of $\calC_{A/\,/Y}$.)
\end{itemize}
In this case, we will write $f \perp g$.\index{not}{fperpg@$f \perp g$}
\end{definition}

\begin{remark}
More informally: a morphism $f: A \rightarrow B$ in an $\infty$-category $\calC$ is left orthogonal to another morphism $g: X \rightarrow Y$ if, for every commutative diagram
$$ \xymatrix{ A \ar[d]^{f} \ar[r] & X \ar[d]^{g} \\
B \ar[r] \ar@{-->}[ur] & Y, }$$
the space of dotted arrows rendering the diagram commutative is contractible.
\end{remark}

\begin{remark}\label{spack}
Let $f: A \rightarrow B$ and $g: X \rightarrow Y$ be morphisms in an $\infty$-category $\calC$.
Fix a morphism $A \rightarrow Y$, which we can identify with an object
$\overline{Y} \in \calC_{A/}$. Lifting $g: X \rightarrow Y$ to an object of
$\widetilde{X} \in \calC_{A/ \, /Y}$ is equivalent to lifting $g$ to a morphism
$\overline{g}: \overline{X} \rightarrow \overline{Y}$ in $\calC_{A/}$.
The map $f: A \rightarrow B$ determines an object $\overline{B} \in \calC_{A/}$, 
and lifting $f$ to an object $\widetilde{B} \in \calC_{A/ \, /Y}$ is equivalent to giving a map 
$h: \overline{B} \rightarrow \overline{Y}$ in $\calC_{A/}$. We therefore have a fiber
sequence of spaces
$$ \bHom_{ \calC_{ A/ \, /Y} }( \widetilde{B}, \widetilde{X} )
\rightarrow \bHom_{\calC_{A/}}( \overline{B}, \overline{X} )
\rightarrow \bHom_{\calC_{A/}}( \overline{B}, \overline{Y} ),$$
where the fiber is taken over the point $h$. Consequently, condition
$(\ast)$ of Definition \ref{defperp} can be reformulated as follows:
for every morphism $\overline{g}: \overline{X} \rightarrow \overline{Y}$ in
$\calC_{A/}$ lifting $g$, composition with $\overline{g}$ induces a homotopy equivalence
$$ \bHom_{\calC_{A/}}( \overline{B}, \overline{X} ) \rightarrow \bHom_{\calC_{A/}}( \overline{B}, \overline{Y}).$$
\end{remark}

\begin{notation}\index{not}{Sperp@$S^{\perp}$}\index{not}{perpS@$^{\perp}S$}
Let $\calC$ be an $\infty$-category, and let $S$ be a collection of morphisms in $\calC$. We let
$S^{\perp}$ denote the collection of all morphisms in $\calC$ which are right orthogonal to $S$, and
$^{\perp}S$ the collection of all morphisms in $\calC$ which are left orthogonal to $S$.
\end{notation}

\begin{remark}
Let $\calC$ be an ordinary category containing a pair of morphisms $f$ and $g$. If $f \perp g$, then
$f$ has the left lifting property with respect to $g$, and $g$ has the right lifting property with respect to $f$. It follows that for any collection $S$ of morphisms in $\calC$, we have inclusions
$S^{\perp} \subseteq S_{\perp}$ and $^{\perp} S \subseteq _{\perp}S$, where the latter classes of morphisms are defined in \S \ref{liftingprobs}.
\end{remark}

Applying Remark \ref{spack} to an $\infty$-category $\calC$ and its opposite, we obtain the following result:

\begin{proposition}\label{swimmm}
Let $\calC$ be an $\infty$-category and $S$ a collection of morphisms in $\calC$.
\begin{itemize}
\item[$(1)$] The sets of morphisms $S^{\perp}$ and $^{\perp}S$ contain every equivalence in $\calC$.
\item[$(2)$] The sets of morphisms $S^{\perp}$ and $^{\perp}S$ are closed under the formation of retracts.
\item[$(3)$] Suppose given a commutative diagram
$$ \xymatrix{ & Y \ar[dr]^{g} & \\
X \ar[ur]^{f} \ar[rr]^{h} & & Z }$$
in $\calC$, where $g \in S^{\perp}$. Then $f \in S^{\perp}$ if and only if $h \in S^{\perp}$. In particular, $S^{\perp}$ is closed under composition.

\item[$(4)$] Suppose given a commutative diagram
$$ \xymatrix{ & Y \ar[dr]^{g} & \\
X \ar[ur]^{f} \ar[rr]^{h} & & Z }$$
in $\calC$, where $f \in {}^{\perp}S$. Then $g \in {}^{\perp}S$ if and only if $h \in {}^{\perp}S$. In particular, $^{\perp}S$ is closed under composition.

\item[$(5)$] The set of morphisms $S^{\perp}$ is stable under pullbacks: that is, given a pullback diagram
$$ \xymatrix{ X' \ar[d]^{g'} \ar[r] & X \ar[d]^{g} \\
Y' \ar[r] & Y }$$
in $\calC$, if $g$ belongs to $S^{\perp}$, then $g'$ belongs to $S^{\perp}$.
\item[$(6)$] The set of morphisms $^{\perp}S$ is stable under pushouts: that is, given a pushout diagram
$$ \xymatrix{ A \ar[d]^{f} \ar[r] & A' \ar[d]^{f'} \\
B \ar[r] & B', }$$
if $f$ belongs to $^{\perp}S$, then so does $f'$.
\item[$(7)$] Let $K$ be a simplicial set such that $\calC$ admits $K$-indexed colimits.
Then the full subcategory of $\Fun( \Delta^1, \calC)$ spanned by the elements of
$^{\perp}S$ is closed under $K$-indexed colimits.
\item[$(8)$] Let $K$ be a simplicial set such that $\calC$ admits $K$-indexed limits. Then the full subcategory of $\Fun( \Delta^1, \calC)$ spanned by the elements of $S^{\perp}$ is closed under $K$-indexed limits.
\end{itemize}
\end{proposition}

\begin{remark}\label{smule}
Suppose given a pair of adjoint functors $\Adjoint{F}{\calC}{\calD.}{G}$
Let $f$ be a morphism in $\calC$ and $g$ a morphism in $\calD$. Then
$f \perp G(g)$ if and only if $F(f) \perp g$.
\end{remark}

\begin{definition}[Joyal]\label{spanhun}\index{gen}{factorization system}
Let $\calC$ be an $\infty$-category. A {\it factorization system} on $\calC$ is a pair
$(S_L, S_R)$, where $S_L$ and $S_R$ are collections of morphisms of $\calC$ which satisfy the following axioms:
\begin{itemize}
\item[$(1)$] The collections $S_L$ and $S_R$ are stable under the formation of retracts.
\item[$(2)$] Every morphism in $S_L$ is left orthogonal to every morphism in $S_R$.
\item[$(3)$] For every morphism $h: X \rightarrow Z$ in $\calC$, there exists a commutative
triangle
$$ \xymatrix{ & Y \ar[dr]^{g} & \\
X \ar[ur]^{f} \ar[rr]^{h} & & Z }$$
where $f \in S_L$ and $g \in S_R$. 
\end{itemize}
We will call $S_L$ the {\it left set} of the factorization system, and $S_R$ the
{\it right set} of the factorization system.
\end{definition}

\begin{example}\label{scumm}
Let $\calC$ be an $\infty$-category. Then $\calC$ admits a factorization system
$(S_L, S_R)$, where $S_L$ is the collection of all equivalences in $\calC$, and $S_R$ consists of all morphisms of $\calC$.
\end{example}

\begin{remark}\label{spill}
Let $(S_L, S_R)$ be a factorization system on an $\infty$-category $\calC$. Then
$(S_R, S_L)$ is a factorization system on the opposite $\infty$-category $\calC^{op}$. 
\end{remark}

\begin{proposition}\label{swin}
Let $\calC$ be an $\infty$-category, and let $(S_L, S_R)$ be a factorization system on $\calC$. 
Then $S_L = {^{\perp}S_R}$ and $S_R = S_L^{\perp}$.
\end{proposition}

\begin{proof}
By symmetry, it will suffice to prove the first assertion. The inclusion $S_L \subseteq {^{\perp}S_R}$ follows immediately from the definition. To prove the reverse inclusion, let us suppose that $h: X \rightarrow Z$ is a morphism in $\calC$ which is left orthogonal to every morphism in $S_R$. Choose a commutative triangle
$$ \xymatrix{ & Y \ar[dr]^{g} & \\
X \ar[ur]^{f} \ar[rr]^{h} & & Z }$$
where $f \in S_L$ and $g \in S_R$, and consider the associated diagram
$$ \xymatrix{ X \ar[r]^{f} \ar[d]^{h} & Y \ar[d]^{g} \\
Z \ar[r]^{\id} \ar@{-->}[ur] & Z. }$$
Since $h \perp g$, we can complete this diagram to a $3$-simplex of $\calC$ as indicated. This $3$-simplex exhibits $h$ as a retract of $f$, so that $h \in S_L$ as desired.
\end{proof}

\begin{remark}
It follows from Proposition \ref{swin} that a factorization system $(S_L, S_R)$ on an $\infty$-category $\calC$ is completely determined by {\em either} the left set $S_L$ or the right set $S_R$.
\end{remark}

\begin{corollary}\label{spen}
Let $(S_L, S_R)$ be a factorization system on an $\infty$-category $\calC$. Then
the collections of morphisms $S_L$ and $S_R$ contain all equivalences and are stable under composition.
\end{corollary}

\begin{proof}
Combine Propositions \ref{swin} and \ref{swimmm}.
\end{proof}

\begin{remark}
It follows from Corollary \ref{spen} that a factorization system $(S_L, S_R)$ on $\calC$ determines a pair of subcategories $\calC^{L}, \calC^{R} \subseteq \calC$, each containing all the objects of $\calC$: the morphisms of $\calC^{L}$ are the elements of $S_L$, and the morphisms of $\calC^{R}$ are the elements of $S_R$.
\end{remark}

\begin{example}
Let $p: \calC \rightarrow \calD$ be a coCartesian fibration of $\infty$-categories. Then there is an associated factorization system $(S_L, S_R)$ on $\calC$, where $S_L$ is the class of $p$-coCartesian morphisms of $\calC$, and $S_R$ is the class of morphisms $g$ of $\calC$ such that
$p(g)$ is an equivalence in $\calD$. If $\calD \simeq \Delta^0$, this recovers the factorization system of Example \ref{scumm}; if $p$ is an isomorphism, this recovers the opposite of the factorization system of Example \ref{scumm}.
\end{example}

\begin{example}\label{spink}
Let $\calX$ be an $\infty$-topos and let $n \geq -2$ be an integer. Then there exists a factorization system $(S_L, S_R)$ on $\calX$, where $S_L$ denotes the collection of $(n+1)$-connective morphisms
of $\calX$ and $S_R$ the collection of $n$-truncated morphisms of $\calC$. See \S 
\ref{homotopysheaves}.
\end{example}

Let $(S_L, S_R)$ be a factorization system on an $\infty$-category $\calC$, so that any morphism
$h: X \rightarrow Z$ factors as a composition
$$ \xymatrix{ & Y \ar[dr]^{g} & \\
X \ar[ur]^{f} \ar[rr]^{h} & & Z }$$
where $f \in S_L$ and $g \in S_R$. For many purposes, it is important to know that this factorization is {\em canonical}. More precisely, we have the following result:

\begin{proposition}\label{canfact}
Let $\calC$ be an $\infty$-category and let $S_L$ and $S_R$ be collections of morphisms
in $\calC$. Suppose that $S_L$ and $S_R$ are stable under equivalence in $\Fun( \Delta^1, \calC)$, and contain every equivalence in $\calC$.
The following conditions are equivalent:
\begin{itemize}
\item[$(1)$] The pair $(S_L, S_R)$ is a factorization system on $\calC$.
\item[$(2)$] The restriction map $p: \Fun'( \Delta^2, \calC) \rightarrow \Fun( \Delta^{ \{0,2\} }, \calC)$
is a trivial Kan fibration. Here $\Fun'( \Delta^2, \calC)$ denotes the full subcategory of
$\Fun( \Delta^2, \calC)$ spanned by those diagrams
$$ \xymatrix{ & Y \ar[dr]^{g} & \\
X \ar[ur]^{f} \ar[rr]^{h} & & Z }$$
such that $f \in S_L$ and $g \in S_R$. 
\end{itemize}
\end{proposition}

\begin{corollary}\label{funcsys}
Let $\calC$ be an $\infty$-category equipped with a factorization system $(S_L, S_R)$, and let $K$ be an arbitrary simplicial set. Then the $\infty$-category $\Fun(K, \calC)$ admits a factorization system
$(S_L^{K}, S_R^{K})$, where $S_{L}^{K}$ denotes the collection of all morphisms $f$ in 
$\Fun(K, \calC)$ such that $f(v) \in S_L$ for each vertex $v$ of $K$, and
$S_{R}^{K}$ is defined likewise.
\end{corollary}

The remainder of this section is devoted to the proof of Proposition \ref{canfact}. We begin with a few preliminary results.

\begin{lemma}\label{prefukt}
Let $\calC$ be an $\infty$-category and let $(S_L, S_R)$ be a factorization system on $\calC$.
Let $\calD$ be the full subcategory of $\Fun( \Delta^1, \calC)$ spanned by the elements of $S_R$.
Then:
\begin{itemize}
\item[$(1)$] The $\infty$-category $\calD$ is a localization of $\Fun( \Delta^1, \calC)$; in other words, the inclusion $\calD \subseteq \Fun( \Delta^1, \calC)$ admits a left adjoint.

\item[$(2)$] A morphism $\alpha: h \rightarrow g$ in $\Fun( \Delta^1, \calC)$, corresponding to a commutative diagram
$$ \xymatrix{ X \ar[d]^{h} \ar[r]^{f} & Y \ar[d]^{g} \\
Z' \ar[r]^{e} & Z }$$
exhibits $g$ as a $\calD$-localization of $h$ $($see Definition \ref{locaobj}$)$ if and only if $g \in S_R$, $f \in S_L$, and 
$e$ is an equivalence.
\end{itemize}
\end{lemma}

\begin{proof}
We will prove the ``if'' direction of assertion $(2)$.
It follows from the definition of a factorization system that for every object
$h \in \Fun( \Delta^1, \calC)$, there exists a morphism $\alpha: h \rightarrow g$ satisfying
the condition stated in $(2)$, which therefore exhibits $g$ as a $\calD$-localization of $h$.
Invoking Proposition \ref{testreflect}, we will deduce $(1)$. Because a $\calD$-localization of
$h$ is uniquely determined up to equivalence, we will also deduce the ``only if'' direction of assertion $(2)$.

Suppose given a commutative diagram
$$ \xymatrix{ X \ar[d]^{h} \ar[r]^{f} & Y \ar[d]^{g} \\
Z' \ar[r]^{e} & Z }$$
where $f \in S_L$, $g \in S_R$, and $e$ is an equivalence, and let $\overline{g}: \overline{Y} \rightarrow \overline{Z}$ be another element of $S_R$. We have a diagram of spaces
$$ \xymatrix{ \bHom_{\Fun(\Delta^1, \calC)}( g, \overline{g}) \ar[r]^{\psi} \ar[d] & \bHom_{ \Fun( \Delta^1, \calC)}( h, \overline{g} ) \ar[d] \\
\bHom_{\calC}( Z, \overline{Z}) \ar[r]^{\psi_0} & \bHom_{\calC}( Z', \overline{Z} )}$$
which commutes up to canonical homotopy. We wish to prove that $\psi$ is a homotopy equivalence.

Since $e$ is an equivalence in $\calC$, the map $\psi_0$ is a homotopy equivalence.
It will therefore suffice to show that $\psi$ induces a homotopy equivalence after passing to the homotopy fibers over any point of $\bHom_{\calC}(Z, \overline{Z}) \simeq \bHom_{\calC}(Z', \overline{Z})$. These homotopy fibers can be identified with the homotopy fibers of the vertical arrows in the diagram
$$ \xymatrix{ \bHom_{ \calC}( Y, \overline{Y} ) \ar[r] \ar[d] & \bHom_{\calC}( X, \overline{Y}) \ar[d] \\
\bHom_{ \calC}( Y, \overline{Z} ) \ar[r] & \bHom_{ \calC}( X, \overline{Z} ). }$$
It will therefore suffice to show that this diagram (which commutes up to specified homotopy) is a homotopy pullback. Unwinding the definition, this is equivalent to the assertion that $f$
is left orthogonal to $\overline{g}$, which is part of the definition of a factorization system.
\end{proof}

\begin{lemma}\label{hulfer}
Let $K$, $A$, and $B$ be simplicial sets. Then the diagram
$$ \xymatrix{ K \times B \ar[r] \ar[d] & K \times (A \star B) \ar[d] \\
B \ar[r] & ( K \times A) \star B }$$
is a homotopy pushout square of simplicial sets $($with respect to the Joyal model structure$)$.
\end{lemma}

\begin{proof}
We consider the larger diagram
$$ \xymatrix{ K \times B \ar[r] \ar[d] & K \times (A \diamond B) \ar[r] \ar[d] & K \times (A \star B) \ar[d] \\
B \ar[r] & (K \times A) \diamond B \ar[r] & (K \times A) \star B. }$$
The square on the left is a pushout square in which the horizontal maps are monomorphisms
of simplicial sets, and therefore a homotopy pushout square (since the Joyal model structure is left proper). The square on the right is a homotopy pushout square, since the horizontal arrows are both categorical equivalences (Proposition \ref{rub3}). It follows that the outer square is also a homotopy pushout, as desired.
\end{proof}

\begin{notation}
In the arguments which follow, we let $Q$ denote the simplicial subset of
$\Delta^3$ spanned by all simplices which do not contain $\Delta^{ \{1,2 \} }$. Note that
$Q$ is isomorphic to the product $\Delta^1 \times \Delta^1$ as a simplicial set.
\end{notation}

\begin{lemma}\label{sidewise}
Let $\calC$ be an $\infty$-category, and let $\sigma: Q \rightarrow \calC$ be a diagram, which
we depict as
$$ \xymatrix{ A \ar[r] \ar[d] & X \ar[d] \\
B \ar[r] & Y. }$$
Then there is a canonical categorical equivalence
$$\theta: \Fun( \Delta^3, \calC) \times_{ \Fun( Q, \calC) } \{ \sigma \} \rightarrow \bHom_{ \calC_{ A/ \, /Y }}( B, X)$$
In particular, $\Fun( \Delta^3, \calC) \times_{ \Fun(Q, \calC)} \{ \sigma \}$ is a Kan complex.
\end{lemma}

\begin{proof}
We will identify $\bHom_{ \calC_{A/ \, /Y}}(B,X)$ with the simplicial set $Z$ defined by the following universal property: for every simplicial set $K$, we have a pullback diagram of sets
$$ \xymatrix{ \Hom_{\sSet}(K, Z) \ar[r] \ar[d] & \Hom_{\sSet}( \Delta^0 \star (K \times \Delta^1) \star \Delta^0, \calC) \ar[d] \\
\Delta^0 \ar[r] & \Hom_{\sSet}( \Delta^0 \star (K \times \bd \Delta^1) \star \Delta^0, \calC). }$$
The map $\theta$ is then induced by the natural transformation
$$ K \times \Delta^3 \simeq K \times ( \Delta^0 \star \Delta^1 \star \Delta^0)
\rightarrow \Delta^0 \star (K \times \Delta^1) \star \Delta^0.$$

We wish to prove that $\theta$ is a categorical equivalence. Since $\calC$ is an
$\infty$-category, it will suffice to show that for every simplicial set $K$, the rightmost square of the diagram
$$ \xymatrix{ K \times ( \Delta^{ \{0\} } \coprod \Delta^{ \{3 \} }) \ar[r] \ar[d] & K \times C \ar[r] \ar[d] & K \times \Delta^3 \ar[d] \\
\Delta^{ \{0\} } \coprod \Delta^{ \{3\} } \ar[r] & \Delta^0 \star (K \times \bd \Delta^1) \star \Delta^0 \ar[r] & \Delta^0
\star (K \times \Delta^1) \star \Delta^0 }$$
is a homotopy pushout square (with respect to the Joyal model structure). For this we need only verify that the left and outer squares are homotopy pushout diagrams, which follows from repeated application of Lemma \ref{hulfer}.
\end{proof}

\begin{proof}[Proof of Proposition \ref{canfact}]
We first show that $(1) \Rightarrow (2)$. Assume that $(S_L, S_R)$ is a factorization system on $\calC$.
The map $p: \Fun'( \Delta^2, \calC) \rightarrow \Fun( \Delta^{ \{0,2\} }, \calC)$ is obviously a categorical fibration. It will therefore suffice to show that $p$ is a categorical equivalence.

Let $\calD$ be the full subcategory of $\Fun( \Delta^1 \times \Delta^1, \calC)$
spanned by those diagrams of the form
$$\xymatrix{ X \ar[d]^{h} \ar[r]^{f} & Y \ar[d]^{g} \\
Z' \ar[r]^{e} & Z }$$
where $f \in S_L$, $g \in S_R$, and $e$ is an equivalence in $\calC$. The map $p$ factors as a composition
$$ \Fun'(\Delta^2, \calC) \stackrel{p'}{\rightarrow} \calD \stackrel{p''}{\rightarrow} \Fun( \Delta^1, \calC)$$
where $p'$ carries a diagram
$$ \xymatrix{ & Y \ar[dr]^{g} & \\
X \ar[rr]^{h} \ar[ur]^{f} & & Z }$$
to the partially degenerate square
$$\xymatrix{ X \ar[d]^{h} \ar[dr]^{h} \ar[r]^{f} & Y \ar[d]^{g} \\
Z \ar[r]^{\id} & Z, }$$
and $p''$ is given by restriction to the left vertical edge of the diagram. To complete the proof, it will suffice to show that $p'$ and $p''$ are categorical equivalences.

We first show that $p'$ is a categorical equivalence.
The map $p'$ admits a left inverse $q$, given by composition with an inclusion
$\Delta^2 \subseteq \Delta^1 \times \Delta^1$. We note that $q$ is a pullback of the restriction map
$q': \Fun''( \Delta^2, \calC) \rightarrow \Fun( \Delta^{ \{0,2 \} }, \calC),$
where $\Fun''( \Delta^2, \calC)$ is the full subcategory spanned by diagrams of the form
$$ \xymatrix{ X \ar[dr] \ar[d] & \\
Z' \ar[r]^{e} & Z }$$
where $e$ is an equivalence. Since $q'$ is a trivial Kan fibration (Proposition \ref{lklk}), $q$ is a trivial Kan fibration, so that $p'$ is a categorical equivalence as desired.

We now complete the proof by showing that $p''$ is a trivial Kan fibration. Let
$\calE$ denote the full subcategory of $\Fun( \Delta^1, \calC) \times \Delta^1$ spanned by those pairs
$(g,i)$ where either $i=0$ or $g \in S_R$. The projection map $r: \calE \rightarrow \Delta^1$ is a 
Cartesian fibration associated to the inclusion $\Fun'( \Delta^1, \calC) \subseteq \Fun( \Delta^1, \calC)$, where $\Fun'(\Delta^1, \calC)$ is the full subcategory spanned by the elements of $S_R$. 
Using Lemma \ref{prefukt}, we conclude that $r$ is also a coCartesian fibration. Moreover,
we can identify
$$ \calD \subseteq \Fun( \Delta^1 \times \Delta^1, \calC) \simeq
\bHom_{ \Delta^1}( \Delta^1, \calE)$$
with the full subcategory spanned by the coCartesian sections of $r$. In terms of this identification,
$p''$ is given by evaluation at the initial vertex $\{0\} \subseteq \Delta^1$, and is therefore
a trivial Kan fibration as desired. This completes the proof that $(1) \Rightarrow (2)$.

Now suppose that $(2)$ is satisfied, and choose a section $s$ of the trivial Kan fibration $p$.
Let $s$ carry each morphism $f: X \rightarrow Z$ to a commutative diagram
$$ \xymatrix{ & Y \ar[dr]^{s_R(f)} & \\
X \ar[ur]^{ s_L(f)} \ar[rr]^{f} & & Z. }$$
If $s_{R}(f)$ is an equivalence, then $f$ is equivalent to $s_L(f)$ and therefore belongs to
$S_L$. Conversely, if $f$ belongs to $S_L$, then the diagram
$$ \xymatrix{ & Z \ar[dr]^{\id} & \\
X \ar[ur]^{f} \ar[rr]^{f} & & Z }$$
is a preimage of $f$ under $p$, and therefore equivalent to $s(f)$; this implies that
$s_L(f)$ is an equivalence. We have proved the following:
\begin{itemize}
\item[$(\ast)$] A morphism $f$ of $\calC$ belongs to $S_L$ if and only if $s_L(f)$ is an equivalence in $\calC$.
\end{itemize}

It follows immediately from $(\ast)$ that $S_L$ is stable under the formation of retracts; similarly, $S_R$ is stable under the formation of retracts. To complete the proof, it will suffice to show that $f \perp g$ whenever $f \in S_L$ and $g \in S_R$. Fix a commutative diagram $\sigma:$
$$ \xymatrix{ A \ar[d]^{f} \ar[r] & X \ar[d]^{g} \\
B \ar[r] & Y }$$
in $\calC$. In view of Lemma \ref{sidewise}, it will suffice to show that the Kan complex
$\Fun( \Delta^3, \calC) \times_{ \Fun( Q, \calC ) } \{ \sigma \}$ is contractible.

Let $\calD$ denote the full subcategory of $\Fun( \Delta^2 \times \Delta^1, \calC)$ spanned by those diagrams
$$ \xymatrix{ C \ar[r] \ar[d]^{u'} & Z \ar[d]^{v'} \\
C' \ar[r] \ar[d]^{u''} & Z' \ar[d]^{v''} \\
C'' \ar[r] & Z'' }$$
for which $u' \in S_L$, $v'' \in S_R$, and the maps $v'$ and $u''$ are equivalences.
Let us identify $\Delta^3$ with the full subcategory of $\Delta^2 \times \Delta^1$ spanned by
all those vertices except for $(2,0)$ and $(0,1)$. Applying Proposition \ref{lklk} twice,
we deduce that the restriction functor $\Fun( \Delta^2 \times \Delta^1, \calC) \rightarrow \Fun( \Delta^3, \calC)$ induces a trivial Kan fibration from $\calD$ to the full subcategory
$\calD' \subseteq \Fun(\Delta^3, \calC)$ spanned by those diagrams
$$ \xymatrix{ C \ar[d]^{u'} \ar[r] & Z' \ar[d]^{v''} \\
C' \ar[r] \ar[ur] & Z'' }$$
such that $u' \in S_L$ and $v'' \in S_R$. It will therefore suffice to show that the fiber
$\calD \times_{ \Fun(Q, \calC)} \{ \sigma \}$ is contractible.

By construction, the restriction functor $\calD \rightarrow \Fun(Q, \calC)$ is equivalent to the composition
$$ q: \calD \subseteq \Fun( \Delta^2 \times \Delta^1, \calC) \rightarrow
\Fun( \Delta^{ \{0,2\} } \times \Delta^1, \calC).$$
It will therefore suffice to show that $q^{-1} \{ \sigma \}$ is a contractible Kan complex.
Invoking assumption $(2)$ and $(\ast)$, we deduce that $q$ induces an equivalence from
$\calD$ to the full subcategory of $\Fun( \Delta^{ \{0,2\} } \times \Delta^1, \calC)$ spanned by
those diagrams
$$ \xymatrix{ C \ar[d]^{u} \ar[r] & Z \ar[d]^{v} \\
C'' \ar[r] & Z'' }$$
such that $u \in S_L$ and $v \in S_R$. The desired result now follows from our assumption that
$f \in S_L$ and $g \in S_R$.
\end{proof}

\subsection{Application: Automorphisms of $\Cat_{\infty}$}\label{cataut}

In \S \ref{working}, we saw that for every $\infty$-category $\calC$, the opposite simplicial set
$\calC^{op}$ is again an $\infty$-category. We will see below that the construction
$\calC \mapsto \calC^{op}$ determines an equivalence from the $\infty$-category
$\Cat_{\infty}$ to itself. In fact, this is essentially the {\em only} nontrivial self-equivalence of
$\Cat_{\infty}$. More precisely, we have the following result due to To\"{e}n (see \cite{toenchar}):

\begin{theorem}\label{cabbi}\index{gen}{$\infty$-category!opposite}\index{gen}{opposite!$\infty$-category}
Let $\calE$ denote the full subcategory of $\Fun( \Cat_{\infty}, \Cat_{\infty})$ spanned by the equivalences. Then $\calE$ is equivalent to the $($nerve of the$)$ discrete category
$\{ \id, r \}$, where $r: \Cat_{\infty} \rightarrow \Cat_{\infty}$ is a functor which associates to every
$\infty$-category its opposite. 
\end{theorem}

Our goal in this section is to give a proof of Theorem \ref{cabbi}. We first outline the basic strategy. Fix an equivalence $f$ from $\Cat_{\infty}$ to itself. The first step is to argue that $f$ is determined by its restriction to a reasonably small subcategory of $\Cat_{\infty}$. To prove this, we will introduce the notion of a subcategory $\calC^{0} \subseteq \calC$ which {\it strongly generates} $\calC$
(Definitions \ref{coughball} and \ref{ballcough}). We will then show that $\Cat_{\infty}$ is strongly generated by the subcategory consisting of nerves of partially ordered sets (in fact, it is generated
by an even smaller subcategory: see Theorem \ref{upsquare}). This will allow us to reduce to the problem of understanding the category of self-equivalences of the category of partially ordered sets, which is easy to tackle directly: see Proposition \ref{cape}. 

We begin by introducing some definitions.

\begin{definition}\label{coughball}\index{gen}{strongly generates}\index{gen}{generates!strongly}
Let $f: \calC \rightarrow \calD$ be a functor between $\infty$-categories.
We will say that $f$ is {\em strongly generates} the $\infty$-category $\calD$ if the identity transformation
$\id: f \rightarrow f$ exhibits the identity functor $\id_{\calD}$ as a left Kan extension of $f$
along $f$.
\end{definition}

\begin{remark}
In other words, a functor $f: \calC \rightarrow \calD$ strongly generates the $\infty$-category
$\calD$ if and only if, for every object $D \in \calD$, the evident diagram
$( \calC \times_{\calD} \calD_{/D} )^{\triangleright} \rightarrow \calD_{/D}^{\triangleright} \rightarrow \calD$
exhibits $D$ as a colimit of the diagram
$(\calC \times_{\calD} \calD_{/D}) \rightarrow \calC \stackrel{f}{\rightarrow} \calD.$
In particular, this implies that every object of $\calD$ can be obtained as the colimit of a diagram
which factors through $\calC$. Moreover, if $\calC$ is small and $\calD$ is locally small, then the diagram can be assumed small.
\end{remark}

\begin{remark}\label{copse}
Let $f: \calC \rightarrow \calD$ be a functor between $\infty$-categories, where $\calC$ is small, $\calD$ is locally small, and $\calD$ admits small colimits. In view of Theorem \ref{charpresheaf}, we may assume without loss of generality that $f$ factors as a composition
$$ \calC \stackrel{j}{\rightarrow} \calP(\calC) \stackrel{F}{\rightarrow} \calD,$$
where $j$ denotes the Yoneda embedding and $F$ preserves small colimits.
Corollary \ref{coughspaz} implies that $F$ has a right adjoint $G$, given by the composition
$$ \calD \stackrel{j'}{\rightarrow} \Fun( \calD^{op}, \SSet) \stackrel{\circ f}{\rightarrow} \calP(\calC),$$
where $j'$ denotes the Yoneda embedding for $\calD$; moreover, the transformation
$$ f = F \circ j \rightarrow (F \circ (G \circ F)) \circ j \simeq (F \circ G) \circ f$$
exhibits $(F \circ G)$ as a left Kan extension of $f$ along itself. It follows that $f$
strongly generates $\calD$ if and only if the counit map $F \circ G \rightarrow \id_{\calD}$ is an equivalence of functors. This is equivalent to the requirement that the functor $G$ is fully faithful.

In other words, the functor $f: \calC \rightarrow \calD$ strongly generates $\calD$ if and only if the
induced functor $F: \calP(\calC) \rightarrow \calD$ exhibits $\calD$ as a localization of
$\calP(\calC)$. In particular, the Yoneda embedding $\calC \rightarrow \calP(\calC)$
strongly generates $\calP(\calC)$, for any small $\infty$-category $\calC$.
\end{remark}

\begin{remark}\label{copo}
Let $f: \calC \rightarrow \calD$ be as in Remark \ref{copse}, and let $\calE$ be an $\infty$-category which admits small colimits. Let $\Fun^{0}(\calD, \calE)$ denote the full subcategory of
$\Fun(\calD, \calE)$ spanned by those functors which preserve small colimits. Then composition
with $f$ induces a fully faithful functor $\Fun^{0}(\calD, \calE) \rightarrow \Fun(\calC, \calE)$. 
This follows from Theorem \ref{charpresheaf}, Proposition \ref{unichar}, and Remark \ref{copse}.
\end{remark}

\begin{definition}\index{gen}{generates!strongly}\index{gen}{strongly generates}\label{ballcough}
Let $\calC$ be an $\infty$-category. We will say that a full subcategory $\calC^{0} \subseteq \calC$ 
{\it strongly generates} $\calC$ if the inclusion functor $\calC^{0} \rightarrow \calC$ strongly generates $\calC$, in the sense of Definition \ref{coughball}.
\end{definition}

\begin{remark}\label{sobre}
In other words, $\calC^0$ strongly generates $\calC$ if and only if the identity functor
$\id_{\calC}$ is a left Kan extension of $\id_{\calC} | \calC^{0}$. It follows from Proposition \ref{acekan} that if $\calC^{0} \subseteq \calC^{1} \subseteq \calC$ are full subcategories and
$\calC^{0}$ strongly generates $\calC$, then $\calC^{1}$ also strongly generates $\calC$.
\end{remark}

\begin{example}
The $\infty$-category $\SSet$ of spaces is strongly generated by its final object;
this follows immediately from Remark \ref{copse}.
\end{example}

The main ingredient in the proof of Theorem \ref{cabbi} is the following result, which is of interest in its own right:

\begin{theorem}\label{upsquare}
The $\infty$-category $\Cat_{\infty}$ is strongly generated by the full subcategory consisting of the objects $\{ \Delta^n \}_{n \geq 0}$. 
\end{theorem}

\begin{remark}
It follows from Theorem \ref{upsquare} that the theory of $\infty$-categories can be obtained as a suitable localization of the model category $\Fun( \cDelta^{op}, \sSet)$ of bisimplicial sets. It is possible to describe this localization precisely: this leads to Rezk's theory of {\em complete Segal spaces}, which is another model for the theory of higher categories where every $k$-morphism is assumed to be invertible for $k > 1$. For more details, we refer the reader to \cite{completesegal}.
\end{remark}

\begin{proof}[Proof of Theorem \ref{upsquare}]
We can identify the full subcategory in question with $\Nerve( \cDelta)$, the (nerve of) the category of simplices. The fully faithful embedding $f: \Nerve(\cDelta) \rightarrow \Cat_{\infty}$ can be extended (up to equivalence) to a colimit-preserving functor $F: \calP( \Nerve(\cDelta) ) \rightarrow \Cat_{\infty}$, which admits a right adjoint $G$ (this follows from Corollary \ref{adjointfunctor}, but an explicit description will be given below). In view of Remark \ref{copse}, it will suffice to show that the unit transformation $\id \rightarrow F \circ G$ is an equivalence.

We now reformulate the desired conclusion in the language of model categories. We can identify
$\Cat_{\infty}$ with the underlying $\infty$-category $\bfA^{\degree}$ of the simplicial model category
$\bfA = \mSet$ of marked simplicial sets, with the Cartesian model structure described in \S \ref{twuf}. The diagram $f$ is then obtained from a diagram $\overline{f}: \cDelta \rightarrow \bfA$, given by the cosimplicial object $[n] \mapsto (\Delta^n)^{\flat}.$
Since $\bfA$ is a simplicial model category (with respect to the simplicial structure
given by the formula $X \otimes K = X \times K^{\sharp}$), we can extend $\overline{f}$ to a colimit preserving functor
$$\overline{F}: \Fun( \cDelta^{op}, \sSet) \rightarrow \bfA.$$
Here $\Fun( \cDelta^{op}, \sSet)$ can be identified with the category of bisimplicial sets.
Since the cosimplicial object $\overline{f} \in \Fun( \cDelta, \bfA)$ is Reedy cofibrant (see \S \ref{coreed}), the functor $\overline{F}$ is a left Quillen functor if we endow $\Fun( \cDelta^{op}, \sSet)$ with the injective model structure (Example \ref{cabletome}). The functor $\overline{F}$ has a right adjoint
$\overline{G}$, given by the formula
$$ \overline{G}(X)^{m,n} = \Hom_{ \bfA}( (\Delta^m)^{\flat} \times (\Delta^n)^{\sharp}, X).$$
This right adjoint induces a functor from $\bfA^{\degree}$ to $\Fun(\cDelta^{op}, \sSet)^{\degree}$, which
(after passing to the simplicial nerve) is equivalent to the functor $G: \Cat_{\infty}
\rightarrow \calP( \Nerve(\cDelta))$ considered above. Consequently, it will suffice to show that
the counit map $L \overline{F} \circ R \overline{G} \rightarrow \id_{ \h{\bfA}}$ is an equivalence of functors, where $L \overline{F}$ and $R \overline{G}$ denote the left and right derived functors of
$\overline{F}$ and $\overline{G}$, respectively. Since every object of $\Fun( \cDelta^{op}, \sSet)$ is cofibrant, we can identify $\overline{F}$ with its left derived functor. We are therefore reduced to proving the following:
\begin{itemize}
\item[$(\ast)$] Let $\overline{X} = (X,M)$ be a fibrant object of the category $\bfA$ of marked simplicial sets. Then the counit map $\eta_{\overline{X}}: \overline{F} \overline{G} \overline{X} \rightarrow \overline{X}$ is a weak equivalence in $\bfA$.
\end{itemize}
Since $\overline{X}$ is fibrant, the simplicial set $X$ is an $\infty$-category and $M$ is the collection of all equivalences in $X$. Unwinding the definitions, we can identify
$\overline{F} \overline{G} X$ with the marked simplicial set $(Y, N)$ described as follows:
\begin{itemize}
\item[$(a)$] An $n$-simplex of $Y$ is a map of simplicial sets
$\Delta^n \times \Delta^n \rightarrow X$, which carries every morphism of
$\{i\} \times \Delta^n$ to an equivalence in $\calC$, for $0 \leq i \leq n$.
\item[$(b)$] An edge $\Delta^1 \rightarrow Y$ belongs to $N$ if and only if the corresponding
map $\Delta^1 \times \Delta^1 \rightarrow X$ factors through the projection onto the second
factor.
\end{itemize}

In terms of this identification, the map $\eta_{\overline{X}}: (Y,N) \rightarrow (X,M)$ is defined on $n$-simplices by composing with the diagonal map $\Delta^n \rightarrow \Delta^n \times \Delta^n$.

Let $N'$ denote the collection of all edges of $Y$ which correspond to maps from
$(\Delta^1 \times \Delta^1)^{\sharp}$ into $\overline{X}$. The map $\eta_{\overline{X}}$ factors as a composition
$$ (Y, N) \stackrel{i}{\rightarrow} (Y,N') \stackrel{ \eta'_{\overline{X}}}{\rightarrow} (X,M).$$
We claim that the map $i$ is a weak equivalence of marked simplicial sets. To prove this, it
will suffice to show that for every edge $\alpha$ which belongs to $N'$, there exists a 
$2$-simplex  $\sigma:$
$$ \xymatrix{ & y' \ar[dr]^{\alpha''} & \\
y \ar[ur]^{\alpha'} \ar[rr]^{\alpha} & & y'' }$$
in $Y$, where $\alpha'$ and $\alpha''$ belong to $N$. To see this, let us suppose that
$\alpha$ classifies a commutative diagram
$$ \xymatrix{ A \ar[d]^{p} \ar[r]^{q} \ar[dr]^{r} & A' \ar[d]^{p'} \\
B \ar[r]^{q'} & B' }$$
in the $\infty$-category $X$. We wish to construct an appropriate $2$-simplex
$\sigma$ in $Y$, corresponding to a map $\widetilde{\sigma}: \Delta^2 \times \Delta^2 \rightarrow X^0$, where $X^0$ denotes the largest Kan complex contained in $X$. Let $T$ denote the full subcategory
of $\Delta^2 \times \Delta^2$ spanned by all vertices except for $(0,2)$, and let
$\widetilde{\sigma}_0: T \rightarrow X^0$ be the map described by the diagram
$$ \xymatrix{ A \ar[r]^{\id} \ar[d]^{q} & A \ar[d]^{p} \ar[r]^{q} & A' \ar[d]^{\id} \\
A' \ar[r]^{\id} & A' \ar[r]^{\id} \ar[d]^{p'} & A' \ar[d]^{p'} \\
& B' \ar[r]^{\id} & B'. }$$
To prove that $\widetilde{\sigma}_0$ can be extended to a map $\widetilde{\sigma}$ with the desired properties, it suffices to solve an extension problem of the form
$$ \xymatrix{ T \coprod_{ \Delta^{ \{0,2\} } } \Delta^2 \ar[r] \ar[d] & X^0 \\
\Delta^2 \times \Delta^2. \ar@{-->}[ur] & }$$
This is possible because $X^0$ is a Kan complex and the left vertical map is a weak homotopy equivalence. This completes the proof that $i$ is a weak equivalence. By the two-out-of-three property, it will now suffice to show that $\eta'_{\overline{X}}: (Y,N') \rightarrow (X,M)$ is an equivalence of marked simplicial sets.

We now define maps $R_{\leq}, R_{\geq}: \Delta^1 \times Y \rightarrow Y$ as follows.
Consider a map $g: \Delta^n \rightarrow \Delta^1 \times Y$, corresponding to a partition
$[n] = [n]_{-} \cup [n]_{+}$ and a map $\widetilde{g}: \Delta^n \times \Delta^n \rightarrow X$.
We then define $R_{\leq} \circ g$ to be the $n$-simplex of $Y$ corresponding to the map
$\widetilde{g} \circ \tau: \Delta^n \times \Delta^n \rightarrow X$, where
$\tau: \Delta^n \times \Delta^n \rightarrow \Delta^n \times \Delta^n$ is defined on vertices by the formula
$$ \tau(i,j) = \begin{cases} (i,j) & \text{if } i \leq j \\
(i,j) & \text{if } j \in [n]_{-} \\
(i,i) & \text{otherwise.} \end{cases}$$
Similarly, we let $R_{\geq} \circ g$ correspond to the map $\widetilde{g} \circ \tau'$, where
$\tau'$ is given on vertices by the formula
$$ \tau_{i,j} = \begin{cases} (i,j) & \text{if } i \geq j \\
(i,j) & \text{if } j \in [n]_{+} \\
(i,I) & \text{otherwise.} \end{cases}$$
The map $R_{\leq}$ defines a homotopy from $\id_{Y}$ to an idempotent map
$r_{\leq}: Y \rightarrow Y$. Similarly, $R_{\geq}$ defines a homotopy from an idempotent map
$r_{\geq}: Y \rightarrow Y$ to the identity map $\id_{Y}$. Let $Y_{\leq}, Y_{\geq} \subseteq Y$
denote the images of the maps $r_{\leq}$ and $r_{\geq}$, respectively. Let
$N'_{\leq}$ denote the collection of all edges of $Y$ which belong to $N'$, and define
$N'_{\geq}$ similarly. The map $R_{\leq}$ determines a map
$(Y,N') \times (\Delta^1)^{\sharp} \rightarrow (Y,N')$, which exhibits
$(Y_{\leq}, N'_{\leq})$ as a deformation retract of $(Y, N')$ in the category of marked simplicial sets.
Similarly, the map $R_{\geq}$ exhibits $(Y_{\leq} \cap Y_{\geq}, N'_{\leq} \cap N'_{\geq})$ as a
deformation retract of $(Y_{\leq}, N'_{\leq})$. It will therefore suffice to show that the composite map
$$ (Y_{\leq} \cap Y_{\geq}, N'_{\leq} \cap N'_{\geq}) \subseteq (Y, N') \rightarrow (X,M)$$
is a weak equivalence of marked simplicial sets. We now complete the proof by observing that this composite map is an isomorphism.
\end{proof}

%\begin{corollary}
%Let $\calC$ denote the full subcategory of $\Cat_{\infty}$ spanned by those $\infty$-categories which are equivalent to the nerves of partially ordered sets. Then $\calC$ strongly generates the $\infty$-category
%$\Cat_{\infty}$.
%\end{corollary}

%\begin{proof}
%Combine Theorem \ref{upsquare} with Remark \ref{sobre}.
%\end{proof}

It follows from Remark \ref{sobre} and Theorem \ref{upsquare} that $\Cat_{\infty}$ is strongly generated by the full subcategory spanned by those $\infty$-categories of the form $\Nerve P$, where
$P$ is a partially ordered set. Our next step is to describe this subcategory in more intrinsic terms.

\begin{proposition}
Let $\calC$ be an $\infty$-category. The following conditions are equivalent:
\begin{itemize}
\item[$(1)$] The $\infty$-category $\calC$ is equivalent to the nerve of a partially ordered set $P$.
\item[$(2)$] For every $\infty$-category $\calD$ and every pair of functors
$F, F': \calD \rightarrow \calC$ such that $F(x) \simeq F'(x)$ for each object $x \in \calD$, 
the functors $F$ and $F'$ are equivalent as objects of $\Fun(\calD, \calC)$.
\item[$(3)$] For every $\infty$-category $\calD$, the map of sets
$$ \pi_0 \bHom_{\Cat_{\infty}}( \calD, \calC)
\rightarrow \Hom_{ \Set}( \pi_0 \bHom_{ \Cat_{\infty}}( \Delta^0, \calD),
\pi_0 \bHom_{ \Cat_{\infty} }( \Delta^0, \calC) )$$
is injective.
\end{itemize}
\end{proposition}

\begin{proof}
The implication $(1) \Rightarrow (2)$ is obvious, and $(3)$ is just a restatement of $(2)$.
Assume $(2)$; we will show that $(1)$ is satisfied. Let $P$ denote the collection of equivalence
classes of objects of $\calC$, where $x \leq y$ if the space $\bHom_{\calC}(x,y)$ is nonempty. 
There is a canonical functor $\calC \rightarrow \Nerve P$. To prove that this functor is an equivalence, it will suffice to show the following:
\begin{itemize}
\item[$(\ast)$] For every pair of objects $x,y \in \calC$, the space $\bHom_{\calC}(x,y)$ is either empty or contractible.
\end{itemize}
To prove $(\ast)$, we may assume without loss of generality that $\calC$ is the nerve of a fibrant
simplicial category $\overline{\calC}$. Let $x$ and $y$ be objects of $\overline{\calC}$ such that
the Kan complex $K = \bHom_{ \overline{\calC} }(x,y)$ is nonempty.
We define a new (fibrant) simplicial category $\overline{\calD}$ so that $\overline{\calD}$ consists of a pair of objects $\{x', y'\}$, with
$$ \bHom_{\overline{\calD}}(x',x') \simeq \bHom_{\overline{\calD}}(y',y') \simeq \Delta^0$$
$$ \bHom_{\overline{\calD}}(x',y') \simeq K \quad \bHom_{\overline{\calD}}(y',x') \simeq \emptyset.$$
We let $\overline{F}, \overline{F}': \overline{\calD} \rightarrow \overline{\calC}$ be simplicial functors
such that $\overline{F}(x') = \overline{F}'(x') = x$, $\overline{F}(y') = \overline{F}'(y')=y$, where
$\overline{F}$ induces the identity map from $\bHom_{\overline{\calD}}(x',y') = K = \bHom_{\overline{\calC}}(x,y)$ to itself, while $\overline{F}'$ induces a constant map from $K$ to itself.
Then $\overline{F}$ and $\overline{F}'$ induce functors $F$ and $F'$ from $\sNerve( \overline{\calD} )$ to $\calC$. It follows from assumption $(2)$ that the functors $F$ and $F'$ are equivalent, which implies that the identity map from $K$ to itself is homotopic to a constant map; this proves that $K$ is contractible. 
\end{proof}

\begin{corollary}\label{bigegg}
Let $\sigma: \Cat_{\infty} \rightarrow \Cat_{\infty}$ be an equivalence of $\infty$-categories, and let
$\calC$ be an $\infty$-category $($which we regard as an object of $\Cat_{\infty}${}$)$. Then $\calC$ is equivalent to the nerve of a partially ordered set if and only if $\sigma(\calC)$ is equivalent to the nerve of a partially ordered set.
\end{corollary}

\begin{lemma}\label{calcul}
Let $\sigma, \sigma' \in \{ \id_{\cDelta}, r \} \subseteq \Fun( \cDelta, \cDelta)$, where
$r$ denotes the reversal functor from $\cDelta$ to itself. Then
$$ \Hom_{ \Fun(\cDelta, \cDelta)}(\sigma, \sigma') = \begin{cases} \emptyset & \text{if } \sigma= \sigma' \\
\{ \id \} & \text{if } \sigma = \sigma'. \end{cases}$$
\end{lemma}

\begin{proof}
Note that $\sigma$ and $\sigma'$ are both the identity at the level of objects. 
Let $\alpha: \sigma \rightarrow \sigma'$ be a natural transformation. Then, for each $n \geq 0$,
$\alpha_{[n]}$ is a map from $[n]$ to itself. We claim that $\alpha_{[n]}$ is given by the formula
$$ \alpha_{[n]}(i) = \begin{cases} i & \text{if } \sigma = \sigma' \\
n-i & \text{if } \sigma \neq \sigma'.\end{cases}$$
To prove this, we observe that a choice of $i \in [n]$ determines a map $[0] \rightarrow [n]$, which allows us to reduce to the case $n=0$ (where the result is obvious) by functoriality.

It follows from the above argument that the natural transformation $\alpha$ is uniquely determined, if it exists. Moreover, $\alpha$ is a well-defined natural transformation if and only if each $\alpha_{[n]}$ is an order-preserving map from $[n]$ to itself; this is true if and only if $\sigma = \sigma'$.
\end{proof}

\begin{proposition}\label{cape}
Let $\calP$ denote the category of partially ordered sets, and let
$\sigma: \calP \rightarrow \calP$ be an equivalence of categories. Then
$\sigma$ is isomorphic either to the identity functor $\id_{\calP}$ or the functor
$r$ which carries every partially ordered set $X$ to the same set with the opposite ordering.
\end{proposition}

\begin{proof}
Since $\sigma$ is an equivalence of categories, it carries the final object $[0] \in \calP$ to itself (up to canonical isomorphism). It follows that for every
partially ordered set $X$, we have a canonical bijection of sets
$$ \eta_{X}: X \simeq \Hom_{ \calP}( [0], X) \simeq
\Hom_{\calP}( \sigma([0]), \sigma(X) ) \simeq \Hom_{\calP}( [0], \sigma(X)) \simeq \sigma(X).$$

We next claim that $\sigma([1])$ is isomorphic to $[1]$ as a partially ordered set.
Since $\eta_{[1]}$ is bijective, the partially ordered set $\sigma([1])$ has precisely two elements.
Thus $\sigma([1])$ is isomorphic either to $[1]$ or to a partially ordered set $\{x,y\}$ with two elements, neither larger than the other. In the second case, the set $\Hom_{\calP}( \sigma([1]), \sigma([1]))$
has four elements. This is impossible, since $\sigma$ is an equivalence of categories and
$\Hom_{\calP}( [1], [1])$ has only three elements. Let $\alpha: \sigma([1]) \rightarrow [1]$
be an isomorphism (automatically unique, since the ordered set $[1]$ has no automorphisms in $\calP$). 

The map $\alpha \circ \eta_{[1]}$ is a bijection from the set $[1]$ to itself. We will assume that
this map is the identity, and prove that $\sigma$ is isomorphic to the identity functor $\id_{\calP}$.
The same argument, applied to $\sigma \circ r$, will show that if $\alpha \circ \eta_{[1]}$ is not the identity, then $\sigma$ is isomorphic to $r$.

To prove that $\sigma$ is equivalent to the identity functor, it will suffice to show that for every
partially ordered set $X$, the map $\eta_{X}$ is an isomorphism of partially ordered sets.
In other words, we must show that both $\eta_{X}$ and $\eta_{X}^{-1}$ are maps of partially ordered sets. We will prove that $\eta_{X}$ is a map of partially ordered sets; the same argument, applied to 
an inverse to the equivalence $\sigma$, will show that $\eta^{-1}_{X}$ is a map of partially ordered sets.
Let $x,y \in X$ satisfy $x \leq y$; we wish to prove that $\eta_{X}(x) \leq \eta_{X}(y)$ in $\sigma(X)$.
The pair $(x,y)$ defines a map of partially ordered sets $[1] \rightarrow X$. By functoriality, we
may replace $X$ by $[1]$, and thereby reduce to the problem of proving that $\eta_{[1]}$ is a map of partially ordered sets. This follows from our assumption that $\alpha \circ \eta_{[1]}$ is the identity map.
\end{proof}

\begin{proof}[Proof of Theorem \ref{cabbi}]
Let $\calC$ be the full subcategory of $\Cat_{\infty}$ spanned by those $\infty$-categories
which are equivalent to the nerves of partially ordered sets, and let $\calC^{0}$ denote the full subcategory of $\calC$ spanned by the objects $\{ \Delta^n \}_{n \geq 0}$. 
Corollary \ref{bigegg} implies that every object $\sigma \in \calE$ restricts to an equivalence from $\calC$ to itself. According to Proposition \ref{cape}, $\sigma| \calC$ is equivalent either to the identity functor, or to the restriction $r|\calC$. In either case, we conclude that $\sigma$ also induces an equivalence from $\calC^{0}$ to itself.

Using Theorem \ref{upsquare} and Remark \ref{copo}, we deduce that the restriction functor
$\calE \rightarrow \Fun(\calC^{0}, \calC^{0})$ is fully faithful. In particular, any object $\sigma \in \calE$ is determined by the restriction $\sigma | \calC$, so that $\sigma$ is equivalent to either $\id$ or
$r$ by virtue of Proposition \ref{cape}. Since $\calC^{0}$ is equivalent to the nerve of the category
$\cDelta$, Lemma \ref{calcul} implies the existence of a fully faithful embedding from
$\calE$ to the nerve of the discrete category $\{ \id, r \}$. To complete the proof, it will suffice to show that this functor is essentially surjective. In other words, we must show that there exists a functor
$R: \Cat_{\infty} \rightarrow \Cat_{\infty}$ whose restriction to $\calC$ is equivalent to $r$. 

To carry out the details, it is convenient to replace $\Cat_{\infty}$ by an equivalent
$\infty$-category with a slightly more elaborate definition. Recall that $\Cat_{\infty}$ is defined to be the simplicial nerve of a simplicial category $\Cat_{\infty}^{\Delta}$, whose objects are $\infty$-categories, where $\bHom_{\Cat_{\infty}^{\Delta}}(X,Y)$ is the largest Kan complex contained in
$\Fun(X,Y)$. We would like to define $R$ to be induced by the functor $X \mapsto X^{op}$, but this
is not a simplicial functor from $\Cat_{\infty}^{\Delta}$ to itself; instead we have a canonical isomorphism $\bHom_{\Cat_{\infty}^{\Delta}}( X^{op}, Y^{op} )
\simeq \bHom_{\Cat_{\infty}^{\Delta}}(X,Y)^{op}.$
However, if we let $\Cat_{\infty}^{\top}$ denote the topological category obtained by geometrically realizing the morphism spaces in 
$\Cat_{\infty}^{\Delta}$, then $i$ induces an autoequivalence of $\Cat_{\infty}^{\top}$ as a topological category (via the natural homeomorphisms $| K | \simeq |K^{op}|$, which is defined for every simplicial set $K$). We now define $\Cat'_{\infty}$ to be the topological nerve of $\Cat_{\infty}^{\top}$ (see 
Definition \ref{topnerve}). Then $\Cat'_{\infty}$ is an $\infty$-category equipped with a canonical equivalence $\Cat_{\infty} \rightarrow \Cat'_{\infty}$, and the involution $i$ induces
an involution $I$ on $\Cat'_{\infty}$, which carries each object $\calD \in \Cat'_{\infty}$ to 
the opposite $\infty$-category $\calD^{op}$. We now define $R$ to be the composition
$$ \Cat_{\infty} \rightarrow \Cat'_{\infty} \stackrel{I}{\rightarrow} \Cat'_{\infty} \rightarrow \Cat_{\infty},$$
where the last map is a homotopy inverse to the equivalence $\Cat_{\infty} \rightarrow \Cat'_{\infty}$.
It is easy to see that $R$ has the desired properties (moreover, we note that for every
object $\calD \in \Cat_{\infty}$, the image $R \calD$ is canonically equivalent with the opposite $\infty$-category $\calD^{op}$).
\end{proof}
