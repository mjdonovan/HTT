% !TEX root = highertopoi.tex
 
\section{Constructions of $\infty$-Topoi}\label{topcomp}

\setcounter{theorem}{0}

According to Definition \ref{itoposdef}, an $\infty$-category $\calX$ is an $\infty$-topos if and only if $\calX$ arises as an (accessible) left exact localization of a presheaf $\infty$-category $\calP(\calC)$. To complete the analogy with classical topos theory, we would like to have some concrete description of the collection of left exact localizations of $\calP(\calC)$. In \S \ref{leloc}, we will study left exact localization functors in general, and single out a special class which we call {\it topological} localizations. In \S \ref{cough}, we will study topological localizations of $\calP(\calC)$, and show that they are in bijection with {\it Grothendieck topologies} on the $\infty$-category $\calC$, in exact analogy with classical topos theory. In particular, given a Grothendieck topology on $\calC$, one can define an $\infty$-topos $\Shv(\calC) \subseteq \calP(\calC)$ of {\it sheaves on $\calC$}. In \S \ref{surjsurj}, we will characterize $\Shv(\calC)$ by a universal mapping property. 
Unfortunately, not every $\infty$-topos $\calX$ can be obtained as topological localization of an $\infty$-category of presheaves. Nevertheless, in \S \ref{cantopp} we will construct $\infty$-categories of sheaves which closely approximate $\calX$, using the formalism of {\it canonical topologies}. These ideas will be applied in \S \ref{chap6sec3}, to obtain a classification theorem for $n$-topoi.

\subsection{Left Exact Localizations}\label{leloc}

Let $\calX$ be an $\infty$-category. Up to equivalence, a localization $L: \calX \rightarrow \calY$
is determined by the collection $S$ of all morphisms $f: X \rightarrow Y$ in $\calX$ such that
$Lf$ is an equivalence in $\calY$ (Proposition \ref{localloc}). Our first result provides a useful criterion for testing the left-exactness of $L$.

\begin{proposition}\label{charleftloc}\index{gen}{localization!left exact}
Let $L: \calX \rightarrow \calY$ be a localization of $\infty$-categories. 
Suppose that $\calX$ admits finite limits. The following conditions are equivalent:
\begin{itemize}
\item[$(1)$] The functor $L$ is left exact.
\item[$(2)$] For every pullback diagram
$$ \xymatrix{ X' \ar[r] \ar[d]^{f'} & X \ar[d]^{f} \\
Y' \ar[r] & Y }$$ in $\calX$ such that $Lf$ is an equivalence in $\calY$,
$Lf'$ is also an equivalence in $\calY$.
\end{itemize}
\end{proposition}

\begin{proof}
It is clear that $(1)$ implies $(2)$. Suppose that $(2)$ is satisfied. We wish to show that $L$ is
left exact. Let $S$ be the collection of morphisms $f$ in $\calX$ such that $Lf$ is an equivalence.
Without loss of generality, we may identify $\calY$ with the full subcategory of $\calX$ spanned by the $S$-local objects. Since the final object $1 \in
\calX$ is obviously $S$-local, we have $L1 \simeq 1$. Thus it will
suffice to show that $L$ commutes with pullbacks. We observe that given
any diagram $X \rightarrow Y \leftarrow Z$, the pullback $LX  \times_{LY} LZ$ is a limit
of $S$-local objects of $\calX$, and therefore $S$-local. To complete the proof, it will suffice
to show that the natural map $f: X \times_Y Z \rightarrow LX \times_{LY} LZ$ belongs to $S$.
We can write $f$ as a composition
of maps $$X \times_Y Z \rightarrow X \times_{LY} Z \rightarrow LX
\times_{LY} Z \rightarrow LX \times_{LY} LZ.$$ The last two maps
are obtained from $X \rightarrow LX$ and $Z \rightarrow LZ$ by
base change. Assumption $(2)$ implies that they belong to $S$. Thus, it will
suffice to show that $f': X \times_Y Z \rightarrow X \times_{LY}
Z$ belongs to $S$. This map is a pullback of the diagonal $f'': Y \rightarrow Y \times_{LY} Y$, so it will suffice to prove that $f'' \in S$.
Projection to the first factor gives a left homotopy inverse $g: Y \times_{LY} Y
\rightarrow Y$ of $f''$, so it suffices to prove that $g \in S$. 
But $g$ is a base change of the morphism $Y \rightarrow LY$.
\end{proof}

\begin{proposition}\label{swimmer}
Let $\calX$ be a presentable $\infty$-category in which colimits are universal. Let $S$ be a class of morphisms in $\calX$, and let $\overline{S}$ be the strongly saturated class of morphisms generated by $S$. Suppose
that $S$ has the following property: for every pullback diagram
$$ \xymatrix{ X' \ar[r] \ar[d]^{f'} & X \ar[d]^{f} \\
Y' \ar[r] & Y }$$ in $\calX$, if $f \in S$, then $f' \in \overline{S}$. Then $\overline{S}$ is stable under pullbacks.
\end{proposition}

\begin{proof}
Let $S'$ be the set of all morphisms $f$ in $\calX$ with the property that for any pullback diagram
$$ \xymatrix{ X' \ar[r] \ar[d]^{f'} & X \ar[d]^{f} \\
Y' \ar[r] & Y, }$$
the morphism $f'$ belongs to $\overline{S}$. By assumption, $S \subseteq S'$. Using the fact
that colimits are universal, we deduce that $S'$ is strongly saturated. Consequently
$\overline{S} \subseteq S'$, as desired.
\end{proof}

\begin{corollary}\label{sweetums}
Let $\calX$ be a presentable $\infty$-category in which colimits are universal, let $S$ be a $($small$)$ set of morphisms in $\calX$, and let $\overline{S}$ denote the smallest strongly saturated class of morphisms which contains $S$ and is stable under pullbacks. Then $\overline{S}$ is generated $($as a strongly saturated class of morphisms$)$ by a $($small$)$ set.
\end{corollary}

\begin{proof}
Choose a (small) set $U$ of objects of $\calX$ which generates $\calX$ under colimits. Enlarging $U$ if necessary, we may suppose that $U$ contains the codomain of every morphism belonging to $S$.
Let $S'$ be the set of all morphisms $f'$ which fit into a pullback diagram
$$ \xymatrix{ X' \ar[r] \ar[d]^{f'} & X \ar[d]^{f} \\
Y' \ar[r] & Y }$$
where $f \in S$ and $Y' \in U$, and let $\overline{S}'$ denote the strongly saturated class of morphisms generated by $S'$. To complete the proof it will suffice to show that $\overline{S}' = \overline{S}$.
The inclusions $S \subseteq S' \subseteq \overline{S}' \subseteq \overline{S}$ are obvious.
To show that $\overline{S} \subseteq \overline{S}'$, it will suffice to show that $\overline{S}'$ is stable under pullbacks. In view of Proposition \ref{swimmer}, it will suffice to show that for every pullback diagram
$$ \xymatrix{ X'' \ar[r] \ar[d]^{f''} & X' \ar[d]^{f'} \\
Y'' \ar[r] & Y' }$$
such that $f' \in S'$, the morphism $f''$ belongs to $\overline{S}'$. Using our assumption that colimits in $\calX$ are universal and that $U$ generates $\calX$ under colimits, we can reduce to the case where
$Y'' \in U$. In this case, $f'' \in S'$ by construction.
\end{proof}

Recall that a morphism $f: Y \rightarrow Z$ in an $\infty$-category $\calX$ is a {\it monomorphism}\index{gen}{monomorphism}
if it is a $(-1)$-truncated object of the $\infty$-category $\calX_{/Z}$. Equivalently, $f$ is a monomorphism if for every object $X \in \calX$, the induced map
$$ \bHom_{\calX}(X,Y) \rightarrow \bHom_{\calX}(X,Z)$$
exhibits $\bHom_{\calX}(X,Y) \in \calH$ as a summand of $\bHom_{\calX}(X,Z)$ in the homotopy category $\calH$. If we fix $Z \in \calX$, then the collection of equivalence classes of monomorphisms $Y \rightarrow Z$ are partially ordered under inclusion. We will denote this partially ordered collection by $\Sub(Z)$.

\begin{proposition}\label{subobjset}\index{not}{SubX@$\Sub(X)$}
Let $\calX$ be a presentable $\infty$-category, and let $X$ be an object of $\calX$. Then
$\Sub(X)$ is a $($small$)$ partially ordered set.
\end{proposition}

\begin{proof}
By definition, the partially ordered set $\Sub(X)$ is characterized by the existence of
an equivalence 
$$ \tau_{ \leq -1} \calX_{/X} \rightarrow \Nerve( \Sub(X)) .$$ 
Propositions \ref{slicstab} and \ref{maketrunc} imply that $\Nerve(\Sub(X))$ is presentable.
Consequently, there exists a small subset $S \subseteq \Sub(X)$ which generates $\Nerve (\Sub(X))$ under colimits. It follows that every element of $\Sub(X)$ can be written as the supremum of a subset of $S$, so that $\Sub(X)$ is also small.
\end{proof}

\begin{definition}\label{deftoploc}\index{gen}{localization!topological}\index{gen}{topological!localization}\index{gen}{topological!class of morphisms}
Let $\calX$ be a presentable $\infty$-category, and let $\overline{S}$ be a strongly saturated class of morphisms of $\calX$. We will say that $\overline{S}$ is {\it topological} if the following conditions are satisfied:
\begin{itemize}
\item[$(1)$] There exists $S \subseteq \overline{S}$ consisting of {\em monomorphisms} such that
$S$ generates $\overline{S}$ as a strongly saturated class of morphisms.
\item[$(2)$] Given a pullback diagram
$$ \xymatrix{ X' \ar[r] \ar[d]^{f'} & X \ar[d]^{f} \\
Y' \ar[r] & Y }$$
in $\calX$ such that $f$ belongs to $\overline{S}$, the morphism $f'$ also belongs to $\overline{S}$.
\end{itemize}

We will say that a localization $L: \calX \rightarrow \calY$ is {\it topological} if the collection $\overline{S}$ of all morphisms $f: X \rightarrow Y$ in $\calX$ such that $Lf$ is an equivalence
is topological.
\end{definition}

\begin{proposition}\label{toplocsmall}
Let $\calX$ be a presentable $\infty$-category in which colimits are universal, and let
$\overline{S}$ be a strongly saturated class of morphisms of $\calX$ which is topological.
Then there exists a $($small$)$ subset $S_0 \subseteq \overline{S}$ which consists of monomorphisms and generates $\overline{S}$ as a strongly saturated class of morphisms.
\end{proposition}

\begin{proof}
For every object $U \in \calX$, let $\Sub'(U) \subseteq \Sub(U)$ denote the collection of equivalence classes of monomorphisms $U' \rightarrow U$ which belong to $\overline{S}$.
Choose a small collection of objects $\{ U_{\alpha} \}_{\alpha \in A}$ which generates $\calX$ under colimits. For each $\alpha \in A$ and each element
$\widetilde{\alpha} \in \Sub'(U_{\alpha})$, choose a representative monomorphism $f_{\widetilde{\alpha}}: V_{\widetilde{\alpha}} \rightarrow U_{\alpha}$ which belongs to $\overline{S}$. Let 
$$S_0 = \{ f_{\widetilde{\alpha}} | \alpha \in A, \widetilde{\alpha} \in \Sub'(U_{\alpha}) \}.$$
It follows from Proposition \ref{subobjset} that $S_0$ is a (small) set. Let $\overline{S}_0$ denote the strongly saturated class of morphisms generated by $S_0$. We will show that $\overline{S}_0 = \overline{S}$.

Let $\calX^{0}$ be the full subcategory of $\calX$ spanned by objects $U$ with the following property: if $f: V \rightarrow U$ is a monomorphism and $f \in \overline{S}$, then $f \in \overline{S_0}$. By construction, for each $\alpha \in A$, $U_{\alpha} \in \calX^{0}$. 
Since colimits in $\calX$ are universal, it is easy to see that $\calX^{0}$ is stable under colimits in $\calX$. It follows that $\calX^{0} = \calX$, so that every monomorphism which belongs to $\overline{S}$ also belongs to $\overline{S_0}$. Since $\overline{S}$ is generated by monomorphisms, we conclude that $\overline{S} = \overline{S_0}$, as desired.
\end{proof}

\begin{corollary}\label{topaccess}
Let $\calX$ be a presentable $\infty$-category. Every topological localization $L: \calX \rightarrow \calY$ is accessible and left exact. 
\end{corollary}

\subsection{Grothendieck Topologies and Sheaves in Higher Category Theory}\label{cough}

Every ordinary topos is equivalent to the category of sheaves on some Grothendieck site. This can be deduced from the following pair of statements:
\begin{itemize}
\item[$(i)$] Every topos is equivalent to a left exact localization of the some presheaf category
$\Set^{\calC^{op}}$.
\item[$(ii)$] There is a bijective correspondence between left exact localizations of
$\Set^{\calC^{op}}$ and Grothendieck topologies on $\calC$.
\end{itemize}
In \S \ref{chap6sec1}, we proved the $\infty$-categorical analogue of assertion $(i)$. Unfortunately, $(ii)$ is not quite true in the $\infty$-categorical setting. In this section, we will establish a slightly weaker statement: for every $\infty$-category $\calC$, there is a bijective correspondence between Grothendieck topologies on $\calC$ and {\em topological} localizations of $\calP(\calC)$ (Proposition \ref{toprole}).
Our first step is to introduce the $\infty$-categorical analogue of a Grothendieck site. The following definition is taken from \cite{toen}:

\begin{definition}\label{grtop}\index{gen}{sieve}\index{gen}{sieve!covering}
Let $\calC$ be a $\infty$-category. A {\it sieve} on $\calC$ is a full subcategory of $\calC^{(0)} \subseteq \calC$ having the property that if $f: C \rightarrow D$ is a morphism in $\calC$, and $D$ belongs to $\calC^{(0)}$, then $C$ also belongs to $\calC^{(0)}$.\index{not}{calC0@$\calC^{(0)}$}

Observe that if $f: \calC \rightarrow \calD$ is a functor between $\infty$-categories and
$\calD^{(0)} \subseteq \calD$ is a sieve on $\calD$, then $f^{-1} \calD^{(0)} = \calD^{(0)} \times_{\calD} \calC$ is a sieve on $\calC$.
Moreover, if $f$ is an equivalence, then $f^{-1}$ induces a bijection between sieves on
$\calD$ and sieves on $\calC$.

If $C \in \calC$ is an object, then a {\it sieve on $C$} is a sieve on the $\infty$-category $\calC_{/C}$. Given a morphism $f: D \rightarrow C$ and a sieve $\calC^{(0)}_{/C}$ on $C$, we let $f^{\ast} \calC^{(0)}_{/C}$ denote the unique sieve on $D$ such that $f^{\ast} \calC^{(0)}_{/C}
\subseteq \calC_{/D}$ and
$\calC^{(0)}_{/C}$ determine same sieve on $\calC_{/f}$.

A {\it Grothendieck topology}\index{gen}{Grothendieck topology} on an $\infty$-category $\calC$ consists of a specification, for each
object $C$ of $\calC$, of a collection of sieves on $C$, which we will refer to as {\it covering sieves}. The collections of covering sieves are required to possess the following properties:

\begin{itemize}
\item[$(1)$] If $C$ is an object of $\calC$, then the sieve $\calC_{/C} \subseteq \calC_{/C}$ on $C$ is a covering sieve.

\item[$(2)$] If $f: C \rightarrow D$ is a morphism in $\calC$, and $\calC^{(0)}_{/D}$ is a covering sieve on $D$, then $f^{\ast} \calC^{(0)}_{/C}$ is a covering sieve on $C$.

\item[$(3)$] Let $C$ be an object of $\calC$, $\calC^{(0)}_{/C}$ a covering sieve on $C$, and $\calC^{(1)}_{/C}$ an arbitrary sieve on $C$. Suppose that, for each $f: D \rightarrow C$ belonging to the sieve $\calC^{(0)}_{/C}$, the pullback
$f^{\ast} \calC^{(1)}_{/C}$ is a covering sieve on $D$. Then $\calC^{(1)}_{/C}$ is a covering sieve on $C$.
\end{itemize}
\end{definition}

\begin{example}\label{trivvtop}
Any $\infty$-category $\calC$ may be equipped with the {\it trivial topology}, in which a sieve
$\calC^{(0)}_{/C}$ on an object $C$ of $\calC$ is covering if and only if $\calC^{(0)}_{/C} = \calC_{/C}$.
\end{example}

\begin{remark}
In the case where $\calC$ is (the nerve of) an ordinary category, the definition given above reduces to the usual notion of a Grothendieck topology on $\calC$. Even in the general case, a Grothendieck
topology on $\calC$ is just a Grothendieck topology on the homotopy category $h \calC$.
This is not completely obvious, since for an object $C$ of $\calC$, the functor
$$ \eta: \h{(\calC_{/C})} \rightarrow (\h{\calC})_{/C}$$ is usually not an equivalence of categories.
A morphism from on the left hand side corresponds to a commutative triangle
$$ \xymatrix{ D \ar[rr] \ar[dr] & & D' \ar[dl] \\
& C & }$$ given by a {\it specified} $2$-simplex $\sigma: \Delta^2 \rightarrow \calC$ (taken modulo homotopy), while on the right hand side one requires only that the above diagram commutes up to homotopy: this amounts to requiring the existence of $\sigma$, but $\sigma$ itself is not taken as part of the data.

Although $\eta$ need not be an equivalence of categories, $\eta^{\ast}$ does induce
a bijection from the set of sieves on $(\h{\calC})_{/C}$ to the set of sieves on $\h{(\calC_{/C})}$: for this, it suffices to observe that $\eta$ induces surjective maps
$$ \Hom_{\h(\calC_{/C})} (D,D') \rightarrow \Hom_{ (\h{\calC})_{/C} }(D,D')$$
on morphism sets, which is obvious from the description given above.
\end{remark}

The main objective of this section is to prove that for any (small) $\infty$-category $\calC$, there is a bijective correspondence between Grothendieck topologies on $\calC$ and (equivalence classes of) topological localizations of $\calP(\calC)$. We begin by establishing a correspondence between sieves on $\calC$ and $(-1)$-truncated objects of $\calP(\calC)$. 
For each object $U \in \calP(\calC)$, let $\calC^{(0)}(U) \subseteq \calC$ be the full subcategory spanned by those objects $C \in \calC$ such that $U(C) \neq \emptyset$. It is easy to see that
$\calC^{(0)}(U)$ is a sieve on $\calC$. Conversely, given a sieve $\calC^{(0)} \subseteq \calC$, there is a unique map $\calC \rightarrow \Delta^1$ such that $\calC^{(0)}$ is the preimage of
$\{0\}$. This construction determines a bijection between sieves on $\calC$ and functors
$f: \calC \rightarrow \Delta^1$, and we may identify $\Delta^1$ with the full subcategory of
$\SSet^{op}$ spanned by the objects $\emptyset, \Delta^0 \in \Kan$. Since every $(-1)$-truncated Kan complex is equivalent to either $\emptyset$ or $\Delta^0$, we conclude:

\begin{lemma}\label{siffer}
For every small $\infty$-category $\calC$, the construction
$U \mapsto \calC^{(0)}(U)$ determines a bijection between the set of equivalence classes of
$(-1)$-truncated objects of $\calP(\calC)$ and the set of all sieves on $\calC$.
\end{lemma}

We now introduce a relative version of the above construction. Let $\calC$ be a small $\infty$-category as above, and let $j: \calC \rightarrow \calP(\calC)$ be
the Yoneda embedding. Let $C \in \calC$ be an object, and let $i: U \rightarrow j(C)$ be a monomorphism in $\calP(\calC)$. Let $\calC_{/C}(U)$ denote the full subcategory of $\calC$
spanned by those objects $f: D \rightarrow C$ of $\calC_{/C}$ such that there exists a commutative triangle
$$ \xymatrix{ j(D) \ar[rr]^{j(f)} \ar[dr] & & j(C) \\
& U \ar[ur]^{i} & }$$
It is easy to see that $\calC_{/C}(U)$ is a sieve on $C$. Moreover, it clear that if $i: U \rightarrow j(C)$ and $i': U' \rightarrow j(C)$ are equivalent subobjects of $j(C)$, then $\calC_{/C}(U) = \calC_{/C}(U')$. 

\begin{proposition}\label{surry}
Let $\calC$ be a small $\infty$-category containing an object $C$, and let $j: \calC \rightarrow \calP(\calC)$ be the Yoneda embedding. The construction described above yields a bijection
$$ ( i: U \rightarrow j(C) ) \mapsto \calC_{/C}(U)$$
from $\Sub(j(C))$ to the set of all sieves on $C$.
\end{proposition}

\begin{proof}
Use Corollary \ref{swapKK} to reduce to Lemma \ref{siffer}.
\end{proof}

\begin{definition}\label{defsheaff}\index{gen}{sheaf}\index{not}{ShvC@$\Shv(\calC)$}
Let $\calC$ be a $($small$)$ $\infty$-category equipped with a Grothendieck topology.
Let $S$ be the collection of all monomorphisms $U \rightarrow j(C)$ which correspond
to covering sieves $\calC^{(0)}_{/C} \subseteq \calC_{/C}$. An object
$\calF \in \calP(\calC)$ is a {\it sheaf} if it is $S$-local. We let $\Shv(\calC)$ denote the full subcategory of $\calP(\calC)$ spanned by $S$-local objects. 
\end{definition}

\begin{lemma}\label{stokeworth}
Let $\calC$ be a $($small$)$ $\infty$-category equipped with a Grothendieck topology. Then
$\Shv(\calC)$ is a topological localization of $\calP(\calC)$. In particular,
$\Shv(\calC)$ is an $\infty$-topos. 
\end{lemma}

\begin{proof}
By definition, $\Shv(\calC) = S^{-1} \calP(\calC)$, where $S$ is the class of all monomorphisms 
$i: U \rightarrow j(C)$ which correspond to covering sieves on $C \in \calC$. Let $\overline{S}$ be the strongly saturated class of morphisms generated by $S$; we wish to show that $\overline{S}$ is stable under pullback.

Let $S'$ denote the collection of all morphisms $f: X \rightarrow Y$ such that for any
pullback diagram $\sigma: \Delta^1 \times \Delta^1 \rightarrow \calP(\calC)$ depicted as follows:
$$ \xymatrix{ X' \ar[d]^{f'} \ar[r] & X \ar[d]^{f} \\
Y' \ar[r]^{g} & Y, }$$
the morphism $f'$ belongs to $\overline{S}$. Since colimits in $\calP(\calC)$ are universal, it is easy to prove that $S'$ is strongly saturated. We wish to prove that $\overline{S} \subseteq S'$. Since $\overline{S}$ is the smallest saturated class containing $S$, it will suffice to prove that $S \subseteq S'$. We may therefore suppose that $Y = j(C)$ in the diagram above, and that $f: X \rightarrow j(C)$ is the monomorphism corresponding to a covering sieve
$\calC^{(0)}_{/C}$ on $C$. 

Since $\calP(\calC)_{/j(C)} \simeq \calP(\calC_{/C})$ is generated under colimits by the Yoneda embedding, there exists a diagram $p: K \rightarrow \calC_{/C}$ such that the composite
map $j \circ p: K \rightarrow \calP(\calC)_{/j(C)}$ has $g: Y' \rightarrow j(C)$ as a colimit.
Because colimits in $\calP(\calC)$ are universal, we can extend $j \circ p$ to a 
diagram $P: K \rightarrow (\calP(\calC)^{\Delta^1})_{/f}$ which carries each vertex 
$k \in K$ to a pullback diagram,
$$ \xymatrix{ X_k \ar[d]^{f_k} \ar[r] & X \ar[d] \\
j(D_k) \ar[r]^{j(g_k)} & j(C) }$$
such that $\sigma$ is a colimit of $P$. Each $f_k$ is a monomorphism associated to the
covering sieve $g_k^{\ast} \calC^{(0)}_{/C}$, and therefore belongs to $S \subseteq \overline{S}$. It follows that $f'$ is a colimit in $\calP(\calC)^{\Delta^1}$ of morphisms belonging to $\overline{S}$, and therefore itself belongs to $\overline{S}$.
\end{proof}

The next lemma ensures us that we can recover a Grothendieck topology on $\calC$ from
its $\infty$-category of sheaves $\Shv(\calC) \subseteq \calP(\calC)$.

\begin{lemma}\label{recloose}
Let $\calC$ be a $($small $)$ $\infty$-category equipped with a Grothendieck topology, and
let $L: \calP(\calC) \rightarrow \Shv(\calC)$ denote a left adjoint to the inclusion. Let
$j: \calC \rightarrow \calP(\calC)$ denote the Yoneda embedding, and let
$i: U \rightarrow j(C)$ be a monomorphism corresponding to a sieve $\calC^{(0)}_{/C}$
on $C$. Then $Li$ is an equivalence if and only if
$\calC^{(0)}_{/C}$ is a covering sieve.
\end{lemma}

\begin{proof}
It is clear that if $\calC^{(0)}_{/C}$ is a covering sieve, then $Li$ is an equivalence. Conversely, suppose that $Li$ is an equivalence. Then $\tau_{\leq 0}(L i)$ is an equivalence. In view of
Proposition \ref{compattrunc}, we can identify $\tau_{\leq 0}(Li)$ with $L (\tau_{\leq 0} i)$.
The morphism $\tau_{\leq 0} i$ can be identified with a monomorphism 
$\eta: \calF \subseteq \Hom_{\h{\calC}}( \bigdot, C)$ in the ordinary category of presheaves of sets on $\h{\calC}$, where $$\calF(D) = \{ f \in \Hom_{\h{\calC}}(D,C): f \in \calC^{(0)}_{/C} \}.$$ 
If $\eta$ becomes an equivalence after sheafification, then the identity map
$\id_{C}: C \rightarrow C$ belongs to $\calF(C)$ locally; in other words, 
there exists a collection of morphisms $\{ f_{\alpha}: C_{\alpha} \rightarrow C \}$ which
generated a covering sieve on $C$ such that each $f_{\alpha}$ belongs to
$\calF(C_{\alpha})$, and therefore to $\calC^{(0)}_{/C}$. It follows that
$\calC^{(0)}_{/C}$ contains a covering sieve on $C$ and is therefore itself covering.
\end{proof}

We may summarize the results of this section as follows:

\begin{proposition}\label{toprole}
Let $\calC$ be a small $\infty$-category. There is a bijective correspondence
between Grothendieck topologies on $\calC$ and $($equivalence classes of$)$ topological localizations of $\calP(\calC)$.
\end{proposition}

\begin{proof}
According to Lemma \ref{stokeworth}, every Grothendieck topology on $\calC$ determines
a topological localization $\Shv(\calC) \subseteq \calP(\calC)$. Lemma \ref{recloose} shows that two Grothendieck topologies which determine the same $\infty$-categories of sheaves must coincide. To complete the proof, it will suffice to show that {\em every} topological localization of
$\calP(\calC)$ arises in this way. Let $\overline{S}$ be a strongly saturated collection of morphisms in $\calP(\calC)$, and suppose that $\overline{S}$ is topological. Let $S \subseteq \overline{S}$
be the collection of all monomorphisms $U \rightarrow j(C)$ which belong to $\overline{S}$, where
$j: \calC \rightarrow \calP(\calC)$ denotes the Yoneda embedding. Since the objects $\{ j(C) \}_{C \in \calC}$ generate $\calP(\calC)$ under colimits, and colimits in $\calP(\calC)$ are universal, 
we conclude that every monomorphism in $\overline{S}$ is a colimit of elements of $S$. 
Since $\overline{S}$ is generated by monomorphisms, we conclude that $\overline{S}$ is generated by $S$. 

Let us say that a sieve $\calC^{(0)}_{/C} \subseteq \calC_{/C}$ on an object $C \in \calC$
is {\it covering} if the corresponding monomorphism $U \rightarrow j(C)$ belongs to $S$.
We will show that the collection of covering sieves determines a Grothendieck topology on $\calC$. Granting this, we observe that $\overline{S}^{-1} \calP(\calC)$ is the $\infty$-category $\Shv(\calC) \subseteq \calP(\calC)$ of sheaves with respect to this Grothendieck topology, which will complete the proof.

We now verify the axioms $(1)$ through $(3)$ of Definition \ref{grtop}:

\begin{itemize}
\item[$(1)$] Every sieve of the form $\calC_{/C} \subseteq \calC_{/C}$ is covering, since every identity map $\id_{j(C)}: j(C) \rightarrow j(C)$ belongs to $S$.
\item[$(2)$] Let $f: C \rightarrow D$ be a morphism in $\calC$, and let
$\calC^{(0)}_{/D} \subseteq \calC_{/D}$ be a covering sieve, corresponding to a monomorphism
$i: U \rightarrow j(D)$ which belongs to $S$. Then $f^{\ast} \calC^{(0)}_{/C} \subseteq \calC_{/C}$
corresponds to a monomorphism $u: U' \rightarrow j(C)$ which is a pullback of $i$ along
$j(f)$, and therefore belongs to $S$ (since $\overline{S}$ is stable under pullbacks).
\item[$(3)$] Let $C$ be an object of $\calC$, $\calC^{(0)}_{/C}$ a covering sieve on $C$ corresponding to a monomorphism $i: U \rightarrow j(C)$ which belongs to $S$, 
and $\calC^{(1)}_{/C}$ an arbitrary sieve on $C$ corresponding to a monomorphism
$v: U' \rightarrow j(C)$. Suppose that, for each $f: D \rightarrow C$ belonging to the sieve $\calC^{(0)}_{/C}$, the pullback $f^{\ast} \calC^{(1)}_{/C}$ is a covering sieve on $D$.
Since $j': \calC_{/C} \rightarrow \calP(\calC)_{/j(C)}$ is a fully faithful embedding which generates $\calP(\calC)_{/j(C)}$ under colimits (see the proof of Corollary \ref{swapKK}), we conclude
there is a diagram $K \rightarrow \calC_{/C}$ such that $j' \circ K$ has colimit $i'$.
Since colimits in $\calP(\calC)$ are universal, we conclude that the map
$v': U \times_{j(C)} U' \rightarrow U$ is a colimit of morphisms of the form
$j(D) \times_{j(C)} U' \rightarrow j(D)$, which belong to $\overline{S}$ by assumption.
Since $\overline{S}$ is stable under colimits, we conclude that $i''$ belongs to $\overline{S}$.
We now have a pullback diagram
$$ \xymatrix{ U \times_{j(C)} U' \ar[r]^{v'} \ar[d]^{u'} & U \ar[d]^{u} \\
U' \ar[r]^{v} & j(C). }$$
By assumption, $u \in S$. 
Thus $v \circ u' \sim u \circ v' \in \overline{S}$. Since $u'$ is a pullback of
$u$, we conclude that $u' \in \overline{S}$, so that $v \in \overline{S}$.
This implies that $\calC^{(1)}_{/C} \subseteq \calC_{/C}$ is a covering sieve, as we wished to prove.
\end{itemize}
\end{proof}

For later use, we record the following characterization of initial objects in $\infty$-categories of sheaves:

\begin{proposition}\label{suture}\index{gen}{initial object!in an $\infty$-category of sheaves}
Let $\calC$ be a small $\infty$-category equipped with a Grothendieck topology, and let
$\calC' \subseteq \calC$ denote the full subcategory spanned by those objects
$C \in \calC$ such that $\emptyset \subseteq \calC_{/C}$ is a covering sieve on $C$.
An object $\calF \in \Shv(\calC)$ is initial if and only if it satisfies the following conditions:
\begin{itemize}
\item[$(1)$] If $C \in \calC'$, then $\calF(C)$ is contractible.
\item[$(2)$] If $C \notin \calC'$, then $\calF(C)$ is empty.
\end{itemize}
\end{proposition}

\begin{proof}
Let $L: \calP(\calC) \rightarrow \Shv(\calC)$ be a left adjoint to the inclusion, and
let $\emptyset$ be an initial object of $\calP(\calC)$. Then $L \emptyset$ is an initial object of $\Shv(\calC)$. Since $L$ is left exact, it preserves $(-1)$-truncated objects, as does the inclusion
$\Shv(\calC) \subseteq \calP(\calC)$. Thus $L \emptyset$ is $(-1)$-truncated, and corresponds to some sieve $\calC^{(0)} \subseteq \calC$ (Lemma \ref{siffer}). As we saw in the proof of Lemma \ref{recloose}, a sieve $\calC^{(0)}$ classifies an object of
$\Shv(\calC)$ is and only if $\calC^{(0)}$ is saturated in the following sense:
if $C \in \calC$ and the induced sieve $\calC^{(0)} \times_{\calC} \calC_{/C}$ is covering,
then $C \in \calC^{(0)}$. An initial object
of $\Shv(\calC)$ is an initial object of $\tau_{\leq -1} \Shv(\calC)$, and must therefore
correspond to the {\em smallest} saturated sieve on $\calC$. An easy argument shows that this sieve is $\calC'$, and that $\calF \in \calP(\calC)$ is a $(-1)$-truncated object classified by
$\calC'$ if and only if conditions $(1)$ and $(2)$ are satisfied.
\end{proof}

\subsection{Effective Epimorphisms}\label{surjsurj}

In classical topos theory, the assumption that every equivalence relation is effective
leads to a bijective correspondence between equivalence relations on an object $X$
and {\em effective epimorphisms} $X \rightarrow Y$. The purpose of this section is to
generalize the notion of an effective epimorphism to the $\infty$-categorical setting.

Our primary interest is studying the class of effective epimorphisms in an $\infty$-topos $\calX$.
However, we will later need to employ the same ideas when $\calX$ is an $n$-topos, for $n < \infty$. It is therefore convenient to work in a slightly more general setting.

\begin{definition}\index{gen}{semitopos}
An $\infty$-category $\calX$ is a {\it semitopos} if it satisfies the following conditions:
\begin{itemize}
\item[$(1)$] The $\infty$-category $\calX$ is presentable.
\item[$(2)$] Colimits in $\calX$ are universal.
\item[$(3)$] For every morphism $f: U \rightarrow X$, the underlying groupoid
of the \Cech nerve $\mCech(f)$ is effective (see \S \ref{gengroup}). 
\end{itemize}
\end{definition}

\begin{remark}
Every $\infty$-topos is a semitopos; this follows immediately from Theorem \ref{mainchar}.
\end{remark}

\begin{remark}
If $\calX$ is a semitopos, then so is $\calX_{/X}$ for every object $X \in \calX$.
\end{remark}

\begin{proposition}\label{slurpme}
Let $\calX$ be a semitopos. Let $p: U \rightarrow X$ be a morphism
in $\calX$, let $U_{\bigdot}$ be the underlying simplicial object of the \Cech nerve $\mCech(p)$, let
$V \in \calX$ be a colimit of $U_{\bigdot}$. The induced diagram
$$ \xymatrix{ U \ar[rr] \ar[dr]^{p} & & V \ar[dl]^{p'} \\
& X & }$$
identifies $p'$ with a $(-1)$-truncation of $p$ in $\calX_{/X}$.
\end{proposition}

\begin{proof}
We first show that $V$ is $(-1)$-truncated. It suffices to show
that the diagonal map $V \rightarrow V \times_{X} V$ is an
equivalence. We may identify $V$ with $V \times_{V} V$.
Since colimits in $\calX$ are universal, it will suffice to prove
that for each $m,n \geq 0$, the natural map
$$ p_{n.m}: U_m \times_{V} U_n \rightarrow U_{m} \times_{X} U_{n}$$
is an equivalence. We next observe that each $p_{n,m}$ is a pullback of
$$ p_{0,0}: U \times_{V} U \rightarrow U \times_{X} U. $$
Because $U_{\bigdot}$ is an effective groupoid, both sides may be identified with $U_{1}$.

To complete the proof, it suffices to show that the natural map $\bHom_{\calX_{/X}}(p',q)
\rightarrow \bHom_{\calX_{/X}}(p,q)$ is an equivalence whenever $q: E \rightarrow X$
is a monomorphism. Note that both sides are either empty or contractible. We must show that
if $\bHom_{\calX_{/X}}(p,q)$ is nonempty, then so is $\bHom_{\calX_{/X}}(p',q)$. 
We observe that the map $\calX_{/q} \rightarrow \calX_{/X}$ is fully faithful, 
and that its essential image is a sieve on $\calX_{/X}$. If that sieve contains $p$, then
it contains the entire groupoid $U_{\bigdot}$ (viewed as a groupoid in $\calX_{/X}$).
We conclude that there exists a groupoid object $W_{\bigdot}: \Nerve(\cDelta)^{op} \rightarrow \calX_{/q}$ lifting $U_{\bigdot}$. Let $\widetilde{V} \in \calX_{/q}$ be a colimit of
$V_{\bigdot}$. According to Proposition \ref{needed17}, the image of $\widetilde{V}$ in $\calX_{/X}$ can be identified with the map $p': V \rightarrow X$. The existence of $\widetilde{V}$ 
proves that $\bHom_{\calX{/X}}(p',q)$ is nonempty, as desired.
\end{proof}

\begin{corollary}\label{subobj}
Let $\calX$ be a semitopos, and let $f: U \rightarrow X$
be a morphism in $\calX$. The following conditions are equivalent:
\begin{itemize}
\item[$(1)$] If we regard $f$ as an object of the $\infty$-category $\calX_{/X}$, then
$\tau_{\leq -1}(f)$ is a final object of $\calX_{/X}$.
\item[$(2)$] The \Cech nerve $\mCech(f)$ is a simplicial resolution of $X$.
\end{itemize}
\end{corollary}

We will say that a morphism $f: U \rightarrow X$ in an semitopos $\calX$ is
an {\it effective epimorphism} if it satisfies the equivalent conditions of Corollary \ref{subobj}.\index{gen}{effective epimorphism} There is a one-to-one correspondence between effective epimorphisms and effective groupoids. More precisely, let $\Res_{\Eff}(\calX)$ denote the full subcategory of the $\infty$-category
$\calX_{\Delta_{+}}$ spanned by those augmented simplicial objects $U_{\bigdot}$ which
are both \Cech nerves and simplicial resolutions. The restriction functors
$$ \xymatrix{ & \calX_{\Delta_{+}} \ar[dr] \ar[dl] & \\
\calX_{\Delta} & & \Fun(\Delta^1, \calX) }$$
induce equivalences of $\infty$-categories from $\Res_{\Eff}(\calX)$ to the full subcategory
of $\calX_{\Delta}$ spanned by the effective groupoids, and from $\Res_{\Eff}(\calX)$ to the full subcategory of $\Fun(\Delta^1, \calX)$ spanned by the effective epimorphisms.

\begin{remark}\label{geoeff}
Let $f_{\ast}: \calX \rightarrow \calY$ be a geometric morphism of $\infty$-topoi, and let
$u: U \rightarrow Y$ be an effective epimorphism in $\calY$. Then $f^{\ast}(u)$ is an effective epimorphism in $\calX$. To see this, choose a \Cech nerve $U_{\bigdot}$ of $u$. Since
$u$ is an effective epimorphism, $U_{\bigdot}$ is a colimit diagram. The left exactness of $f^{\ast}$ implies that $f^{\ast} \circ U_{\bigdot}$ is a \Cech nerve of $f^{\ast}(u)$. Since $f^{\ast}$ is a left adjoint, we conclude that $f^{\ast} \circ U_{\bigdot}$ is a colimit diagram so that $f^{\ast}(u)$ is an effective epimorphism.
\end{remark}

The following result summarizes a few basic properties of effective epimorphisms:

\begin{proposition}\label{sinn}
Let $\calX$ be a semitopos.

\begin{itemize}
\item[$(1)$] Any equivalence $X \rightarrow Y$ in $\calX$ is an effective epimorphism.

\item[$(2)$] If $f,g: X \rightarrow Y$ are homotopic morphisms in $\calX$, then
$f$ is an effective epimorphism if and only if $g$ is an effective epimorphism.

\item[$(3)$] If $F: \calY \rightarrow \calX$ is a left exact presentable functor between semitopoi, and $f: U \rightarrow X$ is an effective epimorphism in $\calX$, then $F(f)$ is an effective epimorphism
in $\calY$.
\end{itemize}
\end{proposition}

\begin{proof}
Assertions $(1)$ and $(2)$ are obvious. To prove $(3)$, we observe that $f$ is an effective epimorphism if and only if it can be extended to an augmented simplicial object $U_{\bigdot}$ which is both a simplicial resolution and a \Cech nerve. Since $F$ is left exact, it preserves the property of being a \Cech nerve; since $F$ preserves colimits, it preserves the property of being a simplicial resolution.
\end{proof}

\begin{remark}
Let $\calX$ be a semitopos, and let $f: X \rightarrow T$ be an effective epimorphism in $\calX$
Applying part $(3)$ of Proposition \ref{sinn} to the geometric morphism $f: \calX_{/S} \rightarrow \calX_{/T}$ induced by a morphism $S \rightarrow T$ in $\calX$, we deduce that any base change
$X \times_{T} S \rightarrow X$ of $f$ is also an effective epimorphism.
\end{remark}

In order to verify other basic properties of the class of effective epimorphisms, such as stability under composition, we will need to reformulate the property of surjectivity in terms of subobjects.
Let $\calX$ be presentable $\infty$-category. For each $X \in \calX$, the $\infty$-category
$\tau_{\leq -1} \calX_{/X}$ of subobjects of $X$ is equivalent to the nerve of a partially
ordered set which we will denote by $\Sub(X)$; we may identify $\Sub(X)$ with the 
set of equivalence classes of monomorphisms $U \rightarrow X$.
A morphism $f: X \rightarrow Y$ in $\calX$ induces a left exact pullback functor $\calX_{/X} \rightarrow \calX_{/Y}$. This functor preserves $(-1)$-truncated objects by Proposition \ref{eaa}, and therefore induces a map $f^{\ast}: \Sub(Y) \rightarrow \Sub(X)$ of partially ordered sets. 

\begin{remark}\label{summep}
Let $\calX$ be a presentable $\infty$-category in which colimits are universal. Then any monomorphism $u: U \rightarrow \coprod X_{\alpha}$ can be obtained as a coproduct
of maps $u_{\alpha}: U_{\alpha} \rightarrow X_{\alpha}$, where each $u_{\alpha}$ is a pullback of $u$ and therefore also a monomorphism. It follows that the natural map
$$ \theta: \Sub( \coprod X_{\alpha} ) \rightarrow \prod \Sub(X_{\alpha})$$
is a monomorphism of partially ordered sets. If coproducts in $\calX$ are disjoint, then
$\theta$ is bijective.
\end{remark}

\begin{proposition}\label{charsurj}
Let $\calX$ be a semitopos.
A morphism $f: U \rightarrow X$ in $\calX$ is an effective epimorphism if and
only if $f^{\ast}: \Sub(X) \rightarrow \Sub(U)$ is injective.
\end{proposition}

\begin{proof}
Suppose first that $f^{\ast}$ is injective. Let $U_{\bigdot}$
be the underlying groupoid of a \Cech nerve of $f$, let
$V$ be a colimit of $U_{\bigdot}$, let $u: V \rightarrow X$
be the corresponding monomorphism, and let $[V]$ denote the corresponding element
of $\Sub(X)$. Since $f$ factors through $u$, we conclude that $f^{\ast}[V] = f^{\ast} [X] = [U] \in \Sub(U)$. Invoking the injectivity of $f^{\ast}$, we conclude that $[V] = [X]$ so that $u$ is an equivalence.

For the converse, let us suppose that $f$ is an effective epimorphism.
Let $[V]$ and $[V']$ be elements of $\Sub(X)$, represented by monomorphisms
$u: V \rightarrow X$ and $u': V' \rightarrow X$, and suppose that
$f^{\ast} [V] = f^{\ast} [V']$. We wish to prove that $[V] = [V']$. 
Since $f^{\ast}$ is a left exact functor, we have
$f^{\ast}( [V] \cap [V'] ) = f^{\ast} [ V \times_{X} V' ]$. It will suffice
to prove that $[V'] = [V \times_{X} V']$; the same argument will then establish
that $[V] = [V \times_{X} V']$ and the proof will be complete. In other words, we may assume without loss of generality that $[V] \subseteq [V']$ so that there is a commutative diagram
$$ \xymatrix{ V_0 \ar[dr]^{u} \ar[rr]^{g} & & V' \ar[dl]^{u'} \\
& X. & }$$
We wish to show that $g$ is an equivalence. The map $g$ induces
a natural transformation of augmented simplicial objects
$$ \alpha_{\bigdot}: u^{\ast} \circ \mCech(f) \rightarrow {u'}^{\ast} \circ \mCech(f).$$
We observe that $g$ can be identified with $\alpha_{-1}$. Since
$f$ is an effective epimorphism, $\mCech(f)$ is a colimit diagram.
Since colimits in $\calX$ are universal, we conclude that
$\alpha_{-1}$ is a colimit of $\alpha | \Nerve(\cDelta)^{op}$. Consequently, to
prove that $\alpha_{-1}$ is an equivalence, it will suffice to prove that $\alpha_n$
is an equivalence for $n \geq 0$. Since each $\alpha_n$ is a pullback of $\alpha_0$, it will
suffice to prove that $\alpha_0$ is an equivalence. But this is simply a reformulation
of the condition that $f^{\ast} [V] = f^{\ast} [V']$.
\end{proof}

From this we immediately deduce some corollaries.

\begin{corollary}\label{sumepi}
Let $\calX$ be a semitopos, and let $\{ f_{\alpha}: X_{\alpha} \rightarrow Y_{\alpha} \}$ be a $($small$)$ collection of effective epimorphisms in $\calX$. Then the induced map
$$ f: \coprod_{\alpha} X_{\alpha} \rightarrow \coprod_{\alpha} Y_{\alpha} $$
is an effective epimorphism.
\end{corollary}

\begin{proof}
Combine Proposition \ref{charsurj} with Remark \ref{summep}.
\end{proof}

\begin{corollary}\label{composite}
Let $\calX$ be a semitopos containing a diagram
$$ \xymatrix{ & Y \ar[dr]^{g} & \\
X \ar[ur]^{f} \ar[rr]^{h} & & Z. }$$
\begin{itemize}
\item[$(1)$] If $f$ and $g$ are effective epimorphisms, then so is $h$.
\item[$(2)$] If $h$ is an effective epimorphism, then so is $g$.
\end{itemize}
\end{corollary}

\begin{proof}
This follows immediately from Proposition \ref{charsurj}, and the observation
that we have an equality $f^{\ast} \circ g^{\ast} = h^{\ast}$ of functions
$\Sub(Z) \rightarrow \Sub(X)$.
\end{proof}

The theory of effective epimorphism is a mechanism for
proving theorems by descent.

\begin{lemma}\label{epie}
Let $\calX$ be a semitopos, let $\overline{p}: K^{\triangleright} \rightarrow \calX$ be
a colimit diagram, let $\infty$ denote the cone point of $K^{\triangleright}$. Then
the associated map
$$ \coprod_{x \in K_0} p(x) \rightarrow p(\infty)$$
$($which is well-defined up to homotopy$)$ is an effective epimorphism.
\end{lemma}

\begin{proof}
For each vertex $x$ of $K^{\triangleright}$, let $Z_x =p(x)$, and if $x$ belongs to $K$ we will 
denote the corresponding map $Z_{x} \rightarrow Z_{\infty}$ by $f_{x}$.
Let $E'' \subseteq E' \in \Sub(Z_{\infty})$ be such that
$f_{x}^{\ast}E'' = f_{x}^{\ast} E'$ for each vertex $x$ of $K$; we wish to show that $E'' = E'$. 
We can represent $E''$ and $E'$ by a $2$-simplex $\sigma_{\infty}: \Delta^2 \rightarrow \calX$, which we depict as
$$ \xymatrix{ & Z'_{\infty} \ar[dr] & \\ 
Z''_{\infty} \ar[ur] \ar[rr] & & Z_{\infty}. }$$
Lift the above diagram to a $2$-simplex $\sigma: \Delta^2 \rightarrow 
\Fun( K^{\triangleright}, \calX)$
$$ \xymatrix{ & p' \ar[dr]^{g''} & \\
p'' \ar[ur]^{g'} \ar[rr]^{g} & & p }$$
where $g$, $g'$, and $g''$ are Cartesian transformations. Our assumption guarantees that
the restriction of $g'$ induces an equivalence $p'' | K \rightarrow p' | K$. Since colimits in
$\calX$ are universal, $g'$ is itself an equivalence, so that $E'' = E'$ as desired.
\end{proof}

\begin{proposition}\label{torque}
Let $\calX$ be an $\infty$-topos, and let $S$ be a collection of morphisms of $\calX$ which is stable under pullbacks and coproducts. The following conditions are equivalent:
\begin{itemize}
\item[$(1)$] The class $S$ is local $($Definition \ref{localitie}$)$. 

\item[$(2)$] Given a pullback diagram
$$ \xymatrix{ X' \ar[r] \ar[d]^{f'} & X \ar[d]^{f} \\
Y' \ar[r]^{g} & Y }$$
where $g$ is an effective epimorphism and $f' \in S$, we have $f \in S$.
\end{itemize}
\end{proposition}

\begin{proof}
We first show that $(1) \Rightarrow (2)$. 
Let $Y_{\bigdot}: \Nerve(\cDelta_{+})^{op} \rightarrow \calX$ be a \Cech nerve of
the map $g$, and choose a Cartesian transformation $f_{\bigdot}: X_{\bigdot} \rightarrow Y_{\bigdot}$ of augmented simplicial objects which extends $f$. Then we can identify
$f'$ with $f_{0}: X_0 \rightarrow Y_0$. Each $f_n$ is a pullback of $f_0$, and therefore belongs to $S$. Applying Lemma \ref{ib2}, we deduce that $f$ belongs to $S$ as well.

Conversely, suppose that $(2)$ is satisfied. We will show that $S$ satisfies criterion $(3)$ of Lemma \ref{ib3}. Let $$ \xymatrix{ u \ar[r]^{\alpha} \ar[d]^{\beta} & v \ar[d]^{\beta'} \\
u' \ar[r]^{\alpha'} & v' }$$
be a pushout diagram in $\calO_{\calX}$, where $\alpha$ and $\beta$ are Cartesian
and $u,v,u' \in S$. Since $\calX$ is an $\infty$-topos, we conclude that $\alpha'$ and $\beta'$ are also Cartesian. To complete the proof, it will suffice to show that $v' \in S$. For this, we observe that
there is a pullback diagram
$$ \xymatrix{ X \amalg X' \ar[d]^{v \amalg u'} \ar[r] & X'' \ar[d]^{v'} \\
Y \amalg Y' \ar[r]^{g} & Y'' }$$
where $g$ is an effective epimorphism (Lemma \ref{epie}) and apply hypothesis $(2)$.
\end{proof}


\begin{proposition}\label{hintdescent1}
Let $\calX$ be a semitopos, and suppose given a pullback square
$$ \xymatrix{ X' \ar[r]^{g'} \ar[d]^{f'} & X \ar[d]^{f} \\
S' \ar[r]^{g} & S }$$
in $\calX$. If $f$ is an effective epimorphism, then so is $f'$. 
The converse holds if $g$ is an effective epimorphism.
\end{proposition}

\begin{proof}
Let $g^{\ast}: \calX^{/S} \rightarrow \calX^{/S'}$ be a pullback functor. Without loss of 
generality we may suppose that $f' = g^{\ast} f$. 
Let $U_{\bigdot}: \Nerve(\cDelta_{+})^{op} \rightarrow \calX$ be a \Cech nerve of $f$.
Since $g^{\ast}$ is left exact (being a right adjoint), we conclude that
$g^{\ast} \circ U_{\bigdot}$ is a \Cech nerve of $f'$. If $f$ is an effective epimorphism, then
$U_{\bigdot}$ is a colimit diagram. Because colimits in $\calX$ are universal, $g^{\ast} \circ U_{\bigdot}$ is also a colimit diagram, so that $f'$ is an effective epimorphism.

Conversely, suppose that $f'$ and $g$ are effective epimorphisms. Corollary \ref{composite} implies that $g \circ f'$ is an effective epimorphism. The commutativity of the diagram
implies that $f \circ g'$ is an effective epimorphism, so that $f$ is an effective epimorphism (Corollary \ref{composite} again).
\end{proof}

\begin{lemma}\label{hint0}
Let $\calX$ be a semitopos, and suppose given a pullback square
$$ \xymatrix{ X' \ar[r]^{g'} \ar[d]^{f'} & X \ar[d]^{f} \\
S' \ar[r]^{g} & S }$$
in $\calX$. Suppose that $f'$ is an equivalence that $g$ is an effective epimorphism.
Then $f$ is an equivalence.
\end{lemma}

\begin{proof}
Let $U_{\bigdot}$ be a \Cech nerve of $g'$, and let $V_{\bigdot}$ be a \Cech nerve of $g$.
The above diagram induces a transformation $\alpha_{\bigdot}: U_{\bigdot} \rightarrow V_{\bigdot}$. The map $\alpha_0$ can be identified with $f'$, and is therefore an equivalence.
For $n \geq 0$, $\alpha_n: U_n \rightarrow V_n$ is a pullback of $\alpha_0$, and therefore also an equivalence. Since $g$ is an effective epimorphism, $V_{\bigdot}$ is a colimit diagram.
Applying Proposition \ref{hintdescent1}, we conclude that $g'$ is also an effective epimorphism so that $U_{\bigdot}$ is a colimit diagram. It follows that $f = \alpha_{-1}$ is a colimit of
equivalences, and is therefore an equivalence.
\end{proof}

\begin{proposition}\label{hintdescent0}
Let $\calX$ be a semitopos, and suppose given a pullback square
$$ \xymatrix{ X' \ar[r] \ar[d]^{f'} & X \ar[d]^{f} \\
S' \ar[r]^{g} & S }$$
in $\calX$. If $f$ is $n$-truncated, then so is $f'$. The converse holds if $g$ is an effective epimorphism.
\end{proposition}

\begin{proof}
Let $g^{\ast}: \calX^{/S} \rightarrow \calX^{/S'}$ be a pullback functor. The first part of
$(1)$ asserts that $g^{\ast}$ carries $n$-truncated objects to $n$-truncated objects.
This follows immediately from Proposition \ref{eaa}, since $g^{\ast}$ is a right adjoint and therefore left exact. We will prove the converse in a slightly stronger form: if $i: U \rightarrow V$
is a morphism in $\calX^{/S}$ such that then $g^{\ast}(i)$ is an $n$-truncated morphism in
$\calX^{/S'}$, then $i$ is $n$-truncated. The proof is by induction on $n$. If $n \geq -1$, we
can use Lemma \ref{trunc} to reduce to the problem of showing that the diagonal map
$\delta: U \rightarrow U \times_{V} U$ is $(n-1)$-truncated. Since $g^{\ast}$ is left exact, 
we can identify $g^{\ast}(\delta)$ with the diagonal map $g^{\ast} U \rightarrow g^{\ast} U \times_{ g^{\ast} V} g^{\ast} U$, which is $(n-1)$-truncated according to Lemma \ref{trunc}; the desired result then follows from the inductive hypothesis. In the case $n=-2$, we have a pullback
diagram
$$ \xymatrix{ g^{\ast} U \ar[r] \ar[d]^{g^{\ast} i} & U \ar[d]^{i} \\
g^{\ast} V \ar[r]^{g'} \ar[d] & V \ar[d] \\
S' \ar[r]^{g} & S. }$$
Proposition \ref{hintdescent1} implies that $g'$ is an effective epimorphism, and
$g^{\ast} i$ is an equivalence, so that $i$ is also an equivalence by Lemma \ref{hint0}.
\end{proof}

Let $\calC$ be a small $\infty$-category equipped with a Grothendieck topology. 
Our final goal in this section is to use the language of effective epimorphisms to characterize
the $\infty$-topos $\Shv(\calC)$ by a universal property.

\begin{lemma}\label{stubba}
Let $\calC$ be a $($small$)$ $\infty$-category containing an object $C$, let
$\{ f_{\alpha}: C_{\alpha} \rightarrow C \}_{\alpha \in A}$ be a collection of morphisms indexed by a set $A$ and let $\calC^{(0)}_{/C} \subseteq \calC_{/C}$ be the sieve on $C$ that they generate.
Let $j: \calC \rightarrow \calP(\calC)$ denote the Yoneda embedding and $i: U \rightarrow j(C)$
a monomorphism corresponding to the sieve $\calC^{(0)}_{/C}$. Then $i$ can be identified with a $(-1)$-truncation of the induced map $\coprod_{\alpha \in A} j(C_{\alpha}) \rightarrow j(C)$
in the $\infty$-topos $\calP(\calC)_{/C}$.
\end{lemma}

\begin{proof}
Using Proposition \ref{surry}, we can identify the equivalence classes of $(-1)$-truncated object $U \in \calP(\calC)_{/j(C)}$ with sieves $\calC^{(0)}_{/C} \subseteq \calC_{/C}$. It is not difficult to see that $j( f_{\alpha})$ factors through $U$ if and only if $f_{\alpha} \in \calC^{(0)}_{/C}$. 
Consequently, the $(-1)$-truncation of $\coprod_{\alpha \in A} j(C_{\alpha}) \rightarrow j(C)$
is associated to the {\em smallest} sieve on $\calC$ which contains each $f_{\alpha}$.
\end{proof}

\begin{lemma}\label{pregrute}
Let $\calX$ be an $\infty$-topos, $\calC$ a small $\infty$-category equipped with a Grothendieck topology, and $f_{\ast}: \calX \rightarrow \calP(\calC)$ a functor with a left exact
left adjoint $f^{\ast}: \calP(\calC) \rightarrow \calX$.

The following conditions are equivalent:
\begin{itemize}
\item[$(1)$] The functor $f_{\ast}$ factors through $\Shv(\calC) \subseteq \calP(\calC)$.

\item[$(2)$] For every collection of morphisms $\{ v_{\alpha}: C_{\alpha} \rightarrow C\}$ which generate
a covering sieve in $\calC$, the induced map
$$ \coprod f^{\ast}( j(C_{\alpha}) ) \rightarrow f^{\ast} ( j(C) )$$
is an effective epimorphism in $\calX$, where $j: \calC \rightarrow \calP(\calC)$ denotes the Yoneda embedding.
\end{itemize}
\end{lemma}

\begin{proof}
Suppose first that $(1)$ is satisfied, and let $\{ v_{\alpha}: C_{\alpha} \rightarrow C\}$ be a collection of morphisms as in the statement of $(2)$. Let $L: \calP(\calC) \rightarrow \Shv(\calC)$ be a left adjoint to the inclusion. Then we have an equivalence of functors $f^{\ast} \simeq (f^{\ast}|\Shv(\calC)) \circ L$. Applying Remark \ref{geoeff}, we are reduced to showing that if
$$ u: \coprod j(C_{\alpha}) \rightarrow j(C)$$
is the natural map, then $Lu$ is an effective epimorphism in $\calP(\calC)$. 
Factor $u$ as a composition
$$ \coprod j(C_{\alpha}) \stackrel{u'}{\rightarrow} U \stackrel{u''}{\rightarrow} j(C)$$
where $u'$ is an effective epimorphism and $u''$ is a monomorphism. We wish to show that
$Lu''$ is an equivalence. Lemma \ref{stubba} allows us to identify $u''$ with the monomorphism
associated to the sieve $\calC^{(0)}_{/C}$ on $C$ generated by the maps $v_{\alpha}$. By assumption, this is a covering sieve, so that $L u''$ is an equivalence in $\Shv(\calC)$ by construction.

Conversely, suppose that $(2)$ is satisfied. Let $C \in \calC$ and let $\calC^{(0)}_{/C} \subseteq \calC_{/C}$ be a covering sieve on $C$ associated to a monomorphism
$u'': U \rightarrow j(C)$. We wish to show that $f^{\ast} u''$ is an equivalence.
According to Lemma \ref{stubba}, we have a factorization
$$ \coprod_{\alpha} j(C_{\alpha}) \stackrel{u'}{\rightarrow} U \stackrel{u''}{\rightarrow} j(C), $$
where the maps $v_{\alpha}: C_{\alpha} \rightarrow C$ are chosen to generate the sieve
$\calC^{(0)}_{/C}$, and $u'$ is an effective epimorphism. Let $u$ be a composition of $u'$ and $u''$. Then $f^{\ast} u'$ is an effective epimorphism (Remark \ref{geoeff}), and $f^{\ast} u$ is an effective epimorphism by assumption $(2)$. Corollary \ref{composite} now shows that
$f^{\ast} u''$ is an effective epimorphism. Since $f^{\ast} u''$ is also a monomorphism, we conclude that $f^{\ast} u''$ is an equivalence as desired.
\end{proof}

\begin{proposition}\label{igrute}
Let $\calX$ be an $\infty$-topos, and let $\calC$ be a small $\infty$-category equipped with a Grothendieck topology. Let
$L: \calP(\calC) \rightarrow \Shv(\calC)$ denote a left adjoint to the inclusion, and
$j: \calC \rightarrow \calP(\calC)$ the Yoneda embedding. Let
$\Fun^{\ast}( \Shv(\calC), \calX )$ denote the $\infty$-category of left exact, colimit-preserving functors from $\Shv(\calC)$ to $\calX$ (Definition \ref{defhomst}). The composition
$$ J: \Fun^{\ast}(\Shv(\calC), \calX) \stackrel{L}{\rightarrow} \Fun^{\ast}( \calP(\calC), \calX)
\stackrel{j}{\rightarrow} \Fun( \calC, \calX)$$
is fully faithful. Suppose furthermore that $\calC$ admits finite limits. Then
a functor $f: \calC \rightarrow \calX$ belongs to the essential image of $J$ if and only if
the following conditions are satisfied:
\begin{itemize}
\item[$(1)$] The functor $f$ is left exact.
\item[$(2)$] For every collection of morphisms $\{ C_{\alpha} \rightarrow C\}_{\alpha \in A}$ which generate a covering sieve on $C$, the associated morphism
$$ \coprod_{\alpha \in A} f(C_{\alpha}) \rightarrow f(C)$$ is an effective epimorphism in $\calX$.
\end{itemize}
\end{proposition}

\begin{proof}
If the topology on $\calC$ is trivial, then Theorem \ref{charpresheaf} implies that
$J$ is fully faithful, and the description of the essential image of $J$ follows from Proposition \ref{natash}. In the general case, Proposition \ref{unichar} implies that
composition with $L$ induces a fully faithful embedding
$$ J': \Fun^{\ast}( \Shv(\calC) ,\calX) \rightarrow \Fun^{\ast}( \calP(\calC), \calX),$$
so that $J$ is a composition of $J'$ with a fully faithful functor
$$ J'': \Fun^{\ast}(\calP(\calC),\calX) \rightarrow \Fun(\calC, \calX).$$ 
Suppose that $\calC$ admits finite limits and that $f$ satisfies $(1)$, so that
$f$ is equivalent to $J''(u^{\ast})$ for some left exact, colimit preserving
$u^{\ast}: \calP(\calC) \rightarrow \calX$. The functor $u^{\ast}$ is unique up to equivalence,
and Lemma \ref{pregrute} ensures that $u^{\ast}$ belongs to the essential image of $J'$ if
and only if condition $(2)$ is satisfied.
\end{proof}

\begin{remark}
It is possible to formulate a generalization of Proposition \ref{igrute} which describes the essential image of $J$ even when $\calC$ does not admit finite limits. The present version will be sufficient for the applications in this book. 
\end{remark}

\subsection{Canonical Topologies}\label{cantopp}

Let $\calX$ be an $\infty$-topos. Suppose that we wish to identify $\calX$ with an $\infty$-category of sheaves. The first step is to choose a pair of adjoint functors
$$ \Adjoint{F}{\calP(\calC)}{\calX}{G}$$
where $F$ is left exact. According to Theorem \ref{charpresheaf}, $F$ is determined up
to equivalence by the composition
$$ f: \calC \stackrel{j}{\rightarrow} \calP(\calC) \stackrel{F}{\rightarrow} \calX.$$
We might then try to choose a topology on $\calC$ such that $G$ factors as a composition
$$ \calX \stackrel{G'}{\rightarrow} \Shv(\calC) \subseteq \calP(\calC).$$
Though it is not always possible to guarantee that $G'$ is an equivalence, we will show that for an appropriately chosen topology (Definition \ref{defcantop}), the $\infty$-topos $\Shv(\calC)$ is a close approximation to $\calX$ (Proposition \ref{preciselate}). 

\begin{definition}\label{defcantop}\index{gen}{canonical covering}
Let $\calX$ be a semitopos, $\calC$ a small $\infty$-category which admits finite limits, and
$f: \calC \rightarrow \calX$ a left exact functor. We will say that a sieve $\calC^{(0)}_{/C} \subseteq \calC_{/C}$ on an object $C \in \calC$ is a {\it canonical covering relative to $f$} if there exists
a collection of morphisms $\{ u_{\alpha}: C_{\alpha} \rightarrow C \}$ belonging to $\calC^{(0)}_{/C}$ such that the induced map $ \coprod f(C_{\alpha}) \rightarrow f(C)$ is an effective epimorphism in $\calX$.
\end{definition}

Our first goal is to verify that the canonical topology is actually a Grothendieck topology on $\calC$.

\begin{proposition}\label{cantop}\index{gen}{canonical topology}
Let $f: \calC \rightarrow \calX$ be as in Definition \ref{defcantop}. 
The collection of canonical coverings relative to $f$ determine a Grothendieck topology on $\calC$.
\end{proposition}

\begin{proof}
Since any identity map $\id_{f(C)}: f(C) \rightarrow f(C)$ is an effective epimorphism, it is clear that
the sieve $\calC_{/C}$ is a canonical covering of $C$ for every $C \in \calC$. Suppose
that $\calC_{/C}^{(0)} \subseteq \calC_{/C}$ is a canonical covering of $C$, and that
$g: D \rightarrow C$ is a morphism in $\calC$. We wish to prove that
the induced sieve $g^{\ast} \calC_{/C}^{(0)}$ is a canonical covering. Choose a 
collection of objects $u_{\alpha}: C_{\alpha} \rightarrow C$ of $\calC_{/C}^{(0)}$ 
such that the induced map $\coprod_{\alpha} f(C_{\alpha}) \rightarrow f(C)$ is an
effective epimorphism, and form pullback diagrams
$$ \xymatrix{ D_{\alpha} \ar[r]^{v_{\alpha}} \ar[d] & D \ar[d]^{g} \\
C_{\alpha} \ar[r]^{u_{\alpha}} & C }$$
in $\calC$. Using the fact that $f$ is left exact and that colimits in $\calX$ are universal, we conclude that the diagram
$$ \xymatrix{ \coprod f(D_{\alpha}) \ar[r] \ar[d] & f(D) \ar[d] \\
\coprod f(C_{\alpha}) \ar[r] & f(C) }$$
is a pullback, so that the upper  horizontal map is an effective epimorphism by Proposition \ref{hintdescent1}. Since each $v_{\alpha}$ belongs to $g^{\ast} \calC_{/C}^{(0)}$, it follows that
$g^{\ast} \calC_{/C}^{(0)}$ is a canonical covering.

Now suppose that $\calC_{/C}^{(0)}$ and $\calC_{/C}^{(1)}$ are sieves on $C \in \calC$, where
$\calC_{/C}^{(0)}$ is a canonical covering, and for each $g: D \rightarrow C$ in $\calC_{/C}^{(0)}$, the covering $g^{\ast} \calC_{/C}^{(1)}$ is a canonical covering of $D$. Choose a collection
of morphisms $g_{\alpha}: D_{\alpha} \rightarrow C$ belonging to $\calC_{/C}^{(0)}$ with the property that $\coprod f(D_{\alpha}) \rightarrow f(C)$ is an effective epimorphism. For each
$D_{\alpha}$, choose a collection of morphisms $h_{\alpha,\beta}: E_{\alpha,\beta} \rightarrow D_{\alpha}$ belonging to $g_{\alpha}^{\ast} \calC_{/C}^{(1)}$ such that the map
$\coprod_{\beta} f(E_{\alpha,\beta}) \rightarrow f(D_{\alpha})$ is an effective epimorphism.
Using Corollary \ref{sumepi}, we conclude that the map
$$ \coprod_{\alpha,\beta} f(E_{\alpha,\beta}) \rightarrow \coprod_{\alpha} f(D_{\alpha})$$ is an effective epimorphism. Since effective epimorphisms are stable under composition (Corollary \ref{composite}), we have an effective epimorphism $\coprod_{\alpha,\beta} f(E_{\alpha,\beta}) \rightarrow f(C)$, induced by the collection of compositions $g_{\alpha} \circ h_{\alpha,\beta}: E_{\alpha,\beta} \rightarrow C$. Each of these compositions belong to $\calC_{/C}^{(1)}$, so
that $\calC_{/C}^{(1)}$ is a canonical covering of $C$.
\end{proof}

For later use, we record a few features of the canonical topology:

\begin{lemma}\label{caninit}
Let $f: \calC \rightarrow \calX$ be as in Definition \ref{defcantop}, and regard
$\calC$ as endowed with the canonical topology relative to $f$. Let $j: \calC \rightarrow \calP(\calC)$ denote the Yoneda embedding and let $L: \calP(\calC) \rightarrow \Shv(\calC)$
be a left adjoint to the inclusion. Suppose that $C \in \calC$ is such that $f(C)$ is an initial object of $\calX$. Then $L j(C)$ is an initial object of $\Shv(\calC)$.
\end{lemma}

\begin{proof}
If $f(C)$ is an initial object of $\calX$, then the empty sieve $\emptyset \subseteq \calC_{/C}$
is a covering sieve with respect to the canonical topology. By construction, the associated monomorphism $\emptyset \rightarrow j(C)$ becomes an equivalence after applying $L$, so that $L j(C)$ is initial in $\Shv(\calC)$.
\end{proof}

\begin{lemma}\label{canonicalcoproducts}
Let $f: \calC \rightarrow \calX$ be as in Definition \ref{defcantop}. Suppose that $f$ is fully faithful, coproducts in $\calX$ are disjoint, and let $\{ u_{\alpha}: C_{\alpha} \rightarrow C \}$ be a small collection of morphisms in $\calC$ such that the morphisms $f(u_{\alpha})$ exhibit
$f(C)$ as a coproduct of the family $\{ f(C_{\alpha}) \}$. Let $\calF: \calC^{op} \rightarrow \SSet$
be a sheaf on $\calC$ $($with respect to the canonical topology induced by $f${}$)$. Then the morphisms
$\{ \calF(u_{\alpha}) \}$ exhibit $\calF(C)$ as a product of $\{ \calF(C_{\alpha}) \}$ in $\SSet$.
\end{lemma}

\begin{proof}
We wish to show that the natural map $\calF(C) \rightarrow \prod \calF(C_{\alpha} )$ is an isomorphism in the homotopy category $\calH$. We may identify the left hand side with
$\bHom_{\calP(\calC)}( j(C), \calF)$, and the right hand side with
$\bHom_{\calP(\calC)}( \coprod j(C_{\alpha}), \calF)$. Consequently, it will suffice to show that the natural map
$$ v: \coprod j(C_{\alpha}) \rightarrow j(C)$$
becomes an equivalence after applying the localization functor $L: \calP(\calC) \rightarrow \Shv(\calC)$. Choose a factorization of $v$ as a composite
$$ \coprod j(C_{\alpha}) \stackrel{v'}{\rightarrow} U \stackrel{v''}{\rightarrow} j(C)$$
where $v'$ is an effective epimorphism, and $v''$ is a monomorphism. We observe that 
$v''$ is the monomorphism associated to the
sieve $\calC^{(0)}_{/C} \rightarrow \calC$ generated by the morphisms $u_{\alpha}$.
This is clearly a covering sieve with respect to the canonical topology, so that $Lv''$
is an equivalence in $\Shv(\calC)$. It follows that $Lv$ is equivalent to $Lv'$, and is therefore an effective epimorphism (Remark \ref{geoeff}). 
Form a pullback diagram
$$ \xymatrix{ V \ar[r]^{\overline{v}} \ar[d] & \coprod j(C_{\beta}) \ar[d]^{v} \\
\coprod j(C_{\alpha}) \ar[r]^{v} & j(C) }$$
We wish to prove that $Lv$ is an equivalence. According to Lemma \ref{hint0}, it will suffice
to show that $L \overline{v}$ is an equivalence. Since colimits in $\calP(\calC)$ are universal, we may identify $\overline{v}$ with a coproduct of morphisms 
$$\overline{v}_{\beta}: V_{\beta} \rightarrow j(C_{\beta}),$$ 
where $V_{\beta}$ can be written as a coproduct $\coprod_{\alpha} j( C_{\alpha} \times_{C} C_{\beta})$. Using Lemma \ref{sumdescription}, we can identify the summand
$ j( C_{\beta} \times_{C} C_{\beta})$ of $V_{\beta}$ with $j(C_{\beta})$, and the restriction
of $\overline{v}_{\beta}$ to this summand is an equivalence. To complete the proof, it will suffice to show that for every other summand $D_{\alpha, \beta} = j( C_{\alpha} \times_{C} C_{\beta} )$,
the localization $LD$ is an initial object of $\Shv(\calC)$. To prove this, we observe Lemma \ref{sumdescription} implies that $f( C_{\alpha} \times_{C} C_{\beta} )$ is an initial object
of $\calX$, and apply Lemma \ref{caninit}.
\end{proof}


\begin{lemma}\label{charepii}
Let $\calC$ be a small $\infty$-category equipped with a Grothendieck topology, and let
$u: \calF' \rightarrow \calF$ be a morphism in $\Shv(\calC)$. Suppose that, for each
$C \in \calC$ and each $\eta \in \pi_0 \calF(C)$, there exists a collection of morphisms
$\{ C_{\alpha} \rightarrow C \}$ which generates a covering sieve on $C$ and
a collection of $\eta_{\alpha} \in \pi_0 \calF'(C_{\alpha})$ such that $\eta$ and
$\eta_{\alpha}$ have the same image in $\pi_0 \calF(C_{\alpha})$. Then $u$ is an effective epimorphism.
\end{lemma}

\begin{proof}
Replacing $\calF$ by its image in $\calF'$ if necessary, we may suppose that
$u$ is a monomorphism. Let $L: \calP(\calC) \rightarrow \Shv(\calC)$ be a left adjoint to the inclusion, and let $\calD$ be the full subcategory of $\calP(\calC)$ spanned by those objects
$\calG$ such that, for every pullback diagram
$$ \xymatrix{ \calG' \ar[r]^{u'} \ar[d] & \calG \ar[d] \\
\calF' \ar[r]^{u} & \calF }$$
in $\calP(\calC)$, $Lu'$ is an equivalence in $\Shv(\calC)$. To prove that $u$ is an equivalence, it will suffice to show that the equivalent morphism $Lu$ is an equivalence. For this, it will suffice to prove that $\calF \in \calD$. We will in fact prove that $\calD = \calP(\calC)$. We first observe
that, since colimits in $\calP(\calC)$ are universal and $L$ commutes with colimits, 
$\calD$ is stable under colimits in $\calP(\calC)$. Since $\calP(\calC)$ is generated under colimits by the image of the Yoneda embedding, it will suffice to prove that $j(C) \in \calD$, for each $C \in \calC$. Choose a map $j(C) \rightarrow \calF$, classified up to homotopy by $\eta \in \pi_0 \calF(C)$, and form a pullback diagram 
$$ \xymatrix{ U \ar[r]^{u'} \ar[d] & j(C) \ar[d] \\
\calF' \ar[r]^{u} & \calF }$$
as above. Then $u'$ is a monomorphism; according to Proposition \ref{surry} it is classified
by a sieve $\calC^{(0)}_{/C}$ on $\calC$. Our hypothesis guarantees that $\calC^{(0)}_{/C}$ contains a collection of morphisms $\{ C_{\alpha} \rightarrow C \}$ which generate a covering sieve, so that $\calC^{(0)}_{/C}$ is itself covering. It follows immediately from the construction of $\Shv(\calC)$ that $Lu'$ is an equivalence.
\end{proof}

We close with the following result, which implies that any $\infty$-topos is closely approximated by an $\infty$-category of sheaves.

\begin{proposition}\label{preciselate}
Let $\calX$ be a semitopos, $\calC$ a small $\infty$-category which admits finite limits, and
$$ \Adjoint{F}{\calP(\calC)}{\calX}{G}$$ a pair of adjoint functors. Suppose that the composition
$$ f: \calC \stackrel{j}{\rightarrow} \calP(\calC) \stackrel{F}{\rightarrow} \calX$$
is left exact, and regard $\calC$ as endowed with the canonical topology relative to $f$.
Then:

\begin{itemize}
\item[$(1)$] The functor $G$ factors through $\Shv(\calC)$.

\item[$(2)$] Suppose that $f$ is fully faithful and generates $\calX$ under colimits. Then $G$ carries effective epimorphisms in $\calX$ to effective epimorphisms in $\Shv(\calC)$. 

\end{itemize}
\end{proposition}

\begin{proof}
In view of the definition of the canonical topology, $(1)$ is equivalent to the following assertion: given a collection of morphisms $\{ u_{\alpha}: C_{\alpha} \rightarrow C \}$ in
$\calC$ such that the induced map $u: \coprod_{\alpha} C_{\alpha} \rightarrow C$ is an effective epimorphism in $\calX$, if $i: U \rightarrow j(C)$ is the monomorphism in $\calP(\calC)$
corresponding to the sieve $\calC^{(0)}_{/C} \subseteq \calC_{/C}$ generated by
the collection $\{ u_{\alpha} \}$, then $F(i)$ is an equivalence in $\calX$.
Let $u': \coprod_{\alpha} j(C_{\alpha}) \rightarrow j(C)$ be the coproduct of the family
$\{ j( u_{\alpha} ) \}$, and let $V_{\bigdot}: \cDelta_{+}^{op} \rightarrow \calP(\calC)$ be a \Cech nerve of $u'$. Then $i$ can be identified with the induced map from the colimit of 
$V_{\bigdot} | \Nerve(\cDelta)^{op}$ to $V_{-1}$. Since $F$ preserves colimits, to show that
$F(i)$ is an equivalence, it will suffice to show that $F \circ V_{\bigdot}$ is a colimit diagram.
Since $u$ is an effective epimorphism, it suffices to observe that $F \circ V_{\bigdot}$ is equivalent to the \Cech nerve of $u$.

We now prove $(2)$. Suppose that $u: Y \rightarrow Z$ is an effective epimorphism in $\calX$.
We wish to prove that $Gu$ is an effective epimorphism in $\Shv(\calC)$. We will show that the criterion of Lemma \ref{charepii} is satisfied. Choose an object $C \in \calC$ and a point
$\eta \in \pi_0 \bHom_{\calP(\calC)}( j(C), GZ) \simeq \pi_0 \bHom_{\calX}(f(C), Z)$. Form a pullback diagram
$$ \xymatrix{ Y' \ar[r]^{u'} \ar[d]^{s} & f(C) \ar[d] \\
Y \ar[r]^{u} & Z }$$
so that $u'$ is an effective epimorphism. Since $f(\calC)$ generates $\calX$ under colimits, there
exists an effective epimorphism $u'': \coprod_{\alpha} f(C_{\alpha}) \rightarrow Y$. The composition
$u' \circ u''$ is an effective epimorphism, and corresponds to a family of maps
$w_{\alpha}: f(C_{\alpha}) \rightarrow f(C)$ in $\calX$. Since $f$ is fully faithful, we may suppose
that each $w_{\alpha} = fv_{\alpha}$ for some map $v_{\alpha}: C_{\alpha} \rightarrow C$ in $\calC$. It follows that the collection of maps $\{ v_{\alpha} \}$ generate a covering sieve on $C$ with respect to the canonical topology. Moreover, each of the compositions
$$f(C_{\alpha}) \rightarrow \coprod_{\alpha} f(C_{\alpha}) \rightarrow Y $$
gives rise to a point $\eta_{\alpha} \in \pi_0 \bHom_{\calX}( f(C_{\alpha}), Y) \simeq
\pi_0 \bHom_{\calP(\calC)}( j( C_{\alpha}), G(Y))$ with the desired properties.
\end{proof}