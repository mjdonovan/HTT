% !TEX root = highertopoi.tex
\section{The Language of Higher Category Theory}\label{langur}


\setcounter{theorem}{0}

One of the main goals of this book is to demonstrate that many ideas from classical category theory can be adapted to the setting of higher categories. In this section, we will survey some of the simplest examples.

\subsection{The Opposite of an $\infty$-Category}\label{working}

If $\calC$ is an ordinary category, then the opposite category
$\calC^{op}$ is defined in the following way:

\begin{itemize}\index{gen}{opposite!of a category}
\item The objects of $\calC^{op}$ are the objects of $\calC$.
\item For $X,Y \in \calC$, we have $\Hom_{\calC^{op}}(X,Y) =
\Hom_{\calC}(Y,X)$. Identity morphisms and composition are defined
in the obvious way.
\end{itemize}

This definition generalizes without change to the setting of topological or simplicial categories. Adapting this definition to the setting of $\infty$-categories requires a few additional words.
We may define more generally the {\it opposite} of a simplicial set
$S$ as follows: For any finite, nonempty, linearly ordered set
$J$, we set $S^{op}(J) = S(J^{op})$, where $J^{op}$ denotes the
same set $J$ endowed with the opposite ordering. More concretely,
we have $S^{op}_n = S_n$, but the face and degeneracy maps on
$S^{op}$ are given by the formulas
$$ (d_i: S^{op}_n \rightarrow S^{op}_{n-1}) = (d_{n-i}: S_n
\rightarrow S_{n-1})$$
$$ (s_i: S^{op}_n \rightarrow S^{op}_{n+1}) = (s_{n-i}: S_n
\rightarrow S_{n+1}).$$\index{gen}{opposite!of a simplicial set}

The formation of opposite categories is fully compatible
with all of the constructions we have introduced for passing back
and forth between different models of higher category theory.

It is clear from the definition that a simplicial set $S$ is an $\infty$-category if and only if its opposite $S^{op}$ is an $\infty$-category: for $0 < i < n$, $S$ has the extension property
with respect to the horn inclusion $\Lambda^n_i \subseteq \Delta^n$ if
and only if $S^{op}$ has the extension property with respect to the horn inclusion
$\Lambda^n_{n-i} \subseteq \Delta^n$.
\begin{shaded}
To take the opposite of an $\infty$-category, one needs to reverse the ordering on the simplicial set. That is, we reverse the order of the $d_i$ and $s_i$ at each level. There's no difficulty in doing this.

Check out the following cool fact:
\end{shaded}

The construction $S \mapsto S^{op}$ determines an automorphism of the $\infty$-category
of $\infty$-categories. We will later see that this is (essentially) the {\em only} nontrivial automorphism
(see Theorem \ref{cabbi}).

\subsection{Mapping Spaces in Higher Category Theory}\label{prereq1}

If $X$ and $Y$ are objects of an ordinary category $\calC$, then one has a well-defined
set $\Hom_{\calC}(X,Y)$ of morphisms from $X$ to $Y$. In higher category theory, one has instead a morphism {\em space} $\bHom_{\calC}(X,Y)$. In the setting of topological or simplicial
categories, this morphism space (either a topological space or a simplicial set) is an inherent feature of the formalism. In the setting of $\infty$-categories, it is not so obvious
how $\bHom_{\calC}(X,Y)$ should be defined. However, it is at least clear what to do on the level of the homotopy category.

\begin{definition}\label{morspace}\index{not}{MapS@$\bHom_{S}(X,Y)$}
Let $S$ be a simplicial set containing vertices $x$ and $y$, and let
$\calH$ denote the homotopy category of spaces. We define
$\bHom_{S}(x,y) = \bHom_{\h{S}}(x,y) \in \calH$ to be the object of $\calH$ representing
the space of maps from $x$ to $y$ in $S$. Here $\h{S}$ denotes the homotopy category of $S$, regarded as a $\calH$-enriched category (Definition \ref{tulkas}). 
\end{definition}

\begin{warning}
Let $S$ be a simplicial set. The notation $\bHom_{S}( X, Y)$ has two {\em very} different meanings.
When $X$ and $Y$ are vertices of $S$, then our notation should be interpreted in the sense of Definition \ref{morspace}, so that $\bHom_{S}(X,Y)$ is an object of $\calH$. If $X$ and $Y$ are objects of $(\sSet)_{/S}$, then we instead let $\bHom_{S}(X,Y)$ denote the simplicial mapping object
$$ Y^{X} \times_{ S^{X} } \{ \phi \} \in \sSet,$$
where $\phi$ denotes the structural morphism $X \rightarrow S$. We trust that it will be clear in context which of these two definitions applies in a given situation.
\end{warning}

We now consider the following question: given a simplicial set $S$ containing a pair of vertices $x$ and $y$, how can we compute $\bHom_{S}(x,y)$? We have defined
$\bHom_{S}(x,y)$ as an object of the homotopy category $\calH$, but for many purposes it is important to choose a simplicial set $M$ which represents $\bHom_{S}(x,y)$. 
The most obvious candidate for $M$ is the simplicial set
$\bHom_{\sCoNerve[S]}(x,y)$. The advantages of this definition are that it works in all cases (that is, $S$ does not need to be an $\infty$-category), and comes equipped with an associative composition law. However, the construction of the simplicial set
$\bHom_{\sCoNerve[S]}(x,y)$ is quite complicated. Furthermore,
$\bHom_{\sCoNerve[S]}(x,y)$ is usually not a Kan complex, so it
can be difficult to extract algebraic invariants like homotopy
groups, even when a concrete description of its simplices is known. 

In order to address these shortcomings, we will introduce another simplicial set which
represents the homotopy type $\bHom_{S}(x,y) \in \calH$, at least when $S$ is an $\infty$-category. 
We define a new simplicial
set $\Hom^{\rght}_S(x,y)$, the space of {\it right morphisms} from
$x$ to $y$, by letting $\Hom_{\sSet}( \Delta^n, \Hom^{\rght}_S(x,y))$ denote the set of
all $z: \Delta^{n+1} \rightarrow S$ such that $z| \Delta^{ \{n+1 \}} = y$ and $z|
\Delta^{ \{0, \ldots, n\} }$ is a constant simplex at the vertex
$x$. The face and degeneracy operations on
$\Hom^{\rght}_S(x,y)_n$ are defined to coincide with corresponding operations on 
$S_{n+1}$.\index{not}{HomR@$\Hom^{\rght}_{S}(X,Y)$}

We first observe that when $S$ is an $\infty$-category, $\Hom^{\rght}_S(x,y)$ really is a ``space'':

\begin{proposition}\label{gura}
Let $\calC$ be an $\infty$-category containing a pair of objects $x$ and $y$. The simplicial set
$\Hom^{\rght}_{\calC}(x,y)$ is a Kan complex.
\end{proposition}

\begin{proof}
It is immediate from the definition that if $\calC$ is a
$\infty$-category, then $M=\Hom^{\rght}_{\calC}(x,y)$ satisfies the Kan
extension condition for every horn inclusion $\Lambda^n_i
\subseteq \Delta^n$ where $0 < i \leq n$. This implies that $M$ is
a Kan complex (Proposition \ref{greenwich}).
\end{proof}

\begin{remark}\label{needie}
If $S$ is a simplicial set and $x,y,z \in S_0$, then there is no
obvious composition law
$$\Hom^{\rght}_{S}(x,y) \times \Hom^{\rght}_{S}(y,z) \rightarrow \Hom^{\rght}_{S}(x,z).$$
We will later see that if $S$ is an $\infty$-category, then there is a
composition law which is well-defined up to a
contractible space of choices. The absence of a canonical choice for a composition
law is the main drawback of $\Hom^{\rght}_{S}(x,y)$, in comparison with
$\bHom_{\sCoNerve[S]}(x,y).$
The main goal of \S \ref{valencequi} is to show that, if $S$ is an $\infty$-category, then there is a 
(canonical) isomorphism between $\Hom^{\rght}_{S}(x,y)$ and $\bHom_{\sCoNerve[S]}(x,y)$
in the homotopy category $\calH$. In particular, we will conclude that
$\Hom^{\rght}_{S}(x,y)$ represents $\bHom_{S}(x,y)$, whenever $S$ is an $\infty$-category.
\end{remark}

\begin{remark}\index{not}{HomLS@$\Hom^{\lft}_{S}(X,Y)$}\label{swink}
The definition of $\Hom^{\rght}_{S}(x,y)$ is not self-dual: that
is, $\Hom^{\rght}_{S^{op}}(x,y) \neq \Hom^{\rght}_{S}(y,x)$ in
general. Instead we define $\Hom^{\lft}_S(x,y) =
\Hom^{\rght}_{S^{op}}(y,x)^{op}$, so that $\Hom^{\lft}_{S}(x,y)_n$ is
the set of all $z \in S_{n+1}$ such that $z|\Delta^{ \{0\} } = x$
and $z | \Delta^{ \{1, \ldots, n+1\} }$ is the constant simplex at
the vertex $y$. \end{remark}

Although the simplicial sets $\Hom^{\lft}_S(x,y)$ and
$\Hom^{\rght}_S(x,y)$ are generally not isomorphic to one another,
they are homotopy equivalent whenever $S$ is an $\infty$-category. To
prove this, it is convenient to define a third, self-dual, space
of morphisms: let $\Hom_{S}(x,y) = \{x\} \times_S S^{\Delta^1}
\times_S \{y\}$. In other words, to give an $n$-simplex of
$\Hom_{S}(x,y)$, one must give a map $f: \Delta^n \times \Delta^1
\rightarrow S$, such that $f| \Delta^n \times \{0\}$ is constant
at $x$ and $f| \Delta^n \times \{1\}$ is constant at $y$. We
observe that there exist natural inclusions
$$ \Hom^{\rght}_{S}(x,y) \hookrightarrow \Hom_S(x,y) \hookleftarrow
\Hom^{\lft}_{S}(x,y),$$ 
which are induced by retracting the cylinder $\Delta^n \times \Delta^1$ onto certain maximal dimensional simplices. We will later show (Corollary
\ref{homsetsagree}) that these inclusions are homotopy
equivalences, provided that $S$ is an $\infty$-category.\index{not}{HomS@$\Hom_{S}(X,Y)$}

\subsection{The Homotopy Category}\label{hcat}

For every ordinary category $\calC$, the nerve $\Nerve(\calC)$ is an $\infty$-category. 
Informally, we can describe the situation as follows: the nerve functor is a fully faithful inclusion from the bicategory of categories to the $\infty$-bicategory of $\infty$-categories.
Moreover, this inclusion has a left adjoint:

\begin{proposition}\label{leftadj}
The nerve functor $\Cat \rightarrow \sSet$ is right adjoint to the functor
$\h: \sSet \rightarrow \Cat$, which associates to every simplicial set $S$ its
homotopy category $\h{S}$ $($here we ignore the $\calH$-enrichment of $\h{S}${}$)$.
\end{proposition}

\begin{proof}
Let us temporarily distinguish between the nerve functor $\Nerve: \Cat \rightarrow \sSet$
and the simplicial nerve functor $\Nerve': \sCat \rightarrow \sSet$. These two functors are related by the fact that $\Nerve$ can be written as a composition
$$ \Cat \stackrel{i}{\subseteq} \sCat \stackrel{\Nerve'}{\rightarrow} \sSet.$$
The functor $\pi_0: \sSet \rightarrow \Set$ is a left adjoint to the inclusion functor
$\Set \rightarrow \sSet$, so the functor
$$ \sCat \rightarrow \Cat$$
$$ \calC \mapsto \h{\calC}$$
is left adjoint to $i$. It follows that $\Nerve = \Nerve' \circ i$ has a left adjoint, given by the composition
$$ \sSet \stackrel{ \sCoNerve[ \bigdot] }{\rightarrow} \sCat \stackrel{ \h{}}{\rightarrow} \Cat,$$
which coincides with the homotopy category functor $\h{}: \sSet \rightarrow \Cat$ by definition.
\end{proof}

\begin{remark}
The formation of the homotopy category is literally left adjoint to
the inclusion $\Cat \subseteq \sCat$. The analogous assertion is not
quite true in the setting of topological categories, since the
functor $\pi_0: \CG \rightarrow \Set$ is a left adjoint only when
restricted to locally path connected spaces.
\end{remark}

\begin{warning}
If $\calC$ is a simplicial category, then we do not necessarily
expect that $\h{\calC} \simeq \h{\sNerve(\calC)}$. However, this is always
the case when $\calC$ is {\it fibrant}, in the sense that every
simplicial set $\bHom_{\calC}(X,Y)$ is a Kan complex.
\end{warning}

\begin{remark}
If $S$ is a simplicial set, Joyal (\cite{joyalnotpub}) refers to the category $\h{S}$ as the {\it fundamental
category} of $S$. This is motivated by the observation that if $S$ is a Kan complex, then $\h{S}$ is
the fundamental groupoid of $S$ in the usual sense.
\end{remark}

Our objective, for the remainder of this section, is to obtain a more explicit understanding of
the homotopy category $\h{S}$ of a simplicial set $S$. Proposition \ref{leftadj} implies that $\h{S}$ 
admits the following presentation by generators and relations:

\begin{itemize}
\item The objects of $\h{S}$ are the vertices of $S$.

\item For every edge $\phi: \Delta^1 \rightarrow S$, there is a morphism $\overline{\phi}$ from
$\phi(0)$ to $\phi(1)$.

\item For each $\sigma: \Delta^2 \rightarrow S$, we have $\overline{d_0(\sigma)} \circ \overline{d_2(\sigma)} = \overline{d_1(\sigma)}$.

\item For each vertex $x$ of $S$, the morphism $\overline{s_0 x}$ is the identity $\id_x$. 
\end{itemize}

If $S$ is an $\infty$-category, there is a much more satisfying construction of the category $\h{S}$.
We will describe this construction in detail, since it nicely illustrates the utility of the weak Kan condition of Definition \ref{qqcc}.

Let $\calC$ be an $\infty$-category. We will construct a category $\pi(\calC)$\index{gen}{homotopy category!of an $\infty$-category}
(which we will eventually show to be equivalent to the homotopy category $\h{\calC}$). The objects of $\pi(\calC)$ are the vertices of $\calC$. Given an
edge $\phi: \Delta^1 \rightarrow \calC$, we shall say that $\phi$ has {\it source}
$C= \phi(0)$ and {\it target} $C' = \phi(1) $, and write $\phi:
C \rightarrow C'$. For each object $C$ of $\calC$, we
let $\id_{C}$ denote the degenerate edge $s_0(C): C \rightarrow C$.

Let $\phi: C \rightarrow C'$ and $\phi': C \rightarrow C'$ be a pair of edges of $\calC$ having
the same source and target. We will say that $\phi$ and $\phi'$ are {\it homotopic} if
there is a $2$-simplex $\sigma: \Delta^2 \rightarrow \calC$, which we depict as follows:
$$ \xymatrix{
& C' \ar[dr]^{\id_{C'}} & \\
C \ar[ur]^{\phi} \ar[rr]^{\phi'} & & C'.}$$
In this case, we say that $\sigma$ is a {\it homotopy} between $\phi$ and $\phi'$.\index{gen}{homotopy!between morphisms of $\calC$}

\begin{proposition}\label{extneeded}
Let $\calC$ be an $\infty$-category, and let $C$ and $C'$ be objects of
$\pi(\calC)$. Then the relation of homotopy is an equivalence relation
on the edges joining $C$ to $C'$.
\end{proposition}

\begin{proof}
Let $\phi: \Delta^1 \rightarrow \calC$ be an edge. Then $s_1(\phi)$ is a homotopy from $\phi$ to
itself. Thus homotopy is a reflexive relation.

Suppose next that $\phi, \phi', \phi'': C \rightarrow C'$ are edges with the same source and target. Let $\sigma$
be a homotopy from $\phi$ to $\phi'$ and $\sigma'$ a homotopy from
$\phi$ to $\phi''$. Let $\sigma'': \Delta^2 \rightarrow \calC$ denote the
constant map at the vertex $C'$. 
We have a commutative diagram
$$ \xymatrix{ \Lambda^3_1 \ar@{^{(}->}[d] \ar[rr]^{ (\sigma'', \bigdot, \sigma', \sigma) } & & \calC \\
\Delta^3 \ar@{-->}[urr]^{\tau}. & & }$$
Since $\calC$ is an $\infty$-category, there exists a $3$-simplex $\tau: \Delta^3 \rightarrow \calC$ as indicated by the dotted arrow in the diagram. It is easy to see that
$d_1(\tau)$ is a homotopy from $\phi'$ to $\phi''$.

As a special case, we may take $\phi=\phi''$; we then deduce that
the relation of homotopy is symmetric. It then follows immediately
from the above that the relation of homotopy is also transitive.
\end{proof}

\begin{remark}\label{cello}
The definition of homotopy that we have given is not evidently self-dual; in other words, 
it is not immediately obvious a homotopic pair of edges $\phi, \phi': C \rightarrow C'$ of an $\infty$-category $\calC$ remain homotopic when regarded as edges in the opposite $\infty$-category $\calC^{op}$. To prove this, let $\sigma$ be a homotopy from $\phi$ to $\phi'$, and consider
the commutative diagram
$$ \xymatrix{ \Lambda^3_2 \ar@{^{(}->}[d] \ar[rr]^{ ( \sigma, s_1 \phi, \bigdot, s_0 \phi)} & & \calC \\
\Delta^3 \ar@{-->}[urr]^{\tau}. & & }$$
The assumption that $\calC$ is an $\infty$-category guarantees a $3$-simplex
$\tau$ rendering the diagram commutative. The face $d_2 \tau$ may be regarded as a homotopy
from $\phi'$ to $\phi$ in $\calC^{op}$.
\end{remark}

We can now define the morphism sets of the category $\pi(\calC)$: given vertices $X$ and $Y$ of $\calC$, we let $\Hom_{\pi(\calC)}(X,Y)$ denote the set of homotopy
classes of edges $\phi: X \rightarrow Y$ in $\calC$. For each edge
$\phi: \Delta^1 \rightarrow \calC$, we let $[ \phi ]$ denote the corresponding morphism in
$\pi(\calC)$.

We define a composition law on $\pi(\calC)$ as follows. Suppose that $X$, $Y$, and $Z$ are vertices of $\calC$, and that we are given 
edges $\phi: X \rightarrow Y$, $ \psi : Y \rightarrow Z$.
The pair $(\phi, \psi)$ determines a map $\Lambda^2_1 \rightarrow \calC$. Since $\calC$ is an $\infty$-category, this map extends to a $2$-simplex $\sigma: \Delta^2 \rightarrow \calC$. We now define
$[ \psi] \circ [ \phi ] = [ d_1 \sigma ]$.

\begin{proposition}\label{trug}
Let $\calC$ be an $\infty$-category.
The composition law on $\pi(\calC)$ is well-defined. In other words, the homotopy class $[\psi] \circ [\phi]$ does not depend on the choice of $\psi$ representing $[\psi]$, the choice of $\phi$ representing $[ \phi]$, or the choice of the the $2$-simplex $\sigma$.
\end{proposition}

\begin{proof}
We begin by verifying the independence of the choice of $\sigma$. Suppose that we are given two $2$-simplices $\sigma, \sigma': \Delta^2 \rightarrow \calC$, satisfying
$$ d_0 \sigma = d_0 \sigma' = \psi $$
$$ d_2 \sigma = d_2 \sigma' = \phi.$$
Consider the diagram
$$ \xymatrix{ \Lambda^3_1 \ar@{^{(}->}[d] \ar[rr]^{( s_1 \psi, \bigdot, \sigma', \sigma)} & & \calC \\
\Delta^3 \ar@{-->}[urr]^{\tau}. & & }$$
Since $\calC$ is an $\infty$-category, there exists a $3$-simplex $\tau$ as indicated by the dotted arrow. It follows that $d_1 \tau$ is a homotopy from $d_1 \sigma$ to $d_1 \sigma'$.

We now show that $[\psi ] \circ [\phi]$ depends only on $\psi$ and $\phi$ only up to homotopy.
In view of Remark \ref{cello}, the assertion is symmetric with respect to $\psi$ and $\phi$; it will therefore suffice to show that $[ \psi ] \circ [ \phi ]$ does not change if we replace $\phi$ by a morphism $\phi'$ which is homotopic to $\phi$. Let $\sigma$ be a $2$-simplex with
$d_0 \sigma = \psi$, $d_2 \sigma = \phi$, and let $\sigma'$ be a homotopy from $\phi$ to $\phi'$.
Consider the diagram
$$ \xymatrix{ \Lambda^3_1 \ar@{^{(}->}[d] \ar[rr]^{ (s_0 \psi, \bigdot, \sigma, \sigma') } & & \calC \\
\Delta^3. \ar@{-->}[urr]^{\tau} & & }$$
Again, the hypothesis that $\calC$ is an $\infty$-category guarantees the existence of a $3$-simplex $\tau$ as indicated in the diagram. Let $\sigma'' = d_1 \tau$. Then 
$[ \psi ] \circ [ \phi' ] = [ d_1 \sigma' ]$. But $d_1 \sigma = d_1 \sigma'$ by construction, so that
$[ \psi ] \circ [ \phi ] = [ \psi ] \circ [ \phi' ]$ as desired.
\end{proof}

\begin{proposition}
If $\calC$ is an $\infty$-category, then $\pi(\calC)$ is a category.
\end{proposition}

\begin{proof}
Let $C$ be a vertex of $\calC$. We first verify that $[\id_{C}]$ is an identity with respect to the composition law on $\pi(\calC)$. For every edge $\phi: C' \rightarrow C$ in $\calC$, the
$2$-simplex $s_1(\phi)$ verifies the equation
$$ [ \id_{C} ] \circ [\phi] = [\phi].$$
This proves that $\id_{C}$ is a left
identity; the dual argument (Remark \ref{cello}) shows that $[\id_{C}]$ is a right
identity.

The only other thing we need to check is the associative law for
composition in $\pi(\calC)$. Suppose given a composable sequence of edges
$$ C \stackrel{\phi}{\rightarrow} C' \stackrel{\phi'}{\rightarrow} C'' \stackrel{\phi''}{\rightarrow} C'''.$$
Choose $2$-simplices $\sigma, \sigma', \sigma'': \Delta^2 \rightarrow \calC$, corresponding to diagrams
$$ \xymatrix{ & C' \ar[dr]^{\phi'} & \\
C \ar[ur]^{\phi} \ar[rr]^{\psi} & & C'' }$$
$$ \xymatrix{ & C'' \ar[dr]^{\phi''} & \\
C \ar[ur]^{\psi} \ar[rr]^{\theta} & & C''' }$$
$$ \xymatrix{ & C'' \ar[dr]^{\phi''} & \\
C' \ar[ur]^{\phi'} \ar[rr]^{\psi'} & & C''',}$$
respectively. Then $[\phi'] \circ [\phi] = [\psi]$, $[\phi''] \circ [\psi] = [\theta]$, and
$[\phi''] \circ [\phi'] = [\psi']$. Consider the diagram
$$ \xymatrix{
\Lambda^3_2 \ar[rr]^{(\sigma'', \sigma', \bigdot, \sigma)} \ar@{^{(}->}[d] & & \calC \\
\Delta^3. \ar@{-->}[urr]^{\tau} & & }$$
Since $\calC$ is an $\infty$-category, there exists a $3$-simplex $\tau$ rendering the diagram commutative. Then $d_2(\tau)$ verifies the equation
$[\psi'] \circ [\phi] = [\theta]$, so that
$$([\phi''] \circ [\phi']) \circ [\phi] = [\theta] = [\phi''] \circ
[\psi] = [\phi''] \circ ([\phi'] \circ [\phi])$$ as desired.
\end{proof}

We now show that if $\calC$ is an $\infty$-category, then $\pi(\calC)$ is
naturally equivalent (in fact isomorphic) to $\h{\calC}$.

\begin{proposition}
Let $\calC$ be an $\infty$-category. There exists a unique functor $F: \h{\calC} \rightarrow \pi(\calC)$ with the following properties:
\begin{itemize}
\item[$(1)$] On objects, $F$ is the identity map.
\item[$(2)$] For every edge $\phi$ of $\calC$, $F( \overline{\phi} ) = [\phi]$.
\end{itemize}
Moreover, $F$ is an isomorphism of categories.
\end{proposition}

\begin{proof}
The existence and uniqueness of $F$ follows immediately from our presentation
of $\h{\calC}$ by generators and relations. It is obvious that $F$ is bijective on objects and surjective on morphisms. To complete the proof, it will suffice to show that $F$ is faithful.

We first show that every morphism $f: x \rightarrow y$ in $\h{\calC}$ may be written as $\overline{\phi}$ for some $\phi \in \calC$. Since the morphisms in $\h{\calC}$ are generated by morphisms having the form $\overline{\phi}$ under composition, it suffices to show that the set of such morphisms contains all identity morphisms and is stable under composition. The first assertion is clear, since $\overline{ s_0 x}=\id_x$. For the second, we note that if $\phi: x \rightarrow y$ and $\phi': y \rightarrow z$ are composable edges, then there exists a 2-simplex $\sigma: \Delta^2 \rightarrow \calC$, which we may depict as follows:
$$ \xymatrix{ & y \ar[dr]^{\phi'} & \\
x \ar[ur]^{\phi} \ar[rr]^{\psi} & & z. }$$
Thus $\overline{ \phi' } \circ \overline{\phi} = \overline{ \psi }$.

Now suppose that $\phi,\phi': x \rightarrow y$ are such that $[\phi]=[\phi']$; we wish to show that $\overline{\phi}=\overline{\phi'}$. By definition, there exists a homotopy $\sigma: \Delta^2 \rightarrow \calC$ joining $\phi$ and $\phi'$. The existence of $\sigma$ entails the relation
$$ \id_y \circ \overline{\phi} = \overline{\phi'} $$ in the homotopy category $\h{S}$, so that
$\overline{\phi} = \overline{\phi'}$ as desired.
\end{proof}

\subsection{Objects, Morphisms and Equivalences}\label{obmor}
\begin{Didn't Read}

As in ordinary category theory, we may speak of {\it objects} and {\it morphisms} in a higher category $\calC$. If $\calC$ is a topological (or simplicial) category, these
should be understood literally as the objects and morphisms in the
underlying category of $\calC$. We may also apply this terminology
to $\infty$-categories (or even more general simplicial sets): if $S$ is a simplicial set, then the {\it
objects} of $S$ are the vertices $\Delta^0 \rightarrow S$, and the {\it morphisms}
of $S$ are edges $\Delta^1 \rightarrow S$. A morphism $\phi: \Delta^1 \rightarrow S$ is said to have {\it source} $X= \phi(0)$ and {\it target} $Y= \phi(1)$;
we will often denote this by writing $\phi: X \rightarrow Y$.\index{gen}{object!of an $\infty$-category}\index{gen}{morphism!in an $\infty$-category}
If $X: \Delta^0 \rightarrow S$ is an object of $S$, we will write
$\id_X = s_0(X): X \rightarrow X$ and refer to this as the {\it identity morphism} of $X$.

If $f,g: X \rightarrow Y$ are two morphisms in a higher category $\calC$, then $f$ and $g$
are {\it homotopic} if they determine the same morphism in the homotopy category $\h{\calC}$. In the setting of $\infty$-categories, this coincides with the notion of homotopy introduced in the previous section. In the setting of
topological categories, this simply means that $f$ and $g$ lie
in the same path component of $\bHom_{\calC}(X,Y)$. In either case,
we will sometimes indicate this relationship between $f$ and $g$ by writing $f
\simeq g$.

A morphism $f: X \rightarrow Y$ in an $\infty$-category $\calC$ is
said to be an {\it equivalence} if it determines an isomorphism in the homotopy category
$\h{\calC}$. We say that $X$ and $Y$ are {\it equivalent} if there is
an equivalence between them (in other words, if they are
isomorphic as objects of $\h{\calC}$).\index{gen}{equivalence!in an $\infty$-category}

If $\calC$ is a topological category, then the
requirement that a morphism $f: X \rightarrow Y$ be an equivalence
is quite a bit weaker than the requirement that $f$ be an
isomorphism. In fact, we have the following:

\begin{proposition}\label{rooot}\index{gen}{equivalence!in a topological category}
Let $f: X \rightarrow Y$ be a morphism in a topological category.
The following conditions are equivalent:

\begin{itemize}
\item[$(1)$] The morphism $f$ is an equivalence.

\item[$(2)$] The morphism $f$ has a {\it homotopy inverse} $g: Y
\rightarrow X$; that is, a morphism $g$ such that $f \circ g
\simeq \id_Y$ and $g \circ f \simeq \id_X$.

\item[$(3)$] For every object $Z \in \calC$, the induced map
$\bHom_{\calC}(Z,X) \rightarrow \bHom_{\calC}(Z,Y)$ is a homotopy
equivalence.

\item[$(4)$] For every object $Z \in \calC$, the induced map
$\bHom_{\calC}(Z,X) \rightarrow \bHom_{\calC}(Z,Y)$ is a weak
homotopy equivalence.

\item[$(5)$] For every object $Z \in \calC$, the induced map
$\bHom_{\calC}(Y,Z) \rightarrow \bHom_{\calC}(X,Z)$ is a homotopy
equivalence.

\item[$(6)$] For every object $Z \in \calC$, the induced map
$\bHom_{\calC}(Y,Z) \rightarrow \bHom_{\calC}(X,Z)$ is a weak
homotopy equivalence.
\end{itemize}
\end{proposition}

\begin{proof}
It is clear that $(2)$ is merely a reformulation of $(1)$. We will
show that $(2) \Rightarrow (3) \Rightarrow (4) \Rightarrow (1)$;
the implications $(2) \Rightarrow (5) \Rightarrow (6) \Rightarrow
(1)$ follow using the same argument.

To see that $(2)$ implies $(3)$, we note that if $g$ is a homotopy
inverse to $f$, then composition with $g$ gives a map
$\bHom_{\calC}(Z,Y) \rightarrow \bHom_{\calC}(Z,X)$ which is
homotopy inverse to composition with $f$. It is clear that $(3)$
implies $(4)$. Finally, if $(4)$ holds, then we note that $X$ and
$Y$ represent the same functor on $\h{\calC}$ so that $f$ induces an
isomorphism between $X$ and $Y$ in $\h{\calC}$.
\end{proof}

\begin{example}
Let $\calC$ be the category of CW-complexes, considered as a topological category by endowing
each of the sets $\Hom_{\calC}(X,Y)$ with the (compactly generated) compact open topology. A pair of objects $X,Y \in \calC$ are equivalent (in the sense defined above) if and only if they are homotopy equivalent (in the sense of classical topology).
\end{example}

If $\calC$ is an $\infty$-category (topological category, simplicial category), then we
shall write $X \in \calC$ to mean that $X$ is an object of
$\calC$. We will generally understand that all meaningful properties of
objects are invariant under equivalence. Similarly, all
meaningful properties of morphisms are invariant under
homotopy and under composition with equivalences.

In the setting of $\infty$-categories, there is a very useful characterization of equivalences which is due to Joyal.

\begin{proposition}[Joyal \cite{joyalnotpub}]\label{greenlem}\index{gen}{equivalence!in an $\infty$-category}
Let $\calC$ be an $\infty$-category, and $\phi: \Delta^1 \rightarrow \calC$ a morphism of $\calC$. Then $\phi$ is an equivalence if and only if, for every $n \geq 2$ and every map
$f_0: \Lambda^n_0 \rightarrow \calC$ such that $f_0 | \Delta^{\{0,1\}} = \phi$,
there exists an extension of $f_0$ to $\Delta^n$.
\end{proposition}

The proof requires some ideas which we have not yet introduced, and will be given in \S \ref{leftfib}.

\subsection{$\infty$-Groupoids and Classical Homotopy Theory}

Let $\calC$ be an $\infty$-category. We will say that $\calC$ is an {\it $\infty$-groupoid} if the homotopy category $\h{\calC}$ is a groupoid: in other words, if every morphism in $\calC$ is an equivalence. In \S \ref{highcat}, we asserted that the theory of $\infty$-groupoids is equivalent to classical homotopy theory. We can now formulate this idea in a very precise way:

\begin{proposition}[Joyal \cite{joyalpub}]\label{greenwich}\index{gen}{$\infty$-groupoid}
Let $\calC$ be a simplicial set. The following conditions are
equivalent:

\begin{itemize}
\item[$(1)$] The simplicial set $\calC$ is an $\infty$-category and its homotopy category $\h{\calC}$ is a groupoid.

\item[$(2)$] The simplicial set $\calC$ satisfies the extension condition
for all horn inclusions $\Lambda^n_i \subseteq \Delta^n$ for $0 \leq i < n$.

\item[$(3)$] The simplicial set $\calC$ satisfies the extension condition
for all horn inclusions $\Lambda^n_i \subseteq \Delta^n$ for $0 < i \leq n$.

\item[$(4)$] The simplicial set $\calC$ is a Kan complex; in other words, it
satisfies the extension condition for all horn inclusions
$\Lambda^n_i \subseteq \Delta^n$ for $0 \leq i \leq n$.
\end{itemize}
\end{proposition}

\begin{proof}
The equivalence $(1) \Leftrightarrow (2)$ follows immediately from
Proposition \ref{greenlem}.
Similarly, the equivalence $(1) \Leftrightarrow (3)$ follows by applying Proposition \ref{greenlem} to $\calC^{op}$. We conclude by observing that $(4) \Leftrightarrow (2) \wedge (3)$.
\end{proof}

\begin{remark}
The assertion that we can identify $\infty$-groupoids with spaces is less obvious in other formulations of higher category theory. For example, suppose that $\calC$ is a topological category whose homotopy category $\h{\calC}$ is a groupoid. For simplicity, we will assume furthermore that $\calC$ has a single object $X$. We may then identify $\calC$ with
the topological monoid $M=\Hom_{\calC}(X,X)$. The assumption that
$\h{ \calC}$ is a groupoid is equivalent to the assumption
that the discrete monoid $\pi_0 M$ is a group. In this case, one can show that the 
unit map $M \rightarrow \Omega BM$ is a weak homotopy
equivalence, where $BM$ denotes the classifying space of the
topological monoid $M$. In other words, up to equivalence,
specifying $\calC$ (together with the object $X$) is equivalent to
specifying the space $BM$ (together with its base point).
\end{remark}

Informally, we might say that the inclusion functor $i$ from Kan
complexes to $\infty$-categories exhibits the $\infty$-category of
(small) $\infty$-groupoids as a full subcategory of
the $\infty$-bicategory of (small) $\infty$-categories. Conversely, every
$\infty$-category $\calC$ has an ``underlying'' $\infty$-groupoid, which is obtained by discarding the noninvertible morphisms of $\calC$:\index{gen}{$\infty$-groupoid!underlying an $\infty$-category}

\begin{proposition}[\cite{joyalnotpub}]\label{lumba}
Let $\calC$ be an $\infty$-category. Let $\calC' \subseteq \calC$ be the largest simplicial
subset of $\calC$ having the property that every edge of $\calC'$ is an equivalence in $\calC$.
Then $\calC'$ is a Kan complex. It may be characterized by the
following universal property: for any Kan complex $K$, the induced
map $\Hom_{\sSet}(K,\calC') \rightarrow \Hom_{\sSet}(K,\calC)$ is a
bijection.
\end{proposition}

\begin{proof}
It is straightforward to check that $\calC'$ is an $\infty$-category. Moreover, if
$f$ is a morphism in $\calC'$, then $f$ has a homotopy inverse $g \in \calC$. Since
$g$ is itself an equivalence in $\calC$, we conclude that $g$ belongs to $\calC'$ and
is therefore a homotopy inverse to $f$ in $\calC'$. In other words, every morphism in $\calC'$ is an equivalence, so that $\calC'$ is a Kan complex by Proposition \ref{greenwich}. To prove the last assertion, we observe that if $K$ is an $\infty$-category, then any map of simplicial sets
$\phi: K \rightarrow \calC$ carries equivalences in $K$ to equivalences in $\calC$. In particular, if
$K$ is a Kan complex, then $\phi$ factors (uniquely) through $\calC'$.
\end{proof}

It follows from Proposition \ref{lumba} that the functor
$\calC \mapsto \calC'$ is right adjoint to
the inclusion functor from Kan complexes to $\infty$-categories. It is easy to see that this right adjoint is an invariant notion: that is, a categorical equivalence of $\infty$-categories $\calC \rightarrow \calD$ induces a homotopy equivalence
$\calC' \rightarrow \calD'$ of Kan complexes. 

\begin{remark}
It is easy to give analogous constructions in the case of topological or simplicial categories. For example, if $\calC$ is a topological category, then we can define $\calC'$ to be another topological category with the same objects as $\calC$, where $\bHom_{\calC'}(X,Y) \subseteq \bHom_{\calC}(X,Y)$ is the subspace consisting of equivalences in $\bHom_{\calC}(X,Y)$, equipped with the subspace topology.
\end{remark}

\begin{remark}
We will later introduce a relative version of the construction described in Proposition \ref{lumba}, which applies to certain families of $\infty$-categories (Corollary \ref{relativeKan}).
\end{remark}

Although the inclusion functor from Kan complexes to $\infty$-categories does not
literally have a left adjoint, it does have a left adjoint in a higher-categorical sense. This left adjoint is computed by any ``fibrant replacement'' functor (for the usual model structure) from $\sSet$ to itself, for
example the functor $S \mapsto \Sing |S|$.
The unit map $u: S \rightarrow
\Sing |S|$ is always a weak homotopy equivalence, but generally not a categorical equivalence. For example, if $S$ is an $\infty$-category, then $u$ is a categorical equivalence if and only if $S$ is a Kan complex. In general, $\Sing |S|$ may be regarded as the $\infty$-groupoid
obtained from $S$ by freely adjoining inverses to all the
morphisms in $S$.

\begin{remark}
The inclusion functor $i$ and its homotopy-theoretic left adjoint
may also be understood using the formalism of {\it
localizations of model categories}. In addition to its usual model
category structure, the category $\sSet$ of simplicial sets may be
endowed with the {\it Joyal model structure} which we will define in \S
\ref{compp3}. These model structures have the same cofibrations (in both cases, the
cofibrations are simply the monomorphisms of simplicial sets).
However, the Joyal model structure has fewer weak equivalences
(categorical equivalences, rather than weak homotopy equivalences)
and consequently more fibrant objects (all $\infty$-categories,
rather than only Kan complexes). It follows that the usual
homotopy theory of simplicial sets is a
localization of the homotopy theory of $\infty$-categories. The
identity functor from $\sSet$ to itself determines a Quillen
adjunction between these two homotopy theories, which plays the
role of $i$ and its left adjoint.
\end{remark}

\subsection{Homotopy Commutativity versus Homotopy Coherence}\label{comcoh}

Let $\calC$ be an $\infty$-category (topological category, simplicial category). 
To a first approximation,
working in $\calC$ is like working in its homotopy category $\h{
\calC}$: up to equivalence, $\calC$ and $\h{\calC}$ have the same
objects and morphisms. The main difference between $\h{\calC}$ and
$\calC$ is that in $\calC$, one must not ask whether or not
morphisms are {\em equal}; instead one should ask whether or not they are {\it homotopic}. If so, the homotopy itself is an additional datum which we will need to consider. Consequently, the notion of a commutative diagram in
$\h{\calC}$, which corresponds to a {\it homotopy commutative}
diagram in $\calC$, is quite unnatural and usually needs to be
replaced by the more refined notion of a {\it homotopy coherent}
diagram in $\calC$.\index{gen}{homotopy coherence}\index{gen}{diagram!homotopy commutative}\index{gen}{diagram!homotopy coherent}

To understand the problem, let us suppose that $F: \calI
\rightarrow \calH$ is a functor from an ordinary category $\calI$
into the homotopy category of spaces $\calH$. In other words, $F$
assigns to each object $X \in \calI$ a space (say, a CW complex)
$F(X)$, and to each morphism $\phi: X \rightarrow Y$ in $\calI$ a
continuous map of spaces $F(\phi): F(X) \rightarrow F(Y)$ (well-defined
up to homotopy), such that $F(\phi \circ \psi)$ is homotopic to
$F(\phi) \circ F(\psi)$ for any pair of composable morphisms $\phi,
\psi$ in $\calI$. In this situation, it may or may not be possible
to {\em lift} $F$ to an actual functor $\widetilde{F}$ from
$\calI$ to the ordinary category of topological spaces, such that
$\widetilde{F}$ induces a functor $\calI \rightarrow \calH$ which
is naturally isomorphic to $F$. In general there are obstructions
to both the existence and the uniqueness of the lifting
$\widetilde{F}$, even up to homotopy. To see this, let us suppose for a moment that
$\widetilde{F}$ exists, so that there exist homotopies
$k_{\phi}: \widetilde{F}(\phi) \simeq F(\phi)$. These homotopies determine {\em additional}
data on $F$: namely, one obtains a canonical homotopy $h_{\phi,\psi}$ from $F(\phi \circ \psi)$
to $F(\phi) \circ F(\psi)$ by composing
$$ F(\phi \circ \psi) \simeq \widetilde{F}(\phi \circ \psi) = \widetilde{F}(\phi) \circ \widetilde{F}(\psi)
\simeq F(\phi) \circ F(\psi).$$
The functor $F$ to the homotopy category
$\calH$ should be viewed as a first approximation to $\widetilde{F}$; we obtain a
second approximation when we take into account the homotopies
$h_{\phi, \psi}$. These homotopies are not arbitrary: the
associativity of composition gives a relationship between
$h_{\phi, \psi}, h_{\psi, \theta}, h_{\phi, \psi \circ \theta}$
and $h_{\phi \circ \psi, \theta}$, for a composable triple of
morphisms $(\phi, \psi, \theta)$ in $\calI$. This relationship may
be formulated in terms of the existence of a certain higher
homotopy, which is once again canonically determined by
$\widetilde{F}$ (and the homotopies $k_{\phi}$). To obtain the next approximation to
$\widetilde{F}$, we should take these higher homotopies into
account, and formulate the associativity properties that {\em
they} enjoy, and so on. Roughly speaking, a {\it homotopy coherent} diagram in
$\calC$ is a functor $F: \calI \rightarrow
\h{\calC}$, together with all of the extra data that would be
available if we were able to lift $F$ to a functor $\widetilde{F}: \calI \rightarrow \calC$.

The distinction between homotopy commutativity and homotopy coherence is arguably the {\em main} difficulty in working with higher categories. The idea of homotopy coherence is simple enough, and can be made precise in the setting of a general topological category. However, the amount of data required to specify a homotopy coherent diagram is considerable,
so the concept is quite difficult to employ in practical situations.

\begin{remark}
Let $\calI$ be an ordinary category and $\calC$ a topological category. Any functor
$F: \calI \rightarrow \calC$ determines a homotopy coherent diagram in $\calC$ (with all of the homotopies involved being constant). For many topological categories $\calC$, the converse fails: not every homotopy-coherent diagram in $\calC$ can be obtained in this way, even up to equivalence. In these cases, it is the notion of {\it homotopy coherent} diagram which is fundamental; a homotopy coherent diagram should be regarded as ``just as good'' as a strictly commutative diagram, for $\infty$-categorical purposes. As evidence for this, we remark that given an equivalence $\calC' \rightarrow \calC$, a strictly commutative diagram $F: \calI \rightarrow \calC$ cannot always be lifted to a strictly commutative diagram in $\calC'$; however it can always be lifted (up to equivalence) to a homotopy coherent diagram in $\calC'$.
\end{remark}

One of the advantages of working with $\infty$-categories is that the
definition of a homotopy coherent diagram is easy to formulate. We can simply define a homotopy coherent diagram in an $\infty$-category $\calC$ to
be a map of simplicial sets $f: \Nerve(\calI) \rightarrow \calC$. The
restriction of $f$ to simplices of low dimension encodes the
induced map on homotopy categories. Specifying $f$ on
higher-dimensional simplices gives precisely the ``coherence
data'' that the above discussion calls for.

\begin{remark}\label{remmt}
Another possible approach to the problem of homotopy coherence is
to restrict our attention to simplicial (or topological) categories $\calC$ in
which every homotopy coherent diagram is equivalent to a strictly commutative diagram. For example, this is always true when $\calC$ arises from a simplicial model category (Proposition \ref{gumby444}). Consequently, in the framework of model categories it is possible to ignore the theory of homotopy coherent diagrams, and work with strictly commutative diagrams instead. This approach is quite powerful, particularly when combined with the observation that every simplicial category $\calC$
admits a fully faithful embedding into a simplicial model category (for example, one can use
a simplicially enriched version of the Yoneda embedding). This idea can be used to show that
every homotopy coherent diagram in $\calC$ can be ``straightened'' to a commutative diagram, possibly after replacing $\calC$ by an equivalent simplicial category (for a more precise version
of this statement, we refer the reader to Corollary \ref{strictify}).
\end{remark}

\subsection{Functors between Higher Categories}\label{funcback}

The notion of a homotopy coherent diagram in an higher category
$\calC$ is a special case of the more general notion of a functor
$F: \calI \rightarrow \calC$ between higher categories
(specifically, it is the special case in which $\calI$ is assumed
to be an ordinary category). Just as the collection of all
ordinary categories forms a bicategory (with functors as
morphisms and natural transformations as $2$-morphisms), the
collection of all $\infty$-categories can
be organized into an $\infty$-bicategory. In particular, for any
$\infty$-categories $\calC$ and $\calC'$, we expect to be able to construct
an $\infty$-category $\Fun(\calC,\calC')$ of functors from $\calC$ to
$\calC'$.

In the setting of topological categories, the construction of an appropriate mapping object $\Fun(\calC, \calC')$ is quite difficult. The naive guess is that $\Fun(\calC, \calC')$ should be 
a category of topological functors from $\calC$ to $\calC'$: that is, functors which induce continuous maps between morphism spaces. However, we saw in \S \ref{comcoh} that this notion is generally too rigid, even in the special case where $\calC$ is an ordinary category.

\begin{remark}
Using the language of model categories, one might say that the
problem is that not every topological category is {\it cofibrant}.
If $\calC$ is a ``cofibrant'' topological category (for example, if $\calC =
|\sCoNerve[S]|$ where $S$ is a simplicial set), then the collection of topological
functors from $\calC$ to $\calC'$ is large enough to contain representatives for
every $\infty$-categorical functor from $\calC$ to $\calC'$. Most
ordinary categories are not cofibrant when viewed as topological categories.
More importantly, the property of being cofibrant is not stable under products, so that naive attempts to construct a mapping object $\Fun(\calC, \calC')$ need not give the correct answer even when $\calC$ itself is assumed cofibrant (if $\calC$ is cofibrant, then we are guaranteed to have ``enough'' topological functors $\calC \rightarrow \calC'$ to represent all functors between the underlying $\infty$-categories, but not necessarily enough natural transformations between them; note that the product $\calC \times [1]$ is usually not cofibrant, even in the simplest nontrivial case where $\calC = [1]$.) This is arguably the most important technical disadvantage of the theory of topological (or simplicial) categories as an approach to higher category theory.
\end{remark}

The construction of functor categories is much easier to describe in the framework of $\infty$-categories. If $\calC$ and $\calD$ are
$\infty$-categories, then we can simply define a {\it functor} from
$\calC$ to $\calD$ to be a map $p: \calC \rightarrow \calD$ of simplicial sets.\index{gen}{functor!between $\infty$-categories}

\begin{notation}\index{not}{Fun@$\Fun(\calC, \calC')$}
Let $\calC$ and $\calD$ be simplicial sets. We let $\Fun(\calC, \calD)$ denote the
simplicial set $\bHom_{\sSet}(\calC, \calD)$ parametrizing maps from $\calC$ to $\calD$.
We will use this notation only when $\calD$ is an $\infty$-category (the simplicial set $\calC$ will often, but not always, be an $\infty$-category as well). We will refer to $\Fun(\calC, \calD)$ as the
{\it $\infty$-category of functors from $\calC$ to $\calD$} (see Proposition \ref{tyty} below).
We will refer to morphisms in $\Fun(\calC, \calD)$ as {\it natural transformations} of functors, and
equivalences in $\Fun(\calC, \calD)$ as {\it natural equivalences}.\index{gen}{natural transformation}\index{gen}{natural equivalence}
\end{notation}

\begin{proposition}\label{tyty}
Let $K$ be an arbitrary simplicial set.
\begin{itemize}
\item[$(1)$] For every $\infty$-category $\calC$, the simplicial set $\Fun(K,\calC)$ is an $\infty$-category.

\item[$(2)$] Let $\calC \rightarrow \calD$ be a categorical equivalence of $\infty$-categories. Then the induced map $\Fun(K,\calC) \rightarrow \Fun(K,\calD)$ is a categorical equivalence.

\item[$(3)$] Let $\calC$ be an $\infty$-category, and $K \rightarrow K'$ a categorical equivalence of simplicial sets. Then the induced map $\Fun(K',\calC) \rightarrow \Fun(K,\calC)$ is a categorical equivalence.
\end{itemize}
\end{proposition}

The proof makes use of the Joyal model structure on $\sSet$, and will be given in \S \ref{compp3}.

\subsection{Joins of $\infty$-Categories}\label{join}

Let $\calC$ and $\calC'$ be ordinary categories. We will define a
new category $\calC \join \calC'$, called the {\it join} of $\calC$ and
$\calC'$. An object of $\calC \join \calC'$ is either an object of
$\calC$ or an object of $\calC'$. The morphism sets are given as follows:
 $$\Hom_{\calC \join \calC'}(X,Y) = \begin{cases} \Hom_{\calC}(X,Y) & \text{if } X,Y \in \calC \\
\Hom_{\calC'}(X,Y) & \text{if } X,Y \in \calC' \\
\emptyset & \text{if } X \in \calC', Y \in \calC \\
\ast & \text{if } X \in \calC, Y \in \calC'. \end{cases}$$\index{gen}{join!of categories}
Composition of morphisms in $\calC \join \calC'$ is defined in the
obvious way. 

The join construction described above is often useful when discussing diagram categories, limits, and colimits. In this section, we will introduce a generalization of this construction to the $\infty$-categorical setting.

\begin{definition}
If $S$ and $S'$ are
simplicial sets, then the simplicial set $S \star S'$\index{not}{Star@$S \star S'$} is defined as
follows: for each nonempty finite linearly ordered set $J$, we set
$$(S \star S')(J) = \coprod_{J = I \cup I'} S(I) \times
S'(I'),$$ where the union is taken over all decompositions of $J$ into disjoint subsets $I$ and $I'$, satisfying $i < i'$ for all $i \in I$, $i' \in I'$. Here we allow the
possibility that either $I$ or $I'$ is empty, in which case we agree to
the convention that $S(\emptyset) = S'(\emptyset) = \ast$.\index{gen}{join!of simplicial sets}
\end{definition}

More concretely, we have $$(S \star S')_{n} =
S_n \cup S'_n \cup \bigcup_{i+j = n-1} S_i \times S'_j.$$

The join operation endows $\sSet$ with the
structure of a monoidal category (see \S \ref{monoidaldef}).
The identity for the join operation is
the empty simplicial set $\emptyset = \Delta^{-1}$. More generally, we have
natural isomorphisms $\phi_{ij}: \Delta^{i-1} \star \Delta^{j-1} \simeq
\Delta^{(i+j)-1}$, for all $i, j \geq 0$.

\begin{remark}
The operation $\star$ is essentially determined by the isomorphisms
$\phi_{ij}$, together with its behavior under the formation of
colimits: for any fixed simplicial set $S$, the functors
$$ T \mapsto T \star S$$
$$ T \mapsto S \star T$$
commute with colimits, when regarded as functors from $\sSet$ to
the undercategory $(\sSet)_{S/}$ of simplicial sets {\em under} $S$.
\end{remark}

Passage to the nerve carries joins of
categories into joins of simplicial sets. More precisely, for every pair of
categories $\calC$ and $\calC'$, there is
a canonical isomorphism $$\Nerve( \calC \join \calC') \simeq
\Nerve(\calC) \join \Nerve(\calC').$$ (The existence of this
isomorphism persists when we allow $\calC$ and $\calC'$ to be a simplicial or
topological categories and apply the appropriate generalization of
the nerve functor.) This suggests that the join operation on
simplicial sets is the appropriate $\infty$-categorical analogue of
the join operation on categories.

We remark that the formation of joins does not commute with the
functor $\sCoNerve[\bigdot]$. However, the simplicial category $\sCoNerve[S \star S']$
contains $\sCoNerve[S]$ and $\sCoNerve[S']$ as full (topological)
subcategories, and contains no morphisms from objects of
$\sCoNerve[S']$ to objects of $\sCoNerve[S]$. Consequently, there is unique map $\phi: \sCoNerve[S \star S'] \rightarrow \sCoNerve[S] \star \sCoNerve[S']$ which reduces to the identity on
$\sCoNerve[S]$ and $\sCoNerve[S']$. We will later show that $\phi$ is an equivalence of simplicial categories (Corollary \ref{diamond3}).

We conclude by recording a pleasant property of the join
operation:

\begin{proposition}[Joyal \cite{joyalnotpub}]
If $S$ and $S'$ are $\infty$-categories, then $S \star S'$ is an $\infty$-category.
\end{proposition}

\begin{proof}
Let $p: \Lambda^n_i \rightarrow S \star S'$ be a map, with $0 < i
< n$. If $p$ carries $\Lambda^n_i$ entirely into $S \subseteq S
\star S'$ or into $S' \subseteq S \star S'$, then we deduce
the existence an extension of $p$ to $\Delta^n$ by invoking
the assumption that $S$ and $S'$ are $\infty$-categories. Otherwise,
we may suppose that $p$ carries the vertices $\{0, \ldots, j\}$
into $S$, and the vertices $\{ j+1, \ldots, n\}$ into $S'$. 
We may now restrict $p$ to obtain maps
$$ \Delta^{ \{ 0, \ldots, j \} } \rightarrow S$$
$$ \Delta^{ \{ j+1, \ldots, n \} } \rightarrow S',$$
which together determine a map $\Delta^n \rightarrow S \star S'$ extending $p$.
\end{proof}

\begin{notation}
Let $K$ be a simplicial set. The {\it left cone} $K^{\triangleleft}$ is defined to be
the join $\Delta^0 \join K$. Dually, the {\it right cone} $K^{\triangleright}$ is defined to be the join $K \join \Delta^0$. Either cone contains a distinguished vertex (belonging to $\Delta^0$), which we will refer to as the {\it cone point}.\index{gen}{cone point}\index{not}{Kleft@$K^{\triangleleft}$}\index{not}{Kright@$K^{\triangleright}$}
\end{notation}

\subsection{Overcategories and Undercategories}\label{slices}

Let $\calC$ be an ordinary category, and $X \in \calC$ an object.
The {\it overcategory} $\calC_{/X}$ is defined as follows:\index{gen}{overcategory}
the objects of $\calC_{/X}$ are
morphisms $Y \rightarrow X$ in $\calC$ having target $X$.
Morphisms are given by commutative triangles
$$\xymatrix{ Y \ar[dr] \ar[rr] & & Z \ar[dl] \\
& X }$$
and composition is defined in the obvious way.

One can rephrase the definition of the overcategory as follows.
Let $[0]$ denote the category with a single object, possessing
only an identity morphism. Then specifying an object $X \in \calC$
is equivalent to specifying a functor $x: [0] \rightarrow
\calC$. The overcategory $\calC_{/X}$ may then be described by
the following universal property: for any category $\calC'$, we
have a bijection
$$ \Hom(\calC', \calC_{/X}) \simeq \Hom_{x}(\calC' \join [0],
\calC),$$ where the subscript on the right hand side indicates
that we consider only those functors $\calC' \join [0]
\rightarrow \calC$ whose restriction to $[0]$ coincides with
$x$.

We would like to generalize the construction of overcategories to the $\infty$-categorical setting. 
Let us begin by working in the framework of topological categories. In this case, there is a natural candidate for the relevant overcategory. Namely, if $\calC$ is a topological category containing an object $X$, then the overcategory $\calC_{/X}$ (defined as above) has the structure of a topological category, where each morphism space $\bHom_{ \calC_{/X} }(Y,Z)$ is topologized as a subspace
of $\bHom_{\calC}(Y,Z)$ (here we are identifying an object of $\calC_{/X}$ with its image in $\calC$). This topological category is usually {\em
not} a model for the correct $\infty$-categorical slice
construction. The problem is that a morphism in $\calC_{/X}$ consists
of a commutative triangle
$$\xymatrix{ Y \ar[dr] \ar[rr] & & Z \ar[dl] \\
& X }$$
of objects over $X$. To obtain the correct notion, we should allow also
triangles which commute only up to homotopy. 

\begin{remark}
In some cases, the naive overcategory $\calC_{/X}$ is a good approximation
to the correct construction: see Lemma \ref{tulmand}.
\end{remark}

In the setting of $\infty$-categories, Joyal has given a much simpler description of the desired construction (see \cite{joyalpub}). This description will play a vitally important role throughout this book. We begin by noting the following:

\begin{proposition}[\cite{joyalpub}]\label{tyrii}
Let $S$ and $K$ be simplicial sets, and $p: K \rightarrow S$ an
arbitrary map. There exists a simplicial set $S_{/p}$ with the following universal property:
$$\Hom_{\sSet}(Y, S_{/p}) = \Hom_{p}( Y \join K, S),$$
where the subscript on the right hand side indicates that we
consider only those morphisms $f: Y \join K \rightarrow S$ such
that $f|K = p$.
\end{proposition}

\begin{proof}
One defines $(S_{/p})_n$ to be $\Hom_{p}(\Delta^n \join K, S)$.
The universal property holds by definition when $Y$ is a simplex.
It holds in general because both sides are compatible with the
formation of colimits in $Y$.
\end{proof}

Let $p: K \rightarrow S$ be as in Proposition \ref{tyrii}. If $S$ is an $\infty$-category, we will refer to $S_{/p}$ as an {\it overcategory} of $S$, or as the {\it $\infty$-category of objects of $S$ over $p$}.
The following result guarantees that the operation of passing to overcategories is well-behaved:\index{gen}{overcategory!of an $\infty$-category}\index{not}{calC/p@$\calC_{/p}$}

\begin{proposition}\label{gorban3}
Let $p: K \rightarrow \calC$ be a map of simplicial sets, and suppose
that $\calC$ is an $\infty$-category. Then $\calC_{/p}$ is an $\infty$-category.
Moreover, if $q: \calC \rightarrow \calC'$ is a categorical equivalence of
$\infty$-categories, then the induced map $\calC_{/p} \rightarrow \calC'_{/qp}$ is a categorical equivalence
as well.
\end{proposition}

The proof requires a number of ideas which we have not yet introduced, and will be postponed (see Proposition \ref{gorban4} for the first assertion and \S \ref{slim} for the second).

\begin{remark}
Let $\calC$ be an $\infty$-category.
In the particular case where $p: \Delta^n \rightarrow \calC$ classifies an
$n$-simplex $\sigma \in \calC_n$, we will often write $\calC_{/\sigma}$ in place of of $\calC_{/p}$.
In particular, if $X$ is an object of $\calC$, we let $\calC_{/X}$ denote the overcategory
$\calC_{/p}$, where $p: \Delta^0 \rightarrow \calC$ has image $X$.\index{not}{calC/X@$\calC_{/X}$}
\end{remark}

\begin{remark} Let $p: K \rightarrow \calC$ be a map of simplicial sets. The
preceding discussion can be dualized, replacing $Y \star K$
by $K \star Y$; in this case we denote the corresponding simplicial set
by $\calC_{p/}$ which (if $\calC$ is an $\infty$-category) we will refer to as an
{\it undercategory} of $\calC$. In the special case where $K = \Delta^n$ and $p$ classifies a simplex $\sigma \in \calC_{n}$, we will also write $\calC_{\sigma/}$ for $\calC_{p/}$; in particular, we will write $\calC_{X/}$ when $X$ is an object of $\calC$.\index{not}{calCp/@$\calC_{p/}$}\index{not}{calCX/@$\calC_{X/}$}\index{gen}{undercategory!of an $\infty$-category}
\end{remark}

\begin{remark}
If $\calC$ is an ordinary category and $X \in \calC$,
then there is a canonical isomorphism $\Nerve(\calC)_{/X} \simeq \Nerve (\calC_{/X})$. In other words, the overcategory construction for $\infty$-categories can be regarded as a {\em generalization} of the relevant construction from classical category theory.
\end{remark}

\subsection{Fully Faithful and Essentially Surjective Functors}

\begin{definition}\label{faulfa}\index{gen}{essentially surjective}
Let $F: \calC \rightarrow \calD$ be a functor between topological categories (simplicial categories, simplicial sets). We will say that $F$ is {\it essentially surjective} if the induced functor 
$\h{F}: \h{\calC} \rightarrow \h{\calD}$ is essentially surjective (that is, if every object of $\calD$ is
equivalent to $F(X)$ for some $X \in \calC$). 

We will say that $F$ is {\it fully faithful} if $\h{F}$ is a fully faithful functor of $\calH$-enriched categories. In other words, $F$ is fully faithful if and only if, for every pair of objects
$X, Y \in \calC$, the induced map
$\bHom_{\h{\calC}}(X,Y) \rightarrow \bHom_{\h{\calD}}(F(X),F(Y))$ is an isomorphism in the homotopy category $\calH$. \index{gen}{fully faithful}\index{gen}{functor!fully faithful}
\end{definition}

\begin{remark}
Because Definition \ref{faulfa} makes reference only to the homotopy categories of $\calC$ and $\calD$, it is invariant under equivalence and under operations which pass between
the various models for higher category theory that we have introduced.
\end{remark}

Just as in ordinary category theory, a functor $F$ is an equivalence if and only if it is fully faithful and essentially surjective.

\subsection{Subcategories of $\infty$-Categories}

Let $\calC$ be an $\infty$-category, and let $(\h{\calC})' \subseteq \h{\calC}$ be a subcategory of its homotopy category. We can then form a pullback diagram of simplicial sets
$$ \xymatrix{ \calC' \ar[r] \ar[d] & \calC \ar[d] \\
\Nerve (\h{\calC})' \ar[r] & \Nerve (\h{ \calC}). }$$
We will refer to $\calC'$ as the {\it subcategory of $\calC$ spanned by $(\h{\calC})'$}. In general, we will say that a simplicial subset $\calC' \subseteq \calC$ is a {\it subcategory} of $\calC$ if it arises via this construction.\index{gen}{subcategory!of an $\infty$-category}

\begin{remark}
We say ``subcategory'', rather than ``sub-$\infty$-category'', in
order to avoid awkward language. The terminology is not meant to
suggest that $\calC'$ is itself a category, or isomorphic to the nerve
of a category.
\end{remark}

In the case where $(\h{\calC})'$ is a full subcategory of $\h{\calC}$, we will say that
$\calC'$ is a {\it full subcategory} of $\calC$. In this case, $\calC'$ is determined by the set
$\calC'_0$ of those objects $X \in \calC$ which belong to $\calC'$. We will then say that
$\calC'$ is the {\it full subcategory of $\calC$ spanned by $\calC'_0$}.\index{gen}{subcategory!full}

It follows from Remark \ref{needie} that
the inclusion $\calC' \subseteq \calC$ is fully faithful. In general, any fully faithful functor $f: \calC'' \rightarrow \calC$ factors as a composition
$$ \calC'' \stackrel{f'}{\rightarrow} \calC' \stackrel{f''}{\rightarrow} \calC,$$
where $f'$ is an equivalence of $\infty$-categories and $f''$ is the inclusion of the full subcategory
$\calC' \subseteq \calC$ spanned by the set of objects $f( \calC''_0 ) \subseteq \calC_0$.

\subsection{Initial and Final Objects}

If $\calC$ is an ordinary category, then an object $X \in \calC$
is said to be {\it final} if $\Hom_{\calC}(Y,X)$ consists of a single element, for every $Y \in \calC$. Dually, an object $X \in \calC$ is {\it initial} if it is final when viewed as an object of $\calC^{op}$.
The goal of this section is to generalize these definitions to the $\infty$-categorical setting.\index{gen}{final object!of a category}\index{gen}{object!final}

If $\calC$ is a topological category, then a candidate definition immediately presents itself: we could ignore the topology on the morphism spaces, and consider those objects of $\calC$ which are final when $\calC$ is regarded as an ordinary category. This requirement is unnaturally strong. For example, the category $\CG$ of compactly generated Hausdorff spaces has a final object: the topological space $\ast$, consisting of a single point. However, there are objects of $\CG$ which are equivalent to $\ast$ (any contractible space) but
not isomorphic to $\ast$ (and therefore not final objects of $\CG$, at least in the classical sense). Since any reasonable $\infty$-categorical notion is stable under equivalence, we need to find a weaker condition.

\begin{definition}\label{inuy}\index{gen}{final object!of an $\infty$-category}\index{gen}{object!final}
Let $\calC$ be a topological category (simplicial category, simplicial set). An object $X \in \calC$ is {\it final} if it is final in the homotopy category
$\h{\calC}$, regarded as a category enriched over $\calH$. In other words, $X$ is final if and only if
for each $Y \in \calC$, the mapping space $\bHom_{ \h{\calC}}(Y,X)$ is weakly contractible
(that is, a final object of $\calH$).
\end{definition}

\begin{remark}
Since the Definition \ref{inuy} makes reference only to the homotopy category $\h{\calC}$, it is invariant under equivalence and under passing between the various models for higher category theory.
\end{remark}

In the setting of $\infty$-categories, it is convenient to employ a slightly more
sophisticated definition, which we borrow from \cite{joyalpub}.

\begin{definition}\label{strongfin}\index{gen}{strongly final}
Let $\calC$ be a simplicial set. A
vertex $X$ of $\calC$ is {\it strongly final} if the projection $\calC_{/X}
\rightarrow \calC$ is a trivial fibration of simplicial sets. 
\end{definition}

In other words, a vertex $X$ of $\calC$ is strongly final if and only if any
map $f_0: \bd \Delta^n \rightarrow \calC$ such that
$f_0(n) = X$ can be extended to a map $f: \Delta^n \rightarrow S$.

\begin{proposition}\label{harry}
Let $\calC$ be an $\infty$-category containing an object $Y$. The object
$Y$ is strongly final if and only if, for every object $X \in \calC$, the Kan
complex $\Hom^{\rght}_{\calC}(X,Y)$ is contractible.
\end{proposition}

\begin{proof}
The ``only if'' direction is clear: the space
$\Hom^{\rght}_{\calC}(X,Y)$ is the fiber of the projection $p: \calC_{/Y}
\rightarrow \calC$ over the vertex $X$. If $p$ is a trivial
fibration, then the fiber is a contractible Kan complex. Since $p$
is a right fibration (Proposition \ref{sharpen}), the converse holds as well (Lemma \ref{toothie}).
\end{proof}

\begin{corollary}
Let $\calC$ be a simplicial set. Every strongly final object of $\calC$ is a final object of $\calC$. The converse holds if $\calC$ is an $\infty$-category.
\end{corollary}

\begin{proof}
Let $[0]$ denote the category with a single object and a single morphism.
Suppose that $Y$ is a strongly final vertex of $\calC$. Then there exists a retraction of
$\calC^{\triangleright}$ onto $\calC$, carrying the cone point to $Y$. Consequently, we obtain a retraction of ($\calH$-enriched) homotopy categories from $(\h{\calC}) \star [0]$ to $\h{\calC}$, carrying the unique object of $[0]$ to $Y$. This implies that $Y$ is final in $\h{\calC}$, so that $Y$ is a final object of $\calC$.

To prove the converse, we note that if $\calC$ is an $\infty$-category then $\Hom_{\calC}^{\rght}(X,Y)$ represents the homotopy type $\bHom_{\calC}(X,Y) \in \calH$; by Proposition \ref{harry} this space is contractible for all $X$ if and only if $Y$ is strongly final.
\end{proof}

\begin{remark}
The above discussion dualizes in an evident way, so that we have a notion of {\em initial} objects of an $\infty$-category $\calC$.
\end{remark}

\begin{example}
Let $\calC$ be an ordinary category containing an object $X$. Then $X$ is a final (initial)
object of the $\infty$-category $\Nerve(\calC)$ if and only if it is a final (initial) object of $\calC$, in the usual sense.
\end{example}

\begin{remark}
Definition \ref{strongfin} is only natural in the case where $\calC$ is an $\infty$-category. For example, if $\calC$ is not an $\infty$-category, then the collection of strongly final vertices of $\calC$ need not be stable under equivalence.
\end{remark}

An ordinary category $\calC$ may have more than one final object,
but any two final objects are uniquely isomorphic to one another.
In the setting of $\infty$-categories, an analogous statement holds,
but is slightly more complicated because the word ``unique'' needs to be
interpreted in a homotopy theoretic sense:

\begin{proposition}[Joyal]\label{initunique}\index{gen}{final object!uniqueness}
Let $\calC$ be a $\infty$-category, and let $\calC'$ be the full subcategory
of $\calC$ spanned by the final vertices of $\calC$. Then $\calC'$ is either empty or
a contractible Kan complex.
\end{proposition}

\begin{proof}
We wish to prove that every map $p: \bd \Delta^n \rightarrow \calC'$
can be extended to an $n$-simplex of $\calC'$. If $n = 0$, this is
possible unless $\calC'$ is empty. For $n > 0$, the desired extension exists
because $p$ carries the $n$th vertex of $\bd \Delta^n$ to a final
object of $\calC$.
\end{proof}

\subsection{Limits and Colimits}\label{limitcolimit}

An important consequence of the distinction between homotopy
commutativity and homotopy coherence is that the appropriate
notions of limit and colimit in a higher category
$\calC$ do not coincide with the notion of limit and colimit in the homotopy category $\h{\calC}$ (where limits and colimits often do
not exist). Limits and colimits in
$\calC$ are often referred to as {\it homotopy limits} and
{\it homotopy colimits}, to avoid confusing them with ordinary limits
and colimits.

Homotopy limits and
colimits can be defined in a topological category, but the
definition is rather complicated. We will review a few special cases here, and discuss the general definition in the appendix (\S \ref{qlim7}).

\begin{example}\label{examprod}\index{gen}{homotopy product}\index{gen}{product!homotopy}
Let $\{ X_{\alpha} \}$ be a family of objects in a topological
category $\calC$. A {\it homotopy product} $X = \prod_{\alpha}
X_{\alpha}$ is an object of $\calC$ equipped with morphisms
$f_{\alpha}: X \rightarrow X_{\alpha}$ which induce a weak
homotopy equivalence
$$ \bHom_{\calC}(Y,X) \rightarrow \prod_{\alpha} \bHom_{\calC}(Y,
X_{\alpha})$$ for every object $Y \in \calC$.

Passing to path components and using the fact that $\pi_0$
commutes with products, we deduce that $$\Hom_{\h{\calC}}(Y,X) \simeq
\prod_{\alpha} \Hom_{\h{\calC}}(Y, X_{\alpha}),$$ so that any product in $\calC$ is
also a product in $\h{\calC}$. In particular, the object $X$ is
determined up to canonical isomorphism in $\h{\calC}$.

In the special case where the index set is empty, we recover the
notion of a final object of $\calC$: an object $X$ for which each
of the mapping spaces $\bHom_{\calC}(Y,X)$ is weakly contractible.
\end{example}

\begin{example}\label{exampull}\index{gen}{homotopy pullback}\index{gen}{pullback!homotopy}
Given two morphisms $\pi: X \rightarrow Z$ and $\psi: Y \rightarrow Z$
in a topological category $\calC$, let us define $\bHom_{\calC}(W,
X \times^h_Z Y)$ to be the space consisting of points $p \in
\bHom_{\calC}(W,X)$, $q \in \bHom_{\calC}(W,Y)$, together with a path $r: [0,1]
\rightarrow \bHom_{\calC}(W,Z)$ joining $\pi \circ p$ to $\psi \circ q$. We
endow $\bHom_{\calC}(W, X \times^h_Z Y)$ with the obvious topology,
so that $X \times^h_Z Y$ can be viewed presheaf of topological spaces
on $\calC$. A {\it homotopy fiber product for $X$ and $Y$ over
$Z$} is an object of $\calC$ which represents this presheaf, up to
weak homotopy equivalence. In other words, it is an object $P \in \calC$
equipped with a point $p \in \bHom_{\calC}(P, X \times^h_Z Y)$ which
induces weak homotopy equivalences $\bHom_{\calC}(W,P) \rightarrow
\bHom_{\calC}(W, X \times^h_Z Y)$ for every $W \in \calC$.

We note that, if there exists a fiber product (in the ordinary sense) $X \times_Z Y$ in the category
$\calC$, then this ordinary fiber product admits a (canonically determined) map to the homotopy fiber product (if the homotopy fiber product exists). This map need not be an equivalence, but it is an equivalence in many good cases. We also note that a homotopy fiber product $P$ comes equipped with a map
to the fiber product $X \times_Z Y$ taken in the category $\h{\calC}$ (if this fiber product exists); this map is usually not an isomorphism.
\end{example}

\begin{remark}
Homotopy limits and colimits in general may be described in
relation to homotopy limits of topological spaces. The homotopy
limit $X$ of a diagram of objects $\{X_{\alpha} \}$ in an
arbitrary topological category $\calC$ is determined, up to
equivalence, by the condition that there exist a natural weak homotopy
equivalence
$$\bHom_{\calC}(Y,X) \simeq \holim \{ \bHom_{\calC}(Y, X_{\alpha})
\}.$$ Similarly, the homotopy colimit of the diagram is characterized by
the existence of a natural weak homotopy equivalence
$$\bHom_{\calC}(X,Y) \simeq \holim \{ \bHom_{\calC}(X_{\alpha},Y)
\}.$$
For a more precise discussion, we refer the reader to Remark \ref{curble}.
\end{remark}

In the setting of $\infty$-categories, limits and colimits
are quite easy to define:

\begin{definition}[Joyal \cite{joyalpub}]\label{defcolim}\index{gen}{colimit}\index{gen}{limit}
Let $\calC$ be an $\infty$-category and let $p: K \rightarrow \calC$ be an
arbitrary map of simplicial sets. A {\it colimit} for $p$ is
an initial object of $\calC_{p/}$ and a {\it limit} for $p$ is a final
object of $\calC_{/p}$.
\end{definition}

\begin{remark}
According to Definition \ref{defcolim}, a colimit of a diagram $p: K \rightarrow \calC$
is an object of $\calC_{p/}$. We may identify this object with a map
$\overline{p}: K^{\triangleright} \rightarrow \calC$ extending $p$. In general, we will say that a map $\overline{p}: K^{\triangleright} \rightarrow \calC$ is a {\it colimit diagram} if it is a colimit
of $p = \overline{p} | K$. In this case, we will also abuse terminology by referring to
$\overline{p}(\infty) \in \calC$ as a {\it colimit of $p$}, where $\infty$ denotes the cone point of
$K^{\triangleright}$.\index{gen}{colimit!diagram}\index{gen}{limit!diagram}\index{gen}{diagram!(co)limit}
\end{remark}

If $p: K \rightarrow \calC$ is a diagram, we will sometimes
write $\varinjlim(p)$ to denote a colimit of $p$ (considered either as an object of
$\calC_{p/}$ or of $\calC$), and $\varprojlim(p)$ to denote a limit of $p$ (as either an object of
$\calC_{/p}$ or an object of $\calC$). This notation is slightly abusive, since $\varinjlim(p)$ is not uniquely determined by $p$. This phenomenon is familiar in classical category theory: the colimit of a diagram is not unique, but is determined up to canonical isomorphism. In the $\infty$-categorical setting, we have a similar uniqueness result: Proposition \ref{initunique} implies that the collection of candidates for $\varinjlim(p)$, if nonempty, is parametrized by a contractible Kan complex.

\begin{remark}
In \S \ref{quasilimit4}, we will show
that Definition \ref{defcolim} agrees with the classical theory of homotopy (co)limits, when
we specialize to the case where $\calC$ is the nerve of a topological category.
\end{remark}

\begin{remark}
Let $\calC$ be an $\infty$-category, $\calC' \subseteq \calC$ a full subcategory,
and $p: K \rightarrow \calC'$ a diagram. Then $\calC'_{p/} = \calC' \times_{\calC}
\calC_{p/}$. In particular, if $p$ has a colimit in $\calC$, and that
colimit belongs to $\calC'$, then the same object may be regarded as a
colimit for $p$ in $\calC'$.
\end{remark}

Let $f: \calC \rightarrow \calC'$ be a map between $\infty$-categories. Let $p: K
\rightarrow \calC$ be a diagram in $\calC$, having a colimit $x \in \calC_{p/}$.
The image $f(x) \in \calC'_{f p/}$ may or may not be a colimit for the composite map
$f \circ p$. If it is, we will say that $f$ {\it preserves} the colimit of the diagram $p$.\index{gen}{colimit!preservation of}\index{gen}{limits!preservation of}
Often we will apply this terminology not to a particular diagram $p$ but some class of diagrams: for example, we may speak of maps $f$ which
preserve coproducts, pushouts, or filtered colimits (see \S \ref{coexample} for a discussion of special classes of colimits). Similarly, we may ask whether or not a map $f$
preserves the limit of a particular diagram, or various families of diagrams.

We conclude this section by giving a simple example of a colimit-preserving functor.

\begin{proposition}\label{needed17}
Let $\calC$ be an $\infty$-category, $q: T \rightarrow \calC$ and $p: K \rightarrow \calC_{/q}$ two diagrams. Let $p_0$ denote the composition of $p$ with the projection 
$\calC_{/q} \rightarrow \calC$. Suppose that $p_0$ has a colimit in $\calC$. Then:
\begin{itemize}
\item[$(1)$] The diagram $p$ has a colimit in $\calC_{/q}$, and that colimit is preserved by the projection $\calC_{/q} \rightarrow \calC$.

\item[$(2)$] An extension $\widetilde{p}: K^{\triangleright} \rightarrow \calC_{/q}$ is a colimit
of $p$ if and only if the composition
$$K^{\triangleright} \rightarrow \calC_{/q} \rightarrow \calC$$
is a colimit of $p_0$.

\end{itemize}
\end{proposition}

\begin{proof}
We first prove the ``if'' direction of $(2)$. Let $\widetilde{p}: K^{\triangleright} \rightarrow \calC_{/q}$ be such that the composite map $\widetilde{p_0}: K^{\triangleright} \rightarrow \calC$ is a colimit of $p_0$. We wish to show that $\widetilde{p}$ is a colimit of $p$. We may identify $\widetilde{p}$ with a map $K \join \Delta^0 \join T \rightarrow \calC$. For this, it suffices to show that
for any inclusion $A \subseteq B$ of simplicial sets, it is possible to solve the lifting problem depicted in the following diagram:
$$ \xymatrix{ (K \join B \join T) \coprod_{ K \join A \join T} ( K \join \Delta^0 \join A \join T )
\ar@{^{(}->}[d] \ar[r] & \calC \\
K \join \Delta^0 \join B \join T. \ar@{-->}[ur] & }$$
Because $\widetilde{p_0}$ is a colimit of $p_0$, the projection
$$ \calC_{\widetilde{p_0}/} \rightarrow \calC_{p_0/}$$ is a trivial fibration of simplicial
sets and therefore has the right lifting property with respect to the inclusion
$A \join T \subseteq B \join T$.

We now prove $(1)$. Let $\widetilde{p_0}: K^{\triangleright} \rightarrow \calC$ be a colimit of $p_0$.
Since the projection $\calC_{\widetilde{p_0}/} \rightarrow \calC_{p_0/}$ is a trivial fibration, it has the right lifting property with respect to $T$: this guarantees the existence of an extension
$\widetilde{p}: K^{\triangleright} \rightarrow \calC$ lifting $\widetilde{p_0}$. The preceding analysis proves that $\widetilde{p}$ is a colimit of $p$.

Finally, the ``only if'' direction of $(2)$ follows from $(1)$, since any two colimits of $p$ are equivalent.
\end{proof}

\subsection{Presentations of $\infty$-Categories}

Like many other types of mathematical structures, $\infty$-categories can be described by generators and relations. In particular, it makes sense to speak of a {\it
finitely presented} $\infty$-category $\calC$. Roughly speaking, $\calC$ is finitely presented
if it has finitely many objects and its morphism spaces are determined
by specifying a finite number of generating morphisms, a finite
number of relations among these generating morphisms, a finite
number of relations among the relations, and so forth (a finite
number of relations in all).

\begin{example}\label{infinitemorphisms}
Let $\calC$ be the free higher category generated by a single
object $X$ and a single morphism $f: X \rightarrow X$. Then
$\calC$ is a finitely presented $\infty$-category with a single
object, and $\Hom_{\calC}(X,X) = \{ 1, f, f^2, \ldots \}$ is
infinite and discrete. In particular, we note that the finite
presentation of $\calC$ does not guarantee finiteness properties
of the morphism spaces.
\end{example}

\begin{example}
If we identify $\infty$-groupoids with spaces, then giving a
presentation for an $\infty$-groupoid corresponds to giving a cell
decomposition of the associated space. Consequently, the finitely
presented $\infty$-groupoids correspond precisely to the finite
cell complexes.
\end{example}

\begin{example}
Suppose that $\calC$ is a higher category with only two objects
$X$ and $Y$, and that $X$ and $Y$ have contractible endomorphism
spaces and that $\Hom_{\calC}(X,Y)$ is empty. Then $\calC$ is
completely determined by the morphism space $\Hom_{\calC}(Y,X)$,
which may be arbitrary. In this case, $\calC$ is finitely
presented if and only if $\Hom_{\calC}(Y,X)$ is a finite cell
complex (up to homotopy equivalence).
\end{example}

The idea of giving a presentation for an $\infty$-category is very
naturally encoded in the theory of simplicial sets; more
specifically, in Joyal's model structure on $\sSet$, which we will discuss in
\S \ref{compp2}. This model structure can be described as follows:

\begin{itemize}\index{gen}{model category!Joyal}
\item The fibrant objects of $\sSet$ are precisely the
$\infty$-categories.

\item The weak equivalences in $\sSet$ are precisely those maps
$p: S \rightarrow S'$ which induce equivalences $\sCoNerve[S] \rightarrow \sCoNerve[S']$
of simplicial categories.
\end{itemize}

If $S$ is an arbitrary simplicial set, we can
choose a ``fibrant replacement'' for $S$; that is, a categorical
equivalence $S \rightarrow \calC$ where $\calC$ is an $\infty$-category. 
For example, we can take $\calC$ to be the nerve of the topological
category $| \sCoNerve[S] |$. 
The $\infty$-category $\calC$ is
well-defined up to equivalence, and we may
regard it as an $\infty$-category which is ``generated by'' $S$. The simplicial set $S$ itself can be thought of as a ``blueprint'' for building $\calC$. We may view $S$ as generated from the empty (simplicial) set by adjoining nondegenerate simplices. Adjoining a $0$-simplex to $S$ has the effect of adding an object to the $\infty$-category $\calC$, and adjoining a $1$-simplex to $S$ has the effect of adjoining a morphism to $\calC$. Higher dimensional simplices can be thought of as encoding relations among the morphisms.

\subsection{Set-Theoretic Technicalities}

In ordinary category theory, one frequently encounters categories in which the collection of objects
is too large to form a set. Generally speaking, this does not create
any difficulties so long as we avoid doing anything which is obviously illegal
(such as considering the ``category of all categories'' as an object of itself).

The same issues arise in the setting of higher category theory, and are
in some sense even more of a nuisance. In ordinary category
theory, one generally allows a category $\calC$ to have a proper
class of objects, but still requires $\Hom_{\calC}(X,Y)$ to be a
{\em set} for fixed objects $X,Y \in \calC$. The formalism of $\infty$-categories treats
objects and morphisms on the same footing (they are both simplices of a simplicial set), and it is somewhat unnatural (though certainly possible) to directly impose the analogous condition; see \S \ref{locbrend} for a discussion.

There are several means of handling the technical difficulties
inherent in working with large objects (in either classical or higher category theory):

\begin{itemize}
\item[$(1)$] One can employ some set-theoretic device which enables one
to distinguish between ``large'' and ``small''. Examples include:
\begin{itemize}
\item Assuming the existence of a sufficient supply of
(Grothendieck) universes.

\item Working in an axiomatic framework which allows both sets and
{\it classes} (collections of sets which are possibly too large to
themselves be considered sets).

\item Working in a standard set-theoretic framework (such as
Zermelo-Frankel), but incorporating a theory of classes through
some ad-hoc device. For example, one can define a class to be a
collection of sets which is defined by some formula in the
language of set theory.
\end{itemize}

\item[$(2)$] One can work exclusively with ``small'' categories, and
mirror the distinction between ``large'' and ``small'' by keeping
careful track of relative sizes.

 \item[$(3)$] One can simply ignore the set-theoretic difficulties
 inherent in discussing ``large'' categories.

\end{itemize}

Needless to say, approach $(2)$ yields the most refined information. However, it has the disadvantage of burdening our exposition with an additional layer of technicalities. On the other hand, approach $(3)$ will sometimes be inadequate, since we will need to make arguments which play off the distinction between a ``large'' category and a ``small'' subcategory which determines it. Consequently, we shall officially adopt approach $(1)$ for the remainder of this book. More specifically, we assume that for every
cardinal $\kappa_0$, there exists a strongly inaccessible cardinal $\kappa \geq \kappa_0$.
We then let $\calU(\kappa)$ denote the collection of all sets having rank $< \kappa$, so that
$\calU(\kappa)$ is a {\it Grothendieck universe}: in other words, $\calU(\kappa)$ satisfies all of the usual axioms of set theory. We will refer to a mathematical object as {\it small} if it belongs to $\calU(\kappa)$ (or is isomorphic to such an object), and {\it essentially small} if it is equivalent (in whatever relevant sense) to a small object. Whenever it is convenient to do so, we will choose another strongly inaccessible cardinal $\kappa' > \kappa$, to obtain a larger Grothendieck universe
$\calU(\kappa')$ in which $\calU(\kappa)$ becomes small.\index{gen}{small}\index{gen}{essentially small}\index{gen}{Grothendieck universe}

For example, an $\infty$-category $\calC$ is essentially small if and only if it satisfies the following conditions:
\begin{itemize}
\item The set of isomorphism classes of objects in the homotopy
category $\h{\calC}$ has cardinality $< \kappa$.

\item For every morphism $f: X \rightarrow Y$ in $\calC$ and every $i \geq 0$, the homotopy
set $\pi_{i}( \Hom^{\rght}_{\calC}(X,Y), f)$ has cardinality $< \kappa$.
\end{itemize}

For a proof and further discussion, we refer the reader to \S \ref{locbrend}.

\begin{remark}
The existence of the strongly inaccessible cardinal $\kappa$ cannot be proven from the standard axioms of set theory, and the assumption that $\kappa$ exists cannot be proven consistent with the standard axioms for set theory. However, it should be clear that assuming the existence of $\kappa$ is merely the most convenient of the devices mentioned above; none of the results proven in this book will depend on this assumption in an essential way.
\end{remark}

\subsection{The $\infty$-Category of Spaces}\label{introducingspaces}

The category of sets plays a central role in classical category theory. The main reason
is that {\em every} category $\calC$ is enriched over sets: given a pair of objects
$X,Y \in \calC$, we may regard $\Hom_{\calC}(X,Y)$ as an object of $\Set$.
In the higher categorical setting, the proper analogue of $\Set$ is the
$\infty$-category $\SSet$ of {\it spaces}, which we will now
introduce.

\begin{definition}\label{defsset}\index{not}{Kan@$\Kan$}
Let $\Kan$ denote the full subcategory of $\sSet$ spanned by the collection of Kan complexes. 
We will regard $\Kan$ as a simplicial category. Let $\SSet = \Nerve(\Kan)$ denote the (simplicial) nerve of $\Kan$. We will refer to $\SSet$ as the {\it $\infty$-category of spaces}.\index{not}{SSet@$\SSet$}\index{gen}{$\infty$-category!of spaces}
\end{definition}

\begin{remark}
For every pair of objects $X,Y \in \Kan$, the simplicial set
$\bHom_{\Kan}(X,Y) = Y^X$ is a Kan complex. It follows that
$\SSet$ is an $\infty$-category (Proposition \ref{toothy}).
\end{remark}

\begin{remark}
There are many other ways to obtain a suitable ``$\infty$-category of
spaces''. For example, we could instead define $\SSet$ to be the (topological) nerve
of the category of CW-complexes and continuous maps.
All that really matters is that we have a $\infty$-category which is equivalent to $\SSet = \Nerve(\Kan)$. 
We have selected Definition \ref{defsset} for definiteness and to simplify our discussion of the
Yoneda embedding in \S \ref{presheaf1}.
\end{remark}

\begin{remark}
We will occasionally need to distinguish between ``large'' spaces and ``small'' spaces.
In such contexts, we will let $\SSet$ denote the $\infty$-category of small spaces (defined
by taking the simplicial nerve of the category of small Kan complexes), and $\widehat{\SSet}$ the $\infty$-category of large spaces (defined by taking the simplicial nerve of the
category of {\em all} Kan complexes). We observe that $\SSet$ is a large $\infty$-category, and that
$\widehat{\SSet}$ is even bigger.\index{not}{SSethat@$\widehat{\SSet}$}
\end{remark}

\end{Didn't Read}