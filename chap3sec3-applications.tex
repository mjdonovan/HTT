
\section{Applications}\label{hugr}
\setcounter{theorem}{0}

The purpose of this section is to survey some applications of technology developed in
\S \ref{twuf} and \S \ref{strsec}. 
In \S \ref{apps}, we give some applications to the theory of Cartesian fibrations. In \S \ref{universalfib}, we will introduce the language of {\it classifying maps} which will allow us to exploit the Quillen equivalence provided by Theorem \ref{straightthm}. Finally, in \S \ref{catlim} and \S \ref{catcolim}, we will use Theorem \ref{straightthm} to give explicit constructions of limits and colimits in the $\infty$-category $\Cat_{\infty}$ (and also in the $\infty$-category $\SSet$ of spaces).

%\subsection{Straightening and Unstraightening in the Unmarked Case}\label{unm}

%Let $S$ be a simplicial set and $\phi: \sCoNerve[S] \rightarrow \calC^{op}$ a simplicial functor.
%In \S \ref{contrasec}, we defined straightening and unstraightening functors
%$$ \Adjoint{ \St_{\phi}}{ (\sSet)_{/S} }{(\sSet)^{\calC} }{ \Un_{\phi}}.$$
%In \S \ref{strsec}, we studied the corresponding constructions in the context of {\em marked} simplicial sets, and proved that $(\St_{\phi}^{+}, \Un_{\phi}^{+})$ is a Quillen equivalence provided that $\phi$ is an equivalence of simplicial categories. This result can be regarded as a marked version of Theorem \ref{struns}, which asserts that $(\St_{\phi}, \Un_{\phi})$ is a Quillen equivalence (under the same hypotheses). Our goal in this section is to use Theorem \ref{straightthm} to deduce Theorem \ref{struns}. The following statement summarizes both results, and the relationship between them:

%\begin{theorem}\label{strungss}
%Let $S$ be a simplicial set, $\calC$ a simplicial category, and $\phi: \sCoNerve[S]^{op} \rightarrow \calC$ a simplicial functor. We have a commutative diagram of model categories and left Quillen functors
%$$ \xymatrix{ (\mSet)_{/S} \ar[r]^{\St^{+}_{\phi}} \ar[d] & (\mSet)^{\calC} \ar[d] \\
%(\sSet)_{/S} \ar[r]^{ \St_{\phi} } \ar[r] & (\sSet)^{\calC}, }$$
%where the vertical arrows are given by forgetting the markings, and $(\sSet)_{/S}$ is endowed with the contravariant model structure. Moreover, if $\phi$ is an equivalence of simplicial categories, then the horizontal arrows are Quillen equivalences.
%\end{theorem}

%The proof of Theorem \ref{strungss} uses Proposition \ref{romb}, which in turn rests on the following
%characterization of the contravariantly fibrant objects of $(\sSet)_{/S}$:

%\begin{proof}[Proof of Theorem \ref{strungss}]
%The commutativity of the diagram is an immediate consequence of the construction of
%$\St_{\phi}^{+}$. We have already shown that $\St_{\phi}^{+}$ is a left Quillen functor, and it follows from Theorem \ref{bigdiag} that the vertical arrows are left Quillen functors. The first nontrivial point to check is that $\St_{\phi}$ is a left Quillen functor. Since $\St_{\phi}$ clearly preserves cofibrations, it suffices to show that $\St_{\phi}$ preserves weak equivalences.
%Let $X, X' \in (\sSet)_{/S}$, and let $\calF = \St_{\phi} X$, $\calF'  = \St_{\phi} \calF'$ be the associated functors
%$\calC \rightarrow \sSet$. Suppose $f: X \rightarrow X'$ is a contravariant equivalence. By Proposition \ref{romb}, the induced map $f: X^{\sharp} \rightarrow {X'}^{\sharp}$ is a Cartesian equivalence in
%$(\mSet)_{/S}$. By Corollary \ref{spek6}, the induced map
%$$\calF^{\sharp} = \St^{+}_{\phi} X^{\sharp} \rightarrow \St^{+}_{\phi} {X'}^{\sharp} \rightarrow
%(\calF')^{\sharp}$$
%is a weak equivalence of functors $\calC \rightarrow \mSet$. Applying Proposition \ref{romb} again, we deduce that $\calF \rightarrow \calF'$ is an equivalence of functors $\calC \rightarrow \sSet$.

%To complete the proof, it suffices to show that if $\phi$ is a categorical equivalence, then
%$(\St_{\phi}, \Un_{\phi})$ is a Quillen equivalence.  Let $X$ be an object of $\sSet_{/S}$ and
%$\calF$ a fibrant object of $(\sSet)^{\calC}$. We need to show that a map
%$$u: X \rightarrow \Un_{\phi} \calF$$ is a contravariant equivalence in $\sSet_{/S}$
%if and only if the adjoint map
%$$v: \St_{\phi}^{\sharp} X \rightarrow \calF$$ is an equivalence in $(\sSet)^{\calC}$. 
%By Proposition \ref{romb}, $u$ is an equivalence if and only if the associated map
%$$u': X^{\sharp} \rightarrow \Un^{+}_{\phi} \calF^{\sharp}$$ is an equivalence in $(\mSet)_{/S}$, and
%$v$ is an equivalence if and only if the associated map
%$$ v': \St^{+}_{\phi} X^{\sharp} \rightarrow \calF^{\sharp}$$ 
%is an equivalence in $(\mSet)^{\calC}$. Theorem \ref{straightthm} asserts that $u'$ is an equivalence if and only if $v'$ is an equivalence, which completes the proof.
%\end{proof}

\subsection{Structure Theory for Cartesian Fibrations}\label{apps}

The purpose of this section is to prove that Cartesian fibrations between simplicial sets enjoy several pleasant properties. For example, every Cartesian fibration is a categorical fibration (Proposition \ref{funkyfibcatfib}), and categorical equivalences are stable under pullbacks by Cartesian fibrations (Proposition \ref{basechangefunky}). These results are fairly easy to prove for Cartesian fibrations $X \rightarrow S$ in the case where $S$ is an $\infty$-category. Theorem \ref{straightthm} provides a method for reducing to this special case:

\begin{proposition}\label{pretrokee}
Let $p: S \rightarrow T$ be a categorical equivalence of simplicial sets. Then the forgetful functor $$p_{!}: (\mSet)_{/S} \rightarrow (\mSet)_{/T}$$ and its right adjoint $p^{\ast}$ induce a Quillen equivalence between $(\mSet)_{/S}$ and $(\mSet)_{/T}$.
\end{proposition}

\begin{proof}
Let $\calC = \sCoNerve[S]^{op}$ and $\calD = \sCoNerve[T]^{op}$.
Consider the diagram of model categories and left Quillen functors:
$$\xymatrix{ (\mSet)_{/S} \ar[r]^{p_{!}} \ar[d]^{ \St^{+}_{S} } & (\mSet)_{/T} \ar[d]^{\St^{+}_{T}} \\
\calC \ar[r]^{ \sCoNerve[p]_{!} } & \calD }.$$ According to Proposition \ref{formall}, this diagram commutes (up to natural isomorphism). Theorem \ref{straightthm} implies that the vertical arrows are Quillen equivalences. Since $p$ is a categorical equivalence, $\sCoNerve[p]$ is an equivalence of simplicial categories, so that $\sCoNerve[p]_{!}$ is a Quillen equivalence (Proposition \ref{lesstrick}). It follows that $(p_!, p^{\ast})$ is a Quillen equivalence as well. 
\end{proof}

\begin{corollary}\label{tttroke}
Let $p: X \rightarrow S$ be a Cartesian fibration of simplicial sets, and let
$S \rightarrow T$ be a categorical equivalence. Then there exists a Cartesian fibration $Y \rightarrow T$, and an equivalence of $X$ with $S \times_T Y$ $($as Cartesian fibrations over $X${}$)$.
\end{corollary}

\begin{proof}
Proposition \ref{pretrokee} implies that the right derived functor $R p^{\ast}$ is essentially surjective.
\end{proof}

As we explained in Remark \ref{rightprop}, the Joyal model structure on $\sSet$ is {\em not} right proper. In other words, it is possible to have a categorical fibration $X \rightarrow S$ and 
a categorical equivalence $T \rightarrow S$ such that the induced map $X \times_{S} T \rightarrow X$ is not a categorical equivalence. This poor behavior of categorical fibrations is one of the reason that they do not play a prominent role in the theory of $\infty$-categories. Working with a stronger notion of fibration corrects the problem:

\begin{proposition}\label{basechangefunky}\index{gen}{Cartesian fibration!and pullbacks}
Let $p: X \rightarrow S$ be a Cartesian fibration, and let $T \rightarrow S$ be a categorical equivalence. Then the induced map $X \times_{S} T \rightarrow X$ is a categorical equivalence.
\end{proposition}

\begin{proof}
We first suppose that the map $T \rightarrow S$ is inner anodyne. By means of a simple argument, we may reduce to the case where $T \rightarrow S$ is a middle horn inclusion $\Lambda^n_i \subseteq \Delta^n$, where $0 < i < n$. 
According Proposition \ref{simplexplay}, there exists a sequence of maps
$$ \phi: A^0 \leftarrow \ldots \leftarrow A^n $$ and a map
$M(\phi) \rightarrow X$ which is a categorical equivalence, such that
$M(\phi) \times_{S} T \rightarrow X \times_{S} T$ is also a categorical equivalence.
Consequently, it suffices to show that the inclusion $M(\phi) \times_{S} T \subseteq M(\phi)$ is a categorical equivalence. But this map is a pushout of the inclusion
$A^n \times \Lambda^n_i \subseteq A^n \times \Delta^n$, which is inner anodyne.

We now treat the general case. Choose an inner anodyne map $T \rightarrow T'$ where $T'$ is an $\infty$-category. Then choose an inner anodyne map $T' \coprod_{T} S \rightarrow S'$, where $S'$ is also an $\infty$-category. The map $S \rightarrow S'$ is inner anodyne; in particular it is a categorical equivalence, so by
Corollary \ref{tttroke} there is a Cartesian fibration $X' \rightarrow S'$ and an equivalence
$X \rightarrow X' \times_{S'} S$ of Cartesian fibrations over $S$. We have a commutative diagram
$$ \xymatrix{ & X' \times_{S'} T \ar[r]^{u'} & X' \times_{S'} T' \ar[dr]^{u''} & \\
X \times_{S} T \ar[ur]^{u} \ar[dr]^{v} & & & X'. \\
& X \ar[r]^{v'} & X' \times_{S'} S \ar[ur]^{v''} & \\}$$
Consequently, to prove that $v$ is a categorical equivalence, it suffices to show that every other arrow in the diagram is a categorical equivalence. The maps $u$ and $v'$ are equivalences of Cartesian fibrations, and therefore categorical equivalences. The other three maps are special cases of the assertion we are trying to prove., For the map $u''$, we are in the special case of the map $S' \rightarrow T'$, which is an equivalence of $\infty$-categories: in this case we simply apply Corollary \ref{usesec}. For the maps $u'$ and $v''$, we need to know that the assertion of the proposition is valid in the special case of the maps $S \rightarrow S'$ and $T \rightarrow T'$. Since these maps are inner anodyne, the proof is complete.
\end{proof}

\begin{corollary}\label{basety}
Let 
$$ \xymatrix{ X \ar[r] \ar[d] & X' \ar[d]^{p'} \\
S \ar[r] & S' }$$
be a pullback diagram of simplicial sets, where $p'$ is a Cartesian fibration. Then the diagram is homotopy Cartesian $($with respect to the Joyal model structure$)$.
\end{corollary}

\begin{proof}
Choose a categorical equivalence $S' \rightarrow S''$, where $S''$ is an $\infty$-category.
Using Proposition \ref{pretrokee}, we may assume without loss of generality that
$X' \simeq X'' \times_{S''} S'$, where $X'' \rightarrow S''$ is a Cartesian fibration. Now choose
a factorization
$$ S \stackrel{\theta'}{\rightarrow} T \stackrel{\theta''}{\rightarrow} S''$$
where $\theta'$ is a categorical equivalence and $\theta''$ is a categorical fibration. 
The diagram
$$ T \rightarrow S'' \leftarrow X''$$ is fibrant. Consequently, the desired conclusion is equivalent to the assertion that the map $X \rightarrow T \times_{S''} X''$ is a categorical equivalence, which follows immediately from Proposition \ref{basechangefunky}.
\end{proof}

We now prove a stronger version of Corollary \ref{usefir}, which does not require that the base $S$ is a $\infty$-category.

\begin{proposition}\label{apple1}
Suppose given a diagram of simplicial sets
$$ \xymatrix{ X \ar[dr]^{p} \ar[rr]^{f} & & Y \ar[dl]^{q} \\
& S & }$$ 
where $p$ and $q$ are Cartesian fibrations, and $f$ carries $p$-Cartesian edges
to $q$-Cartesian edges.
The following conditions are equivalent:
\begin{itemize}
\item[$(1)$] The map $f$ is a categorical equivalence.
\item[$(2)$] For each vertex $s$ of $S$, $f$ induces a categorical equivalence $X_{s} \rightarrow Y_{s}$.
\item[$(3)$] The map $X^{\natural} \rightarrow Y^{\natural}$ is a Cartesian equivalence in
$(\mSet)_{/S}$. 
\end{itemize}

\end{proposition}

\begin{proof}
The equivalence of $(2)$ and $(3)$ follows from Proposition \ref{crispy}. We next show that $(2)$ implies $(1)$. In virtue of Proposition \ref{tulky}, we may reduce to the case where $S$ is a simplex. Then $S$ is an $\infty$-category and the desired result follows from Corollary \ref{usefir}. (Alternatively, we could observe that $(2)$ implies that $f$ has a homotopy inverse.)

To prove that $(1)$ implies $(3)$, we choose an inner anodyne map $j: S \rightarrow S'$, where $S'$ is an $\infty$-category. Let $X^{\natural}$ denote the object of $(\mSet)_{/S}$ associated to the Cartesian fibration $p: X \rightarrow S$, and let $j_{!} X^{\natural}$ denote the same marked simplicial set, regarded as an object of $(\mSet)_{/T}$. Choose a marked anodyne map
$j_{!} X^{\natural} \rightarrow {X'}^{\natural}$, where $X' \rightarrow S'$ is a Cartesian fibration.
By Proposition \ref{pretrokee}, the map $X^{\natural} \rightarrow j^{\ast} {X'}^{\natural}$ is a Cartesian equivalence, so that $X \rightarrow X' \times_{S'} S$ is a categorical equivalence. According to Proposition \ref{basechangefunky}, the map $X' \times_{S'} S \rightarrow X'$ is a categorical equivalence; thus the composite map $X \rightarrow X'$ is a categorical equivalence.

Similarly, we may choose a marked anodyne map $${X'}^{\natural} \coprod_{ j_{!} X^{\natural}} j_{!} Y^{\natural} \rightarrow {Y'}^{\natural}$$ for some Cartesian fibration $Y' \rightarrow S'$. Since
the Cartesian model structure is left-proper, the map $j_{!} Y^{\natural} \rightarrow {Y'}^{\natural}$
is a Cartesian equivalence, so we may argue as above to deduce that $Y \rightarrow Y'$ is a categorical equivalence. Now consider the diagram
$$ \xymatrix{ X \ar[r]^{f} \ar[d] & Y \ar[d] \\
X' \ar[r]^{f'} \ar[r] & Y'. }$$
We have argued that the vertical maps are categorical equivalences. The map $f$ is a categorical equivalence by assumption. It follows that $f'$ is a categorical equivalence. Since $S'$ is an $\infty$-category, we may apply Corollary \ref{usefir} to deduce that $X'_{s} \rightarrow Y'_{s}$
is a categorical equivalence for each object $s$ of $S'$. It follows that ${X'}^{\natural}
\rightarrow {Y'}^{\natural}$ is a Cartesian equivalence in $(\mSet)_{/S}$, so that we have a commutative diagram
$$ \xymatrix{ X^{\natural} \ar[r] \ar[d] & Y^{\natural} \ar[d] \\
j^{\ast} {X'}^{\natural} \ar[r] & j^{\ast} {Y'}^{\natural} }$$
where the vertical and bottom horizontal arrows are Cartesian equivalences in $(\mSet)_{/S}$. It follows that the top horizontal arrow is a Cartesian equivalence as well, so that $(3)$ is satisfied.
\end{proof}


\begin{corollary}\label{ruy}
Let
$$ \xymatrix{ W \ar[r] \ar[d] & X \ar[d] &  \\
Y \ar[r] & Z \ar[r] & S }$$
be a diagram of simplicial sets. Suppose that every morphism in this diagram is a right fibration, and that the square is a pullback. Then the diagram is homotopy Cartesian with respect to the contravariant model structure on $(\sSet)_{/S}$.
\end{corollary}

\begin{proof}
Choose a fibrant replacement
$$ X' \rightarrow Y' \leftarrow Z'$$
for the diagram
$$ X \rightarrow Y \leftarrow Z$$
in $(\sSet)_{/S}$, and let $W' = X' \times_{Z'} Y'$. We wish to show that the induced map
$i: W \rightarrow W'$ is a covariant equivalence in $(\sSet)_{/S}$. According to Corollary \ref{prefibchar}, it will suffice to show that for each vertex $s$ of $S$, the map of fibers
$W_{s} \rightarrow W'_{s}$ is a homotopy equivalence of Kan complexes. 
To prove this, we observe that we have a natural transformation of diagrams from
$$ \xymatrix{ W_{s} \ar[r] \ar[d] & X_s \ar[d] \\
Y_{s} \ar[r] & Z_{s} }$$
to
$$ \xymatrix{ W'_{s} \ar[r] \ar[d] & X'_s \ar[d] \\
Y'_{s} \ar[r] & Z'_{s} }$$
which induces homotopy equivalences 
$$X_{s} \rightarrow X'_{s} \quad \quad Y_{s} \rightarrow Y'_{s} \quad \quad Z_{s} \rightarrow Z'_{s}$$ 
(Corollary \ref{prefibchar}), where both diagrams are homotopy Cartesian (Proposition \ref{dent}).
\end{proof}

\begin{proposition}\label{funkyfibcatfib}\index{gen}{Cartesian fibration!and categorical fibrations}
Let $p: X \rightarrow S$ be a Cartesian fibration of simplicial sets. Then $p$ is a categorical fibration.
\end{proposition}

\begin{proof}
Consider a diagram
$$ \xymatrix{ A \ar[r] \ar@{^{(}->}[d]^{i} & X \ar[d]^{p} \\
B \ar[r] \ar@{-->}[ur]^{f} \ar[r]^{g}  & S} $$
of simplicial sets where $i$ is an inclusion and a categorical equivalence. We must demonstrate the existence of the indicated dotted arrow. Choose a categorical equivalence
$j: S \rightarrow T$, where $T$ is an $\infty$-category. By Corollary \ref{tttroke}, there exists
a Cartesian fibration $q: Y \rightarrow T$
such that $Y \times_{T} S$ is equivalent to $X$. Thus, there exist maps $$ u: X \rightarrow Y \times_{T} S $$
$$ v: Y \times_{T} S \rightarrow X$$
such that $u \circ v$ and $v \circ u$ are homotopic to the identity (over $S$).

Consider the induced diagram
$$ \xymatrix{ A \ar[r] \ar@{^{(}->}[d]^{i} & Y \\
B. \ar@{-->}[ur]_{f'} &} $$
Since $Y$ is an $\infty$-category, there exists a dotted arrow $f'$ making the diagram commutative. Let $g' = q \circ f': B \rightarrow T$. We note that $g'|A = (j \circ g)|A$. Since $T$ is an $\infty$-category and $i$ is a categorical equivalence, there exists a homotopy
$B \times \Delta^1 \rightarrow T$ from $g'$ to $j \circ g$ which is fixed on $A$. Since
$q$ is a Cartesian fibration, this homotopy lifts to a homotopy from $f'$ to some map
$f'': B \rightarrow Y$, so that we have a commutative diagram 
$$ \xymatrix{ A \ar[r] \ar@{^{(}->}[d]^{i} & Y \ar[d]^{q} \\
B \ar[r] \ar@{-->}[ur]_{f''} \ar[r] & T.} $$

Consider the composite map
$$ f''': B \stackrel{(f'',g)}{\rightarrow} Y \times_{T} S \stackrel{v}{\rightarrow} X.$$
Since $f'$ is homotopic to $f''$, and $v \circ u$ is homotopic to the identity, we conclude that
$f'''|A$ is homotopic to $f_0$ (via a homotopy which is fixed over $S$). Since $p$ is a Cartesian fibration, we can extend $h$ to a homotopy from $f'''$ to the desired map $f$.
\end{proof}

In general, the converse to Proposition \ref{funkyfibcatfib} fails: a categorical fibration of simplicial sets $X \rightarrow S$ need not be a Cartesian fibration. This is clear, since the property of being a categorical fibration is self-dual while the condition of being a Cartesian fibration is not. However, in the case where $S$ is a Kan complex, the theory of Cartesian fibrations {\em is} self-dual, and we have the following result:

\begin{proposition}\label{groob}
Let $p: X \rightarrow S$ be a map of simplicial sets, where $S$ is a Kan complex.
The following assertions are equivalent:
\begin{itemize}
\item[$(1)$] The map $p$ is a Cartesian fibration.
\item[$(2)$] The map $p$ is a coCartesian fibration.
\item[$(3)$] The map $p$ is a categorical fibration.
\end{itemize}
\end{proposition}

\begin{proof}
We will prove that $(1)$ is equivalent to $(3)$; the equivalence of $(2)$ and $(3)$ follows from a dual argument. Proposition \ref{funkyfibcatfib} shows that $(1)$ implies $(3)$ (for this implication, the assumption that $S$ is a Kan complex is not needed).

Now suppose that $(3)$ holds. Then $X$ is an $\infty$-category. Since every edge of $S$ is an equivalence, the $p$-Cartesian edges of $X$ are precisely the equivalences in $X$. It therefore suffices to show that if if $y$ is a vertex of $X$ and $\overline{e}: \overline{x} \rightarrow p(y)$ is an edge of $S$, then $\overline{e}$ lifts to an equivalence $e: x \rightarrow y$ in $S$. Since
$S$ is a Kan complex, we can find a contractible Kan complex $K$ and a map
$\overline{q}: K \rightarrow S$ such that $\overline{e}$ is the image of an edge $e': x' \rightarrow y'$ in $K$. 
The inclusion $\{y'\} \subseteq K$ is a categorical equivalence; since $p$ is a categorical fibration, we can lift $\overline{q}$ to a map $q: K \rightarrow X$ with $q(y')=y$. Then $e=q(e')$ has the desired properties.
\end{proof}

\subsection{Universal Fibrations}\label{universalfib}

In this section, we will apply Theorem \ref{straightthm} to construct a {\em universal} Cartesian fibration. Recall that $\Cat_{\infty}$ is defined to be the nerve of the simplicial category
$\Cat_{\infty}^{\Delta} = ( \mSet)^{\degree}$ of $\infty$-categories. In particular, we may regard the inclusion $\Cat_{\infty}^{\Delta} \hookrightarrow \mSet$ as a (projectively) fibrant object 
$\calF \in (\mSet)^{ \Cat_{\infty}^{\Delta} }$. Applying the unstraightening functor
$\Un^{+}_{\Cat_{\infty}^{op}}$, we obtain a fibrant object of $(\mSet)_{/\Cat_{\infty}^{op}}$, which we may identify with Cartesian fibration $q: \calZ \rightarrow \Cat_{\infty}^{op}$. We will refer to $q$
as the {\it universal Cartesian fibration}.\index{gen}{Cartesian fibration!universal} We observe that the objects of $\Cat_{\infty}$ can be identified with $\infty$-categories, and that the fiber of $q$ over an $\infty$-category $\calC$ can be identified with $U(\calC)$, where $U$ is the functor described in Lemma \ref{utest}. In particular, there is a canonical equivalence of $\infty$-categories $$\calC \rightarrow U(\calC) =
\calZ \times_{ \Cat_{\infty}^{op} } \{ \calC \}.$$ Thus we may think of $q$ as a Cartesian fibration which associates to each object of $\Cat_{\infty}$ the associated $\infty$-category.\index{gen}{universal!Cartesian fibration}

\begin{remark}
The $\infty$-categories $\Cat_{\infty}$ and $\calZ$ are {\em large}. However, the universal Cartesian fibration $q$ is small in the sense that for any small simplicial set $S$ and any map $f: S \rightarrow \Cat_{\infty}^{op}$, the fiber product $S \times_{ \QC^{op}} \calZ$ is small. This is because the fiber product can be identified with $\Un^{+}_{\phi}(\calF| \sCoNerve[S])$, where $\phi: \sCoNerve[S] \rightarrow \mSet$ is the composition of $\sCoNerve[f]$ with the inclusion.
\end{remark}

\begin{definition}\label{classer}\index{gen}{Cartesian fibration!classified by $f: S \rightarrow \Cat_{\infty}^{op}$}\index{gen}{classifying map!for a (co)Cartesian fibration}
Let $p: X \rightarrow S$ be a Cartesian fibration of simplicial sets. We will say that
a functor $f: S \rightarrow \Cat_{\infty}^{op}$ {\it classifies $p$} if there is an equivalence of Cartesian fibrations $X \rightarrow \calZ \times_{ \Cat^{op}_{\infty}} S
\simeq \Un^{+}_{S} f$. 

Dually, if $p: X \rightarrow S$ is a coCartesian fibration, then we will say that a functor
$f: S \rightarrow \Cat_{\infty}$ {\it classifies $p$} if $f^{op}$ classifies the
Cartesian fibration $p^{op}: X^{op} \rightarrow S^{op}$.\index{gen}{coCartesian fibration!classified by $f: S \rightarrow \Cat_{\infty}$}
\end{definition}

\begin{remark}\label{uniright}
Every Cartesian fibration $X \rightarrow S$ between {\em small} simplicial sets admits a classifying map $\phi: S \rightarrow \Cat_{\infty}^{op}$, which is uniquely determined up to equivalence. 
This is one expression of the idea that $\calZ \rightarrow \Cat_{\infty}^{op}$ is a {\it universal} Cartesian fibration. However, it is not immediately obvious that this property characterizes $\Cat_{\infty}$ up to equivalence, because $\Cat_{\infty}$ is not itself small. To remedy the situation, let us consider an arbitrary uncountable regular cardinal $\kappa$, and let $\Cat_{\infty}(\kappa)$ denote the full subcategory of $\Cat_{\infty}$ spanned by the $\kappa$-small $\infty$-categories. We then deduce the following:
\begin{itemize}
\item[$(\ast)$] Let $p: X \rightarrow S$ be a Cartesian fibration between small simplicial sets.
Then $p$ is classified by a functor $\chi: S \rightarrow \Cat_{\infty}(\kappa)^{op}$ if and only if,
for every vertex $s \in S$, the fiber $X_{s}$ is essentially $\kappa$-small. In this case, $\chi$ is determined uniquely up to homotopy.
\end{itemize}
Enlarging the universe and applying $(\ast)$ in the case where $\kappa$ is the supremum of all small cardinals, we deduce the following property:
\begin{itemize}
\item[$(\ast')$] Let $p: X \rightarrow S$ be a Cartesian fibration between simplicial sets which are not necessarily small. Then $p$ is classified by a functor $\chi: S \rightarrow \Cat_{\infty}^{op}$ if and only if, for every vertex $s \in S$, the fiber $X_{s}$ is essentially small. In this case, $\chi$ is determined uniquely up to homotopy.
\end{itemize}
This property evidently determines the $\infty$-category $\Cat_{\infty}$ (and the
Cartesian fibration $q: \calZ \rightarrow \Cat_{\infty}^{op}$) up to equivalence.
\end{remark}

\begin{warning}
The terminology of Definition \ref{classer} has the potential to cause confusion in the case where
$p: X \rightarrow S$ is both a Cartesian fibration and a coCartesian fibration. In this case,
$p$ is classified both by a functor $S \rightarrow \Cat_{\infty}^{op}$ (as a Cartesian fibration)
and by a functor $S \rightarrow \Cat_{\infty}$ (as a coCartesian fibration).
\end{warning}

The category $\Kan$ of Kan complexes can be identified with a full (simplicial) subcategory of $\Cat_{\infty}^{\Delta}$. Consequently we may identify the $\infty$-category
$\SSet$ of spaces with the full simplicial subset of $\Cat_{\infty}$, spanned by the vertices which represent $\infty$-groupoids. We let $\calZ^0 = \calZ \times_{ \Cat_{\infty}^{op}} \SSet^{op}$ be the restriction of the universal Cartesian fibration. The fibers of $q^0: \calZ^0 \rightarrow \SSet^{op}$
are Kan complexes (since they are equivalent to the $\infty$-categories represented by the vertices of $\SSet$). It follows from Proposition \ref{goey} that $q^0$ is a right fibration. We will refer to
$q^0$ as the {\it universal right fibration}.\index{gen}{universal!right fibration}\index{gen}{right fibration!universal}

Proposition \ref{goey} translates immediately into the following characterization of right fibrations:

\begin{proposition}
Let $p: X \rightarrow S$ be a Cartesian fibration of simplicial sets. The following conditions are equivalent:
\begin{itemize}
\item[$(1)$] The map $p$ is a right fibration.
\item[$(2)$] Every functor $f: S \rightarrow \Cat_{\infty}^{op}$ which classifies $p$ factors
through $\SSet^{op} \subseteq \Cat_{\infty}^{op}$.
\item[$(3)$] There exists a functor $f: S \rightarrow \SSet^{op}$ which classifies $p$.
\end{itemize}
\end{proposition}

Consequently, we may speak of right fibrations $X \rightarrow S$ being classified by functors
$S \rightarrow \SSet^{op}$, and left fibrations being classified by functors $S \rightarrow \SSet$.\index{gen}{right fibration!classifed by $S \rightarrow \SSet^{op}$}\index{gen}{left fibration!classified by $S \rightarrow \SSet$}\index{gen}{classifying map!for a right fibration}\index{gen}{classifying map!for a left fibration}

The $\infty$-category $\Delta^0$ corresponds to a vertex of $\Cat_{\infty}$ which we will denote by $\ast$. The fiber of $q$ over this point may be identified with $U \Delta^0 \simeq \Delta^0$; consequently, there is a unique vertex $\ast_{\calZ}$ of $\calZ$ lying over $\ast$. 
We note that $\ast$ and $\ast_{\calZ}$ belong to the subcategories $\SSet$ and $\calZ^0$. Moreover, we have the following:

\begin{proposition}\label{unifinal}
Let $q^0: \calZ^0 \rightarrow \SSet^{op}$ be the universal right fibration. The vertex $\ast_{\calZ}$ is a final object of the $\infty$-category $\calZ^0$.
\end{proposition}

\begin{proof}
Let $n > 0$, and let $f_0: \bd \Delta^n \rightarrow \calZ^0$ have the property that $f_0$ carries the final vertex of $\Delta^n$ to $\ast_{\calZ}$. We wish to show that there exists an extension
$$ \xymatrix{ \bd \Delta^n \ar[r]^{f_0} \ar@{^{(}->}[d] & \calZ \\
\Delta^n \ar@{-->}[ur]^{f} }$$
(in which case the map $f$ automatically factors through $\calZ^0$).

Let $\calD$ denote the simplicial category containing $\SSet^{op}_{\Delta}$ as a full subcategory, together with one additional object $X$, with the morphisms given by
$$ \bHom_{\calD}( K, X) = K $$
$$ \bHom_{\calD}(X,X) =  \ast $$
$$ \bHom_{\calD}(X,K) = \emptyset$$
for all $K \in \SSet^{op}_{\Delta}$. Let $\calC = \sCoNerve[ \Delta^n \star \Delta^0 ]$, and let
$\calC_0$ denote the full subcategory $\calC_0 = \sCoNerve[ \bd \Delta^n \star \Delta^0 ]$. 
We will denote the objects of $\calC$ by $\{ v_0, \ldots, v_{n+1} \}$.
Giving the map $f_0$ is tantamount to giving a simplicial functor $F_0: \calC_0 \rightarrow \calD$
with $F_0(v_{n+1})=X$, and constructing $f$ amounts to giving a simplicial functor $F: \calC \rightarrow \calD$ which extends $F_0$.

We note that the inclusion $\bHom_{\calC_0}( v_i, v_j) \rightarrow \bHom_{\calC}(v_i,v_j)$ is an isomorphism, unless $i=0$ and $j \in \{n, n+1\}$. Consequently, to define $F$, it suffices to find extensions
$$ \xymatrix{ \bHom_{\calC_0}(v_0, v_n) \ar[r] \ar@{^{(}->}[d] & \bHom_{\calD}(F_0(v_0), F_0(v_n)) \\ \bHom_{\calC}(v_0,v_n) \ar@{-->}[ur]^{j} }$$
$$ \xymatrix{ \bHom_{\calC_0}(v_0, v_{n+1}) \ar[r] \ar@{^{(}->}[d] & \bHom_{\calD}(F_0(v_0), F_0(v_{n+1})) \\ \bHom_{\calC}(v_0,v_{n+1}) \ar@{-->}[ur]^{j'} }$$
such that the following diagram commutes:
$$ \xymatrix{ \bHom_{\calC}(v_0,v_n) \times \bHom_{\calC}(v_n,v_{n+1}) \ar[r] \ar[d] &
\bHom_{\calD}(F_0(v_0), F_0(v_n)) \times \bHom_{\calD}(F_0(v_n), F_0(v_{n+1})) \ar[d] \\
\bHom_{\calC}(v_0, v_{n+1}) \ar[r] & \bHom_{\calD}(F_0(v_0), F_0(v_{n+1})). }$$

We note that $\bHom_{\calC}(v_{n},v_{n+1})$ is a point. In view of the assumption that
$f_0$ carries the final vertex of $\Delta^n$ to $\ast_{\calZ}$, we see that
$\bHom_{\calD}( F(v_n), F(v_{n+1}))$ is a point. It follows that, for any fixed choice of $j'$, there is a unique choice of $j$ for which the above diagram commutes. It therefore suffices to show that $j'$ exists. Since $\bHom_{ \calD }( F_0(v_0), X )$ is a Kan complex, it will suffice to show that the inclusion $\bHom_{\calC_0}( v_0, v_{n+1}) \rightarrow \bHom_{\calC}(v_0,v_{n+1})$ is an anodyne map of simplicial sets. In fact, it is isomorphic to the inclusion
$$ (\{1\} \times (\Delta^1)^{n-1}) \coprod_{ \{1\} \times \bd (\Delta^1)^{n-1} }
(\Delta^1 \times \bd (\Delta^1)^{n-1}) \subseteq \Delta^1 \times \Delta^{n-1},$$
which is the smash product of the cofibration $\bd (\Delta^1)^{n-1} \subseteq (\Delta^1)^{n-1}$ with the anodyne inclusion $\{1\} \subseteq \Delta^1$.
\end{proof}

\begin{corollary}\label{grt}
The universal right fibration $q^0: \calZ^0 \rightarrow \SSet^{op}$ is representable
by the final object of $\SSet$.
\end{corollary}

\begin{proof}
Combine Propositions \ref{unifinal} and \ref{reppfunc}.
\end{proof}

\begin{corollary}\label{unipull}
Let $p: X \rightarrow S$ be a left fibration between small simplicial sets. Then there exists a map
$S \rightarrow \SSet$ and an equivalence of left fibrations $X \simeq S \times_{\SSet} \SSet_{\ast/}$. 
\end{corollary}

\begin{proof}
Combine Corollary \ref{grt} with Remark \ref{uniright}.
\end{proof}

\subsection{Limits of $\infty$-Categories}\label{catlim}

The $\infty$-category $\Cat_{\infty}$ can be identified with the simplicial nerve of
$(\mSet)^{\degree}$. It follows from Corollary \ref{limitsinmodel} that $\Cat_{\infty}$ admits (small) limits and colimits, which can be computed in terms of homotopy (co)limits in the model category $\mSet$. For many applications, it is convenient to be able to construct limits and colimits while working entirely in the setting of $\infty$-categories. We will describe the construction of limits in this section; the case of colimits will be discussed in \S \ref{catcolim}.

Let $p: S^{op} \rightarrow \Cat_{\infty}$ be a diagram in $\Cat_{\infty}$. Then $p$ classifies
a Cartesian fibration $q: X \rightarrow S$. We will show (Corollary \ref{blurt} below) that
the limit $\projlim(p) \in \Cat_{\infty}$ can be identified with the $\infty$-category of {\em Cartesian} sections of $q$. We begin by proving a more precise assertion:

\begin{proposition}\label{charcatlimit}\index{gen}{limit!of $\infty$-categories}
Let $K$ be a simplicial set, $\overline{p}: K^{\triangleright} \rightarrow \Cat^{op}_{\infty}$ a diagram
in the $\infty$-category of spaces, $\overline{X} \rightarrow K^{\triangleright}$ a Cartesian fibration classified by $\overline{p}$, and $X = \overline{X} \times_{K^{\triangleright}} K$. 
The following conditions are equivalent:
\begin{itemize}
\item[$(1)$] The diagram $\overline{p}$ is a colimit of $p = \overline{p} | K$.

\item[$(2)$] The restriction map
$$ \theta: \bHom^{\flat}_{K^{\triangleright}}( (K^{\triangleright})^{\sharp}, \overline{X}^{\natural}) \rightarrow \bHom^{\flat}_{K}(K^{\sharp}, X^{\natural})$$
is an equivalence of $\infty$-categories.
\end{itemize}
\end{proposition}

\begin{proof}
According to Proposition \ref{cofinalcategories}, there exists a small category $\calC$ and a cofinal map $f: \Nerve(\calC) \rightarrow K$; let $\overline{\calC}= \calC \star [0]$ be the category obtained from $\calC$ by adjoining a new final object, and let
$\overline{f}: \Nerve(\overline{\calC}) \rightarrow K^{\triangleright}$ be the induced map (which is also cofinal). The maps $f$ and $\overline{f}$ are contravariant equivalences in
$(\sSet)_{/ K^{\triangleright} }$, and therefore induce Cartesian equivalences
$$ \Nerve(\calC)^{\sharp} \rightarrow K^{\sharp}$$
$$ \Nerve(\overline{\calC})^{\sharp} \rightarrow (K^{\triangleright})^{\sharp}.$$
We have a commutative diagram
$$ \xymatrix{ \bHom^{\flat}_{K^{\triangleright}}( (K^{\triangleright})^{\sharp}, 
\overline{X}^{\natural} ) \ar[r]^{\theta} \ar[d] & 
\bHom^{\flat}_{K^{\triangleright}}( K^{\sharp}, 
\overline{X}^{\natural} ) \ar[d] \\
\bHom^{\flat}_{K^{\triangleright}}( \Nerve(\overline{\calC})^{\sharp}, 
\overline{X}^{\natural} ) \ar[r]^{\theta'} & 
\bHom^{\flat}_{K^{\triangleright}}( \Nerve(\calC)^{\sharp}, 
\overline{X}^{\natural} ). }$$
The vertical arrows are categorical equivalences. Consequently, condition $(2)$ holds for 
$\overline{p}: K^{\triangleright} \rightarrow \Cat_{\infty}^{op}$ if and only if condition $(2)$
holds for the composition $\Nerve(\overline{\calC}) \rightarrow K^{\triangleright} \rightarrow
\Cat_{\infty}^{op}$. We may therefore assume without loss of generality that
$K = \Nerve(\calC)$. 

Using Corollary \ref{strictify}, we may further suppose that $\overline{p}$ is obtained as the simplicial nerve of a functor $\overline{\calF}: \overline{\calC}^{op} \rightarrow (\mSet)^{\degree}$. 
Changing $\overline{\calF}$ if necessary, we may suppose that it is a {\em strongly} fibrant
diagram in $\mSet$. Let $\calF = \overline{\calF}|\calC^{op}$. 
Let $\overline{\phi}: \sCoNerve[K^{\triangleright}]^{op} \rightarrow \overline{\calC}^{op}$
be the counit map, and $\phi: \sCoNerve[K]^{op} \rightarrow \calC^{op}$ the restriction of $\overline{\phi}$. We may assume without loss of generality that $\overline{X} = \St^{+}_{\phi} \overline{\calF}$. We have a (not strictly commutative) diagram of categories and functors
$$ \xymatrix{ \mSet \ar[d]^{\St^{+}_{\ast}} \ar[r]^{\times K^{\sharp}} & (\mSet)_{/K} \ar[d]^{\St^{+}_{\phi}} \\
\mSet \ar[r]^{\delta} & (\mSet)^{\calC^{op}}, }$$
where $\delta$ denotes the diagonal functor. This diagram commutes up to a natural transformation
$$ \St^{+}_{\phi}( K^{\sharp} \times Z) \rightarrow
\St^{+}_{\phi}(K^{\sharp}) \boxtimes \St^{+}_{\ast}(Z) \rightarrow \delta( \St^{+}_{\ast} Z ).$$
Here the first map is a weak equivalence by Proposition \ref{spek3}, and the second map is a weak  equivalence because $L \St^{+}_{\phi}$ is an equivalence of categories (Theorem \ref{straightthm}) and therefore carries the final object $K^{\sharp} \in \h{(\mSet)_{/K}}$ to a final object of
$\h{(\mSet)^{\calC^{op}}}$. We therefore obtain a diagram of {\em right} derived functors
$$ \xymatrix{ \h{\mSet} & \h{(\mSet)_{/K}} \ar[l]^{\Gamma} \\
\h{\mSet} \ar[u]^{R \Un^{+}_{\ast}} & \h{(\mSet)^{\calC^{op}}} \ar[l] \ar[u]^{R \Un^{+}_{\phi}}, }$$
which commutes up to natural isomorphism, where we regard $(\mSet)^{\calC^{op}}$ as equipped with the {\em injective} model structure described in \S \ref{quasilimit3}. Similarly, we have a commutative diagram
$$ \xymatrix{ \h{\mSet}& \h{(\mSet)_{/K^{\triangleright}}} \ar[l]^{\Gamma'} \\
\h{\mSet} \ar[u]^{R \Un^{+}_{\ast}} & \h{(\mSet)^{\overline{\calC}^{op}}} \ar[l] \ar[u]^{R \Un^{+}_{\overline{\phi}}}. }$$
Condition $(2)$ is equivalent to the assertion that the restriction map
$\Gamma'(\overline{X}^{\natural}) \rightarrow \Gamma(X^{\natural})$ is an isomorphism in
$\h{\mSet}$. Since the vertical functors in both diagrams are equivalences of categories (Theorem \ref{straightthm}), this is equivalent to the assertion that the map
$$ \varprojlim \overline{\calF} \rightarrow \varprojlim \calF$$
is a weak equivalence in $\mSet$. Since $\overline{\calC}$ has an initial object $v$, $(2)$ is equivalent to the assertion that $\overline{\calF}$ exhibits $\overline{\calF}(v)$ as a homotopy limit
of $\calF$ in $(\mSet)^{\degree}$. Using Theorem \ref{colimcomparee}, we conclude that
$(1) \Leftrightarrow (2)$ as desired.
\end{proof}

It follows from Proposition \ref{charcatlimit} that limits in $\Cat_{\infty}$ are computed by forming $\infty$-categories of Cartesian sections:

\begin{corollary}\label{blurt}
Let $p: K \rightarrow \Cat^{op}_{\infty}$ be a diagram in the $\infty$-category $\Cat_{\infty}$ of spaces and let
$X \rightarrow K$ be a Cartesian fibration classified by $p$. There
is a natural isomorphism 
$$ \projlim(p) \simeq \bHom^{\flat}_{K}( K^{\sharp}, X^{\natural} ) $$
in the homotopy category $\h{\Cat_{\infty}}$.
\end{corollary}

\begin{proof}
Let $\overline{p}: (K^{\triangleright})^{op} \rightarrow \Cat_{\infty}^{op}$ be a limit of $p$, and let
$X' \rightarrow K^{\triangleright}$ be a Cartesian fibration classified by
$\overline{p}$. Without loss of generality we may suppose $X \simeq X' \times_{K^{\triangleright}} K$. We have maps
$$ \bHom^{\flat}_{K}( K^{\sharp}, X^{\natural}) \leftarrow \bHom^{\flat}_{K^{\triangleright}}( (K^{\triangleright})^{\sharp}, {X'}^{\natural})
\rightarrow \bHom^{\flat}_{K^{\triangleright} } ( \{v\}^{\sharp}, {X'}^{\natural}),$$
where $v$ denotes the cone point of $K^{\triangleright}$. Proposition \ref{charcatlimit} implies
that the left map is an equivalence of $\infty$-categories. Since the inclusion
$\{v\}^{\sharp} \subseteq (K^{\triangleright})^{\sharp}$ is marked anodyne, the map on the right is a trivial fibration. We now conclude by observing that the space
$ \bHom^{\flat}_{K^{\triangleright} }( \{v\}^{\sharp}, {X'}^{\natural}) \simeq
X' \times_{ K^{\triangleright} } \{v\}$ can be identified with $\overline{p}(v) = \varprojlim(p)$.
\end{proof}

Using Proposition \ref{charcatlimit}, we can easily deduce an analogous characterization of
limits in the $\infty$-category of spaces.

\begin{corollary}\label{charspacelimit}\index{gen}{limit!of spaces}
Let $K$ be a simplicial set, $\overline{p}: K^{\triangleleft} \rightarrow \SSet$ a diagram
in the $\infty$-category of spaces, and $X \rightarrow K^{\triangleleft}$ a left fibration classified by $\overline{p}$. The following conditions are equivalent:
\begin{itemize}
\item[$(1)$] The diagram $\overline{p}$ is a limit of $p = \overline{p} | K$.

\item[$(2)$] The restriction map 
$$ \bHom_{K^{\triangleleft}}( K^{\triangleleft}, X) \rightarrow
\bHom_{K^{\triangleleft}}(K, X)$$
is a homotopy equivalence of Kan complexes.
\end{itemize}
\end{corollary}

\begin{proof}
The usual model structure on $\sSet$ is a localization of the Joyal model structure. It follows that the inclusion $\Kan \subseteq \Cat^{\Delta}_{\infty}$ preserves homotopy limits (of diagrams indexed by categories). Using Theorem \ref{colimcomparee}, Proposition \ref{cofinalcategories}, and Corollary \ref{strictify}, we conclude that the inclusion $\SSet \subseteq \Cat_{\infty}$ preserves (small) limits.
The desired equivalence now follows immediately from Proposition \ref{charcatlimit}.
\end{proof}

\begin{corollary}\label{needta}
Let $p: K \rightarrow \SSet$ be a diagram in the $\infty$-category $\SSet$ of spaces, and let
$X \rightarrow K$ be a left fibration classified by $p$. There
is a natural isomorphism
$$ \projlim(p) \simeq \bHom_{K}( K, X ) $$
in the homotopy category $\calH$ of spaces.
\end{corollary}

\begin{proof}
Apply Corollary \ref{blurt}.
\end{proof}

\begin{remark}
It is also possible to adapt the proof of Proposition \ref{charcatlimit} to give a direct proof of
Corollary \ref{charspacelimit}. We leave the details to the reader.
\end{remark}

\subsection{Colimits of $\infty$-Categories}\label{catcolim}

In this section, we will address the problem of constructing {\em colimits} in
the $\infty$-category $\Cat_{\infty}$. Let $p: S^{op} \rightarrow \Cat_{\infty}$ be
diagram, classifying a Cartesian fibration $f: X \rightarrow S$. In \S \ref{catlim}, we saw that
$\projlim(p)$ can be identified with the $\infty$-category of Cartesian sections of $f$. To construct the colimit $\injlim(p)$, we need to find an $\infty$-category which admits a map {\em from} each fiber $X_{s}$. The natural candidate, of course, is $X$ itself. However, because $X$ is generally not an $\infty$-category, we must take some care to formulate a correct statement.

\begin{lemma}\label{wilkins}
Let 
$$\xymatrix{
X' \ar[r] \ar[d] & X \ar[d]^{p} \\
S' \ar[r]^{q} & S }$$
be a pullback diagram of simplicial sets, where $p$ is a Cartesian fibration and
$q^{op}$ is cofinal. The induced map ${X'}^{\natural} \rightarrow X^{\natural}$ is
a Cartesian equivalence $($in $\mSet${}$)$.
\end{lemma}

\begin{proof}
Choose a cofibration $S' \rightarrow K$, where $K$ is a contractible Kan complex.
The map $q$ factors as a composition
$$ S' \stackrel{q'}{\rightarrow} S \times K \stackrel{q''}{\rightarrow} S.$$
It is obvious that the projection $X^{\natural} \times K^{\sharp} \rightarrow X^{\natural}$ is a Cartesian equivalence. We may therefore replace $S$ by $S \times K$ and $q$ by $q'$, thereby reducing to the case where $q$ is a cofibration. Proposition \ref{cofbasic} now implies that
$q$ is left-anodyne. It is easy to see that the collection of cofibrations $q: S' \rightarrow S$ for which the desired conclusion holds is weakly saturated. We may therefore reduce to the case where
$q$ is a horn inclusion $\Lambda^n_i \subseteq \Delta^n$, where $0 \leq i < n$.

We now apply Proposition \ref{simplexplay} to choose a sequence of composable maps
$$ \phi: A^0 \leftarrow \ldots \leftarrow A^n $$ and
a quasi-equivalence $M(\phi) \rightarrow X$. We have a commutative diagram of marked simplicial sets
$$ \xymatrix{ M^{\natural}(\phi) \times_{ (\Delta^n)^{\sharp} } (\Lambda^n_i)^{\sharp}
\ar@{^{(}->}[d]^{i} \ar[r] & {X'}^{\natural} \ar@{^{(}->}[d] \\
M^{\natural}(\phi) \ar[r] & X, }$$
Using Proposition \ref{halfy}, we deduce that the horizontal maps are Cartesian equivalences. To complete the proof, it will suffice to show that $i$ is a Cartesian equivalence. We now observe that $i$ is a pushout of the inclusion $i'': (\Lambda^n_i)^{\sharp} \times (A^n)^{\flat}
\subseteq (\Delta^n)^{\sharp} \times (A^n)^{\flat}$. It will therefore suffice to prove that $i''$
is a Cartesian equivalence. Using Proposition \ref{urlt}, we are reduced to proving that the inclusion
$(\Lambda^n_i)^{\sharp} \subseteq (\Delta^n)^{\sharp}$ is a Cartesian equivalence. According to Proposition \ref{strstr}, this is equivalent to the assertion that the horn inclusion
$\Lambda^n_i \subseteq \Delta^n$ is a weak homotopy equivalence, which is obvious.
\end{proof}

\begin{proposition}\label{charcatcolimit}\index{gen}{colimit!of $\infty$-categories}
Let $K$ be a simplicial set, $\overline{p}: K^{\triangleleft} \rightarrow \Cat_{\infty}^{op}$ be a diagram in the $\infty$-category $\Cat_{\infty}$, 
$\overline{X} \rightarrow K^{\triangleleft}$ a Cartesian fibration classified
by $\overline{p}$, and $X = \overline{X} \times_{ K^{\triangleleft} } K$.
The following conditions are equivalent:
\begin{itemize}
\item[$(1)$] The diagram $\overline{p}$ is a limit of $p = \overline{p} | K$.
\item[$(2)$] The inclusion $X^{\natural} \subseteq \overline{X}^{\natural}$ is a Cartesian equivalence in $(\mSet)_{/K^{\triangleleft}}$.
\item[$(3)$] The inclusion $X^{\natural} \subseteq \overline{X}^{\natural}$
is a Cartesian equivalence in $\mSet$.
\end{itemize}
\end{proposition}

\begin{proof}
Using the small object argument, we can construct a factorization
$$ X \stackrel{i}{\rightarrow} Y \stackrel{j}{\rightarrow} K^{\triangleleft} $$
where $j$ is a Cartesian fibration, $i$ induces a marked anodyne map
$X^{\natural} \rightarrow Y^{\natural}$, and 
$X \simeq Y \times_{ K^{\triangleleft} } K$. 
Since $i$ is marked anodyne, we can solve the lifting problem
$$ \xymatrix{ X^{\natural} \ar@{^{(}->}[d]^{i} \ar[r] & \overline{X}^{\natural} \ar[d] \\
Y^{\natural} \ar[r] \ar@{-->}[ur]^{q} & (K^{\triangleleft})^{\sharp}. }$$
Since $i$ is a Cartesian equivalence in $(\mSet)_{/K^{\triangleleft}}$, condition $(2)$ is equivalent to the assertion that $q$ is an equivalence of Cartesian fibrations over $K^{\triangleleft}$. Since $q$
induces an isomorphism over each vertex of $K$, this is equivalent to:
\begin{itemize}
\item[$(2')$] The map $q_{v}: Y_{v} \rightarrow \overline{X}_{v}$ is an equivalence of $\infty$-categories, where $v$ denotes the cone point of $K^{\triangleleft}$.
\end{itemize}
We have a commutative diagram
$$ \xymatrix{ Y^{\natural}_{v} \ar[r]^{q_v} \ar@{^{(}->}[d] & \overline{X}^{\natural}_v \ar@{^{(}->}[d] \\
Y^{\natural} \ar[r]^{q} & \overline{X}^{\natural}. }$$
Lemma \ref{wilkins} implies that the vertical maps are Cartesian equivalences. It follows that $(2') \Leftrightarrow (3)$, so that $(2) \Leftrightarrow (3)$.

To complete the proof, we will show that $(1) \Leftrightarrow (2)$.
According to Proposition \ref{cofinalcategories}, there exists a small category $\calC$ and a map $p: \Nerve(\calC) \rightarrow K$ such that $p^{op}$ is cofinal. Let $\overline{\calC} = [0] \star \calC$ be the category obtained by adjoining an initial object to $\calC$. 
Consider the diagram
$$ \xymatrix{ (X \times_{K} \Nerve(\calC))^{\natural} \ar@{^{(}->}[r] \ar[d] & (\overline{X} \times_{K^{\triangleleft}} \Nerve(\overline{\calC}))^{\natural} \ar[d] \\
X^{\natural} \ar@{^{(}->}[r] & \overline{X}^{\natural}. }$$
Lemma \ref{wilkins} implies that the vertical maps are Cartesian equivalences (in $\mSet$).
It follows that the upper horizontal inclusion is a Cartesian equivalence if and only if the lower horizontal inclusion is a Cartesian equivalence. Consequently, it will suffice to prove the
equivalence $(1) \Leftrightarrow (2)$ after replacing $K$ by $\Nerve(\calC)$.

Using Corollary \ref{strictify}, we may further suppose that $\overline{p}$ is the nerve
of a functor $\calF: \overline{\calC} \rightarrow (\mSet)^{\degree}$. Let $\overline{\phi}: \sCoNerve[K^{\triangleleft}] \rightarrow \overline{\calC}$ be the counit map, and let $\phi: \sCoNerve[K] \rightarrow \calC$ be the restriction of $\overline{\phi}$. Without loss of generality, we may suppose that $\overline{X} = \Un_{\overline{\phi}} \calF$. We have a commutative diagram of homotopy categories and right derived functors
$$ \xymatrix{ \h{(\mSet)^{\overline{\calC}}} \ar[r]^{G} \ar[d]^{R \Un^{+}_{\overline{\phi}}} &  \h{(\mSet)^{\calC}}
\ar[d]^{ R \Un^{+}_{\phi} } \\
\h{(\mSet)_{/(K^{\triangleleft})}} \ar[r]^{G'} & \h{(\mSet)_{/K}} }$$
where $G$ and $G'$ are restriction functors. Let $F$ and $F'$ be the left adjoints
to $G$ and $G'$, respectively. According to Theorem \ref{colimcomparee}, assumption $(1)$ is equivalent to the assertion that $\calF$ lies in the essential image of $F$. Since each of the vertical functors is equivalence of categories (Theorem \ref{straightthm}), this is equivalent to the assertion that $\overline{X}$ lies in the essential image of $F'$. 
Since $F'$ is fully faithful, this is equivalent to the assertion that the counit map
$$ F' G' \overline{X} \rightarrow \overline{X}$$
is an isomorphism in $\h{(\mSet)_{/K^{\triangleleft})}}$, which is clearly a reformulation of $(2)$.
\end{proof}

\begin{corollary}\label{tolbot}
Let $p: K^{op} \rightarrow \Cat_{\infty}$ be a diagram, classifying a Cartesian fibration
$X \rightarrow K$. Then there is a natural isomorphism
$\varinjlim(p) \simeq X^{\natural}$ in the homotopy category
in $\h{\Cat_{\infty}}$.
\end{corollary}

\begin{proof}
Let $\overline{p}: (K^{op})^{\triangleright} \rightarrow \Cat_{\infty}$ be a colimit of $p$, classifying a Cartesian fibration $\overline{X} \rightarrow K^{\triangleleft}$. Let $v$ denote the cone point of
$K^{\triangleleft}$, so that $\varinjlim(p) \simeq \overline{X}_{v}$. We now observe that the inclusions
$$\overline{X}^{\natural}_{v} \hookrightarrow \overline{X}^{\natural} \hookleftarrow X^{\natural}$$
are both Cartesian equivalences (Lemma \ref{wilkins} and Proposition \ref{charcatcolimit}). 
\end{proof}

\begin{warning}
In the situation of Corollary \ref{tolbot}, the marked simplicial set $X^{\natural}$ is usually not a fibrant object of $\mSet$, even when $K$ is an $\infty$-category.
\end{warning}

Using exactly the same argument, we can establish a version of Proposition \ref{charcatcolimit} which describes colimits in the $\infty$-category of spaces:

\begin{proposition}\label{charspacecolimit}\index{gen}{colimit!of spaces}
Let $K$ be a simplicial set, $\overline{p}: K^{\triangleright} \rightarrow \SSet$ be a diagram in the $\infty$-category of spaces, $\overline{X} \rightarrow K^{\triangleright}$ a left fibration classified
by $\overline{p}$, and $X = \overline{X} \times_{ K^{\triangleright} } K$.
The following conditions are equivalent:
\begin{itemize}
\item[$(1)$] The diagram $\overline{p}$ is a colimit of $p = \overline{p} | K$.
\item[$(2)$] The inclusion $X \subseteq \overline{X}$ is a covariant equivalence in
$(\sSet)_{/K^{\triangleright}}$.
\item[$(3)$] The inclusion $X \subseteq \overline{X}$
is a weak homotopy equivalence of simplicial sets.
\end{itemize}
\end{proposition}

\begin{proof}
Using the small object argument, we can construct a factorization
$$ X \stackrel{i}{\hookrightarrow} Y \stackrel{j}{\rightarrow} K^{\triangleright} $$
where $i$ is left anodyne, $j$ is a left fibration, and the inclusion
$X \subseteq Y \times_{K^{\triangleright}} K$ is an isomorphism. Choose a dotted arrow
$q$ as indicated in the diagram
$$ \xymatrix{ X \ar@{^{(}->}[d]^{i} \ar[r] & \overline{X} \ar[d] \\
Y \ar[r] \ar@{-->}[ur]^{q} & K^{\triangleright}. }$$
Since $i$ is a covariant equivalence in $(\sSet)_{/K^{\triangleright}}$, condition $(2)$ is equivalent to the assertion that $q$ is an equivalence of left fibrations over $K^{\triangleright}$. Since $q$
induces an isomorphism over each vertex of $K$, this is equivalent to the assertion that
$q_{v}: Y_{v} \rightarrow \overline{X}_{v}$ is an equivalence, where $v$ denotes the cone point
of $K^{\triangleright}$. We have a commutative diagram
$$ \xymatrix{ Y_{v} \ar[r]^{q_v} \ar[d] & \overline{X}_v \ar[d] \\
Y \ar[r]^{q} & \overline{X}. }$$
Proposition \ref{strokhop} implies that the vertical maps are right anodyne, and therefore weak homotopy equivalences. Consequently, $q_{v}$ is a weak homotopy equivalence if and only if $q$ is a weak homotopy equivalence. Since the inclusion $X \subseteq Y$ is a weak homotopy equivalence, this proves that $(2) \Leftrightarrow (3)$.

To complete the proof, we will show that $(1) \Leftrightarrow (2)$.
According to Proposition \ref{cofinalcategories}, there exists a small category $\calC$ and a cofinal map $\Nerve(\calC) \rightarrow K$. Let $\overline{\calC} = \calC \star [0]$ be the category obtained from $\calC$ by adjoining a new final object.
Consider the diagram
$$ \xymatrix{ X \times_{K} \Nerve(\calC) \ar@{^{(}->}[r] \ar[d] & \overline{X} \times_{K^{\triangleright}} \Nerve(\overline{\calC}) \ar[d] \\
X \ar@{^{(}->}[r] & \overline{X}. }$$
Proposition \ref{strokhop} implies that $\overline{X} \rightarrow K^{\triangleright}$ is smooth, so that the vertical arrows in the above diagram are cofinal. In particular, the vertical arrows are weak homotopy equivalences, so that the upper horizontal inclusion is a weak homotopy equivalence
if and only if the lower horizontal inclusion is a weak homotopy equivalence. Consequently, 
it will suffice to prove the equivalence $(1) \Leftrightarrow (2)$ after replacing $K$ by $\Nerve (\calC)$.

Using Corollary \ref{strictify}, we may further suppose that $\overline{p}$ is obtained as the nerve of a functor $\calF: \overline{\calC} \rightarrow \Kan$. Let $\overline{\phi}: \sCoNerve[K^{\triangleright}] \rightarrow \overline{\calC}$ be the counit map, and let $\phi: \sCoNerve[K] \rightarrow \calC$ be the restriction of $\overline{\phi}$. Without loss of generality, we may suppose that $\overline{X}^{op} = \Un_{\overline{\phi}} \calF$. We have a commutative diagram of homotopy categories and right derived functors
$$ \xymatrix{ \h{(\sSet)^{\overline{\calC}}} \ar[r]^{G} \ar[d]^{R \Un_{\overline{\phi}}} & \h{(\sSet)^{\calC}}
\ar[d]^{ R \Un_{\phi} } \ar[d] \\
\h{ (\sSet)_{/ (K^{\triangleright})^{op}}} \ar[r]^{G'} & \h{ (\sSet)_{/K}} }$$
where $G$ and $G'$ are restriction functors. Let $F$ and $F'$ be the left adjoints
to $G$ and $G'$, respectively. According to Theorem \ref{colimcomparee}, assumption $(1)$ is equivalent to the assertion that $\calF$ lies in the essential image of $F$. Since each of the vertical functors is equivalence of categories (Theorem \ref{struns}), this is equivalent to the assertion that
$\overline{X}^{op}$ lies in the essential image of $F'$. Since $F'$ is fully faithful, this is equivalent to the assertion that the counit map
$$ F' G' \overline{X}^{op} \rightarrow \overline{X}^{op}$$
is an isomorphism in $\h{ (\sSet)_{/ (K^{\triangleright})^{op}}}$, which is clearly equivalent to $(2)$.
This shows that $(1) \Leftrightarrow (2)$ and completes the proof.
\end{proof}

\begin{corollary}\label{needka}
Let $p: K \rightarrow \SSet$ be a diagram which classifies a left fibration
$\widetilde{K} \rightarrow K$, and let $X \in \SSet$ be a colimit of $p$. Then
there is a natural isomorphism
$$ \widetilde{K} \simeq X$$
in the homotopy category $\calH$.
\end{corollary}

\begin{proof}
Let $\overline{p}: K^{\triangleright} \rightarrow \SSet$ be a colimit diagram which extends $p$, and
$\widetilde{K}' \rightarrow K^{\triangleright}$ a left fibration classified by $\overline{p}$. Without loss of generality, we may suppose that $\widetilde{K} = \widetilde{K}' \times_{K^{\triangleright}} K$ and
$X = \widetilde{K}' \times_{K^{\triangleright}} \{v\}$, where $v$ denotes the cone point of $K^{\triangleright}$. Since the inclusion $\{v\} \subseteq K^{\triangleright}$ is right anodyne
and the map $\widetilde{K}' \rightarrow K^{\triangleright}$ is a left fibration, Proposition \ref{strokhop} implies that the inclusion
$X \subseteq \widetilde{K}'$ is right anodyne, and therefore a weak homotopy equivalence.
On the other hand, Proposition \ref{charspacecolimit} implies that the inclusion
$\widetilde{K} \subseteq \widetilde{K}'$ is a weak homotopy equivalence. The
composition
$$ X \simeq \widetilde{K}' \simeq \widetilde{K}$$
is the desired isomorphism in $\calH$.
\end{proof}
