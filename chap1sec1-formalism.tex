% !TEX root = highertopoi.tex
\section{Foundations for Higher Category Theory}
\subsection{Goals and Obstacles}\label{highcat}

Recall that a {\it category}\index{gen}{category} $\calC$ consists of the following data:
\begin{itemize}
\item[$(1)$] A collection $\{ X, Y, Z, \ldots \}$ whose members are the {\it objects} of $\calC$. We typically
write $X \in \calC$ to indicate that $X$ is an object of $\calC$.
\item[$(2)$] For every pair of objects $X,Y \in \calC$, a set $\Hom_{\calC}(X,Y)$ of {\it morphisms} from
$X$ to $Y$. We will typically write $f: X \rightarrow Y$ to indicate that $f \in \Hom_{\calC}(X,Y)$, and say that $f$ {\it is a morphism from $X$ to $Y$}.
\item[$(3)$] For every object $X \in \calC$, an {\it identity morphism} $\id_{X} \in \Hom_{\calC}(X,X)$.

\item[$(4)$] For every triple of objects $X,Y, Z \in \calC$, a composition map
$$ \Hom_{\calC}(X,Y) \times \Hom_{\calC}(Y,Z) \rightarrow \Hom_{\calC}(X,Z).$$
Given morphisms $f: X \rightarrow Y$ and $g: Y \rightarrow Z$, we will usually denote the image of
the pair $(f,g)$ under the composition map by $gf$ or $g \circ f$. 
\end{itemize}

These data are furthermore required to satisfy the following conditions, which guarantee that composition is unital and associative:

\begin{itemize}
\item[$(5)$] For every morphism $f: X \rightarrow Y$, we have
$\id_Y \circ f = f = f \circ \id_{X}$ in $\Hom_{\calC}(X,Y)$.
\item[$(6)$] For every triple of composable morphisms
$$ W \stackrel{f}{\rightarrow} X \stackrel{g}{\rightarrow} Y \stackrel{h}{\rightarrow} Z,$$
we have an equality $h \circ (g \circ f) = (h \circ g) \circ f$ in $\Hom_{\calC}(W,Z)$.
\end{itemize}

The theory of categories has proven to be a valuable organization tool in many areas of mathematics. Mathematical structures of virtually any type can be viewed as the objects of a suitable category $\calC$, where the morphisms in $\calC$ are given by structure-preserving maps. There is a veritable legion of examples of categories which fit this paradigm:
\begin{itemize}
\item The category $\Set$ whose objects are sets and whose morphisms are maps of sets.
\item The category $\Group$ whose objects are groups and whose morphisms are group homomorphisms.
\item The category $\Top$ whose objects are topological spaces and whose morphisms are continuous maps.
\item The category $\Cat$ whose objects are (small) categories and whose morphisms
are functors. (Recall that a functor $F$ from $\calC$ to $\calD$ is a map which assigns to each object
$C \in \calC$ another object $FC \in \calD$ and to each morphism $f: C \rightarrow C'$ in
$\calC$ a morphism $F(f): FC \rightarrow FC'$ in $\calD$, so that $F( \id_C) = \id_{FC}$ and
$F(g \circ f) = F(g) \circ F(f)$.)
\item \ldots
\end{itemize}

In general, the existence of a morphism $f: X \rightarrow Y$ in a category $\calC$ reflects some relationship that exists between the objects $X,Y \in \calC$. In some contexts, these relationships themselves become basic objects of study, and can themselves be fruitfully organized into categories:

\begin{example}\label{2cat1}
Let $\Group$ be the category whose objects are groups and whose morphisms are group homomorphisms. In the theory of groups, one is often concerned only with group homomorphisms
{\em up to conjugacy}. The relation of conjugacy can be encoded as follows: for every pair of
groups $G, H \in \Group$, there is a category $\bHom(G,H)$ whose objects are group homomorphisms
from $G$ to $H$ (that is, elements of $\Hom_{\Group}(G,H)$), where a morphism from $f: G \rightarrow H$ to $f': G \rightarrow H$ is an element $h \in H$ such that $h f(g) h^{-1} = f'(g)$ for all $g \in G$.
Note that two group homomorphisms $f,f': G \rightarrow H$ are conjugate if and only if they are isomorphic when viewed as objects of $\bHom(G,H)$.
\end{example}

\begin{example}\label{2cat2}
Let $X$ and $Y$ be topological spaces, and let $f_0, f_1: X \rightarrow Y$ be continuous maps.
Recall that a {\it homotopy} from $f_0$ to $f_1$ is a continuous map $f: X \times [0,1] \rightarrow Y$
such that $f | X \times \{0\}$ coincides with $f_0$ and $f | X \times \{1\}$ coincides with $f_1$.
In algebraic topology, one is often concerned not with the category $\Top$ of topological spaces,
but with its {\it homotopy category}: that is, the category obtained by identifying those pairs of
morphisms $f_0, f_1: X \rightarrow Y$ which are homotopic to one another. For many purposes, it is better to do something a little bit more sophisticated: namely, one can form a category
$\bHom(X,Y)$ whose objects are continuous maps $f: X \rightarrow Y$ and whose morphisms
are given by (homotopy classes of) homotopies.
\end{example}

\begin{example}\label{2cat3}
Given a pair of categories $\calC$ and $\calD$, the collection of all functors from
$\calC$ to $\calD$ is itself naturally organized into a category $\Fun( \calC, \calD)$, where
the morphisms are given by {\it natural transformations}. (Recall that, given a pair of functors
$F,G: \calC \rightarrow \calD$, a natural transformation $\alpha: F \rightarrow G$ is a collection
of morphisms $\{ \alpha_{C}: F(C) \rightarrow G(C) \}_{C \in \calC}$ which satisfy the following
condition: for every morphism $f: C \rightarrow C'$ in $\calC$, the diagram
$$ \xymatrix{ F(C) \ar[r]^{F(f)} \ar[d]^{\alpha_C} & F(C') \ar[d]^{\alpha_{C'} } \\
G(C) \ar[r]^{G(f)} & G(C') }$$
commutes in $\calD$.)
\end{example}

In each of these examples, the objects of interest can naturally be organized into what is
called a {\it $2$-category} (or {\it bicategory}): we have not only a collection of objects and a notion of morphisms between objects, but also a notion of morphisms between morphisms, which
are called {\it $2$-morphisms}. The vision of higher category theory is that there should exist a good notion of $n$-category for all $n \geq 0$, in which we have not only objects,
morphisms, and $2$-morphisms, but also $k$-morphisms for all $k
\leq n$. Finally, in some sort of limit we might hope to obtain a theory
of $\infty$-categories, where there are morphisms of all orders.\index{gen}{$2$-category}\index{gen}{bicategory}

\begin{example}\label{grape}
Let $X$ be a topological space, and $0 \leq n \leq \infty$. We
can extract an $n$-category $\pi_{\leq n} X$ (roughly) as follows.
The objects of $\pi_{\leq n} X$ are the points of $X$. If $x,y \in
X$, then the morphisms from $x$ to $y$ in $\pi_{\leq n} X$ are
given by continuous paths $[0,1] \rightarrow X$ starting at $x$
and ending at $y$. The $2$-morphisms are given by homotopies of
paths, the $3$-morphisms by homotopies between homotopies, and so
forth. Finally, if $n < \infty$, then two $n$-morphisms of
$\pi_{\leq n} X$ are considered to be the same if and only if they are homotopic to one another.\index{not}{pileqnX@$\pi_{\leq n}X$}\index{gen}{fundamental $n$-groupoid}

If $n = 0$, then $\pi_{\leq n} X$ can be identified with the set $\pi_0 X$ of path
components of $X$. If $n=1$, then our definition of $\pi_{\leq n}
X$ agrees with usual definition for the fundamental groupoid of
$X$. For this reason, $\pi_{\leq n} X$ is often called the {\it
fundamental $n$-groupoid of $X$}. It is an {\it $n$-groupoid} (rather than a mere $n$-category) 
because every $k$-morphism of $\pi_{\leq k} X$ has an inverse (at least ``up to
homotopy'').
\end{example}
\begin{shaded}
(Roughly,) the fundamental $n$-groupoid of a topological space is an example of an $n$-category. It's called an $n$-groupoid as every $k$-morphism has an inverse ``up to homotopy''.
\end{shaded}

There are many approaches to realizing the theory of higher categories.
We might begin by defining a $2$-category to be a
``category enriched over $\Cat$.'' In other words, we consider a
collection of objects together with a {\em category} of morphisms
$\Hom(A,B)$ for any two objects $A$ and $B$, and composition {\em
functors} $c_{ABC}: \Hom(A,B) \times \Hom(B,C) \rightarrow
\Hom(A,C)$ (to simplify the discussion, we will ignore identity
morphisms for a moment). These functors are required to satisfy an
associative law, which asserts that for any quadruple $(A,B,C,D)$
of objects, the diagram
$$ \xymatrix{ \Hom(A,B) \times \Hom(B,C) \times \Hom(C,D) \ar[d]
 \ar[r] & \Hom(A,C) \times \Hom(C,D) \ar[d] \\
 \Hom(A,B) \times \Hom(B,D) \ar[r] & \Hom(A,D) }$$
 commutes; in other words, one has an {\em equality} of functors
$$c_{ACD} \circ (c_{ABC} \times 1) = c_{ABD} \circ (1
\times c_{BCD})$$
from $\Hom(A,B) \times \Hom(B,C) \times
\Hom(C,D)$ to $\Hom(A,D)$. This leads to the definition of
a {\it strict $2$-category}.\index{gen}{$2$-category!strict}
\begin{shaded}
To define a 2-category as a category enriched over $\Cat$ gives the notion of a strict 2-category: any two objects have a category of morphisms, with everything strict as can be. This is naughty --- the arrows in the above ``associativity of composition'' diagram are functors, and we're demanding that the diagram commutes. To ask two functors to be \textbf{equal} is very naughty.

We could define a weak 2-category to be one with natural isomorphisms in the above diagram, satisfying higher associativity conditions.
\end{shaded}


At this point, we should object that the definition of a strict
$2$-category violates one of the basic philosophical principles of
category theory: one should never demand that two functors $F$ and $F'$ be
equal to one another. Instead one should postulate the existence of a natural
isomorphism between $F$ and $F'$. This means that the associative
law should not take the form of an equation, but of additional
structure: a collection of isomorphisms $\gamma_{ABCD}: c_{ACD} \circ
(c_{ABC} \times 1) \simeq c_{ABD} \circ (1 \times c_{BCD})$. We
should further demand that the isomorphisms $\gamma_{ABCD}$ be
functorial in the quadruple $(A,B,C,D)$ and satisfy
certain higher associativity conditions, which generalize the ``Pentagon axiom''
described in \S \ref{monoidaldef}. After formulating the
appropriate conditions, we arrive at the definition of a {\it weak
$2$-category}.\index{gen}{$2$-category!weak}


Let us contrast the notions of ``strict $2$-category'' and ``weak
$2$-category.'' The former is easier to define, since we do not
have to worry about the higher associativity conditions satisfied
by the transformations $\gamma_{ABCD}$. On the other hand, the
latter notion seems more natural if we take the philosophy of
category theory seriously. In this case, we happen to be lucky:
the notions of ``strict $2$-category'' and ``weak $2$-category''
turn out to be equivalent. More precisely, any weak $2$-category
is equivalent (in the relevant sense) to a strict $2$-category. The
choice of definition can therefore be regarded as a question of
aesthetics.

We now plunge onward to $3$-categories. Following the above
program, we might define a {\it strict $3$-category} to consist of a
collection of objects together with strict $2$-categories
$\Hom(A,B)$ for any pair of objects $A$ and $B$, together with a
strictly associative composition law. Alternatively, we could seek
a definition of {\it weak $3$-category} by allowing $\Hom(A,B)$ to
be only a weak $2$-category, requiring associativity only up to
natural $2$-isomorphisms, which satisfy higher associativity laws
up to natural $3$-isomorphisms, which in turn satisfy still higher
associativity laws of their own. Unfortunately, it turns out that
these notions are {\em not} equivalent.
\begin{shaded}
Strict 2-categories and weak 2-categories actually turn out to be equivalent in the appropriate sense. On the other hand, we can define:
\begin{itemise}
\item A ``strict 3-category'' to be a collection of objects with a strict 2-category of maps between any two, with strictly associative composition; and
\item A ``strict 3-category'' to be a collection of objects with a weak 2-category of maps between any two, with composition associative up to natural 2-isomorphisms. These natural 2-isomorphisms would have to satisfy higher associativity laws, up to natural 3-isomorphisms, and this seems to go on forever.
\end{itemise}
These notions are not equivalent, but are both awful.
\end{shaded}


Both of these approaches have serious drawbacks. The obvious
problem with weak $3$-categories is that an explicit definition is
extremely complicated (see \cite{tricat}, where a definition is given along these lines), to the point where it is
essentially unusable. On the other hand, strict $3$-categories
have the problem of not being the correct notion: most of the weak
$3$-categories which occur in nature are not equivalent to
strict $3$-categories. For example, the fundamental $3$-groupoid of
the $2$-sphere $S^2$ cannot be described using the language of
strict $3$-categories. The situation only gets worse (from either
point of view) as we pass to $4$-categories and beyond.

Fortunately, it turns out that major simplifications can be
introduced if we are willing to restrict our attention to
$\infty$-categories in which most of the higher morphisms are
invertible. Let us henceforth use the term {\it $(\infty,n)$-category}\index{gen}{$(\infty,n)$-category}
to refer to $\infty$-categories in which all $k$-morphisms are
invertible for $k > n$. The $\infty$-categories described in
Example \ref{grape} (when $n=\infty$) are all
$(\infty,0)$-categories. The converse, which asserts that every
$(\infty,0)$-category has the form $\pi_{\leq \infty} X$ for some
topological space $X$, is a generally accepted principle of higher
category theory. Moreover, the $\infty$-groupoid $\pi_{\leq \infty} X$ encodes the entire homotopy type of $X$. In other words, $(\infty,0)$-categories (that is,
$\infty$-categories in which {\em all} morphisms are invertible)
have been extensively studied from another point of view: they are
essentially the same thing as ``spaces'' in the sense of homotopy
theory, and there are many equivalent ways to describe them (for
example, we can use CW complexes or simplicial sets).

\begin{convention}\index{gen}{$\infty$-groupoid}\index{gen}{$\infty$-bicategory}\index{gen}{$\infty$-category}
We will often refer to $(\infty,0)$-categories as {\it $\infty$-groupoids} and $(\infty,2)$-categories as {\it $\infty$-bicategories}. Unless otherwise specified, the generic term {\it $\infty$-category} will mean $(\infty,1)$-category. 
\end{convention}
\begin{shaded}
It is simpler to work with $(\infty,n)$-categories, $\infty$-categories all of whose $k$-morphisms for $k>n$ are invertible (what exactly is meant by invertible?). $(\infty,0)$-categories, called $\infty$-groupoids, are essentially the same as spaces. $(\infty,1)$-categories will be called $\infty$-categories. [Somehow, assuming that the higher morphisms are all invertible is like pretending they are not there?] Likewise $(\infty,2)$-categories will be called $\infty$-bicategories.
\end{shaded}

In this book, we will restrict our attention almost entirely to the theory of $\infty$-categories (in which we have only invertible $n$-morphisms for $n \geq 2$). Our reasons are threefold:
\begin{itemize}
\item[$(1)$] Allowing noninvertible $n$-morphisms for $n > 1$ introduces a number of additional complications to the theory, at both technical and conceptual levels. As we will see throughout this book, many ideas from category theory generalize to the $\infty$-categorical setting in a natural way. However, these generalizations are not so straightforward if we allow noninvertible $2$-morphisms. For example, one must distinguish between strict and lax fiber products, even in the setting of ``classical'' $2$-categories.

\item[$(2)$] For the applications studied in this book, we will not need to consider $(\infty,n)$-categories for $n > 2$. The case $n=2$ is of some relevance, because the collection of (small) $\infty$-categories can naturally be viewed as a (large) $\infty$-bicategory. However, we will generally be able to exploit this structure in an ad-hoc manner, without developing any general theory of $\infty$-bicategories.

\item[$(3)$] For $n > 1$, the theory of $(\infty,n)$-categories is most naturally viewed as a special case of {\em enriched} (higher) category theory. Roughly speaking, an $n$-category can be viewed as a category enriched over $(n-1)$-categories. As we explained above, this point of view is inadequate because it requires that composition satisfies an associative law up to equality, while in practice the associativity only holds up to isomorphism or some weaker notion of equivalence.
In other words, to obtain the correct definition we need to view the collection of $(n-1)$-categories
as an $n$-category, not as an ordinary category. Consequently, the naive approach is circular:  though it does lead to the correct theory of $n$-categories, we can only make sense of it if the theory of $n$-categories is already in place.
\begin{shaded}
We'd like to say that an $n$-category is just a category enriched over $(n-1)$-categories, but should instead view the collection of $(n-1)$-categories as an $n$-category! This is circular, but not inconsistent.
\end{shaded}

Thinking along similar lines, we can view an $(\infty,n)$-category as an $\infty$-category which is {\em enriched over $(\infty,n-1)$-categories}. The collection of $(\infty,n-1)$-categories is itself organized into an $(\infty,n)$-category $\Cat_{(\infty,n-1)}$, so at a first glance this definition suffers from the same problem of circularity. However, because the associativity properties of composition are required to hold up to {\em equivalence}, rather than up to arbitrary natural transformation, the noninvertible $k$-morphisms in $\Cat_{(\infty,n-1)}$ are irrelevant for $k > 1$. We may therefore view an $(\infty,n)$-category as a category enriched over $\Cat_{(\infty,n-1)}$, where the latter is regarded as an $\infty$-category by discarding noninvertible $k$-morphisms for $2 \leq k \leq n$.
In other words, the naive inductive definition of higher category theory is reasonable, {\em provided that we work in the $\infty$-categorical setting from the outset}. 
We refer the reader to \cite{tamsamani} for a definition of $n$-categories which follows this line of thought.
\begin{shaded}
\begin{enumerate}\squishlist
\item We'd like to say that an $(\infty,n)$-category is an $\infty$-category enriched over $(\infty,n-1)$-categories, but $\Cat_{(\infty,n-1)}$ is an $(\infty,n)$-category, so this sounds circular.
\item Associativity in an $(\infty,n)$-category should hold up to an \textit{equivalence} of $(\infty,n-1)$-functors, being the two routes around the following diagram in $\Cat_{(\infty,n-1)}$:
$$ \xymatrix{ \Hom(A,B) \times \Hom(B,C) \times \Hom(C,D) \ar[d]
 \ar[r] & \Hom(A,C) \times \Hom(C,D) \ar[d] \\
 \Hom(A,B) \times \Hom(B,D) \ar[r] & \Hom(A,D) }$$ 
\item Thus, all the 2-morphisms which are employed in the enriching category $\Cat_{(\infty,n-1)}$ are invertible, so we might as well enrich over the subcategory of $\Cat_{(\infty,n-1)}$ obtained by discarding all non-invertible $k$-morphisms for $k\geq 2$. This subcategory is an $\infty$-category.
\end{enumerate}
Thus, as long as we always work with $\infty$-categories, we can build a good theory of $(\infty,n)$-categories.
\end{shaded}

The theory of {\em enriched} $\infty$-categories is a useful and important one, but will not be treated in this book. Instead we refer the reader to \cite{DAG} for an introduction using the same language and formalism we employ here. 
\end{itemize}

Though we will not need a theory of $(\infty,n)$-categories for $n > 1$, the case $n=1$ is the main subject matter of this book. Fortunately, the above discussion suggests a definition. Namely, an $\infty$-category $\calC$ should be consist of a collection of objects, and an $\infty$-groupoid 
$\bHom_{\calC}(X,Y)$ for every pair of objects $X,Y \in \calC$. These $\infty$-groupoids can be identified with ``spaces'', and should be equipped with an associative composition law.
As before, we are faced with two choices as to how to make this
precise: do we require associativity on the nose, or only up to (coherent)
homotopy? Fortunately, the answer turns out to be irrelevant:
as in the theory of $2$-categories, any $\infty$-category
with a coherently associative multiplication can be replaced by an
equivalent $\infty$-category with a strictly associative
multiplication. We are led to the following:

\begin{definition}\label{ic}\index{gen}{topological category}\index{gen}{category!topological}
A {\it topological category} is a category which is enriched over
$\CG$, the category of compactly generated (and weakly Hausdorff) topological
spaces. The category of topological categories will be denoted by
$\tCat$.\index{not}{CG@$\CG$}\index{not}{Cattop@$\tCat$}
\end{definition}

More explicitly, a topological category $\calC$ consists of a
collection of objects, together with a (compactly generated)
topological space $\bHom_{\calC}(X,Y)$ for any pair of objects $X,Y
\in \calC$. These mapping spaces must be equipped with an
associative composition law, given by continuous maps
$$\bHom_{\calC}(X_0, X_1) \times \bHom_{\calC}(X_1, X_2) \times
\ldots \bHom_{\calC}(X_{n-1},X_n) \rightarrow
\bHom_{\calC}(X_0,X_n)$$ (defined for all $n \geq 0$). Here the
product is taken in the category of compactly generated
topological spaces.
\begin{shaded}
An $(\infty,1)$-category should be a category enriched over $(\infty,0)$-categories, a.k.a.\ $\infty$-groupoids, which can be identified with ``spaces'' with an associative composition law. The law should really be ``up to coherent homotopy'', but any $\infty$-category
with a coherently associative multiplication can be strictified, so we use the strict version, defining the notion of ``topological category''.

This can be used as the definition of $\infty$-category, but there are technical disadvantages.
\end{shaded}

\begin{remark}
The decision to work with compactly generated topological spaces,
rather than arbitrary spaces, is made in order to facilitate the comparison with more
combinatorial approaches to homotopy theory. This is a purely
technical point which the reader may safely ignore.
\end{remark}

It is possible to use Definition \ref{ic} as a foundation for higher category theory: that is, to {\em define} an $\infty$-category to be a topological category. However, this approach has a number of technical disadvantages. We will describe an alternative (though equivalent) formalism in the next section.

\subsection{$\infty$-Categories}\label{qqqc}

Of the numerous formalizations of higher category theory, Definition \ref{ic} is the quickest and most transparent. However, it is one of the most difficult to actually work with: many of the basic constructions of higher category theory give rise most naturally to $(\infty,1)$-categories for which the composition of
morphisms is associative only up to (coherent) homotopy (for several examples of this phenomenon, we refer the reader to \S \ref{langur}). In order to remain in the world of topological categories, it is necessary to combine these constructions with a ``straightening'' procedure which
produces a strictly associative composition law. Although it is always possible to do this
(see Theorem \ref{biggier}), it is much more technically convenient to work from the outset
within a more flexible theory of $(\infty,1)$-categories. Fortunately, there are many candidates
for such a theory, including the theory of Segal categories (\cite{simpson2}), the theory of complete Segal spaces (\cite{completesegal}), and the theory of model categories (\cite{hovey}, \cite{hirschhorn}).
To review all of these notions and their interrelationships would
involve too great a digression from the main purpose of this book.
However, the frequency with which we will encounter sophisticated
categorical constructions necessitates the use of {\em one} of
these more efficient approaches. We will employ the theory of {\it weak Kan complexes}, which goes back to Boardman-Vogt (\cite{quasicat}). These objects have subsequently been studied more extensively by Joyal (\cite{joyalpub} and \cite{joyalnotpub}), who calls them {\it quasicategories}. We will simply call them {\it $\infty$-categories}.\index{gen}{$\infty$-category}\index{gen}{quasicategory}
\begin{shaded}
Although topological categories are easy to define, in order to perform constructions therein we need to have a method of strictifying coherently associative composition laws. This is tricky.

Instead, we'll work with a more flexible theory. There are many, but we'll work with \textit{weak Kan complexes}.
\end{shaded}


To get a feeling for what an $\infty$-category $\calC$ should be, it is useful to consider two extreme cases. If {\em every} morphism in $\calC$ is invertible, then $\calC$ is equivalent to the fundamental $\infty$-groupoid of a topological space $X$. In this case, higher category theory reduces to classical homotopy theory. On the other hand, if $\calC$ has no nontrivial $n$-morphisms for $n > 1$, then $\calC$ is equivalent to an ordinary category. A general formalism must capture the features of both of these examples. In other words, we need
a class of mathematical objects which can behave both like categories and like topological spaces. In \S \ref{highcat}, we achieved this by ``brute force'': namely, we directly amalgamated the theory of topological spaces and the theory of categories, by considering topological categories.
However, it is possible to approach the problem more directly using the theory of
{\em simplicial sets}. 
We will assume that the reader has some familiarity with the theory of simplicial sets; a brief review of this theory is included in \S \ref{simpset}, and a more extensive introduction can be found in \cite{goerssjardine}.
\begin{shaded}
Any $(\infty,0)$-category is an $(\infty,1)$-category, so that $(\infty,1)$-categories can behave like ``spaces''. An $(\infty,1)$-category without nontrivial higher morphisms is just a category, so $(\infty,1)$-categories can behave like categories.

Obviously, we saw both of these behaviors in the setting of topological categories. Instead we'll use simplicial sets.
\end{shaded}

The theory of simplicial sets originated as a combinatorial approach to homotopy theory. Given any topological space $X$, one can associate a simplicial set $\Sing X$, whose $n$-simplices are precisely the continuous maps $| \Delta^n | \rightarrow X$, where $|\Delta^n| = \{ (x_0, \ldots, x_n) \in [0,1]^{n+1} | x_0 + \ldots + x_n =1 \}$ is the standard $n$-simplex. Moreover, the topological space $X$ is {\em determined}, up to weak homotopy equivalence, by $\Sing X$. More precisely, the singular complex functor $X \mapsto \Sing X$
admits a left adjoint, which carries every simplicial set $K$ to its {\it geometric realization} $|K|$. For every topological space $X$, the counit map \index{gen}{geometric realization!of simplicial sets}\index{not}{|K|@$|K|$}\index{not}{SingX@$\Sing X$} $| \Sing X | \rightarrow X$ is a weak homotopy equivalence. Consequently, if one is only interested in studying topological spaces up to weak homotopy equivalence, one might as well work simplicial sets instead.
\begin{shaded}
Sanity: the adjunction \smash{$\xymatrix@R=.3cm@C=1cm{
\sSet  \ar@<.6ex>[r]&
\Top  \ar@<.4ex>[l]\\
%X \ar@{|->}[r] & FX\\
%GY             & Y \ar@{|->}[l]
}$}
is a Quillen equivalence. Thus the counit, $| \Sing X | \rightarrow X$, being adjoint to the weak equivalence $\id_{\Sing X}$, must be a weak equivalence of topological spaces.

For a further sanity check, as a right adjoint preserves limits, it preserves the terminal object, and as a right Quillen functor preserves fibrations, it must preserve fibrant objects. Everything in $\Top$ is fibrant, so that $\Sing X$ is always a Kan complex.
\end{shaded}

If $X$ is a topological space, then the simplicial set $\Sing X$ has an important property, which is captured by the following definition:

\begin{definition}\label{strongkan}
Let $K$ be a simplicial set. We say that $K$ is a {\it Kan complex} if, for any $0 \leq i \leq n$ and any diagram of solid arrows\index{gen}{Kan complex}
$$ \xymatrix{ \Lambda^n_i \ar[r] \ar@{^{(}->}[d] & K \\
\Delta^n \ar@{-->}[ur],& }$$
there exists a dotted arrow as indicated rendering the diagram commutative. Here $\Lambda^n_i \subseteq \Delta^n$ denotes the $i$th horn, obtained from the simplex $\Delta^n$ by deleting the interior and the face opposite the $i$th vertex.
\end{definition}

The singular complex of any topological space $X$ is a Kan complex: this follows from the fact that the horn $| \Lambda^n_i |$ is a retract of the simplex $| \Delta^n |$ in the category of topological spaces. Conversely, any Kan complex $K$ ``behaves like'' a space: for example, there are simple combinatorial recipes for extracting homotopy groups from $K$ (which turn out be isomorphic to the homotopy groups of the topological space $|K|$). According to a theorem of Quillen (see \cite{goerssjardine} for a proof), the singular complex and geometric realization provide mutually inverse equivalences between the homotopy category of CW complexes and the homotopy category of Kan complexes.

The formalism of simplicial sets is also closely related to category theory.\index{gen}{nerve!of a category}\index{not}{NervecalC@$\Nerve(\calC)$} To any category $\calC$, we can associate a simplicial set $\Nerve(\calC)$, called the {\it nerve} of $\calC$. For each $n \geq 0$, we 
let $\Nerve(\calC)_{n} = \bHom_{\sSet}(\Delta^n, \Nerve(\calC))$ denote the set of all functors $[n]
\rightarrow \calC$. Here $[n]$ denotes the linearly ordered set $\{ 0, \ldots, n \}$, regarded as a category in the obvious way. More
concretely, $\Nerve(\calC)_n$ is the set of all composable
sequences of morphisms
$$ C_0 \stackrel{f_1}{\rightarrow} C_1 \ldots \stackrel{f_n}{\rightarrow} C_n$$ having length $n$.
In this description, the face map $d_i$ carries the above sequence
to
$$C_0 \stackrel{f_1}{\rightarrow} C_1 \ldots \stackrel{f_{i-1}}{\rightarrow}
C_{i-1} \stackrel{ f_{i+1} \circ f_i }{\rightarrow} C_{i+1}
\stackrel{f_{i+2}}{\rightarrow} \ldots
\stackrel{f_{n}}{\rightarrow} C_n$$ while the degeneracy $s_i$
carries it to $$C_0 \stackrel{f_1}{\rightarrow} C_1 \ldots
\stackrel{f_i}{\rightarrow} C_i \stackrel{\id_{C_i}}{\rightarrow}
C_i \stackrel{f_{i+1}}{\rightarrow} C_{i+1}
\stackrel{f_{i+2}}{\rightarrow} \ldots \stackrel{f_n}{\rightarrow}
C_n.$$

It is more or less clear from this description that the simplicial
set $\Nerve(\calC)$ is just a fancy way of encoding the structure
of $\calC$ as a category. More precisely, we note that the
category $\calC$ can be recovered (up to isomorphism) from its
nerve $\Nerve(\calC)$. The objects of $\calC$ are simply the {\it
vertices} of $\Nerve(\calC)$; that is, the elements of $\Nerve
(\calC)_0$. A morphism from $C_0$ to $C_1$ is given by an edge $\phi
\in \Nerve(\calC)_1$ with $d_1(\phi) = C_0$ and $d_0(\phi)=C_1$. The
identity morphism from an object $C$ to itself is given by the
degenerate simplex $s_0(C)$. Finally, given a diagram $C_0
\stackrel{\phi}{\rightarrow} C_1 \stackrel{\psi}{\rightarrow}
C_2$, the edge of $\Nerve(\calC)$ corresponding to $\psi \circ
\phi$ may be uniquely characterized by the fact that there exists
a $2$-simplex $\sigma \in \Nerve(\calC)_2$ with $d_2(\sigma)=\phi$,
$d_0(\sigma)=\psi$, and $d_1(\sigma)=\psi \circ \phi$.
\begin{shaded}
Given a category, one can form the nerve $\Nerve(\calC)$. $\calC$ can be recovered (up to isomorphism) from $\Nerve(\calC)$.

To recover the composition, one notes that a pair of composable arrows in $\calC$ determine a map $\Lambda^2_1\rightarrow\Nerve(\calC)$. This horn has a unique filler $\sigma\in \Nerve(\calC)_2$, whose $1^\textup{st}$ face $d_1(\sigma)$ is the composite.

In fact a \textbf{unique} filling condition on \textbf{inner} horns characterises the simplicial sets arising as $\Nerve(\calC)$:
\end{shaded}

It is not difficult to characterize those simplicial sets which arise as the nerve of a category:
\begin{proposition}\label{ruko}
Let $K$ be a simplicial set. Then the following conditions are equivalent:
\begin{itemize}
\item[$(1)$] There exists a small category $\calC$ and an isomorphism $K \simeq \Nerve(\calC)$.
\item[$(2)$] For each $0 < i < n$ and each diagram
$$ \xymatrix{ \Lambda^n_i \ar@{^{(}->}[d] \ar[r] & K \\
\Delta^n \ar@{-->}[ur], & \\}$$
there exists a {\em unique} dotted arrow rendering the diagram commutative.
\end{itemize}
\end{proposition}

\begin{proof}
We first show that $(1) \Rightarrow (2)$. Let $K$ be the nerve of a small category
$\calC$, and let $f_0: \Lambda^n_i \rightarrow K$ be a map of simplicial sets where
$0 < i < n$. We wish to show that $f_0$ can be extended uniquely to a map $f: \Delta^n \rightarrow K$.
For $0 \leq k \leq n$, let $X_i \in \calC$ be the image of the vertex $\{k\} \subseteq \Lambda^n_i$.
For $0 < k \leq n$, let $g_k: X_{k-1} \rightarrow X_{k}$ be the morphism in $\calC$ determined by the restriction $f_0 | \Delta^{ \{k-1,k\} }$. The composable chain of morphisms
$$ X_0 \stackrel{ g_1}{\rightarrow} X_1 \stackrel{g_2}{\rightarrow} \ldots \stackrel{g_n}{\rightarrow} X_n$$
determines an $n$-simplex $f: \Delta^n \rightarrow K$. We will show that $f$ is the desired solution to our extension problem (the uniqueness of this solution is evident: if $f': \Delta^n \rightarrow K$ is
any other map with $f' | \Lambda^n_i = f_0$, then $f'$ must correspond to the same chain of morphisms
in $\calC$, so that $f' = f$). It will suffice to prove the following, for every $0 \leq j \leq n$:
\begin{itemize}
\item[$(\ast_j)$] If $j \neq i$, then 
$$ f | \Delta^{ \{0, \ldots, j-1, j+1, \ldots, n \} } = f_0 | \Delta^{ \{0, \ldots, j-1, j+1, \ldots, n \} }.$$
\end{itemize}
To prove $(\ast_j)$, it will suffice to show that $f$ and $f_0$ have the same
restriction to $\Delta^{ \{k,k' \} }$, where $k$ and $k'$ adjacent
elements of the linearly ordered set $\{ 0, \ldots, j-1, j+1, \ldots, n \} \subseteq [n]$.
If $k$ and $k'$ are adjacent in $[n]$, then this follows by construction. In particular,
$(\ast)$ is automatically satisfied if $j=0$ or $j=n$.
Suppose instead that $k = j-1$ and $k' = j+1$, where $0 < j < n$. If $n = 2$, then $j=1=i$ and we obtain a contradiction. We may therefore assume that $n > 2$, so that either $j-1 > 0$ or
$j+1 < n$. Without loss of generality, $j-1 > 0$, so that $\Delta^{ \{ j-1, j+1 \} }
\subseteq \Delta^{ \{1, \ldots, n \} }$. The desired conclusion now follows from $(\ast_0)$.

We now prove the converse. Suppose that the simplicial set $K$ satisfies $(2)$; we claim that
$K$ is isomorphic to the nerve of a small category $\calC$. We construct the category
$\calC$ as follows:
\begin{itemize}
\item[$(i)$] The objects of $\calC$ are the vertices of $K$.

\item[$(ii)$] Given a pair of objects $x, y \in \calC$, we let
$\Hom_{\calC}(x,y)$ denote the collection of all edges
$e: \Delta^1 \rightarrow K$ such that $e|\{0\} =x$ and $e| \{1\} = y$.

\item[$(iii)$] Let $x$ be an object of $\calC$, Then the identity morphism
$\id_{x}$ is the edge of $K$ defined by the composition
$$ \Delta^1 \rightarrow \Delta^0 \stackrel{e}{\rightarrow} K.$$

\item[$(iv)$] Let $f: x \rightarrow y$ and $g: y \rightarrow z$ be morphisms
in $\calC$. Then $f$ and $g$ together determine a map
$\sigma_0: \Lambda^2_1 \rightarrow K$. In view of condition $(2)$, the map
$\sigma_0$ can be extended uniquely to a $2$-simplex $\sigma: \Delta^2 \rightarrow K$.
We define the composition $g \circ f$ to be the morphism from $x$ to $z$ in $\calC$ corresponding
to the edge given by the composition
$$ \Delta^1 \simeq \Delta^{ \{0,2\} } \subseteq \Delta^2 \stackrel{\sigma}{\rightarrow} K.$$
\end{itemize}

We first claim that $\calC$ is a category. To prove this, we must verify the following axioms:
\begin{itemize}
\item[$(a)$] For every object $y \in \calC$, the identity $\id_{y}$ is a unit with respect to composition.
In other words, for every morphism $f: x \rightarrow y$ in $\calC$ and every morphism
$g: y \rightarrow z$ in $\calC$, we have $\id_y \circ f = f$ and $g \circ \id_y = g$.
These equations are ``witnessed'' by the $2$-simplices $s_1(f), s_0(g) \in \Hom_{\sSet}( \Delta^2, K)$.

\item[$(b)$] Composition is associative. That is, for every sequence of composable morphisms
$$ w \stackrel{f}{\rightarrow} x \stackrel{g}{\rightarrow} y \stackrel{h}{\rightarrow} z,$$
we have $h \circ (g \circ f) = (h \circ g) \circ f$. To prove this, let us first choose
$2$-simplices $\sigma_{0 1 2}$ and $\sigma_{1 2 3}$ as indicated below:
$$ \xymatrix{ & x \ar[dr]^{g} & & & y \ar[dr]^{h} & \\
w \ar[ur]^{f} \ar[rr]^{g \circ f} & & y & x \ar[ur]^{g} \ar[rr]^{h \circ g} & & z. }$$
Now choose a $2$-simplex $\sigma_{0 2 3}$ corresponding to a diagram
$$ \xymatrix{ & y \ar[dr]^{h} & \\
w \ar[ur]^{ g \circ f} \ar[rr]^{ h \circ (g \circ f)} &  & z. }$$
These three $2$-simplices together define a map $\tau_0: \Lambda^3_2 \rightarrow K$.
Since $K$ satisfies condition $(2)$, we can extend $\tau_0$ to a $3$-simplex
$\tau: \Delta^3 \rightarrow K$. The composition
$$ \Delta^2 \simeq \Delta^{ \{0,1,3\}} \subseteq \Delta^3 \stackrel{\tau}{\rightarrow} K$$
corresponds to the diagram
$$ \xymatrix{ & x \ar[dr]^{h \circ g} & \\
w \ar[ur]^{f} \ar[rr]^{ h \circ (g \circ f) } & & z, }$$
which ``witnesses'' the associativity axiom $h \circ (g \circ f) = (h \circ g) \circ f$.
\end{itemize}

It follows that $\calC$ is a well-defined category. By construction, we have a canonical
map of simplicial sets $\phi: K \rightarrow \Nerve \calC$. To complete the proof, it will suffice
to show that $\phi$ is an isomorphism. We will prove, by induction on
$n \geq 0$, that $\phi$ induces a bijection $\Hom_{ \sSet}( \Delta^n, K) \rightarrow
\Hom_{\sSet}( \Delta^n, \Nerve \calC)$. For $n=0$ and $n=1$, this is obvious from
the construction. Assume therefore that $n \geq 2$, and choose an integer $i$ such
that $0 < i < n$. We have a commutative diagram
$$ \xymatrix{ \Hom_{\sSet}( \Delta^n, K ) \ar[r] \ar[d] & \Hom_{\sSet}( \Delta^n, \Nerve \calC) \ar[d] \\
\Hom_{ \sSet}( \Lambda^n_i, K) \ar[r] & \Hom_{\sSet}( \Lambda^n_i, \Nerve \calC ). }$$
Since $K$ and $\Nerve \calC$ both satisfy $(2)$ (for $\Nerve \calC$, this follows
from the first part of the proof), the vertical maps are bijective. It will therefore suffice to show that the lower horizontal map is bijective, which follows from the inductive hypothesis.
\end{proof}

We note that condition $(2)$ of Proposition \ref{ruko} is very similar to Definition \ref{strongkan}. However, it is different in two important respects. First, it requires the extension condition only for {\em inner} horns $\Lambda^n_i$ with $0 < i < n$. Second, the asserted condition is stronger in this case: not only does any map $\Lambda^n_i \rightarrow K$ extend to the simplex $\Delta^n$, but the extension is unique.\index{gen}{horn!inner}\index{gen}{inner horn}

\begin{remark}\label{nottes}
It is easy to see that it is not reasonable to expect condition $(2)$ of Proposition \ref{ruko} to hold for ``outer'' horns $\Lambda^n_i$, $i \in \{0,n\}$. Consider, for example, the case where $i=n=2$, and where $K$ is the nerve of a category $\calC$. Giving a map $\Lambda^2_2 \rightarrow K$
corresponds to supplying the solid arrows in the diagram
$$ \xymatrix{ & C_1 \ar[dr] & \\
C_0 \ar[rr] \ar@{-->}[ur] & & C_2,} $$
and the extension condition would amount to the assertion that one could always find a dotted arrow rendering the diagram commutative. This is true in general only when the category $\calC$ is a {\em groupoid}.
\end{remark}

We now see that the notion of a simplicial set is a flexible one: a simplicial set $K$ can be a good model for an $\infty$-groupoid (if $K$ is a Kan complex), or for an ordinary category (if it satisfies the hypotheses of Proposition \ref{ruko}). Based on these observations, we might expect that some more general class of simplicial sets could serve as models for $\infty$-categories in general.

Consider first an arbitrary simplicial set $K$. We can try to envision $K$ as a generalized category, whose objects are the vertices of $K$ (that is, the elements of $K_0$), and whose morphisms are the edges of $K$ (that is, the elements of $K_1$). A $2$-simplex
$\sigma: \Delta^2 \rightarrow K$ should be thought of as a diagram
$$ \xymatrix{ & Y \ar[dr]^{\psi} & \\
X \ar[ur]^{\phi} \ar[rr]^{\theta} & & Z }$$
together with an identification (or homotopy) between $\theta$ and $\psi \circ \phi$
which renders the diagram ``commutative''. (Note that, in higher category theory, 
this is not merely a condition: the homotopy $\theta \simeq \psi \circ \phi$ is an additional datum.)
Simplices of larger dimension may be thought of as verifying the
commutativity of certain higher-dimensional diagrams.

Unfortunately, for a general simplicial set $K$, the analogy
outlined above is not very strong. The essence of the problem is that, though we may refer
to the $1$-simplices of $K$ as ``morphisms'', there is in general no way to compose them.
Taking our cue from the example of $\Nerve(\calC)$,
we might say that a morphism $\theta:X \rightarrow Z$ is a composition of morphisms
$\phi: X \rightarrow Y$ and $\psi: Y \rightarrow Z$ if there exists a $2$-simplex
$\sigma: \Delta^2 \rightarrow K$ as in the diagram indicated above. 
We must now consider two potential difficulties: the $2$-simplex $\sigma$ may not exist, and
if it does it exist it may not be unique, so that we have more than one choice for the composition $\theta$.
\begin{shaded}
We view a simplicial set $K$ as a generalised category. The 1-simplices are the arrows. A filler $\sigma$ for a $(2,1)$-horn in $K$ should be viewed as evidence that $d_1(\sigma)$ is \textbf{a} composite for $d_0(\sigma)\circ d_2(\sigma)$ --- all composites are on equal footing.

There might be many fillers $\sigma$ with the same faces $d_i(\sigma)$ (for $0\leq i\leq2$), just as there may be many homotopies between two paths. There will certainly not be a canonical choice of filler, nor should there be.

We do, however, want all pairs of composable 1-simplices to have at least one composite, so we demand that $K$ enjoys a \textbf{(non-unique)} inner horn filling condition, and call such a simplicial set an $\infty$-category (a.k.a.\ a \textit{weak Kan complex}, a.k.a.\ a \textit{quasi-category}).

This philosophy works for higher simplices (and thus higher horns), just as it does for 1-simplices.
\end{shaded}

The existence of $\sigma$ can be formulated as an extension condition on the simplicial set $K$.
We note that a composable pair of morphisms $(\psi, \phi)$ determines a map
of simplicial sets
$\Lambda^2_1 \rightarrow K$. Thus, the assertion that $\sigma$ can
always be found may be formulated as a extension property: any
map of simplicial sets $\Lambda^2_1 \rightarrow K$ can be extended to $\Delta^2$, as indicated in the following diagram:
$$ \xymatrix{ \Lambda^2_1\ar[r]\ar@{^{(}->}[d] & K \\
\Delta^2 \ar@{-->}[ur]}$$

The uniqueness of $\theta$ is another matter. It turns out to be
unnecessary (and unnatural) to require that $\theta$ be uniquely
determined. To understand this point, let us return to 
the example of the fundamental groupoid of a topological space
$X$. This is a category whose objects are the points $x \in X$.
The morphisms between a point $x \in X$ and a point $y \in X$ are given by
continuous paths $p: [0,1] \rightarrow X$ such that $p(0)=x$ and
$p(1)=y$. Two such paths are considered to be equivalent if there
is a homotopy between them. Composition in the fundamental
groupoid is given by concatenation of paths. Given paths $p,q:
[0,1] \rightarrow X$ with $p(0)=x$, $p(1)=q(0)=y$, and $q(1)=z$,
the composite of $p$ and $q$ should be a path joining $x$ to $z$.
There are many ways of obtaining such a path from $p$ and $q$. One
of the simplest is to define
$$r(t) = \begin{cases} p(2t) & \text{if } 0 \leq t \leq \frac{1}{2} \\
q(2t-1) & \text{if } \frac{1}{2} \leq t \leq 1. \end{cases}$$
However, we could just as well use the formula
$$r'(t) = \begin{cases} p(3t) & \text{if } 0 \leq t \leq \frac{1}{3} \\
q(\frac{3t-1}{2}) & \text{if } \frac{1}{3} \leq t \leq 1
\end{cases}$$
to define the composite path. Because the paths $r$ and $r'$ are homotopic to one another, it does not matter which one we choose.

The situation becomes more complicated if we try to think
$2$-categorically. We can capture more information about the space
$X$ by considering its {\it fundamental $2$-groupoid}. This is a
$2$-category whose objects are the points of $X$, whose morphisms
are paths between points, and whose $2$-morphisms are given by
homotopies between paths (which are themselves considered modulo
homotopy). In order to have composition of morphisms unambiguously defined, we would have to choose some formula once and for all.
Moreover, there is no particularly compelling choice; for example,
neither of the formulas written above leads to a strictly associative composition law.

The lesson to learn from this is that in higher-categorical
situations, we should not necessarily ask for a uniquely
determined composition of two morphisms. In the fundamental groupoid
example, there are many choices for a composite path but all of
them are homotopic to one another. Moreover, in keeping with the
philosophy of higher category theory, {\em any} path which is
homotopic to the composite should be just as good as the
composite itself. From this point of view, it is perhaps more natural to
view composition as a relation than as a function, and this is
very efficiently encoded in the formalism of simplicial sets: a
$2$-simplex $\sigma: \Delta^2 \rightarrow K$ should be viewed as
``evidence'' that $d_0(\sigma) \circ d_2(\sigma)$ is homotopic to $d_1(\sigma)$.

Exactly what conditions on a simplicial set $K$ will guarantee that it behaves like a higher category? Based on the above argument, it seems reasonable to require that $K$ satisfy an extension condition with respect to certain horn inclusions $\Lambda^n_i$, as in Definition \ref{strongkan}. However, as we observed in Remark \ref{nottes}, this is reasonable only for the inner horns where $0 < i < n$, which appear in the statement of Proposition \ref{ruko}. 

\begin{definition}\label{qqcc}\index{gen}{$\infty$-category}
An {\it $\infty$-category} is a simplicial set $K$ which has the
following property: for any $0 < i < n$, any map $f_0: \Lambda^n_i
\rightarrow K$ admits an extension $f: \Delta^n \rightarrow K$.
\end{definition}

Definition \ref{qqcc} was first formulated by Boardman and Vogt (\cite{quasicat}). They referred to $\infty$-catgories as {\it weak
Kan complexes},\index{gen}{Kan complex!weak} motivated by the obvious analogy with Definition \ref{strongkan}. Our terminology places more emphasis on the analogy with the characterization of ordinary categories given in Proposition \ref{ruko}: we require the same extension conditions, but drop the uniqueness assumption.

\begin{example}
Any Kan complex is an $\infty$-category. In particular, if $X$ is a topological space, then we may view its singular complex $\Sing X$ as an $\infty$-category: this is one way of defining the fundamental $\infty$-groupoid $\pi_{\leq \infty} X$ of $X$, introduced informally in Example \ref{grape}.
\end{example}

\begin{example}
The nerve of any category is an $\infty$-category. We will occasionally abuse terminology by
identifying a category $\calC$ with its nerve $\Nerve(\calC)$; by means of this identification, we may view ordinary category theory as a special case of the study of $\infty$-categories.
\end{example}

The weak Kan condition of Definition \ref{qqcc} leads to a very elegant and powerful version of higher category theory. This theory has been developed by Joyal in the papers \cite{joyalpub} and
\cite{joyalnotpub} (where simplicial sets satisfying the condition of Definition \ref{qqcc} are called
{\it quasi-categories}), and will be used throughout this book.

\begin{notation}
Depending on the context, we will use two different notations in
connection with simplicial sets. When emphasizing their
role as $\infty$-categories, we will often denote
them by
calligraphic letters such as $\calC$, $\calD$, and so forth. When
casting simplicial sets in their different (though related) role
of representing homotopy types, we will employ capital Roman
letters. To avoid confusion, we will also employ the latter notation
when we wish to contrast the theory of $\infty$-categories with some
other other approach to higher category theory, such as the theory
of topological categories.
\end{notation}

\subsection{Equivalences of Topological Categories}\label{stronghcat}

We have now introduced two approaches to higher category theory: one based on topological categories, and one based on simplicial sets. These two approaches turn out to be equivalent to one another. However, the equivalence itself needs to be understood in a higher-categorical sense. We take our cue from classical homotopy theory, in which we can take the basic objects to be either topological spaces or simplicial sets. It is not true that every Kan complex is isomorphic to the singular complex of a topological space, or that every CW complex is homeomorphic to the geometric realization of a simplicial set. However, both of these statements become true if we replace the words ``isomorphic to'' by ``homotopy equivalent to''. We would like to formulate a similar statement regarding our approaches to higher category theory. The first step is to find a concept which replaces ``homotopy equivalence''. If $F: \calC \rightarrow \calD$ is a functor between topological categories, under what circumstances should we regard $F$ as an ``equivalence'' (so that $\calC$ and $\calD$ really represent the same higher category)? 

The most naive answer is that $F$ should be regarded as an equivalence if it is an isomorphism of topological categories. This means that $F$ induces a bijection between the objects of $\calC$ and the objects of $\calD$, and a homeomorphism $\bHom_{\calC}(X,Y) \rightarrow \bHom_{\calD}(F(X),F(Y))$ for every pair of objects $X,Y \in \calC$. However, it is immediately obvious that this condition is far too strong; for example, in the case where $\calC$ and $\calD$ are ordinary categories (which we may view also as topological categories, where all morphism sets are endowed with the discrete topology), we recover the notion of an isomorphism between categories. This notion does not play an important role in category theory. One rarely asks whether or not two categories are isomorphic; instead, one asks whether or not they are equivalent. This suggests the following definition:

\begin{definition}\index{gen}{strong equivalence}
A functor $F: \calC \rightarrow \calD$ between topological categories is a  {\it strong equivalence} if it is an equivalence in the sense of enriched category theory. In other words, $F$ is a strong equivalence if it induces homeomorphisms $\bHom_{\calC}(X,Y) \rightarrow \bHom_{\calD}(F(X),F(Y))$ for every pair of objects $X,Y \in \calC$, and every object of $\calD$ is isomorphic (in $\calD$) to $F(X)$ for some $X \in \calC$.
\end{definition}
\begin{shaded}
Call a functor of topological categories a \textbf{strong equivalence} if it induces homeomorphisms on hom-spaces (like ``fully faithful'') and every object in the target is isomorphic to one in the image (essentially surjective).

In general, this is the definition of equivalence of enriched categories.
However, it is too strict in this context, where homeomorphism should be replaced by homotopy equivalence.
\end{shaded}

The notion of strong equivalence between topological categories has the virtue that, when restricted to ordinary categories, it reduces to the usual notion of equivalence. However, it is still not the right definition: for a pair of objects $X$ and $Y$ of a higher category $\calC$, the morphism space $\bHom_{\calC}(X,Y)$ should itself only be well-defined up to homotopy equivalence. 

\begin{definition}\label{vergen}\index{gen}{homotopy category!of a topological category}
Let $\calC$ be a topological category. The {\it homotopy category} $\h{\calC}$
is defined as follows:\index{not}{hcalC@$\h{\calC}$}
\begin{itemize}
\item The objects of $\h{\calC}$ are the objects of $\calC$.
\item If $X,Y \in \calC$, then we define $\Hom_{\h{\calC}}(X,Y)= \pi_0 \bHom_{\calC}(X,Y)$.
\item Composition of morphisms in $\h{\calC}$ is induced from the composition of morphisms
in $\calC$ by applying the functor $\pi_0$.
\end{itemize}
\end{definition}

\begin{example}\index{gen}{homotopy category!of spaces}
Let $\calC$ be the topological category whose objects are CW-complexes, where
$\bHom_{\calC}(X,Y)$ is the set of continuous maps from $X$ to $Y$, equipped with the
(compactly generated version of the) compact-open topology. We will denote the homotopy category of $\calC$ by $\calH$, and refer to $\calH$ as the {\it homotopy category of spaces}.\index{not}{Hcal@$\calH$}
\end{example}
\begin{shaded}
From a topological category $\calC$ we can form a category $h\calC$, the homotopy category, by taking $\Hom_{h\calC}(X,Y)$ to be $\pi_0\bHom_{\calC}(X,Y)$. The homotopy category of spaces $\calH$ can be defined as the homotopy category of the category of CW-complexes.
\end{shaded}

There is a second construction of the homotopy category $\calH$, which will play an important role in what follows. First, we must recall a bit of terminology from classical homotopy theory.

\begin{definition}\index{gen}{weak homotopy equivalence!of topological spaces}
A map $f: X \rightarrow Y$ between topological spaces is said to be a {\it weak homotopy equivalence}
if it induces a bijection $\pi_0 X \rightarrow \pi_0 Y$, and if 
for every point $x \in X$ and every $i \geq 1$, the induced map of homotopy groups
$$ \pi_{i}(X,x) \rightarrow \pi_i(Y,f(x))$$ is an isomorphism.
\end{definition}

Given a space $X \in \CG$, classical homotopy theory ensures the existence of a CW-complex
$X'$ equipped with a weak homotopy equivalence $\phi: X' \rightarrow X$. Of course,
$X'$ is not uniquely determined; however, it is unique up to canonical homotopy equivalence,
so that the assignment
$$ X \mapsto [X] = X'$$
determines a functor $\theta: \CG \rightarrow \calH$. By construction, $\theta$ carries
weak homotopy equivalences in $\CG$ to isomorphisms in $\calH$. In fact, $\theta$
is universal with respect to this property. In other words, we may describe $\calH$
as the category obtained from $\CG$ by formally inverting all weak homotopy equivalences.
This is one version of Whitehead's theorem, which is usually stated as follows: every weak homotopy equivalence between CW complexes admits a homotopy inverse.\index{gen}{Whitehead's theorem}
\begin{shaded}
Taking CW-replacements gives a functor $\theta: \CG \rightarrow \calH$, from (weak Hausdorff compactly generated) spaces to the homotopy category of spaces.

This functor inverts weak equivalences (by Whitehead's theorem). In fact, $\theta$ the localisation at the class of weak equivalences!
\end{shaded}


We can now improve upon Definition \ref{vergen} slightly. We first observe that
the functor $\theta: \CG \rightarrow \calH$ preserves products. Consequently, we can apply the construction of Remark \ref{laxcon} to convert any topological category $\calC$ into a category enriched over $\calH$. We will denote this $\calH$-enriched category by $\h{\calC}$, and refer to it as the {\it homotopy category} of $\calC$. More concretely, the homotopy category
$\h{\calC}$ may be described as follows:\index{gen}{homotopy category!enriched over $\calH$}
\begin{itemize}
\item[$(1)$] The objects of $\h{\calC}$ are the objects of $\calC$.
\item[$(2)$] For $X,Y \in \calC$, we have
$$ \bHom_{ \h{\calC} }(X,Y) = [ \bHom_{\calC}(X,Y) ].$$
\item[$(3)$] The composition law on $\h{\calC}$ is obtained from the composition law on
$\calC$ by applying the functor $\theta: \CG \rightarrow \calH$.
\end{itemize}

\begin{remark}
If $\calC$ is a topological category, we have now defined $\h{\calC}$ in two different ways: first as an ordinary category, and then as a category enriched over $\calH$. These two definitions are compatible with one another, in the sense that $\h{\calC}$ (as an ordinary category) is the
underlying category of $\h{\calC}$ (as an $\calH$-enriched category). This follows immediately
from the observation that for every topological space $X$, there is a canonical bijection
$\pi_0 X \simeq \bHom_{\calH}( \ast, [X] ).$
\end{remark}
\begin{shaded}
Given a topological category $\calC$, one obtains a category enriched over $\calH$ by applying $\theta$ to the mapping objects (never fear, $\theta$ preserves products). We call this the homotopy category of $\calC$, and its underlying ordinary category coincides with the definition via taking $\pi_0$.

We then call a functor $F: \calC \rightarrow \calD$ between topological categories a \textbf{weak equivalence} or \textbf{equivalence} it the induced functor $\h{ \calC} \rightarrow \h{ \calD}$ is an equivalence of $\calH$-enriched categories (weak equivalences on mapping spaces, essentially surjective).

This is the correct notion of equivalence of topological categories. We have analogous statements:
\begin{enumerate}\squishlist
\item A map of CW-complexes is a homotopy equivalence \Iff it induces isomorphisms on homotopy groups.
\item A functor of topological categories is an equivalence \Iff it induces an equivalence of homotopy categories.
\end{enumerate}


\end{shaded}


If $\calC$ is a topological category, we may imagine that $\h{\calC}$ is the object which is obtained by forgetting the topological morphism spaces of $\calC$ and remembering only their (weak) homotopy types. The following definition codifies the idea that these homotopy types should be ``all that really matter''.

\begin{definition}\label{defequiv}\index{gen}{equivalence!of topological categories}
Let $F: \calC \rightarrow \calD$ be a functor between topological categories. We will say that $F$ is a {\it weak equivalence}, or simply an {\it equivalence}, if the induced functor
$\h{ \calC} \rightarrow \h{ \calD}$ is an equivalence of $\calH$-enriched categories.
\end{definition}

More concretely, a functor $F$ is an equivalence if and only if:

\begin{itemize}
\item For every pair of objects $X,Y \in \calC$, the induced map
$$ \bHom_{\calC}(X,Y) \rightarrow \bHom_{\calD}(F(X),F(Y))$$ is a weak homotopy equivalence of topological spaces.

\item Every object of $\calD$ is isomorphic in $\h{ \calD}$ to $F(X)$, for some $X \in \calC$.
\end{itemize}

\begin{remark}\index{gen}{equivalence!in a topological category}
A morphism $f: X \rightarrow Y$ in $\calD$ is said to be an {\it equivalence} if the induced morphism in $\h{ \calD}$ is an isomorphism. In general, this is much weaker than the condition that $f$ be an isomorphism in $\calD$; see Proposition \ref{rooot}.
\end{remark}

It is Definition \ref{defequiv} which gives the correct notion of equivalence between topological categories (at least, when one is using them to describe higher category theory). We will agree that all relevant properties of topological categories are invariant under this notion of equivalence. We say that two topological categories are {\it equivalent} if there is an equivalence between them, or more generally if there is a chain of equivalences joining them. Equivalent topological categories should be regarded as ``the same'' for all relevant purposes.

\begin{remark}
According to Definition \ref{defequiv}, a functor $F: \calC \rightarrow \calD$ is an equivalence if and only if the induced functor $\h{\calC} \rightarrow \h{ \calD}$ is an equivalence. In other words, the homotopy category $\h{\calC}$ (regarded as a category which is enriched over $\calH$) is an invariant of $\calC$
which is sufficiently powerful to detect equivalences between $\infty$-categories.
This should be regarded as analogous to the more classical fact that the homotopy groups $\pi_i(X,x)$ of a CW complex $X$ are homotopy invariants which detect homotopy equivalences between CW complexes (by Whitehead's theorem). However, it is important to remember that $\h{ \calC}$ does not determine $\calC$ up to equivalence, just as the homotopy type of a CW complex is not determined by its homotopy groups.
\end{remark}

\subsection{Simplicial Categories}\label{compp1}

In the previous sections we introduced two very different approaches to the foundations of higher category theory: one based on topological categories, the other on simplicial sets. In order to prove that they are equivalent to one another, we will introduce a third approach,  which is closely related to the first but shares the combinatorial flavor of the second.

\begin{definition}\index{gen}{simplicial category}\index{gen}{category!simplicial}
A {\it simplicial category} is a category which is enriched over
the category $\sSet$ of simplicial sets. The category of simplicial
categories (where morphisms are given by simplicially enriched functors) will be
denoted by $\sCat$.\index{not}{Catsi@$\sCat$}
\end{definition}

\begin{remark}
Every simplicial category can be regarded as a simplicial object in the category $\Cat$. Conversely, a simplicial object of $\Cat$ arises from a simplicial category if and only if the underlying simplicial set of objects is constant.
\end{remark}

Like topological categories, simplicial categories can be used as models of higher category theory. If $\calC$ is a simplicial category, then we will generally think of the simplicial sets $\bHom_{\calC}(X,Y)$ as ``spaces'', or homotopy types. 

\begin{remark}
If $\calC$ is a simplicial category with the property that each of the simplicial sets
$\bHom_{\calC}(X,Y)$ is an $\infty$-category, then we may view $\calC$ itself as a kind of $\infty$-bicategory. We will not use this interpretation of simplicial categories in this book. Usually we will consider only {\em fibrant} simplicial categories: that is, simplicial categories for which the mapping objects $\bHom_{\calC}(X,Y)$ are Kan complexes.\index{gen}{fibrant!simplicial category}
\end{remark}
\begin{shaded}
A simplicial category is a category enriched over $\sSet$. Its mapping simplicial sets are often thought of as ``spaces'', or homotopy types. We'll usually only consider fibrant simplicial categories --- those whose mapping objects are Kan complexes.
\end{shaded}

The relationship between simplicial categories and topological categories is easy to describe. Let $\sSet$ denote the
category of simplicial sets and $\CG$ the category of compactly
generated Hausdorff spaces. We recall that there exists a pair of
adjoint functors
$$ \Adjoint{||}{\sSet}{\CG}{\Sing}$$
which are called the {\it geometric realization} and {\it
singular complex} functors, respectively. Both of these functors commute
with finite products. Consequently, if $\calC$ is a simplicial
category, we may define a topological category $|\calC|$ in the
following way:\index{gen}{geometric realization!of simplicial categories}\index{not}{|calC|@$|\calC|$}
\index{not}{SingcalC@$\Sing \calC$}

\begin{itemize}
\item The objects of $|\calC|$ are the objects of $\calC$.

\item If $X,Y \in \calC$, then $\bHom_{|\calC|}(X,Y) = |
\bHom_{\calC}(X,Y)|$.

\item The composition law for morphisms in $|\calC|$ is obtained
from the composition law on $\calC$ by applying the geometric
realization functor.
\end{itemize}

Similarly, if $\calC$ is a topological category, we may obtain a
simplicial category $\Sing \calC$ by applying the singular complex
functor to each of the morphism spaces individually. The singular complex and geometric realization functors determine an adjunction between $\sCat$ and $\tCat$.
This adjunction should be understood as determining an ``equivalence'' between the
theory of simplicial categories and the theory of topological categories. This is essentially a formal consequence of the fact that the geometric realization and singular complex functors determine an equivalence between the homotopy theory of topological spaces and the homotopy theory of simplicial sets. More precisely, we recall that a map $f: S \rightarrow T$ of simplicial sets is said to be a {\it weak homotopy equivalence} if the induced map $|S| \rightarrow |T|$ of topological spaces is a weak homotopy equivalence. A theorem of Quillen (see \cite{goerssjardine} for a proof) asserts that the unit and counit morphisms
$$ S \rightarrow \Sing |S| $$
$$ | \Sing X | \rightarrow X$$
are weak homotopy equivalences, for every (compactly generated) topological space $X$ and every simplicial set $S$. It follows that the category obtained from $\CG$ by inverting weak homotopy equivalences (of spaces) is equivalent to the category obtained from $\sSet$ by inverting weak homotopy equivalences. We use the symbol $\calH$ to denote either of these (equivalent) categories.\index{not}{calH@$\calH$}
\begin{shaded}
Using the Quillen equivalence  \smash{$\Adjoint{}{\sSet}{\CG}{}$}, we obtain
an adjunction between $\sCat$ and $\tCat$.
This adjunction should be understood as determining an ``equivalence'' between the
theory of simplicial categories and the theory of topological categories.
\end{shaded}

If $\calC$ is a simplicial category, we let $\h{\calC}$ denote the $\calH$-enriched category obtained by  applying the functor $\sSet \rightarrow \calH$ to each of the morphism spaces of $\calC$. We will refer to $\h{ \calC}$ as the {\it homotopy category of $\calC$}. We note that this is the same notation that was introduced in \S \ref{stronghcat} for the homotopy category of a topological category. However, there is little risk of confusion: the above remarks imply the existence of canonical isomorphisms\index{gen}{homotopy category!of a simplicial category}\index{not}{hcalC@$\h{\calC}$}
$$ \h{\calC} \simeq \h{ |\calC|}$$
$$ \h{\calD} \simeq \h{ \Sing \calD}$$
for every simplicial category $\calC$ and every topological category $\calD$.

\begin{definition}\index{gen}{equivalence!of simplicial categories}
A functor $\calC \rightarrow \calC'$ between simplicial categories is an {\it equivalence} if
the induced functor $\h{ \calC} \rightarrow \h{ \calC'}$ is an equivalence of $\calH$-enriched categories.
\end{definition}

In other words, a functor $\calC \rightarrow \calC'$ between simplicial categories is an equivalence if and only if the geometric realization $|\calC| \rightarrow |\calC'|$ is an equivalence of topological categories. In fact, one can say more. It follows easily from the preceding remarks that the unit and counit maps
$$ \calC \rightarrow \Sing | \calC |$$
$$ | \Sing \calD | \rightarrow \calD$$
induce {\em isomorphisms} between homotopy categories. Consequently, if we are working with topological or simplicial categories {\it up to equivalence}, we are always free to replace a simplicial category $\calC$ by $|\calC|$, or a topological category $\calD$ by $\Sing \calD$. In this sense, the notions of topological and simplicial category are equivalent and either can be used as a foundation for higher category theory.
\begin{shaded}
A functor of simplicial categories is an equivalence if the induced functor on homotopy categories is an equivalence of $\calH$-enriched categories, as we defined earlier for topological categories.

From the Quillen equivalence \smash{$\Adjoint{}{\sSet}{\CG}{}$} we obtain unit and counit maps $ \calC \rightarrow \Sing | \calC |$ and $ | \Sing \calD | \rightarrow \calD$ which induce isomorphisms on homotopy categories. Thus if we work \textit{up to equivalence}, it is equivalent to work with either simplicial or topological categories. All that remains is the following:
\end{shaded}

\subsection{Comparing $\infty$-Categories with Simplicial Categories}\label{theequiv}

In \S \ref{compp1}, we introduced the theory of simplicial categories and explained why (for our purposes) it is equivalent to the theory of topological categories. In this section, we will show that the theory of simplicial categories is also closely related to the theory of $\infty$-categories.
Our discussion requires somewhat more elaborate constructions than were needed in the previous sections; a reader who does not wish to become bogged down in details is urged to skip ahead to \S \ref{working}. \ldots
\begin{shaded}
I haven't read this section fully. However, the following things happened:
\begin{itemize}\squishlist
\item 
\begin{itemize}\squishlist
\item A construction is given of the ``simplicial nerve'' $\Nerve(\calC)$ of a simplicial category $\calC$.
\item By applying $N\circ\Sing$ to a topological category, one obtains the ``topological nerve''.
\item Just like the usual nerve, the simplicial nerve is defined by an adjunction:
\[\Hom_{ \sSet}( \Delta^n, \Nerve(\calC)) =
\Hom_{\sCat}( \sCoNerve[ \Delta^n ], \calC).\]
\item The point here is that $\sCoNerve[ \Delta^n ]$ is a thickened version of the category $[n]$, which is more simplicial.
\item The adjunction may be written as
$\xymatrix@R=.3cm@C=1cm{
\sSet  \ar@<.6ex>[r]^{\sCoNerve[\bigdot]}&
\sCat  \ar@<.4ex>[l]^{\Nerve}
}$.
\end{itemize}
\item
\begin{itemize}\squishlist
\item Let $\calC$ be a simplicial category such that $\bHom_{\calC}(X,Y)$ is a Kan complex for all $X,Y\in\calC$. Then the simplicial nerve $\sNerve(\calC)$ is an $\infty$-category.
\item In particular, the topological nerve of any topological category is an $\infty$-category.
\end{itemize}
\item
\begin{itemize}\squishlist
\item Given a simplicial set $S$, we define the homotopy category $\h S$ of $S$ to be $\h\sCoNerve[S]$ --- the homotopy category of the simplicial category $\sCoNerve[S]$.
\item A map $S\to T$ of simplicial sets is a \textit{categorical equivalence} if it induces an equivalence of $\calH$-enriched categories $\h S\to\h T$.
\item We now observe that the adjoint functors $(|\sCoNerve[ \bigdot ]|, \tNerve )$
determine an equivalence between the theory of simplicial sets (up to categorical equivalence)
and that of topological categories (up to equivalence). 
\begin{itemise}
\item For any
topological category $\calC$ the counit map
$|\sCoNerve[ \tNerve(\calC)] | \rightarrow \calC$ is an equivalence of topological categories;
\item For any simplicial set $S$ the unit map
$S \rightarrow \tNerve |\sCoNerve[S]|$
is a categorical equivalence of simplicial sets.
\end{itemise}
\item Although not every simplicial set is an $\infty$-category, it is categorically equivalent to one, even the nerve of the topological category $|\sCoNerve[S]|$.
\end{itemize}
\end{itemize}


\end{shaded}

%\begin{Didn't Read}

We will relate simplicial categories with simplicial sets by means of the {\it simplicial nerve functor}
$$ \sNerve: \sCat \rightarrow \sSet,$$
originally introduced by Cordier (see \cite{coherentnerve}).
The nerve of an ordinary category $\calC$ is characterized by the formula
$$ \Hom_{ \sSet}( \Delta^n, \Nerve(\calC)) = \Hom_{\Cat}( [n], \calC );$$
here $[n]$ denotes the linearly ordered set $\{ 0, \ldots, n\}$, regarded
as a category. This
definition makes sense also when $\calC$ is a simplicial
category, but is clearly not very interesting: it makes no use of
the simplicial structure on $\calC$. In order to obtain a more interesting construction,
we need to replace the ordinary category $[n]$ by a suitable ``thickening'', a simplicial
category which we will denote by $\sCoNerve[\Delta^n]$\index{not}{CoNerve@$\sCoNerve[S]$}.

\begin{definition}\label{csimp1}
Let $J$ be a finite nonempty linearly ordered set. The simplicial category
$\sCoNerve[\Delta^J]$ is defined as follows:
\begin{itemize}
\item The objects of $\sCoNerve[\Delta^J]$ are the elements of
$J$.

\item If $i,j \in J$, then
$$\bHom_{\sCoNerve[\Delta^J]}(i,j) = \begin{cases} \emptyset & \text{if } j < i \\
\Nerve(P_{i,j}) & \text{if } i \leq j. \end{cases}$$
Here $P_{i,j}$ denotes the partially ordered set $\{ I \subseteq J: (i,j \in I) \wedge (
\forall k \in I) [i \leq k \leq j] ) \}$.

\item If $i_0 \leq i_1 \leq \ldots \leq i_n$, then the composition
$$ \bHom_{\sCoNerve[\Delta^J]}(i_0,i_1) \times \ldots \times
\bHom_{\sCoNerve[\Delta^J]}(i_{n-1},i_n) \rightarrow
\bHom_{\sCoNerve[\Delta^J]}(i_0,i_n)$$ is induced by the map of
partially ordered sets
$$ P_{i_0,i_1} \times \ldots  \times P_{i_{n-1},i_n} \rightarrow P_{i_0,i_n}$$
$$ ( I_1, \ldots, I_n ) \mapsto I_1 \cup \ldots \cup I_n.$$
\end{itemize}
\end{definition}

In order to help digest Definition \ref{csimp1}, let us analyze the structure of the
topological category $| \sCoNerve[ \Delta^n ] |$. The objects of this category
are elements of the set $[n] = \{ 0, \ldots, n\}$. For each
$0 \leq i \leq j \leq n$, the topological space $\bHom_{ |\sCoNerve[\Delta^n]| }(i,j)$ is homeomorphic to a cube; it may be identified with the set of all functions $p: \{ k \in [n]: i \leq k \leq j \} \rightarrow [0,1]$ which satisfy $p(i) = p(j) =
1$. The morphism space $\bHom_{ | \sCoNerve[\Delta^n] |}(i,j)$ is empty when
$j < i$, and composition of morphisms is given by concatenation of functions.

\begin{remark}\label{conervexp}
Let us try to understand better the simplicial category $\sCoNerve[\Delta^n]$ and its relationship to the ordinary category $[n]$. These categories have the same objects, namely the elements of $\{ 0, \ldots, n\}$.
In the category $[n]$, there is a unique morphism $q_{ij}: i \rightarrow j$ whenever $i \leq j$. In virtue of the uniqueness, these elements satisfy $q_{jk} \circ q_{ij} = q_{ik}$ for $i \leq j \leq k$.

In the simplicial category $\sCoNerve[\Delta^n]$, there is a vertex
$p_{ij} \in \bHom_{ \sCoNerve[\Delta^n] }(i,j)$, given by the element
$\{ i, j \} \in P_{ij}$. We note that $p_{jk} \circ p_{ij} \neq p_{ik}$ (unless we are in one of the degenerate cases where
$i =j$ or $j=k$). Instead, the collection of all compositions
$$ p_{i_n i_{n-1}} \circ p_{i_{n-1} i_{n-2}} \circ \ldots \circ p_{i_1 i_0},$$ where 
$i=i_0 < i_1 < \ldots <  i_{n-1} < i_n = j$ constitute all of the different vertices of the cube 
$\bHom_{\sCoNerve[\Delta^n]}(i,j)$. The weak contractibility of $\bHom_{\sCoNerve[\Delta^n]}(i,j)$
expresses the idea that although these compositions do not coincide, they are all canonically homotopic to one another. We observe that there is a (unique) functor
$\sCoNerve[\Delta^n] \rightarrow [n]$ which is the identity on objects, and that this functor
is an equivalence of simplicial categories. We can summarize the situation informally as follows: the simplicial category $\sCoNerve[\Delta^n]$ is a ``thickened version'' of $[n]$, where
we have dropped the strict associativity condition
$$ q_{jk} \circ q_{ij} = q_{ik}$$ and instead have imposed associativity only up to (coherent) homotopy. (We can formulate this idea more precisely by saying that $\sCoNerve[ \Delta^{\bigdot}]$ is a cofibrant replacement for $[\bigdot]$ with respect to a suitable model structure on the category of cosimplicial objects of $\sCat$.)
\end{remark}

The construction $J \mapsto \sCoNerve[\Delta^J]$ is functorial in $J$, as we now explain.

\begin{definition}\label{csimp2}
Let $f: J \rightarrow J'$ be a monotone map between linearly
ordered sets. The simplicial functor $\sCoNerve[f]:
\sCoNerve[\Delta^J] \rightarrow \sCoNerve[\Delta^{J'}]$ is defined
as follows:

\begin{itemize}
\item For each object $i \in \sCoNerve[\Delta^J]$,
$\sCoNerve[f](i) = f(i) \in \sCoNerve[\Delta^{J'}]$.

\item If $i \leq j$ in $J$, then the map
$ \bHom_{\sCoNerve[\Delta^J]}(i,j) \rightarrow
\bHom_{\sCoNerve[\Delta^{J'}]}(f(i),f(j))$ induced by $f$ is the nerve of the map $$P_{i,j}
\rightarrow P_{f(i),f(j)}$$
$$ I \mapsto f(I).$$
\end{itemize}
\end{definition}

\begin{remark}
Using the notation of Remark \ref{conervexp}, we note that Definition \ref{csimp2} has been rigged so that the functor $\sCoNerve[f]$ carries the vertex $p_{ij}
\in \bHom_{\sCoNerve[ \Delta^J ]}(i,j)$ to the vertex
$p_{f(i) f(j)} \in \bHom_{\sCoNerve[\Delta^{J'}]}(f(i), f(j))$.
\end{remark}

It is not difficult to check that the construction described in
Definition \ref{csimp2} is well-defined, and compatible with composition in $f$.
Consequently, we deduce that $\sCoNerve$ determines a functor
$$ \cDelta \rightarrow \sCat$$
$$ \Delta^n \mapsto \sCoNerve[ \Delta^n ],$$
which we may view as a cosimplicial object of $\sCat$.

\begin{definition}\index{gen}{simplicial nerve}\index{gen}{nerve!of a simplicial category}\index{not}{NervecalC@$\Nerve(\calC)$}\label{topnerve}
Let $\calC$ be a simplicial category. The {\it simplicial nerve}
$\Nerve(\calC)$ is the simplicial set described by the
formula
$$ \Hom_{ \sSet}( \Delta^n, \Nerve(\calC)) =
\Hom_{\sCat}( \sCoNerve[ \Delta^n ], \calC).$$ 

If $\calC$ is a topological category, we define the {\it
topological} {\it nerve} $\tNerve(\calC)$ of $\calC$ to be the
simplicial nerve of $\Sing \calC$.\index{gen}{topological nerve}\index{gen}{nerve!of a topological category}
\end{definition}

\begin{remark}
If $\calC$ is a simplicial (topological) category, we will often abuse terminology by referring to the 
simplicial (topological) nerve of $\calC$ simply as the {\em nerve} of $\calC$. 
\end{remark}

\begin{warning}
Let $\calC$ be a simplicial category. Then $\calC$ can be regarded as an ordinary category, by ignoring all simplices of positive dimension in the mapping spaces of $\calC$. The simplicial nerve of $\calC$ does {\em not} coincide with the nerve of this underlying ordinary category. Our notation is therefore potentially ambiguous. We will adopt the following convention: whenever $\calC$ is a simplicial category, $\Nerve(\calC)$ will denote the {\em simplicial} nerve of $\calC$, unless we specify otherwise. Similarly, if $\calC$ is a topological category, then the topological nerve
of $\calC$ does not generally coincide with the nerve of the underlying category; the notation
$\Nerve(\calC)$ will be used to indicate the topological nerve, unless otherwise specified.
\end{warning}

\begin{example}
Any ordinary category $\calC$ may be considered as a simplicial
category, by taking each of the simplicial sets
$\Hom_{\calC}(X,Y)$ to be {\em constant}. In this case, the set of
simplicial functors $\sCoNerve[\Delta^n] \rightarrow \calC$ may be
identified with the set of functors from $[n]$ into $\calC$.
Consequently, the simplicial nerve of $\calC$ agrees with the ordinary nerve of $\calC$, as defined in \S \ref{qqqc}. Similarly, the ordinary nerve of $\calC$ can be identified with the topological nerve of $\calC$, where $\calC$ is regarded as a topological category with discrete morphism spaces.
\end{example}

In order to get a feel for what the nerve of a topological
category $\calC$ looks like, let us explicitly describe its
low-dimensional simplices:

\begin{itemize}
\item The $0$-simplices of $\tNerve(\calC)$ may be identified with
the objects of $\calC$.

\item The $1$-simplices of $\tNerve(\calC)$ may be identified with
the morphisms of $\calC$.

\item To give a map from the boundary of a $2$-simplex into
$\tNerve(\calC)$ is to give a diagram (not necessarily commutative)
$$ \xymatrix{ & Y \ar[dr]^{f_{YZ}} & \\
X \ar[ur]^{f_{XY}} \ar[rr]^{f_{XZ}} & & Z. }$$
To give a $2$-simplex of $\tNerve(\calC)$ having this specified boundary is equivalent to
giving a path from $f_{XZ}$ to $f_{YZ} \circ f_{XY}$ in
$\bHom_{\calC}(X,Z)$.
\end{itemize}

The category $\sCat$ of simplicial categories admits (small)
colimits. Consequently, by formal nonsense, the functor
$\sCoNerve: \cDelta \rightarrow \sCat$ extends uniquely (up to unique isomorphism) to a
colimit-preserving functor $\sSet \rightarrow \sCat$, which we
will denote also by $\sCoNerve$. By construction, the functor
$\sCoNerve$ is left adjoint to the simplicial nerve functor $\sNerve$\index{not}{CoNerve@$\sCoNerve[S]$}. For each simplicial set $S$, we can view $\sCoNerve[S]$ as the simplicial category ``freely generated'' by $S$: every $n$-simplex $\sigma: \Delta^n \rightarrow S$ determines a functor $\sCoNerve[\Delta^n] \rightarrow \sCoNerve[S]$, which we can think of as a homotopy coherent diagram $[n] \rightarrow \sCoNerve[S]$. 

\begin{example}
Let $A$ be a partially ordered set. The simplicial category $\sCoNerve[\Nerve A]$ can be constructed using the following generalization of Definition \ref{csimp1}:
\begin{itemize}
\item The objects of $\sCoNerve[ \Nerve A]$ are the elements of $A$.
\item Given a pair of elements $a,b \in A$, the simplicial set $\bHom_{ \sCoNerve[ \Nerve A]}(a,b)$ can be identified with $\Nerve P_{a,b}$, where $P_{a,b}$ denotes the collection of linearly ordered subsets $S \subseteq A$ with least element $a$ and largest element $b$, partially ordered by inclusion.
\item Given a sequence of elements $a_0, \ldots, a_n \in A$, the composition map
$$ \bHom_{ \sCoNerve[ \Nerve A]}(a_0, a_1) \times \ldots \times \bHom_{ \sCoNerve[ \Nerve A]}( a_{n-1}, a_n) \rightarrow \bHom_{ \sCoNerve[ \Nerve A]}(a_0, a_n)$$
is induced by the map of partially ordered sets
$$ P_{a_0, a_1} \times \ldots \times P_{a_{n-1}, a_n} \rightarrow P_{ a_0, a_n}$$
$$ (S_1, \ldots, S_n) \mapsto S_1 \cup \ldots \cup S_n.$$
\end{itemize}
\end{example}

\begin{proposition}\label{toothy}
Let $\calC$ be a simplicial category having the property that, for every pair of objects
$X,Y \in \calC$, the simplicial set $\bHom_{\calC}(X,Y)$ is a Kan complex. Then the simplicial nerve $\sNerve(\calC)$ is an $\infty$-category.
\end{proposition}

\begin{proof}
We must show that if $0 < i < n$, then $\sNerve(\calC)$ has the right extension property with respect to the inclusion $\Lambda^n_i \subseteq \Delta^n$. Rephrasing this in the language of simplicial categories, we must show that $\calC$ has the right extension property with respect to the simplicial functor $\sCoNerve[\Lambda^n_i] \rightarrow \sCoNerve[\Delta^n].$
To prove this, we make use of the following observations concerning
$\sCoNerve[\Lambda^n_i]$, which we view as a simplicial subcategory 
of $\sCoNerve[\Delta^n]$:

\begin{itemize}
\item The objects of $\sCoNerve[\Lambda^n_i]$ are the objects of
$\sCoNerve[\Delta^n]$: that is, elements of the set $[n]$.

\item For $0 \leq j \leq k \leq n$, the simplicial set
$\bHom_{\sCoNerve[\Lambda^n_i]}(j,k)$ coincides with
$\bHom_{\sCoNerve[\Delta^n]}(j,k)$ unless $j=0$ and $k=n$
(note that this condition fails if $i=0$ or $i=n$).
\end{itemize}

Consequently, every extension problem
$$ \xymatrix{ \Lambda^n_i \ar@{^{(}->}[d] \ar[r]^{F} & \Nerve(\calC) \\
\Delta^n \ar@{-->}[ur] & }$$
is equivalent to
$$\xymatrix{ \bHom_{\sCoNerve[\Lambda^n_i]}(0,n) \ar[d] \ar[r] & \bHom_{\calC}(F(0), F(n)) \\
\bHom_{ \sCoNerve[\Delta^n]}(0,n) \ar@{-->}[ur]. & }$$
Since the simplicial set on the right is a Kan complex by assumption, it suffices to verify that the left vertical map is anodyne. This follows by inspection: the simplicial set $\bHom_{ \sCoNerve[ \Delta^n]}(0,n)$ can be identified with the cube $( \Delta^1 )^{ \{ 1, \ldots, n-1\} }$, and 
$\bHom_{\sCoNerve[ \Lambda^n_i]}(0,n)$ can be identified with the simplicial subset obtained by removing the interior of the cube together with one of its faces.
\end{proof}

\begin{remark}\label{goobrem}
The proof of Proposition \ref{toothy} yields a slightly stronger result: if $F: \calC \rightarrow \calD$ is a functor between simplicial categories which induces Kan fibrations
$\bHom_{\calC}(C,C') \rightarrow \bHom_{\calD}(F(C),F(C'))$ for every pair of objects $C,C' \in \calC$, then the associated map $\sNerve(\calC) \rightarrow \sNerve(\calD)$ is an inner fibration of simplicial sets (see Definition \ref{fibdeff}).
\end{remark}

\begin{corollary}\label{tooky}
Let $\calC$ be a topological category. Then the topological nerve $\tNerve(\calC)$ is an $\infty$-category.
\end{corollary}

\begin{proof}
This follows immediately from Proposition \ref{toothy}, since the singular complex of any topological space is a Kan complex.
\end{proof}

We now cite the
following theorem, which will be proven in \S \ref{compp2} and refined in \S \ref{compp3}:

\begin{theorem}\label{biggie}
Let $\calC$ be a topological category, and let $X, Y \in \calC$ be
objects. Then the counit map
$$|\bHom_{\sCoNerve[\tNerve(\calC)]}(X,Y)| \rightarrow \bHom_{\calC}(X,Y)$$
is a weak homotopy equivalence of topological spaces.
\end{theorem}

Assuming Theorem \ref{biggie}, we can now explain why the theory of $\infty$-categories is 
equivalent to the theory of topological categories (or, equivalently, simplicial categories).
The adjoint functors $\Nerve$ and $| \sCoNerve[ \bigdot ] |$ are not mutually inverse equivalences of categories. However, they {\em are} homotopy inverse to one another. To make this precise, we need to introduce a definition.

\begin{definition}\label{tulkas}\index{gen}{homotopy category!of a simplicial set}
Let $S$ be a simplicial set. The {\it homotopy category} $\h{S}$ is defined to be the homotopy category $\h{ \sCoNerve[S]}$ of the simplicial category $\sCoNerve[S]$.\index{not}{hS@$\h{S}$}
We will often view $\h{S}$ as a category enriched over the homotopy category
$\calH$ of spaces via the construction of \S \ref{compp1}: that is, for
every pair of vertices $x,y \in S$, we have $\bHom_{ \h{S}}(x,y) = [ \bHom_{ \sCoNerve[S]}(x,y) ]$.
A map $f: S \rightarrow T$ of simplicial sets is a {\it categorical equivalence} if 
the induced map $\h{S} \rightarrow \h{T}$ is an equivalence of $\calH$-enriched categories.\index{gen}{categorical equivalence}
\end{definition}

\begin{remark}
In \cite{joyalnotpub}, Joyal uses the term ``weak categorical equivalence'' for what we have called a ``categorical equivalence'', and reserves the term ``categorical equivalence'' for a stronger notion of equivalence.
\end{remark}

\begin{remark}
We have introduced the term ``categorical equivalence'', rather than simply ``equivalence'' or ``weak equivalence'', in order to avoid confusing the notion of categorical equivalence of simplicial sets with the (more classical) notion of weak homotopy equivalence of simplicial sets.
\end{remark}

\begin{remark}\label{gytyt}
It is immediate from the definition that $f: S \rightarrow T$ is a categorical equivalence if and only if
$\sCoNerve[S] \rightarrow \sCoNerve[T]$ is an equivalence (of simplicial categories), if and only if $|\sCoNerve[S]| \rightarrow |\sCoNerve[T]|$ is an equivalence (of topological categories).
\end{remark}

We now observe that the adjoint functors $(|\sCoNerve[ \bigdot ]|, \tNerve )$
determine an equivalence between the theory of simplicial sets (up to categorical equivalence)
and that of topological categories (up to equivalence). In other words, for any
topological category $\calC$ the counit map
$|\sCoNerve[ \tNerve(\calC)] | \rightarrow \calC$ is an equivalence of topological categories, and for any simplicial set $S$ the unit map
$S \rightarrow \tNerve |\sCoNerve[S]|$
is a categorical equivalence of simplicial sets. In view of Remark \ref{gytyt}, the second assertion is a formal consequence of the first. Moreover, the first assertion is merely a reformulation of Theorem \ref{biggie}.

\begin{remark}
The reader may at this point object that we have achieved a comparison between the theory of topological categories with the theory of simplicial sets, but that not every simplicial set is an $\infty$-category. However, every simplicial set is categorically equivalent to an $\infty$-category. In fact, Theorem \ref{biggie} implies that every simplicial set $S$ is categorically equivalent to the nerve of the topological category $| \sCoNerve[S] |$, which is an $\infty$-category (Corollary \ref{tooky}).
\end{remark}



%\end{Didn't Read}