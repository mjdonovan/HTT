% !TEX root = highertopoi.tex
\section{Kan Extensions}\label{relacoim}
\setcounter{theorem}{0}

Let $\calC$ and $\calI$ be ordinary categories. There is an obvious ``diagonal'' functor $\delta: \calC \rightarrow \calC^{\calI}$, which carries an object $C \in \calC$ to the constant diagram $\calI \rightarrow \calC$ taking the value $C$. If $\calC$ admits small colimits, then the functor $\delta$ has a left adjoint $\calC^{\calI} \rightarrow \calC$. This left adjoint admits an explicit description: it carries an arbitrary diagram $f: \calI \rightarrow \calC$ to the colimit $\colim(f)$. Consequently, we can think of the theory of colimits as the study of left adjoints to diagonal functors.\index{gen}{diagonal functor}

More generally, if one is given a functor $i: \calI \rightarrow \calI'$ between diagram categories, then composition with $i$ induces a functor $i^{\ast}: \calC^{\calI'} \rightarrow \calC^{\calI}$. Assuming that $\calC$ has a sufficient supply of colimits, one can construct a left adjoint to $i^{\ast}$.
We then refer to this left adjoint as {\it left Kan extension along $i$}.\index{gen}{Kan extension}

In this section, we will study the $\infty$-categorical analogue of the theory of left Kan extensions. 
In the extreme case where $\calI'$ is the one-object category $\ast$, this theory simply reduces to the theory of colimits introduced in \S \ref{limitcolimit}. Our primary interest will be at the opposite extreme, when $i$ is a fully faithful embedding; this is the subject of \S \ref{kanex}. We will treat the general case in \S \ref{bigkanext}.

With a view toward later applications, we will treat not only the theory of {\em absolute} left Kan extensions, but also a relative notion which works over a base simplicial set $S$. The most basic example is the case of a {\it relative colimit} which we study in \S \ref{relcol}.

\subsection{Relative Colimits}\label{relcol}

In \S \ref{limitcolimit}, we introduced the notions of limit and colimit for a diagram
$p: K \rightarrow \calC$ in an $\infty$-category $\calC$. For many applications, it is convenient to have a {\em relative} version of these notions, which makes reference not to an $\infty$-category $\calC$ but to an arbitrary inner fibration of simplicial sets.  

\begin{definition}\label{relcoldef}\index{gen}{colimit!relative}\index{gen}{$f$-colimit}
Let $f: \calC \rightarrow \calD$ be an inner fibration of simplicial sets, let
$\overline{p}: K^{\triangleright} \rightarrow \calC$ be diagram, and let $p = \overline{p}|K$.
We will say that $\overline{p}$ is an {\it $f$-colimit} of $p$ if the map
$$ \calC_{\overline{p}/} \rightarrow \calC_{p/} \times_{ \calD_{f p/} } \calD_{ f\overline{p}/} $$
is a trivial fibration of simplicial sets. In this case, we will also say that
$\overline{p}$ is an {\it $f$-colimit diagram}.
\end{definition}

\begin{remark}\label{suppwolf}
Let $f: \calC \rightarrow \calD$ and $\overline{p}: K^{\triangleright} \rightarrow \calC$ be
as in Definition \ref{relcoldef}. Then $\overline{p}$ is an $f$-colimit of $p = \overline{p}|K$ if and only if the map 
$$ \phi: \calC_{\overline{p}/} \rightarrow \calC_{p/} \times_{ \calD_{f p/} } \calD_{ f \overline{p}/} $$
is a categorical equivalence. The ``only if'' direction is clear. The converse follows from
Proposition \ref{sharpen} (which implies that $\phi$ is a left fibration), Proposition \ref{funkyfibcatfib} (which implies that $\phi$ is a categorical fibration), and the fact that a categorical fibration which is a categorical equivalence is a trivial Kan fibration.

Observe that Proposition \ref{sharpen} also implies that the map
$$ \calD_{f  \overline{p}/} \rightarrow \calD_{f p/}$$
is a left fibration. Using Propositions \ref{basechangefunky} and \ref{funkyfibcatfib}, we conclude that $ \calC_{p/} \times_{ \calD_{f p/} } \calD_{f \overline{p}/}$ is a homotopy fiber product of
$\calC_{p/}$ with $\calD_{f \overline{p}/}$ over $\calD_{f p/}$ (with respect to the Joyal model structure on $\sSet$). 
Consequently, we deduce that $\overline{p}$ is an $f$-colimit diagram if and only if the
diagram of simplicial sets
$$ \xymatrix{ \calC_{ \overline{p}/} \ar[r] \ar[d] & \calD_{ f \overline{p} / } \ar[d] \\
\calC_{ p/ } \ar[r] & \calD_{f p/ } }$$
is homotopy Cartesian.
\end{remark}

\begin{example}
Let $\calC$ be an $\infty$-category and $f: \calC \rightarrow \ast$ the projection of $\calC$ to a point. Then a diagram $\overline{p}: K^{\triangleright} \rightarrow \calC$ is an $f$-colimit if and only if it is a colimit in the sense of Definition \ref{defcolim}.
\end{example}

\begin{example}\label{exex1}
Let $f: \calC \rightarrow \calD$ be an inner fibration of simplicial sets, and let
$e: \Delta^1 = (\Delta^0)^{\triangleright} \rightarrow \calC$ be an edge of $\calC$. Then $e$ is an $f$-colimit if and only if it is $f$-coCartesian.
\end{example}

The following basic stability properties follow immediately from the definition:

\begin{proposition}\label{basrel}
\begin{itemize}

\item[$(1)$] Let $f: \calC \rightarrow \calD$ be a trivial fibration of simplicial sets. Then
every diagram $\overline{p}: K^{\triangleright} \rightarrow \calC$ is an $f$-colimit.

\item[$(2)$] Let $f: \calC \rightarrow \calD$ and $g: \calD \rightarrow \calE$ be inner fibrations of simplicial sets, and let $\overline{p}: K^{\triangleright} \rightarrow \calC$ be a diagram.
Suppose that $f \circ \overline{p}$ is a $g$-colimit. Then $\overline{p}$ is an $f$-colimit if and only if $\overline{p}$ is a $g \circ f$-colimit.

\item[$(3)$] Let $f: \calC \rightarrow \calD$ be an inner fibration of $\infty$-categories, and let
$\overline{p}, \overline{q}: K^{\triangleright} \rightarrow \calC$ be diagrams which are equivalent when viewed as objects of the $\infty$-category $\Fun(K^{\triangleright}, \calC)$. Then $\overline{p}$ is 
an $f$-colimit if and only if $\overline{q}$ is an $f$-colimit.

\item[$(4)$] Suppose given a Cartesian diagram
$$ \xymatrix{ \calC' \ar[d]^{f'} \ar[r]^{g} & \calC \ar[d]^{f} \\
\calD' \ar[r] & \calD }$$
of simplicial sets, where $f$ (and therefore also $f'$) is an inner fibration. Let $\overline{p}: K^{\triangleright} \rightarrow \calC'$ be a diagram. If $g \circ \overline{p}$ is an $f$-colimit, then
$\overline{p}$ is an $f'$-colimit.
\end{itemize}
\end{proposition}

\begin{proposition}\label{summertoy}
Suppose give a commutative diagram of $\infty$-categories
$$ \xymatrix{ \calC \ar[r]^{f} \ar[d]^{p} & \calC' \ar[d]^{p'} \\
\calD \ar[r] & \calD' }$$
where the horizontal arrows are categorical equivalences and the vertical arrows are inner fibrations.
Let $\overline{q}: K^{\triangleright} \rightarrow \calC$ be a diagram and let $q = \overline{q}|K$
Then $\overline{q}$ is a $p$-colimit of $q$ if and only if
$f \circ \overline{q}$ is a $p'$-colimit of $f \circ q$.
\end{proposition}

\begin{proof}
Consider the diagram
$$ \xymatrix{ \calC_{ \overline{q}/ } \ar[r] \ar[d] & \calC'_{ f  \overline{q}/} \ar[d] \\
\calC_{q/} \times_{\calD_{p  q/}} \calD_{p  \overline{q}/} \ar[r] &
\calC'_{f  q/} \times_{ \calD'_{p'  f  q/}} \calD'_{p'  f  \overline{q}/} }.$$
According to Remark \ref{suppwolf}, it will suffice to show that the left vertical map is a categorical equivalence if and only if the right vertical map is a categorical equivalence. For this, it suffices to show that both of the horizontal maps are categorical equivalences. Proposition \ref{gorban3}
implies that the maps $ \calC_{ \overline{q}/} \rightarrow \calC'_{ f  \overline{q}/ }$,
$ \calC_{q/ } \rightarrow \calC'_{f  q/}$, $\calD_{p  \overline{q}/} \rightarrow \calD'_{p'  f  \overline{q}/}$, and $ \calD_{p  q/} \rightarrow \calD'_{p'  f  q/}$
are categorical equivalences. It will therefore suffice to show that the diagrams
$$ \xymatrix{ \calC_{q/} \times_{ \calD_{p  q/}} \calD_{p  \overline{q}/} \ar[r] \ar[d] &
\calC_{q/} \ar[d] &
\calC'_{ f  q/} \times_{ \calD'_{p'  f  q/} } \calD'_{ p'  f  \overline{q}/}
\ar[r] \ar[d] & \calC'_{ f  q /} \ar[d] \\
\calD_{p  \overline{q}/} \ar[r]^{\psi} & \calD_{p  q/} & \calD'_{p'  f  \overline{q}/}
\ar[r]^{\psi'} & \calD'_{p'  f  q/} }$$
are homotopy Cartesian (with respect to the Joyal model structure). This follows from
Proposition \ref{basechangefunky}, since $\psi$ and $\psi'$ are coCartesian fibrations.
\end{proof}

The next pair of results can be regarded as a generalization of Proposition \ref{gute}. They assert that, when computing relative colimits, we are free to replace any diagram by a cofinal subdiagram.

\begin{proposition}\label{relexists}
Let $p: \calC \rightarrow \calD$ be an inner fibration of $\infty$-categories, let $i: A \rightarrow B$ be a cofinal map, and let $\overline{q}: B^{\triangleright} \rightarrow \calC$ be a diagram.
Then $\overline{q}$ is a $p$-colimit if and only if $\overline{q} \circ i^{\triangleright}$
is a $p$-colimit.
\end{proposition}

\begin{proof}
Recall (Remark \ref{suppwolf}) that $\overline{q}$ is a relative colimit diagram if and only if the diagram
$$ \xymatrix{ \calC_{\overline{q}/} \ar[r] \ar[d] & \calC_{q/} \ar[d] \\
\calD_{ \overline{q}_0/} \ar[r] & \calD_{q_0/} }$$
is homotopy Cartesian with respect to the Joyal model structure. Since $i$ and $i^{\triangleright}$ are both cofinal, this is equivalent to the assertion that the diagram
$$ \xymatrix{ \calC_{\overline{q}  i^{\triangleright} /} \ar[r] \ar[d] & \calC_{q  i/} \ar[d] \\
\calD_{\overline{q}_0  i^{\triangleright} /} \ar[r] & \calD_{q_0  i/} }$$
is homotopy Cartesian, which (by Remark \ref{suppwolf}) is equivalent to the assertion that
$\overline{q} \circ i^{\triangleright}$ is a relative colimit diagram.
\end{proof}

\begin{proposition}\label{relexist}
Let $p: \calC \rightarrow \calD$ be a coCartesian fibration of $\infty$-categories, let $i: A \rightarrow B$ be a cofinal map, and let 
$$ \xymatrix{ B \ar[r]^{q} \ar[d] & \calC \ar[d]^{p} \\
B^{\triangleright} \ar[r]^{\overline{q}_0} & \calD }$$
be a diagram. Suppose that $q \circ i$
has a relative colimit lifting $\overline{q}_0 \circ i^{\triangleright}$. Then $q$ has a relative colimit lifting $\overline{q}_0$.
\end{proposition}

\begin{proof}
Let $q_0 = \overline{q}_0 | B$. We have a commutative diagram
$$ \xymatrix{ \calC_{q/} \ar[r]^-{f} \ar[d] & \calC_{q  i/} \times_{ \calD_{p  q  i/} } \calD_{p  q/} \ar[r] \ar[d] & \calC_{q  i/} \ar[d] \\
\calD_{q_0/} \ar[r] & \calD_{q_0/} \ar[r] & \calD_{q_0  i/} } $$
where the horizontal maps are categorical equivalences (since $i$ is cofinal, and by
Proposition \ref{basechangefunky}). Proposition \ref{werylonger} implies that the vertical maps are coCartesian fibrations, and that $f$ preserves coCartesian edges. Applying Proposition \ref{apple1} to $f$, we deduce that the map
$\phi: \calC_{q/} \times_{ \calD_{q_0/} } \{ \overline{q}_0 \} \rightarrow
\calC_{q  i/ } \times_{ \calD_{q_0  i/} } \{ \overline{q}_0  i^{\triangleright} \}$
is a categorical equivalence. Since $\phi$ is essentially surjective, we conclude that
there exists an extension $\overline{q}: B^{\triangleright} \rightarrow \calC$ of $q$
which covers $\overline{q}_0$, such that $\overline{q} \circ i^{\triangleright}$ is a $p$-colimit diagram. We now apply Proposition \ref{relexists} to conclude that $\overline{q}$ is itself a $p$-colimit diagram.
\end{proof}

Let $p: X \rightarrow S$ be a coCartesian fibration. 
The following results will allow us to reduce the theory of $p$-colimits to the theory of ordinary colimits in the fibers of $p$.

\begin{proposition}\label{chocolatelast}
Let $p: X \rightarrow S$ be an inner fibration of $\infty$-categories, $K$ a simplicial set, and
$\overline{h}: \Delta^1 \times K^{\triangleright} \rightarrow X$ a natural transformation from
$\overline{h}_0 = \overline{h} | \{0\} \times K^{\triangleright}$ to $\overline{h}_1 = \overline{h} | \{1\} \times K^{\triangleright}$.
Suppose that:
\begin{itemize}
\item[$(1)$] For every vertex $x$ of $K^{\triangleright}$, the restriction
$\overline{h} | \Delta^1 \times \{x\}$ is a $p$-coCartesian edge of $X$.
\item[$(2)$] The composition
$$ \Delta^1 \times \{\infty\} \subseteq \Delta^1 \times K^{\triangleright}
\stackrel{\overline{h}}{\rightarrow} X \stackrel{p}{\rightarrow} S$$
is a degenerate edge of $S$, where $\infty$ denotes the cone point of
$K^{\triangleright}$.
\end{itemize}

Then $h_0$ is a $p$-colimit if and only if $h_1$ is a $p$-colimit.
\end{proposition}

\begin{proof}
Let $h = \overline{h} | \Delta^1 \times K$, $h_0 = h | \{0\} \times K$, and $h_1 = h | \{1\} \times K$.
Consider the diagram
$$ \xymatrix{ X_{\overline{h}_0/} \ar[d] & X_{\overline{h}/} \ar[l]_{\phi} \ar[r] \ar[d] & X_{\overline{h}_1/} \ar[d] \\
X_{h_0/} \times_{S_{p  h_0/} } S_{p  \overline{h}_0} &
X_{h/} \times_{S_{p  h/}} S_{p  \overline{h}/} \ar[r] \ar[l]_{\psi} &
X_{h_1/} \times_{S_{p  h_1/}} S_{p  \overline{h}_1/} }$$
According to Remark \ref{suppwolf}, it will suffice to show that the left vertical map is a categorical equivalence if and only if the right vertical map is a categorical equivalence. For this, it will suffice to show that each of the horizontal arrows is a categorical equivalence. Because the inclusions
$\{1\} \times K \subseteq \Delta^1 \times K$ and $\{1\} \times K^{\triangleright} \subseteq
\Delta^1 \times K^{\triangleright}$ are right anodyne, the horizontal maps on the right are trivial fibrations. We are therefore reduced to proving that $\phi$ and $\psi$ are categorical equivalences.

Let $f: x \rightarrow y$ denote the edge of $X$ obtained by restricting $\overline{h}$ to the cone point of $K^{\triangleright}$. The map $\phi$ fits into a commutative diagram
$$ \xymatrix{ X_{\overline{h}/} \ar[r]^{\phi} \ar[d] & X_{h_0/} \ar[d] \\
X_{f/} \ar[r] & X_{x/}. }$$
Since the inclusion of the cone point into $K^{\triangleright}$ is right anodyne, the vertical arrows are trivial fibrations. Moreover, hypotheses $(1)$ and $(2)$ guarantee that $f$ is an equivalence in $X$, so that the map $X_{f/} \rightarrow X_{x/}$ is a trivial fibration. This proves that $\phi$ is a categorical equivalence.

The map $\psi$ admits a factorization
$$ X_{h/} \times_{S_{p  h/}} S_{p  \overline{h}/}
\stackrel{\psi'}{\rightarrow} 
X_{h_0/} \times_{ S_{p  h_0/}} S_{p  \overline{h}/}
\stackrel{\psi''}{\rightarrow}
X_{h_0} \times_{ S_{p  h_0/}} S_{ p  \overline{h}_0/}.$$
To complete the proof, it will suffice to show that $\psi'$ and $\psi''$ are trivial fibrations of simplicial sets. We first observe that $\psi'$ is a pullback of the map
$$X_{h/} \rightarrow X_{h_0/} \times_{S_{p  h_0/} } S_{p  h/},$$
which is a trivial fibration (Proposition \ref{eggwhite}). The map $\psi''$
is a pullback of the left fibration $\psi''_0: S_{p  \overline{h}/} \rightarrow S_{p  \overline{h}_0/}$. It therefore suffices to show that $\psi''_0$ is a categorical equivalence.
To prove this, we consider the diagram
$$ \xymatrix{ S_{p  \overline{h}/} \ar[r]^{\psi''_0} \ar[d] & S_{p  \overline{h}_0/} \ar[d] \\
S_{p(f)/} \ar[r]^{\psi''_1} & S_{p(x)/} }$$
As above, we observe that the vertical arrows are trivial fibrations, and $\psi''_1$ is a trivial fibration because the morphism $p(f)$ is an equivalence in $S$. It follows that $\psi''_0$ is a categorical equivalence, as desired.
\end{proof}

\begin{proposition}\label{relcolfibtest}
Let $q: X \rightarrow S$ be a locally coCartesian fibration of $\infty$-categories, let $s$ be an object of $S$, and let $\overline{p}: K^{\triangleright} \rightarrow X_{s}$ be a diagram. The following conditions are equivalent:
\begin{itemize}
\item[$(1)$] The map $\overline{p}$ is a $q$-colimit diagram.
\item[$(2)$] For every morphism $e: s \rightarrow s'$ in $S$, the associated functor
$e_{!}: X_{s} \rightarrow X_{s'}$ has the property that $e_{!} \circ \overline{p}$ is a colimit
diagram in the $\infty$-category $X_{s'}$.
\end{itemize}
\end{proposition}

\begin{proof}
Assertion $(1)$ is equivalent to the statement that the map
$$ \theta: X_{\overline{p}/} \rightarrow X_{p/} \times_{ S_{qp/} } S_{q \overline{p}/}$$
is a trivial fibration of simplicial sets. Since $\theta$ is a left fibration, it will suffice to show that the fibers of $\theta$ are contractible. Consider an arbitrary vertex of $S_{q \overline{p}/}$, corresponding to a morphism $t: K \star \Delta^1 \rightarrow S$. Since $K \star \Delta^1$
is categorically equivalent to $( K \star \{0\} ) \coprod_{ \{0\} } \Delta^1$ and
$t | K \star \{0\}$ is constant, we may assume without loss of generality that
$t$ factors as a composition
$$ K \star \Delta^1 \rightarrow \Delta^1 \stackrel{e}{\rightarrow} S.$$
Here $e: s \rightarrow s'$ is an edge of $S$. Pulling back by the map $e$, we can reduce to the problem of proving the following analogue of $(1)$ in the case where $S = \Delta^1$:
\begin{itemize}
\item[$(1')$] The projection $h_0: X_{ \overline{p}/} \times_{S} \{s' \} \rightarrow
X_{p/} \times_{S} \{ s' \}$ is a trivial fibration of simplicial sets.
\end{itemize}

Choose a coCartesian transformation $\overline{\alpha}: K^{\triangleright} \times \Delta^1 \rightarrow X$ from $\overline{p}$ to $\overline{p}'$, which covers the projection
$$ K^{\triangleright} \times \Delta^1 \rightarrow \Delta^1 \simeq S.$$
Consider the diagram
$$ \xymatrix{ X_{ \overline{p}/ } \times_{S} \{s' \} 
\ar[d]^{h_0} & X_{ \overline{\alpha} / } \times_{S} \{ s' \}
\ar[l] \ar[r] \ar[d]^{h} & X_{ \overline{p}' /} \times_{ S} \{ s' \}
\ar[d]^{h_1} \\
X_{p/} \times_{S } \{s'\} &
X_{\alpha/} \times_{ S } \{s'\} \ar[r] \ar[l] &
X_{p'/} \times_{ S } \{s'\}. }$$
Note that the vertical maps are left fibrations (Proposition \ref{sharpen}). Since the inclusion
$K^{\triangleright} \times \{1\} \subseteq K^{\triangleright} \times \Delta^1$ is right anodyne,
the upper right horizontal map is a trivial fibration. Similarly, the lower right horizontal map is a trivial fibration. Since $\overline{\alpha}$ is a coCartesian transformation, we deduce that the left  horizontal maps are also trivial fibrations (Proposition \ref{eggwhite}).
Condition $(2)$ is equivalent to the assertion that
$h_1$ is a trivial fibration (for each edge $e: s \rightarrow s'$ of the original simplicial set $S$). Since $h_1$ is a left fibration, and therefore a categorical fibration (Proposition \ref{funkyfibcatfib}), this is equivalent to the assertion that $h_1$ is a categorical equivalence. Chasing through the diagram, we deduce that
$(2)$ is equivalent to the assertion that $h_0$ is a categorical equivalence, which (by the same argument) is equivalent to the assertion that $h_0$ is a trivial fibration.
\end{proof}

\begin{corollary}\label{constrel}
Let $p: X \rightarrow S$ be a coCartesian fibration of $\infty$-categories, and let $K$ be a simplicial set.
Suppose that:
\begin{itemize}
\item[$(1)$] For each vertex $s$ of $S$, the fiber $X_{s} = X \times_{S} \{s\}$ admits colimits
for all diagrams indexed by $K$.
\item[$(2)$] For each edge $f: s \rightarrow s'$, the associated functor
$X_{s} \rightarrow X_{s'}$ preserves colimits of $K$-indexed diagrams.
\end{itemize}
Then for every diagram
$$ \xymatrix{ K \ar[r]^{q} \ar@{^{(}->}[d] & X \ar[d]^{p} \\
K^{\triangleright} \ar[r]^{f} \ar@{-->}[ur]^{\overline{q}} & S }$$
there exists a map $\overline{q}$ as indicated, which is a $p$-colimit.
\end{corollary}

\begin{proof}
Consider the map $K \times \Delta^1 \rightarrow K^{\triangleright}$ which is the identity on $K \times \{0\}$ and carries $K \times \{1\}$ to the cone point of $K^{\triangleright}$. 
Let $F$ denote the composition
$$ K \times \Delta^1 \rightarrow K^{\triangleright} \stackrel{f}{\rightarrow} S,$$
and let $Q: K \times \Delta^1 \rightarrow X$ be a coCartesian lifting of $F$
to $X$, so that $Q$ is a natural transformation from $q$ to a map
$q': K \rightarrow X_{s}$, where $s$ is the image under $f$ of the cone point of
$K^{\triangleright}$. In view of assumption $(1)$, there exists a map
$\overline{q}': K^{\triangleright} \rightarrow X_{s}$ which is a colimit of $q'$.
Assumption $(2)$ and Proposition \ref{relcolfibtest} guarantee that
$\overline{q}'$ is also a $p$-colimit diagram, when regarded as a map
from $K^{\triangleright}$ to $X$. 

We have a commutative diagram
$$ \xymatrix{ (K \times \Delta^1) \coprod_{ K \times \{1\} } (K^{\triangleright}
\times \{1\} ) \ar[rr]^-{(Q,\overline{q}')} \ar@{^{(}->}[d] & & X \ar[d]^{p} \\
(K \times \Delta^1)^{\triangleright} \ar@{-->}[urr]^{r} \ar[rr] & & S. }$$
The left vertical map is an inner fibration, so there exists a morphism
$r$ as indicated, rendering the diagram commutative. We now consider the map
$ K^{\triangleright} \times \Delta^1 \rightarrow (K \times \Delta^1)^{\triangleright}$
which is the identity on $K \times \Delta^1$ and carries the other vertices of
$K^{\triangleright} \times \Delta^1$ to the cone point of $(K \times \Delta^1)^{\triangleright}$.
Let $\overline{Q}$ denote the composition
$$ K^{\triangleright} \times \Delta^1 \rightarrow (K \times \Delta^1)^{\triangleright}
\stackrel{r}{\rightarrow} X, $$
and let $\overline{q} = \overline{Q} | K^{\triangleright} \times \{0\}$.
Then $\overline{Q}$ can be regarded as a natural transformation
$\overline{q} \rightarrow \overline{q}'$ of diagrams $K^{\triangleright} \rightarrow X$.
Since $\overline{q}'$ is a $p$-colimit diagram, Proposition \ref{chocolatelast}
implies that $\overline{q}$ is a $p$-colimit diagram as well.
\end{proof}

\begin{proposition}\label{timal}
Let $p: X \rightarrow S$ be a coCartesian fibration of $\infty$-categories, and let
$\overline{q}: K^{\triangleright} \rightarrow X$ be a diagram. Assume that:
\begin{itemize}
\item[$(1)$] The map $\overline{q}$ carries each edge of $K$ to a $p$-coCartesian
edge of $K$.
\item[$(2)$] The simplicial set $K$ is weakly contractible.
\end{itemize}

Then $\overline{q}$ is a $p$-colimit diagram if and only if 
it carries every edge of $K^{\triangleright}$ to a $p$-coCartesian edge of $X$.
\end{proposition}

\begin{proof}
Let $s$ denote the image under $p \circ \overline{q}$ of the cone point of $K^{\triangleright}$.
Consider the map $K^{\triangleright} \times \Delta^1 \rightarrow K^{\triangleright}$
which is the identity on $K^{\triangleright} \times \{0\}$ and collapses $K^{\triangleright} \times \{1\}$ to the cone point of $K^{\triangleright}$. Let $h$ denote the composition
$$ K^{\triangleright} \times \Delta^1 \rightarrow K^{\triangleright}
\stackrel{\overline{q}}{\rightarrow} X \stackrel{p}{\rightarrow} S,$$
which we regard as a natural transformation from $p \circ \overline{q}$ to the constant
map with value $s$. Let $H: \overline{q} \rightarrow \overline{q}'$ be a coCartesian transformation
from $\overline{q}$ to a diagram $\overline{q}': K^{\triangleright} \rightarrow X_{s}$.
Using Proposition \ref{protohermes}, we conclude that $\overline{q}'$ carries each
edge of $K$ to a $p$-coCartesian edge of $X$, which is therefore an equivalence
in $X_{s}$. 

Let us now suppose that $\overline{q}$ carries {\em every} edge of
$K^{\triangleright}$ to a $p$-coCartesian edge of $X$. Arguing as above, we conclude
that $\overline{q}'$ carries each edge of $K^{\triangleright}$ to an equivalence in $X_{s}$.
Let $e: s \rightarrow s'$ be an edge of $S$ and $e_{!}: X_{s} \rightarrow X_{s'}$ an associated functor. The composition
$$ K^{\triangleright} \stackrel{\overline{q}'}{\rightarrow} X_{s} \stackrel{e_{!}}{\rightarrow}
X_{s'}$$
carries each edge of $K^{\triangleright}$ to an equivalence in $X_{s}$, and 
is therefore a colimit diagram in $X_{s'}$ (Corollary \ref{silt}). Proposition \ref{relcolfibtest} implies
that $\overline{q}'$ is a $p$-colimit diagram, so that Proposition \ref{chocolatelast} implies that $\overline{q}$ is a $p$-colimit diagram as well.

For the converse, let us suppose that $\overline{q}$ is a $p$-colimit diagram.
Applying Proposition \ref{chocolatelast}, we conclude that $\overline{q}'$ is a $p$-colimit diagram. In particular, $\overline{q}'$ is a colimit diagram in the $\infty$-category
$X_{s}$. Applying Corollary \ref{silt}, we conclude that $\overline{q}'$ carries each
edge of $K^{\triangleright}$ to an equivalence in $X_{s}$. Now consider an arbitrary
edge $f: x \rightarrow y$ of $K^{\triangleright}$. If $f$ belongs to $K$, then
$\overline{q}(f)$ is $p$-coCartesian by assumption. Otherwise, we may suppose that
$y$ is the cone point of $K$. The map $H$ gives rise to a diagram
$$ \xymatrix{ \overline{q}(x) \ar[r]^{\overline{q}(f)} \ar[d]^{\phi} & \overline{q}(y) \ar[d]^{\phi'} \\
\overline{q}'(x) \ar[r]^{\overline{q}'(f)} & \overline{q}'(y) }$$
in the $\infty$-category $X \times_{S} \Delta^1$. Here 
$\overline{q}'(f)$ and $\phi'$ are equivalences in $X_{s}$, so that
$\overline{q}(f)$ and $\phi$ are equivalent as morphisms
$\Delta^1 \rightarrow X \times_{S} \Delta^1$. Since $\phi$ is $p$-coCartesian, we conclude
that $\overline{q}(f)$ is $p$-coCartesian, as desired.
\end{proof}

\begin{lemma}\label{gooodbar}
Let $p: \calC \rightarrow \calD$ be an inner fibration of $\infty$-categories, let
$C \in \calC$ be an object, and let $D = p(C)$.  Then $C$ is a $p$-initial object of $\calC$ if and only if $(C, \id_{D})$ is an initial object of $\calC \times_{ \calD} \calD_{D/}$.
\end{lemma}

\begin{proof}
We have a commutative diagram
$$ \xymatrix{ \calC_{C/} \times_{ \calD_{D/} } \calD_{ \id_{D}/} \ar[r]^{\psi} \ar[d]^{\phi} & \calC_{C/} \ar[d]^{\phi'} \\
\calC \times_{ \calD} \calD_{D/} \ar@{=}[r] & \calC \times_{\calD} \calD_{D/} }$$
where the vertical arrows are left fibrations, and therefore categorical fibrations (Proposition \ref{funkyfibcatfib}). We wish to show that $\phi$ is a trivial fibration if and only if $\phi'$ is a trivial fibration. This is equivalent to proving that $\phi$ is a categorical equivalence if and only if $\phi'$ is a categorical equivalence. For this, it will suffice to show that $\psi$ is a categorical equivalence.
But $\psi$ is a pullback of the trivial fibration $\calD_{\id_{D}/} \rightarrow \calD_{D/}$, and therefore itself a trivial fibration.
\end{proof}

\begin{proposition}\label{panna}
Suppose given a diagram of $\infty$-categories
$$ \xymatrix{ \calC \ar[dr]^{q} \ar[rr]^{p} & & \calD \ar[dl]_{r} \\
& \calE & }$$
where $p$ and $r$ are inner fibrations, $q$ is a Cartesian fibration, and
$p$ carries $q$-Cartesian morphisms to $r$-Cartesian morphisms.

Let $C \in \calC$ be an object, $D = p(C)$, and $E = q(C)$.
Let $\calC_{E} = \calC \times_{\calE} \{E\}$, $\calD_{E} = \calD \times_{\calE} \{E\}$, and
$p_{E}: \calC_{E} \rightarrow \calD_{E}$ the induced map. Suppose that
$C$ is a $p_{E}$-initial object of $\calC_{E}$. Then $C$ is a $p$-initial object of $\calC$.
\end{proposition}

\begin{proof}
Our hypothesis, together with Lemma \ref{gooodbar}, implies that
$(C, \id_{D})$ is an initial object of
$$\calC_{E} \times_{\calD_{E}} (\calD_{E})_{D/}
\simeq (\calC \times_{\calD} \calD_{D/} ) \times_{\calE_{E/}} \{ \id_{E} \}.$$
We will prove that the map
$\phi: \calC \times_{\calD} \calD_{D/} \rightarrow \calE_{E/}$ is a Cartesian fibration.
Since $\id_{E}$ is an initial object of $\calE_{E/}$, Lemma \ref{sabreto} will allow us to conclude that
$(C, \id_{D})$ is an initial object of $\calC \times_{\calD} \calD_{D/}$. We can then conclude the proof by applying Lemma \ref{gooodbar} once more.

It remains to prove that $\phi$ is a Cartesian fibration. Let us say that a morphism of
$\calC \times_{\calD} \calD_{D/}$ is {\it special} if its image in $\calC$ is $q$-Cartesian.
Since $\phi$ is obviously an inner fibration, it will suffice to prove the following assertions:

\begin{itemize}
\item[$(1)$] Given an object $X$ of $\calC \times_{\calD} \calD_{D/}$ and
a morphism $\overline{f}: \overline{Y} \rightarrow \phi(X)$ in $\calE_{E/}$, we can write
$\overline{f} = \phi(f)$ where $f$ is a special morphism of $\calC \times_{\calD} \calD_{D/}$.

\item[$(2)$] Every special morphism in $\calC \times_{\calD} \calD_{D/}$ is $\phi$-Cartesian.
\end{itemize}

To prove $(1)$, we first identify $X$ with a pair consisting of an object $C'' \in \calC$ and
a morphism $D \rightarrow p(C'')$ in $\calD$, and $\overline{f}$ with a $2$-simplex
$\overline{\sigma}: \Delta^2 \rightarrow \calE$ which we depict as a diagram:
$$ \xymatrix{ & E' \ar[dr]^{\overline{g}} & \\
E \ar[ur] \ar[rr] & & q(C''). }$$
Since $q$ is a Cartesian fibration, the morphism $\overline{g}$ can be written
as $q(g)$ for some morphism $g: C' \rightarrow C''$ in $\calC$. We now have
a diagram
$$ \xymatrix{ & p(C') \ar[dr]^{p(g)} & \\
D \ar[rr] & & p(C'')  }$$
in $\calD$. Since $p$ carries $q$-Cartesian morphisms to $r$-Cartesian morphisms,
we conclude that $p(g)$ is $r$-Cartesian, so that the above diagram can be completed to a $2$-simplex $\sigma: \Delta^2 \rightarrow \calD$ such that $r(\sigma) = \overline{\sigma}$.

We now prove $(2)$. Suppose $n \geq 2$, and we have a commutative diagram
$$ \xymatrix{ \Lambda^n_n \ar[r]^-{\sigma_0} \ar@{^{(}->}[d] & \calC \times_{\calD} \calD_{D/} \ar[d] \\
\Delta^n \ar[r] \ar@{-->}[ur]^{\sigma} & \calE_{E/} }$$
where $\sigma_0$ carries the final edge of $\Lambda^n_n$ to a special morphism of
$\calC \times_{\calD} \calD_{D/}$. We wish to prove the existence of the morphism $\sigma$
indicated in the diagram. We first let $\tau_0$ denote the composite map
$$ \Lambda^n_n \stackrel{\sigma_0}{\rightarrow} \calC \times_{\calD} \calD_{D/} \rightarrow \calC.$$
Consider the diagram
$$ \xymatrix{ \Lambda^n_n \ar[r]^{\tau_0} \ar@{^{(}->}[d] & \calC \ar[d]^{q} \\
\Delta^n \ar[r] \ar@{-->}[ur]^{\tau} & \calE. }$$
Since $\tau_0( \Delta^{ \{n-1, n\} })$ is $q$-Cartesian, there exists an extension
$\tau$ as indicated in the diagram. The morphisms $\tau$ and $\sigma_0$ together determine a map $\theta_0$ which fits into a diagram
$$ \xymatrix{ \Lambda^{n+1}_{n+1} \ar[r]^{\theta_0} \ar@{^{(}->}[d] & \calD \ar[d]^{r} \\
\Delta^{n+1} \ar[r] \ar@{-->}[ur]^{\theta} & \calE. }$$
To complete the proof, it suffices to prove the existence of the indicated arrow $\theta$.
This follows from the fact that $\theta_0( \Delta^{ \{n,n+1 \} }) = (p \circ \tau_0)( \Delta^{ \{n-1,n\} })$
is an $r$-Cartesian morphism of $\calD$.
\end{proof}

Proposition \ref{panna} immediately implies the following slightly stronger statement:

\begin{corollary}\label{pannaheave}
Suppose given a diagram of $\infty$-categories
$$ \xymatrix{ \calC \ar[dr]^{q} \ar[rr]^{p} & & \calD \ar[dl]_{r} \\
& \calE & }$$
where $q$ and $r$ are Cartesian fibrations, $p$ is an inner fibration, and 
$p$ carries $q$-Cartesian morphisms to $r$-Cartesian morphisms.

Suppose given another $\infty$-category $\calE_0$ equipped with a functor
$s: \calE_0 \rightarrow \calE$. Set $\calC_0 = \calC \times_{\calE} \calE_0$,
$\calD_0 = \calD \times_{\calE} \calE_0$, and let $p_0: \calC_0 \rightarrow \calD_0$ be the functor induced by $p$.
Let $\overline{f}_0: K^{\triangleright} \rightarrow \calC_0$ be a diagram and let
$\overline{f}$ denote the composition $K^{\triangleright} \stackrel{\overline{f}_0}{\rightarrow}
\calC_0 \rightarrow \calC$. Then $\overline{f}_0$ is a $p_0$-colimit diagram if and only if
$\overline{f}$ is a $p$-colimit diagram.
\end{corollary}

\begin{proof}
Let $f_0 = \overline{f}_0 | K$ and $f = \overline{f} | K$. Replacing our diagram by
$$ \xymatrix{ \calC_{f/} \ar[rr] \ar[dr] & & \calD_{pf/} \ar[dl] \\
& \calE_{qf/}, & }$$
we can reduce to the case where $K = \emptyset$. Then $\overline{f}_0$ determines
an object $C \in \calC_0$. Let $E$ denote the image of $C$ in $\calE_0$. We have a commutative diagram
$$ \xymatrix{ \{E\} \ar[rr]^{s'} \ar[dr]^{s''} & & \calE_0 \ar[dl]^{s} \\
& \calE. & }$$
Consequently, to prove Corollary \ref{pannaheave} for the map $s$, it will suffice to prove the
analogous assertions for $s'$ and $s''$; these follow from Proposition \ref{panna}.
\end{proof}

\begin{corollary}\label{superduck}
Let $p: \calC \rightarrow \calE$ be a Cartesian fibration of $\infty$-categories, 
$E \in \calE$ an object, and $\overline{f}: K^{\triangleright} \rightarrow \calC_{E}$ a diagram.
Then $\overline{f}$ is a colimit diagram in $\calC_{E}$ if and only if
it is a $p$-colimit diagram in $\calC$.
\end{corollary}

\begin{proof}
Apply Corollary \ref{pannaheave} in the case where $\calD = \calE$.
\end{proof}

\subsection{Kan Extensions along Inclusions}\label{kanex}

In this section, we introduce the theory of {\em left Kan extensions}. Let $F: \calC \rightarrow \calD$
be a functor between $\infty$-categories, and let $\calC^{0}$ be a full subcategory of $\calC$.
Roughly speaking, the functor $F$ is a left Kan extension of its restriction $F_0 = F | \calC^{0}$ if
the values of $F$ are as ``small'' as possible, given the values of $F_0$. In order to make this precise, we need to introduce a bit of terminology.

\begin{notation}
Let $\calC$ be an $\infty$-category, and let $\calC^{0}$ be a full subcategory. If $p:
K \rightarrow \calC$ is a diagram, we let $\calC^{0}_{/p}$ denote the fiber product
$\calC_{/p} \times_{ \calC} \calC^{0}$. In particular, if $C$ is an object of $\calC$,
then $\calC^{0}_{/C}$ denotes the full subcategory of $\calC_{/C}$ spanned by the morphisms $C' \rightarrow C$ where $C' \in \calC^{0}$.
\end{notation}

\begin{definition}\label{defKan}
Suppose given a commutative diagram of $\infty$-categories
$$ \xymatrix{ \calC^{0} \ar@{^{(}->}[d] \ar[r]^{F_0} & \calD \ar[d]^{p} \\
\calC \ar[r] \ar[ur]^{F} & \calD', }$$
where $p$ is an inner fibration and
the left vertical map is the inclusion of a full subcategory $\calC^{0} \subseteq \calC$.

We will say that $F$ is a {\it $p$-left Kan extension of $F_0$ at $C \in \calC$} if
the induced diagram
$$ \xymatrix{ (\calC^{0}_{/C}) \ar@{^{(}->}[d] \ar[r]^{F_C} & \calD \ar[d]^{p} \\
(\calC_{/C}^{0})^{\triangleright} \ar[ur] \ar[r] & \calD' }$$
exhibits $F(C)$ as a $p$-colimit of $F_{C}$.

We will say that $F$ is a {\it $p$-left Kan extension of $F_0$} if it is a $p$-left Kan extension
of $F_0$ at $C$, for every object $C \in \calC$.

In the case where $\calD' = \Delta^0$, we will omit mention of $p$ simply say that $F$ is a {\it left Kan extension of $F_0$} if the above condition is satisfied.\index{gen}{Kan extension}
\end{definition}

\begin{remark}\label{mozartwatch}
Consider a diagram
$$ \xymatrix{ \calC^{0} \ar@{^{(}->}[d] \ar[r]^{F_0} & \calD \ar[d]^{p} \\
\calC \ar[r] \ar[ur]^{F} & \calD' }$$
as in Definition \ref{defKan}. If
$C$ is an object of $\calC^{0}$, then the functor $F_{C}: (\calC_{/C}^{0})^{\triangleright} \rightarrow \calD$ is automatically a $p$-colimit. To see this, we observe that $\id_{C}: C \rightarrow C$ is a final object of $\calC^{0}_{/C}$. Consequently, the inclusion $\{ \id_C \} \rightarrow (\calC_{/C}^{0})$
is cofinal and we are reduced to proving that $F(\id_{C}): \Delta^1 \rightarrow \calD$ is a colimit
of its restriction to $\{0\}$, which is obvious.
\end{remark}

\begin{example}
Consider a diagram
$$ \xymatrix{ \calC \ar@{^{(}->}[d] \ar[r]^{q} & \calD \ar[d]^{p} \\
\calC^{\triangleright} \ar[r] \ar[ur]^{\overline{q}} & \calD'. }$$
The map $\overline{q}$ is a $p$-left Kan extension of $q$ if and only if it is a $p$-colimit of $q$.
The ``only if'' direction is clear from the definition, and the converse follows immediately from 
Remark \ref{mozartwatch}.
\end{example}

We first note a few basic stability properties for the class of left Kan extensions.

\begin{lemma}\label{switcher}
Consider a commutative diagram of $\infty$-categories
$$ \xymatrix{ \calC^{0} \ar@{^{(}->}[d] \ar[r]^{F_0} & \calD \ar[d]^{p} \\
\calC \ar[r] \ar[ur]^{F} & \calD' }$$
as in Definition \ref{defKan}. Let $C$ and $C'$ equivalent objects of $\calC$.
Then $F$ is a $p$-left Kan extension of $F_0$ at $C$ if and only if 
$F$ is a $p$-left Kan extension of $F_0$ at $C'$.
\end{lemma}

\begin{proof}
Let $f: C \rightarrow C'$ be an equivalence, so that the restriction maps
$$ \calC_{/C} \leftarrow \calC_{/f} \rightarrow \calC_{/C'}$$
are trivial fibrations of simplicial sets. Let $\calC^{0}_{/f} = \calC^{0} \times_{\calC} \calC_{/f}$, so that we have trivial fibrations
$$ \calC^{0}_{/C} \stackrel{g}{\leftarrow} \calC^{0}_{/f} \stackrel{g'}{\rightarrow} \calC^{0}_{/C'}.$$
Consider the associated diagram
$$ \xymatrix{ & (\calC^{0}_{/C})^{\triangleright} \ar[dr]^{F_{C}} & \\
(\calC^{0}_{/f})^{\triangleright} \ar[ur]^{G} \ar[dr]^{G'} & & \calD \\
& (\calC^{0}_{/C'})^{\triangleright} \ar[ur]^{F_{C'}} & .}$$
This diagram does not commute, but the functors
$F_{C} \circ G$ and $F_{C'} \circ G'$ are equivalent in the $\infty$-category
$\calD^{ (\calC^{0}_{/f})^{\triangleright}}$. Consequently,
$F_{C} \circ G$ is a $p$-colimit diagram if and only if $F_{C'} \circ G'$ is a $p$-colimit diagram
(Proposition \ref{basrel}).
Since $g$ and $g'$ are cofinal, we conclude that $F_{C}$ is a $p$-colimit diagram if and only if $F_{C'}$ is a $p$-colimit diagram (Proposition \ref{relexists}).
\end{proof}

\begin{lemma}\label{basekann}
\begin{itemize}
\item[$(1)$] Let $\calC$ be an $\infty$-category, $p: \calD \rightarrow \calD'$ an inner fibration of $\infty$-categories, and $F,F': \calC \rightarrow \calD$ be two functors which are equivalent
in $\calD^{\calC}$. Let $\calC^{0}$ be a full subcategory of $\calC$. Then $F$ is
a $p$-left Kan extension of $F| \calC^{0}$ if and only if $F'$ is a $p$-left Kan extension of $F'| \calC^{0}$.

\item[$(2)$] Suppose given a commutative diagram of $\infty$-categories
$$ \xymatrix{ \calC^{0} \ar[d]^{G_0} \ar[r] & \calC \ar[r]^{F} \ar[d]^{G} & \calD \ar[d] \ar[r]^{p} & \calE \ar[d] \\
{\calC'}^{0} \ar[r] & \calC' \ar[r]^{F'} & \calD' \ar[r]^{p'} & \calE' }$$
be a commutative diagram of $\infty$-categories, where the left horizontal maps are inclusions of full subcategories, the right horizontal maps are inner fibrations,
and the vertical maps are categorical equivalences. Then
$F$ is a $p$-left Kan extension of $F| \calC^{0}$ if and only if $F'$ is a $p'$-left Kan extension of $F' | {\calC'}^{0}$.
\end{itemize}
\end{lemma}

\begin{proof}
Assertion $(1)$ follows immediately from Proposition \ref{basrel}.
Let us prove $(2)$. Choose an object $C \in \calC$, and consider the diagram
$$ \xymatrix{ (\calC^{0}_{/C})^{\triangleright} \ar[r] \ar[d] & \calD \ar[d] \ar[r]^{p} & \calE \ar[d] \\
({\calC'}^{0}_{/G(C)})^{\triangleright} \ar[r] & \calD' \ar[r]^{p'} & \calE' }$$
We claim that the upper left horizontal map is a $p$-colimit diagram if and only if the bottom left horizontal map is a $p'$-colimit diagram.
In view of Proposition \ref{summertoy}, it will suffice to show that each of the vertical maps is an equivalence
of $\infty$-categories. For the middle and right vertical maps, this holds by assumption.
To prove that the left vertical map is a categorical equivalence, we consider the diagram
$$ \xymatrix{ \calC^{0}_{/C} \ar[r] \ar[d] & {\calC'}^{0}_{/G(C)} \ar[d] \\
\calC_{/C} \ar[r] & \calC'_{/G(C)}. }$$
The bottom horizontal map is a categorical equivalence by Proposition \ref{gorban3}, and the vertical maps are inclusions of full subcategories. It follows that the top horizontal map is fully faithful, and its essential image consists of those morphisms $C' \rightarrow G(C)$ where
$C'$ is equivalent (in $\calC'$) to the image of an object of $\calC^{0}$. Since $G_0$ is essentially surjective, this is the whole of ${\calC'}^{0}_{/G(C)}$. 

It follows that if $F'$ is a $p'$-left Kan extension of $F' | { \calC'}^{0}$, then $F$ is a $p$-left Kan extension of $F | \calC^{0}$. Conversely, if $F$ is a $p$-left Kan extension of $F| \calC^{0}$, then $F'$ is a $p'$-left Kan extension of $F' | { \calC'}^{0} $ at $G(C)$, for every object $C \in \calC$. Since $G$ is essentially surjective, Lemma \ref{switcher} implies that $F'$ is a $p'$-left Kan extension of $F' | {\calC'}^{0}$ at every object of $\calC'$. This completes the proof of $(2)$.
\end{proof}

\begin{lemma}\label{kan0}
Suppose given a diagram of $\infty$-categories
$$ \xymatrix{ \calC^{0} \ar@{^{(}->}[d] \ar[r]^{F_0} & \calD \ar[d]^{p} \\
\calC \ar[r] \ar[ur]^{F} & \calD' }$$
as in Definition \ref{defKan}, where $F$ is a left Kan extension of $F_0$ relative
to $p$. Then the induced map
$$ \calD_{F/} \rightarrow \calD'_{p F/ } \times_{ \calD'_{p F_0/} } \calD_{F_0/}$$
is a trivial fibration of simplicial sets. In particular, we may identify $p$-colimits of $F$ with $p$-colimits of $F_0$.
\end{lemma}

\begin{proof}
Using Lemma \ref{basekann}, we may reduce to the case where $\calC$ is minimal.
Let us call a simplicial subset $\calE \subseteq \calC$ {\it complete} if it has the following property:
for any simplex $\sigma: \Delta^n \rightarrow \calC$, if $\sigma| \Delta^{ \{0, \ldots, i\} }$ factors through $\calC^{0}$ and $\sigma| \Delta^{ \{i+1, \ldots, n \} }$ factors through $\calE$, then $\sigma$
factors through $\calE$. Note that if $\calE$ is complete, then $\calC^{0} \subseteq \calE$.
We next define a transfinite sequence of {\em complete} simplicial subsets of $\calC$
$$ \calC^0 \subseteq \calC^1 \subseteq \ldots $$
as follows: if $\lambda$ is a limit ordinal, we let $\calC^{\lambda} = \bigcup_{ \alpha < \lambda} \calC^{\alpha}$. If $\calC^{\alpha} = \calC$, then we set $\calC^{\alpha+1} = \calC$. Otherwise, choose some simplex $\sigma: \Delta^n \rightarrow \calC$ which does not belong to $\calC^{\alpha}$, where the dimension $n$ of $\sigma$ is chosen as small as possible, and let
$\calC^{\alpha+1}$ be the smallest complete simplicial subset of $\calC$ containing
$\calC^{\alpha}$ and the simplex $\sigma$.

Let $F_{\alpha} = F | \calC^{\alpha}$. We will prove that for every
$\beta \leq \alpha$ the projection
$$ \phi_{\alpha,\beta}: \calD_{F_{\alpha}/} \rightarrow \calD'_{p F_{\alpha}/}
\times_{ \calD'_{ p F_{\beta} / }} \calD_{F_{\beta} / }$$
is a trivial fibration of simplicial sets. Taking $\alpha \gg \beta = 0$, we have $\calC^{\alpha} = \calC$ and the proof will be complete.

Our proof proceeds by induction on $\alpha$. If $\alpha = \beta$, then $\phi_{\alpha,\beta}$ is an isomorphism and there is nothing to prove. If $\alpha > \beta$ is a limit ordinal, then the inductive hypothesis implies that $\phi_{\alpha,\beta}$ is the inverse limit of a transfinite tower of trivial fibrations, and therefore a trivial fibration. 
It therefore suffices to prove that if $\phi_{\alpha,\beta}$ is a trivial fibration, then $\phi_{\alpha+1, \beta}$ is a trivial fibration. We observe that $\phi_{\alpha+1,\beta} = \phi'_{\alpha,\beta} \circ \phi_{\alpha+1,\alpha}$, where $\phi'_{\alpha,\beta}$ is a pullback of $\phi_{\alpha,\beta}$ and
therefore a trivial fibration by the inductive hypothesis. Consequently, it will suffice to prove
that $\phi_{\alpha+1,\alpha}$ is a trivial fibration. The result is obvious if
$\calC^{\alpha+1} = \calC^{\alpha}$, so we may assume without loss of generality that 
$\calC^{\alpha+1}$ is the smallest complete simplicial subset of $\calC$ containing
$\calC^{\alpha}$ together with a simplex $\sigma: \Delta^n \rightarrow \calC$,
where $\sigma$ does not belong to $\calC^{\alpha}$. Since $n$ is chosen to be minimal, we may suppose that $\sigma$ is nondegenerate, and that the boundary of $\sigma$ already belongs to
$\calC^{\alpha}$.

Form
a pushout diagram
$$ \xymatrix{ \calC^0_{/ \sigma} \star \bd \Delta^n \ar[r] \ar[d] & \calC^{\alpha} \ar[d] \\
\calC^{0}_{/\sigma} \star \Delta^n \ar[r] & \calC'. }$$
By construction there is an induced map $\calC' \rightarrow \calC$, which is easily shown to be a monomorphism of simplicial sets; we may therefore identify $\calE'$ with its image in $\calC$. Since $\calC$ is minimal, we can apply Proposition \ref{minstrict} to deduce that $\calC'$ is complete, so that $\calC' = \calC^{\alpha+1}$. Let $G$ denote the composition
$$ \calC^0_{/\sigma} \star \Delta^n \rightarrow \calC \stackrel{F}{\rightarrow} \calD $$
and $G_{\bd} = G | \calC^{0}_{/\sigma} \star \bd \Delta^n$.
It follows that $\phi_{\alpha+1,\alpha}$ is a pullback of the induced map
$$ \psi: \calD_{G/} \rightarrow \calD'_{p G/} \times_{ \calD'_{p G_{\bd}/} }
\calD_{ G_{\bd}/}.$$
To complete the proof, it will suffice to show that $\psi$ is a trivial fibration of simplicial sets.

Let $G_0 = G | \calC^0_{/\sigma}$. Let $\calE = \calD_{G_0/}$, 
$\calE' = \calD'_{p \circ G_0/}$, and let $q: \calE \rightarrow \calE'$ be the induced map.
We can identify $G$ with a map $\sigma': \Delta^n \rightarrow \calE$. Let
$\sigma'_0 = \sigma' | \bd \Delta^n$. Then we wish to prove that
the map
$$ \psi': \calE_{ \sigma'/ } \rightarrow \calE'_{ q \sigma' / }
\times_{ \calE'_{ q \sigma'_0 / } } \calE_{ q \sigma'_0 /}$$
is a trivial fibration. Let $C = \sigma(0)$. 

The projection $\calC^0_{/\sigma} \rightarrow \calC^{0}_{/C}$ is a trivial fibration of simplicial sets, and therefore cofinal. Since $F$ is a $p$-left Kan extension of $F_0$ at $C$, we conclude that $\sigma'(0)$ is a $q$-initial object of $\calE$.

To prove that $\psi$ is a trivial fibration, it will suffice to prove that $\psi$ has the right lifting property with respect to the inclusion $\bd \Delta^m \subseteq \Delta^m$, for each $m \geq 0$. Unwinding the definitions, this amounts to the existence of a dotted arrow as indicated in the diagram
$$ \xymatrix{ \bd \Delta^{n+m+1} \ar[r]^{s} \ar@{^{(}->}[d] & \calE \ar[d]^{q} \\
\Delta^{n+m+1}. \ar@{-->}[ur] \ar[r] & \calE' }$$
However, the map $s$ carries the initial vertex of $\Delta^{n+m+1}$ to a vertex of $\calE$ which
is $q$-initial, so that the desired extension can be found.
\end{proof}

\begin{proposition}\label{acekan}
Let $F: \calC \rightarrow \calD$ be a functor between $\infty$-categories, 
$p: \calD \rightarrow \calD'$ an inner fibration of $\infty$-categories, and
$\calC^{0} \subseteq \calC^{1} \subseteq \calC$ full subcategories. Suppose that
$F| \calC^{1}$ is a $p$-left Kan extension of $F| \calC^{0}$. Then $F$ is a $p$-left Kan extension of $F|\calC^{1}$ if and only if $F$ is a $p$-left Kan extension of $F| \calC^{0}$.
\end{proposition}

\begin{proof}
Let $C$ be an object of $\calC$; we will show that $F$ is a $p$-left Kan extension of $F|\calC^{0}$ at $C$ if and only if $F$ is a $p$-left Kan extension of $F| \calC^{1}$ at $C$. Consider the composition
$$ F^0_{C}: (\calC^{0}_{/C})^{\triangleright} \subseteq
(\calC^{1}_{/C})^{\triangleright} \stackrel{F^1_{C}}{\rightarrow} \calD.$$
We wish to show that $F^0_{C}$ is a $p$-colimit diagram
if and only if $F^1_{C}$ is a $p$-colimit
diagram. According to Lemma \ref{kan0}, it will suffice to show that $F^1_{C} | \calC^{1}_{/C}$ is a left Kan extension of $F^0_{C}$. Let $f: C' \rightarrow C$ be an object of
$\calC^{1}_{/C}$. We wish to show that the composite map
$$ (\calC^{0}_{/f})^{\triangleright} \rightarrow (\calC^{0}_{/C'})^{\triangleright} \stackrel{F^{0}_{C'}}{\rightarrow} \calD$$ is a $p$-colimit diagram. Since the projection
$\calC^{0}_{/f} \rightarrow \calC^{0}_{/C'}$ is cofinal (in fact, a trivial fibration), it will suffice to show that $F^0_{C'}$ is a $p$-colimit diagram (Proposition \ref{relexists}). This follows from our hypothesis that $F| \calC^{1}$ is a $p$-left Kan extension of $F| \calC^{0}$.
\end{proof}

\begin{proposition}\label{stormus}
Let $F: \calC \times \calC' \rightarrow \calD$ be a functor between $\infty$-categories, $p:
\calD \rightarrow \calD'$ an inner fibration of $\infty$-categories, 
and $\calC^{0} \subseteq \calC$ a full subcategory. The following conditions are
equivalent:
\begin{itemize}
\item[$(1)$] The functor $F$ is a $p$-left Kan extension of $F| \calC^{0} \times \calC'$.
\item[$(2)$] For each object $C' \in \calC'$, the induced functor
$F_{C'}: \calC \times \{C'\} \rightarrow \calD$ is a $p$-left Kan extension of
$F_{C'}| \calC^{0} \times \{C'\}$.
\end{itemize}
\end{proposition}

\begin{proof}
It suffices to show that $F$ is a $p$-left Kan extension of $F|\calC^{0} \times \calC'$
at an object $(C,C') \in \calC \times \calC'$ if and only if $F_{C'}$ is a $p$-left Kan extension
of $F_{D} | \calC^{0} \times \{D\}$ at $C$. This follows from the observation that the inclusion
$\calC^{0}_{/C} \times \{ \id_{C'} \} \subseteq \calC^{0}_{/C} \times \calC'_{/C'}$ is cofinal
(because $\id_{C'}$ is a final object of $\calC'_{/C'}$). 
\end{proof}

\begin{lemma}\label{kanhalf}
Let $m \geq 0$, $n \geq 1$ be integers, and let
$$ \xymatrix{ (\bd \Delta^m \times \Delta^n ) \coprod_{ \bd \Delta^m \times \bd \Delta^n}
( \Delta^m \times \bd \Delta^n) \ar@{^{(}->}[d] \ar[rr]^-{f_0} & & X \ar[d]^{p} \\
\Delta^m \times \Delta^n \ar[rr] \ar@{-->}[urr]^{f} & & S }$$
be a diagram of simplicial sets, where $p$ is an inner fibration and
$f_0(0,0)$ is a $p$-initial vertex of $X$.
Then there exists a morphism $f: \Delta^m \times \Delta^n \rightarrow X$ rendering the diagram commutative. 
\end{lemma}

\begin{proof}
Choose a sequence of simplicial sets
$$( \bd \Delta^m \times \Delta^n ) \coprod_{ \bd \Delta^m \times \bd \Delta^n }
( \Delta^m \times \bd \Delta^n) = Y(0) \subseteq \ldots \subseteq Y(k) = \Delta^m \times \Delta^n,$$
where each $Y(i+1)$ is obtained from $Y(i)$ by adjoining a single nondegenerate simplex whose boundary already lies in $Y(i)$. We prove by induction on $i$ that $f_0$ can be
extended to a map $f_i$ such that the diagram 
$$ \xymatrix{ Y(i) \ar@{^{(}->}[d] \ar[r]^{f_i} & X \ar[d]^{p} \\
\Delta^m \times \Delta^n \ar[r] & S }$$
is commutative. Having done so, we can then complete the proof by choosing $i = k$.

If $i = 0$, there is nothing to prove. Let us therefore suppose that $f_i$ has been constructed, and consider the problem of constructing $f_{i+1}$ which extends $f_i$. This is equivalent to the lifting problem
$$ \xymatrix{ \bd \Delta^{r} \ar@{^{(}->}[d] \ar[r]^{\sigma_0} & X \ar[d]^{p} \\
\Delta^r \ar@{-->}[ur]^{\sigma} \ar[r] & S. }$$
It now suffices to observe that
where $r > 0$ and $\sigma_0(0) = f_0(0,0)$ is a $p$-initial vertex of $X$
(since every simplex of $\Delta^m \times \Delta^n$ which violates one of these conditions
already belongs to $Y(0)$ ).
\end{proof}

\begin{lemma}\label{sillytech}
Suppose given a diagram of simplicial sets
$$ \xymatrix{ X \ar[rr]^{p} \ar[dr] & & Y \ar[dl] \\
& S, & }$$
where $p$ is an inner fibration. Let $K$ be a simplicial set, let
$q_{S} \in \bHom_{S}(K \times S, X)$, and let $q'_S = p \circ q_{S}$.
Then the induced map
$$ X^{q_{S}/} \rightarrow Y^{q_{S}/}$$
is an inner fibration $($where the above simplicial sets are defined as in \S \ref{consweet}{}$)$.
\end{lemma}

\begin{proof}
Unwinding the definitions, we see that every lifting problem
$$ \xymatrix{ A \ar@{^{(}->}[d]^{i} \ar[r] & X^{q_{S}/} \ar[d] \\
B \ar[r] \ar@{-->}[ur] & Y^{q_{S}/} }$$
is equivalent to a lifting problem
$$ \xymatrix{ (A \times (K \diamond \Delta^0) ) \coprod_{ A \times K} (B \times K) \ar[r] \ar@{^{(}->}[d]^{i'} & X \ar[d]^{p} \\
B \times (K \diamond \Delta^0) \ar@{-->}[ur] \ar[r] & Y. }$$
We wish to show that this lifting problem has a solution, provided that $i$ is inner anodyne.
Since $p$ is an inner fibration, it will suffice to prove that $i'$ is inner anodyne, which follows from 
Corollary \ref{prodprod2}.
\end{proof}

\begin{lemma}\label{kan1}
Consider a diagram of $\infty$-categories
$$ \calC \rightarrow \calD' \stackrel{p}{\leftarrow} \calD,$$
where $p$ is an inner fibration. Let $\calC^{0} \subseteq \calC$ be a full subcategory.
Suppose given $n > 0$ and a commutative diagram
$$ \xymatrix{ \bd \Delta^n \ar@{^{(}->}[d] \ar[r]^-{f_0} & \bHom_{\calD'}(\calC, \calD) \ar[d] \\
\Delta^n \ar[r]^-{g} \ar@{-->}[ur]^-{f} & \bHom_{\calD'}(\calC^{0},\calD) }$$
with the property that the functor $F: \calC \rightarrow \calD$, determined by
evaluating $f_0$ at the vertex $\{0\} \subseteq \bd \Delta^n$, is a $p$-left Kan extension of
$F| \calC^{0}$. Then there exists a dotted arrow $f$ rendering the diagram commutative.
\end{lemma}

\begin{proof}
The proof uses the same strategy as that of Lemma \ref{kan0}. Using Lemma \ref{basekann} and Proposition \ref{princex}, we may replace $\calC$ by a minimal model and thereby assume that $\calC$ is minimal. As in the proof of Lemma \ref{kan0}, let us call a simplicial subset $\calE \subseteq \calC$ {\it complete} if it has the following property:
for any simplex $\sigma: \Delta^n \rightarrow \calC$, if $\sigma| \Delta^{ \{0, \ldots, i\} }$ factors through $\calC^{0}$ and $\sigma| \Delta^{ \{i+1, \ldots, n \} }$ factors through $\calE$, then $\sigma$
factors through $\calE$. Let $P$ denote the partially ordered set of pairs $(\calE, f_{\calE})$, where
$\calE \subseteq \calC$ is complete and $f_{\calE}$ is a map rendering commutative the diagram
$$ \xymatrix{ \bd \Delta^n \ar@{^{(}->}[d] \ar[r]^-{f_0} & \bHom_{\calD'}(\calC,\calD) \ar[d] \\
\Delta^n \ar@{=}[d] \ar[r]^-{f_{\calE}} & \bHom_{\calD'}(\calE, \calD) \ar[d] \\
\Delta^n \ar[r]^-{g} & \bHom_{\calD'}(\calC^{0},\calD). }$$
We partially order $P$ as follows: $(\calE, f_{\calE}) \leq (\calE', f_{\calE'} )$ if
$\calE \subseteq \calE'$ and $f_{\calE} = f_{\calE'} | \calE$. Using Zorn's lemma, we deduce that $P$ has a maximal element $(\calE, f_{\calE})$. If $\calE = \calC$, we may take $f = f_{\calE}$ and the proof is complete. Otherwise, choose a simplex $\sigma: \Delta^m \rightarrow \calC$ which does not belong to $\calE$, where $m$ is as small as possible. It follows that $\sigma$ is nondegenerate, and that the boundary of $\sigma$ belongs to $\calE$. Form a pushout diagram
$$ \xymatrix{ \calC^{0}_{/\sigma} \star \bd \Delta^m \ar[r] \ar@{^{(}->}[d] & \calE \ar[d] \\
\calC^{0}_{/\sigma} \star \Delta^m \ar[r] & \calE'. }$$
As in the proof of Lemma \ref{kan0}, we may identify $\calE'$ with a complete simplicial subset of $\calC$, which strictly contains $\calE$. Since $(\calE, f_{\calE})$ is maximal, we conclude that
$f_{\calE}$ does not extend to $\calE'$. Consequently, we deduce that there does not exist a dotted arrow rendering the diagram
$$ \xymatrix{ \calC^{0}_{/\sigma} \star \bd \Delta^m \ar@{^{(}->}[d] \ar[r] & \Fun(\Delta^n,\calD) \ar[d] \\
\calC^{0}_{/\sigma} \star \Delta^m \ar[r] \ar@{-->}[ur] & \Fun(\Delta^n, \calD')
\times_{ \Fun(\bd \Delta^n, \calD') } \Fun( \bd \Delta^n, \calD) }$$
commutative. Let 
$q: \calC^{0}_{/\sigma} \rightarrow \Fun( \Delta^n, \calD)$ be the restriction of the upper horizontal map,
and let $q': \calC^{0}_{/\sigma} \rightarrow \Fun(\Delta^n, \calD')$, 
$q_{\bd}: \calC^{0}_{/\sigma} \rightarrow \Fun(\bd \Delta^n, \calD)$,
$q'_{\bd}: \calC^{0}_{/\sigma} \rightarrow \Fun(\bd \Delta^n, \calD')$
be defined by composition with $q$. It follows that there exists no solution
to the associated lifting problem
$$ \xymatrix{ \bd \Delta^m \ar[r] \ar@{^{(}->}[d] & \Fun(\Delta^n,\calD)_{q/} \ar[d] \\
\Delta^m \ar[r] \ar@{-->}[ur] & \Fun(\Delta^n, \calD')_{q'/} \times_{
\Fun( \bd \Delta^n, \calD')_{q'_{\bd}/} } \Fun( \bd \Delta^n,\calD)_{q_{\bd}/}. }$$
Applying Proposition \ref{princex}, we deduce also the insolubility
of the equivalent lifting problem
$$ \xymatrix{ \bd \Delta^m \ar[r] \ar[d] & \Fun(\Delta^n, \calD)^{q/} \ar[d] \\
\Delta^m \ar[r] \ar@{-->}[ur] & \Fun(\Delta^n,\calD')^{q'/} \times_{
\Fun(\bd \Delta^n, \calD')^{q'_{\bd}/} } \Fun(\bd \Delta^n,\calD)^{q_{\bd}/}. }$$

Let $q_{\Delta^n}$ denote the map $\calC^0_{/\sigma} \times \Delta^n \rightarrow
\calD \times \Delta^n$ determined by $q$, and let
and let $\calX = (\calD \times \Delta^n)^{q_{\Delta^n}/}$ be the simplicial set constructed in \S \ref{consweet}. Let $q'_{\Delta^n}: \calC^{0}_{/\sigma} \times \Delta^n \rightarrow
\calD' \times \Delta^n$ and $\calX' = ( \calD' \times \Delta^n)^{q'_{\Delta^n}/}$
be defined similarly. We have natural isomorphisms
$$ \Fun(\Delta^n,\calD)^{q/} \simeq \bHom_{\Delta^n}(\Delta^n, \calX)$$
$$ \Fun(\bd \Delta^n, \calD)^{q_{\bd}/} \simeq \bHom_{\Delta^n}( \bd \Delta^n, \calX).$$
$$ \Fun(\Delta^n,\calD')^{q'/} \simeq \bHom_{\Delta^n}(\Delta^n, \calX')$$
$$ \Fun(\bd \Delta^n, \calD')^{q'_{\bd}/} \simeq \bHom_{\Delta^n}( \bd \Delta^n, \calX').$$
These identifications allow us reformulate our insoluble lifting problem once more:
$$ \xymatrix{ ( \bd \Delta^m \times \Delta^n ) \coprod_{ \bd \Delta^m \times \bd \Delta^n }
( \Delta^m \times \bd \Delta^n) \ar[rr]^-{g_0} \ar@{^{(}->}[d] & & \calX \ar[d]^{\psi} \\
\Delta^m \times \Delta^n \ar@{-->}[urr]^{g} \ar[rr] & & \calX'. }$$
We have a commutative diagram
$$ \xymatrix{ \calX \ar[rr]^{\psi} \ar[dr]^{r} & & \calX' \ar[dl]_{r'} \\
& \Delta^n. & }$$
Proposition \ref{colimfam} implies that $r$ and $r'$ are 
Cartesian fibrations, and that $\psi$ carries $r$-Cartesian edges
to $r'$-Cartesian edges. Lemma \ref{sillytech} implies that $\psi$ is an inner fibration.
Let $\psi_0: \calX_{ \{0\} } \rightarrow \calX'_{ \{0\} }$ be the diagram induced by taking
the fibers over the vertex $\{0\} \subseteq \Delta^n$. We have a commutative diagram
$$ \xymatrix{ \calD_{ \calC^{0}_{/\sigma(0)}/ } \ar[d]^{\theta} & \calD_{ \calC^{0}_{/\sigma} }\ar[l] \ar[d] \ar[r] &
\calX_{ \{0\} } \ar[d]^{\psi_0} \\
\calD'_{ \calC^{0}_{/ \sigma(0)}/} & \calD'_{ \calC^{0}_{/\sigma}/} \ar[r] \ar[l] & \calX'_{ \{0\} }}$$
in which the horizontal arrows are categorical equivalences. The vertex
$g_0(0,0) \in \calX'_{ \{0\} }$ lifts to a vertex of 
$\calD_{ \calC^{0}_{/\sigma}/}$ whose image in $\calD_{ \calC^{0}_{/\sigma(0)}/ }$
is $\theta$-initial (in virtue of our assumption that $F$ is a $p$-left Kan extension
of $F| \calC^{0}$). It follows that $g_0(0,0)$ is $\psi_0$-initial when regarded as a vertex of $\calX_{ \{0\} }$. 
Applying Proposition \ref{panna}, we deduce
that $g_0(0,0)$ is $\psi$-initial when regarded as a vertex of $\calX$. 
Lemma \ref{kanhalf} now guarantees the existence of the dotted arrow $g$, contradicting the maximality of $(\calE, f_{\calE})$.
\end{proof}

The following result addresses the existence problem for left Kan extensions:

\begin{lemma}\label{kan2}\label{Kan extension!existence of}
Suppose given a diagram of $\infty$-categories
$$ \xymatrix{ \calC^{0} \ar@{^{(}->}[d] \ar[r]^{F_0} & \calD \ar[d]^{p} \\
\calC \ar[r] \ar@{-->}[ur]^{F} & \calD' }$$
where $p$ is an inner fibration, and the left vertical arrow is the inclusion
of a full subcategory. The following conditions are equivalent:

\begin{itemize}
\item[$(1)$] There exists a functor $F: \calC \rightarrow \calD$ rendering the diagram
commutative, such that $F$ is a $p$-left Kan extension of $F_0$. 

\item[$(2)$] For every object $C \in \calC$, the diagram given by the composition
$$ \calC^{0}_{/C} \rightarrow \calC^{0} \stackrel{F_0}{\rightarrow} \calD$$ admits a $p$-colimit.
\end{itemize}

\end{lemma}

\begin{proof}
It is clear that $(1)$ implies $(2)$. Let us therefore suppose that $(2)$ is satisfied; we wish to prove that $F_0$ admits a left Kan extension. We will follow the basic strategy used in the proofs
of Lemmas \ref{kan0} and \ref{kan1}. 
Using Proposition \ref{princex} and Lemma \ref{basekann}, we can replace the inclusion $\calC^{0} \subseteq \calC$ by any categorically equivalent inclusion
${\calC'}^{0} \subseteq \calC'$. Using Proposition \ref{minimod}, we can choose
$\calC'$ to be a minimal model for $\calC$; we thereby reduce to the case where
$\calC$ is itself a minimal $\infty$-category.

We will say that a simplicial subset $\calE \subseteq \calC$ is {\it complete} if it has the following property:
for any simplex $\sigma: \Delta^n \rightarrow \calC$, if $\sigma| \Delta^{ \{0, \ldots, i\} }$ factors through $\calC^{0}$ and $\sigma| \Delta^{ \{i+1, \ldots, n \} }$ factors through $\calE$, then $\sigma$
factors through $\calE$. Note that if $\calE$ is complete, then $\calC^{0} \subseteq \calE$.
Let $P$ be the set of all pairs $(\calE, f_{\calE})$ where $\calE \subseteq \calC$ is complete, $f_{\calE}$ is a map of simplicial
sets which fits into a commutative diagram
$$ \xymatrix{ \calC^{0} \ar@{^{(}->}[d] \ar[r]^{F_0} & \calD \ar@{=}[d] \\
\calE \ar[r] \ar@{^{(}->}[d] \ar[r]^{f_{\calE}} & \calD \ar[d]^{p} \\
\calC \ar[r] & \calD', }$$
and every object $C \in \calE$, the composite map
$$ (\calC^{0}_{/C})^{\triangleright} \subseteq (\calE_{/C})^{\triangleright} \rightarrow \calE \stackrel{f_{\calE}}{\rightarrow} \calD$$
is a $p$-colimit diagram. We view $P$ as a partially ordered set, with
$(\calE, f_{\calE}) \leq (\calE', f_{\calE'})$ if $\calE \subseteq \calE'$ and
$f_{\calE'}|\calE = f_{\calE}$. This partially ordered set satisfies the hypotheses of Zorn's lemma,
and therefore has a maximal element which we will denote by $(\calE, f_{\calE})$. If
$\calE = \calC$, then $f_{\calE}$ is a $p$-left Kan extension of $F_0$ and the proof is complete.

Suppose that $\calE \neq \calC$. Then there is a simplex $\sigma: \Delta^n \rightarrow \calC$
which does not factor through $\calE$; choose such a simplex where $n$ is as small as possible.
The minimality of $n$ guarantees that $\sigma$ is nondegenerate, that $\sigma | \bd \Delta^n$ factors through $\calE$, and (if $n > 0$) that $\sigma(0) \notin \calC^{0}$. Form
a pushout diagram
$$ \xymatrix{ \calC^0_{/ \sigma} \star \bd \Delta^n \ar[r] \ar@{^{(}->}[d] & \calE \ar[d] \\
\calC^{0}_{/\sigma} \star \Delta^n \ar[r] & \calE'. }$$
This diagram induces a map $\calE' \rightarrow \calC$, which is easily shown to be a monomorphism of simplicial sets; we may therefore identify $\calE'$ with its image in $\calC$.
Since $\calC$ is minimal, we can apply Proposition \ref{minstrict} to deduce that $\calE' \subseteq \calC$ is complete. Since $(\calE, F_{\calE}) \in P$ is maximal, it follows that we cannot extend
$F_{\calE}$ to a functor $F_{\calE'}: \calE' \rightarrow \calD$ such that $(\calE', F_{\calE'}) \in P$.

Let $q$ denote the composition
$$ \calC^{0}_{/\sigma} \rightarrow \calC^{0} \stackrel{F_0}{\rightarrow} \calD.$$
The map $f_{\calE}$ determines a commutative diagram
$$ \xymatrix{ \bd \Delta^n \ar@{^{(}->}[d] \ar[r]^{g_0} & \calD_{q/} \ar[d]^{p'} \\
\Delta^n \ar[r] \ar@{-->}[ur]^{g} & \calD'_{p  q/}. }$$
Extending $f_{\calE}$ to a map $f_{\calE'}$ such that $(\calE', f_{\calE'}) \in P$ is
equivalent to producing a morphism $g: \Delta^n \rightarrow \calD_{q/}$ rendering the above
diagram commutative which, if $n=0$, is a $p$-colimit of $q$. In the case
$n=0$, the existence of such an extension follows from assumption $(2)$.
If $n > 0$, let $C = \sigma(0)$; then the projection $\calC^{0}_{/\sigma} \rightarrow
\calC^{0}_{/C}$ is a trivial fibration $\infty$-categories and $q$ factors as a composition
$$ \calC^{0}_{/\sigma} \rightarrow \calC^{0}_{/C} \stackrel{q'}{\rightarrow} \calD.$$ 
We obtain therefore a commutative diagram
$$ \xymatrix{ \calD_{q/} \ar[r]^{r} \ar[d]^{p'} & \calD_{q'/} \ar[d]^{p''} \\
\calD'_{p  q/} \ar[r] & \calD'_{p  q'/} }$$
where the horizontal arrows are categorical equivalences.
Since $(\calE, f_{\calE}) \in P$, $(r \circ g_0)(0)$ is a $p''$-initial vertex of
$\calD_{q'/}$. Applying Proposition \ref{summertoy}, we conclude that $g_0(0)$ is a $p'$-initial
vertex of $\calD_{q/}$, which guarantees the existence of the desired extension $g$.
This contradicts the maximality of $(\calE, f_{\calE})$ and completes the proof.
\end{proof}

\begin{corollary}\label{kanexistleft}
Let $p: \calD \rightarrow \calE$ be a coCartesian fibration of $\infty$-categories. Suppose
that each fiber of $p$ admits small colimits, and that for every morphism
$E \rightarrow E'$ in $\calE$, the associated functor $\calD_{E} \rightarrow \calD_{E'}$
preserves small colimits. Let $\calC$ be a small $\infty$-category, and 
$\calC^{0} \subseteq \calC$ a full subcategory. Then every functor
$F_0: \calC^{0} \rightarrow \calD$ admits a left Kan extension relative
to $p$.
\end{corollary}

\begin{proof}
This follows immediately from Lemma \ref{kan2} and Corollary \ref{constrel}.
\end{proof}

Combining Lemmas \ref{kan1} and \ref{kan2}, we deduce:

\begin{proposition}\label{lklk}
Suppose given a diagram of $\infty$-categories
$$ \calC \rightarrow \calD' \stackrel{p}{\leftarrow} \calD,$$
where $p$ is an inner fibration. Let $\calC^{0}$ be a full subcategory of $\calC$.
Let $\calK \subseteq \bHom_{\calD'}(\calC, \calD)$ be the full subcategory spanned by those functors $F: \calC \rightarrow \calD$ which are $p$-left Kan extensions of $F | \calC^{0}$.
Let $\calK' \subseteq \bHom_{\calD'}(\calC^{0}, \calD)$ be the full subcategory spanned
by those functors
$F_0: \calC^{0} \rightarrow \calD$ with the property that, for each object
$C \in \calC$, the induced diagram
$ \calC^{0}_{/C} \rightarrow \calD$ has a $p$-colimit. Then the restriction functor
$\calK \rightarrow \calK'$ is a trivial fibration of simplicial sets.
\end{proposition}

\begin{corollary}\label{leftkanextdef}
Suppose given a diagram of $\infty$-categories
$$ \calC \rightarrow \calD' \stackrel{p}{\leftarrow} \calD,$$
where $p$ is an inner fibration. Let $\calC^{0}$ be a full subcategory of $\calC$.
Suppose further that, for every functor $F_0 \in \bHom_{\calD'}(\calC^{0}, \calD)$,
there exists a functor $F \in \bHom_{\calD'}(\calC,\calD)$ which is a $p$-left Kan extension
of $F_0$.
Then the restriction map $i^{\ast}: \bHom_{\calD'}(\calC,\calD) \rightarrow 
\bHom_{\calD'}(\calC^{0},\calD)$ admits a section
$i_{!}$, whose essential image consists of precisely of those functors $F$
which are $p$-left Kan extensions of $F | \calC^{0}$.
\end{corollary}

In the situation of Corollary \ref{leftkanextdef}, we will refer to $i_{!}$ as a {\it left Kan extension functor}. We note that Proposition \ref{lklk} proves not only the existence of $i_{!}$, but also its uniqueness up to homotopy (the collection of all such functors is parametrized by a contractible Kan complex). The following characterization of $i_{!}$ gives a second explanation for its uniqueness:

\begin{proposition}\label{leftkanadj}
Suppose given a diagram of $\infty$-categories
$$ \calC \rightarrow \calD' \stackrel{p}{\leftarrow} \calD,$$
where $p$ is an inner fibration. Let $i: \calC^{0} \subseteq \calC$ be the inclusion of a full subcategory,
and suppose that every functor $F_0 \in \bHom_{\calD'}(\calC^{0},\calD)$ admits a $p$-left Kan extension. Then the left Kan extension functor
$i_{!}: \bHom_{ \calD'}( \calC^{0}, \calD) \rightarrow \bHom_{\calD'}(\calC, \calD)$
is a left adjoint to the restriction functor $i^{\ast}: \bHom_{\calD'}(\calC, \calD) \rightarrow
\bHom_{\calD'}(\calC^{0}, \calD)$.
\end{proposition}

\begin{proof}
Since $i^{\ast} \circ i_{!}$ is the identity functor on $\bHom_{\calD'}(\calC^{0}, \calD)$, there is an obvious candidate for the unit
$$ u: \id \rightarrow i^{\ast} \circ i_{!}$$
of the adjunction: namely, the identity. According to Proposition \ref{storut}, it will suffice to prove
that for every $F \in \bHom_{\calD'}(\calC^{0}, \calD)$,
$G \in \bHom_{\calD'}(\calC, \calD)$, composition with $u$ induces a homotopy equivalence
$$ \bHom_{\bHom_{\calD'}(\calC,\calD)
}(i_{!} F, G) \rightarrow \bHom_{\bHom_{\calD'}(\calC^{0},\calD)}( i^{\ast} i_{!} F,
i^{\ast} G) \stackrel{u}{\rightarrow}\bHom_{\bHom_{\calD'}(\calC^{0},\calD)}( F,
i^{\ast} G)$$
in the homotopy category $\calH$. This morphism in $\calH$ is represented by the restriction map 
$$ \Hom^{\rght}_{ \bHom_{\calD'}(\calC,\calD)}( i_{!}F,G) \rightarrow \Hom^{\rght}_{\bHom_{\calD'}(\calC^{0},\calD)}(F, i^{\ast} G)$$
which is a trivial fibration by Lemma \ref{kan1}.
\end{proof}

\begin{remark}
Throughout this section we have focused our attention on the theory of (relative) {\em left} Kan extensions. There is an entirely dual theory of {\em right} Kan extensions in the $\infty$-categorical setting, which can be obtained from the theory of left Kan extensions by passing to opposite $\infty$-categories.
\end{remark}

\subsection{Kan Extensions along General Functors}\label{bigkanext}

Our goal in this section is to generalize the theory of Kan extensions to the case where the
change of diagram category is not necessarily given by a fully faithful inclusion 
$\calC^{0} \subseteq \calC$. As in \S \ref{kanex}, we will discuss only the theory of {\em left} Kan extensions; a dual theory of right Kan extensions can be obtained by passing to opposite $\infty$-categories.

The ideas introduced in this section are relatively elementary extensions of the ideas of \S \ref{kanex}. However, we will encounter a new complication.
Let $\delta: \calC \rightarrow \calC'$ be a change of diagram $\infty$-category,
$f: \calC \rightarrow \calD$ a functor, and $\delta_{!}(f): \calC' \rightarrow \calD$
its left Kan extension along $\delta$ (to be defined below). Then one does not generally
expect that $\delta^{\ast} \delta_{!}(f)$ to be equivalent to the original functor $f$. Instead, one
has only a unit transformation $f \rightarrow \delta^{\ast} \delta_{!}(f)$. To set up the theory, this unit transformation must be taken as part of the data. Consequently, the theory of Kan extensions in general requires more elaborate notation and terminology than the special case treated in \S \ref{kanex}. We will compensate for this by considering only the case of {\em absolute} left Kan extensions. It is straightforward to set up a relative theory as in \S \ref{kanex}, but we will not need such a theory in this book.

\begin{definition}\index{gen}{left extension}
Let $\delta: K \rightarrow K'$ be a map of simplicial sets, let $\calD$ be an $\infty$-category, and let
$f: K \rightarrow \calD$ be a diagram. A {\it left extension of $f$ along $\delta$} consists of a map
$f': K' \rightarrow \calD$ and a morphism $f \rightarrow f' \circ \delta$ in the $\infty$-category
$\Fun(K,\calD)$. 
\end{definition}

Equivalently, we may view a left extension of $f: K \rightarrow \calD$ along $\delta: K \rightarrow K'$
as a map $F: M^{op}(\delta) \rightarrow \calD$ such that $F|K=f$, where $M^{op}(\delta) = M(\delta^{op})^{op} = (K \times \Delta^1) \coprod_{ K \times \{1\} } K'$ denotes the mapping cylinder of $\delta$.

\begin{definition}\label{genkan}\index{gen}{Kan extension}
Let $\delta: K \rightarrow K'$ be a map of simplicial sets, and let
$F: M^{op}(\delta) \rightarrow \calD$ be a diagram in an $\infty$-category $\calD$
(which we view as a left extension of $f = F|K$ along $\delta$). We will say that $F$ is a {\it left
Kan extension of $f$ along $\delta$} if there exists a commutative diagram
$$ \xymatrix{ M^{op}(\delta) \ar[r]^{F''} \ar[dr] & \calK \ar[r]^{F'} \ar[d]^{p} & \calD \\
& \Delta^1 & }$$
where $F''$ is a categorical equivalence, $\calK$ is an $\infty$-category, 
$F= F' \circ F''$, and $F'$ is a left Kan extension of $F' | \calK \times_{ \Delta^1} \{0\}$. 
\end{definition}

\begin{remark}\label{genka}
In the situation of Definition \ref{genkan}, the map $p: \calK \rightarrow \Delta^1$ is {\em automatically} a coCartesian fibration. To prove this, choose a factorization
$$ M(\delta^{op})^{\natural} \stackrel{i}{\rightarrow} (\calK')^{\sharp} \rightarrow (\Delta^1)^{\sharp}$$
where $i$ is marked anodyne, and $\calK' \rightarrow \Delta^1$ is a Cartesian fibration. Then $i$ is a quasi-equivalence, so that Proposition \ref{simplexplay} implies that $M(\delta^{op}) \rightarrow \calK'$ is a categorical equivalence. It follows that $\calK$ is equivalent to $(\calK')^{op}$ (via an equivalence which respects the projection to $\Delta^1$), so that the projection
$p$ is a coCartesian fibration.
\end{remark}

The following result asserts that the condition of Definition \ref{genkan} is essentially independent of the choice of $\calK$.

\begin{proposition}\label{oave}
Let $\delta: K \rightarrow K'$ be a map of simplicial sets, and let
$F: M^{op}(\delta) \rightarrow \calD$ be a diagram in an $\infty$-category $\calD$ which
is a left Kan extension along $\delta$. Let
$$ \xymatrix{ M^{op} \ar[r]^{F''} \ar[dr] & \calK \ar[d]^{p} \\
& \Delta^1 }$$
be a diagram where $F''$ is both a cofibration and a categorical equivalence of simplicial sets. Then $F= F' \circ F''$, for some map $F': \calK \rightarrow \calD$
which is a left Kan extension of $F' | \calK \times_{\Delta^1} \{0\}$.
\end{proposition}

\begin{proof}
By hypothesis, there exists a commutative diagram
$$ \xymatrix{ M^{op}(\delta) \ar[r]^{G''} \ar[d]^{F''} & \calK' \ar[r]^{G'} \ar[d]^{q} & \calD \\
\calK \ar[r]^{p} \ar@{-->}[ur]^{r} & \Delta^1 & }$$
where $\calK'$ is an $\infty$-category, $F = G' \circ G''$, and $G''$ is a categorical equivalence, and
$G'$ is a left Kan extension of $G' | \calK' \times_{\Delta^1} \{0\}$. Since $\calK'$ is an
$\infty$-category, there exists a map $r$ as indicated in the diagram such that
$G'' = r \circ F''$. We note that $r$ is a categorical equivalence so that the commutativity
of the lower triangle $p = q \circ r$ follows automatically. We now define
$F' = G' \circ r$, and note that part $(2)$ of Lemma \ref{basekann} implies that $F'$ 
is a left Kan extension of $F'| \calK \times_{\Delta^1} \{0\}$.
\end{proof}

We have now introduced two different definitions of left Kan extensions: Definition \ref{defKan}, which applies in the situation of an inclusion $\calC^{0} \subseteq \calC$ of a full subcategory into an $\infty$-category $\calC$, and Definition \ref{genkan} which applies in the case of a general map $\delta: K \rightarrow K'$ of simplicial sets. These two definitions are essentially the same. More precisely, we have the following assertion:

\begin{proposition}\label{compkan}
Let $\calC$ and $\calD$ be $\infty$-categories, and let $\delta: \calC^{0} \rightarrow \calC$ denote the inclusion of a full subcategory.

\begin{itemize}
\item[$(1)$] Let $f: \calC \rightarrow \calD$ be a functor, $f_0$ its restriction to $\calC^{0}$, so that $( f, \id_{f_0})$ can be viewed as a left extension of $f_0$ along $\delta$.
Then $(f, \id_{f_0})$ is a left Kan extension of $f_0$ along $\delta$ if and only if $f$ is a left Kan extension of $f_0$.  

\item[$(2)$] A functor $f_0: \calC^{0} \rightarrow \calD$ has a left Kan extension
if and only if it has a left Kan extension along $\delta$.
\end{itemize}
\end{proposition}

\begin{proof}
Let $\calK$ denote the full subcategory of $\calC \times \Delta^{1}$ spanned by the objects
$(C, \{i\})$ where either $C \in \calC^{0}$ or $i=1$, so that we have inclusions
$$ M^{op}(\delta) \subseteq \calK \subseteq \calC \times \Delta^{1}.$$
To prove $(1)$, suppose that $f: \calC \rightarrow \calD$ is a left Kan extension of 
$f_0 = f | \calC^{0}$ and let $F$ denote the composite map
$$ \calK \subseteq \calC \times \Delta^{1} \rightarrow \calC \stackrel{f}{\rightarrow} \calD.$$
It follows immediately that $F$ is a left Kan extension of $F | \calC^{0} \times \{0\}$, so that
$F| M^{op}(\delta)$ is a left Kan extension of $f_0$ along $\delta$.

To prove $(2)$, we observe that the ``only if'' follows from $(1)$; the converse follows from the existence criterion of Lemma \ref{kan2}.
\end{proof}

Suppose that $\delta: K^{0} \rightarrow K^{1}$ is a map of simplicial sets, $\calD$ an $\infty$-category, and that every diagram $K^{0} \rightarrow \calD$ admits a left Kan extension along $\delta$. Choose a diagram
$$\xymatrix{ M^{op}(\delta) \ar[rr]^{j} \ar[dr] & & \calK \ar[dl] \\
& \Delta^1 & }$$
where $j$ is inner anodyne and $\calK$ is an $\infty$-category, which we regard
as a correspondence from $\calK^{0} = \calK \times_{\Delta^1} \{0\}$ to 
$\calK^{1} = \calK \times_{ \Delta^1} \{1\}$. Let $\calC$ denote the full subcategory
of $\Fun(\calK, \calD)$ spanned by those functors $F: \calK \rightarrow \calD$ such that
$F$ is a left Kan extension of $F_0 = F|\calK^{0}$. The restriction map
$p: \calC \rightarrow \Fun(K^0, \calD)$ can be written as a composition of
$\calC \rightarrow \calD^{\calK^0}$ (a trivial fibration by Proposition \ref{lklk}) and
$\Fun(\calK^{0},\calD) \rightarrow \Fun(K^0,\calD)$ (a trivial fibration since $K^0 \rightarrow \calK^0$ is inner anodyne), and is therefore a trivial fibration. Let $\overline{\delta}_{!}$ be the composition of
a section of $p$ with the restriction map $\calC \subseteq \Fun(\calK, \calD) \rightarrow \Fun( M^{op}(\delta), \calD)$, and let $\delta_{!}$ denote the composition of 
$\overline{\delta}_{!}$ with the restriction map $\Fun( M^{op}(\delta),\calD) \rightarrow 
\Fun(K^1,\calD)$. Then $\overline{\delta}_{!}$ and $\delta_{!}$ are well-defined up to equivalence, at least once $\calK$ has been fixed (independence of the choice of $\calK$ will follow from the characterization given in Proposition \ref{charleftkangen}). We will abuse terminology by referring to {\em both} $\overline{\delta}_{!}$ and $\delta_{!}$ as {\it left Kan extension along $\delta$} (it should be clear from context which of these functors is meant in a given situation). We observe that $\overline{\delta}_{!}$ assigns to each object $f_0: K^0 \rightarrow \calD$ a 
left Kan extension of $f_0$ along $\delta$. 

\begin{example}
Let $\calC$ and $\calD$ be $\infty$-categories, and let
$i: \calC^{0} \rightarrow \calC$ be the inclusion of a full subcategory. Suppose that
$i_{!}: \Fun(\calC^{0}, \calD) \rightarrow \Fun(\calC, \calD)$ is a section of $i^{\ast}$, which satisfies the conclusion of Corollary \ref{leftkanextdef}. Then $i_{!}$ is a left Kan extension along $i$ in the sense defined above; this follows easily from Proposition \ref{compkan}. 
\end{example}

Left Kan extension functors admit the following characterization:

\begin{proposition}\label{charleftkangen}
Let $\delta: K^0 \rightarrow K^1$ be a map of simplicial sets, let $\calD$ be an $\infty$-category, let $\delta^{\ast}: \Fun(K^1,\calD) \rightarrow \Fun(K^0,\calD)$ be the restriction functor, and let
$\delta_{!}: \Fun(K^0,\calD) \rightarrow \Fun(K^1,\calD)$ be a functor of left Kan extension along $\delta$. Then $\delta_{!}$ is a left adjoint of $\delta^{\ast}$.
\end{proposition}

\begin{proof}
The map $\delta$ can be factored as a composition
$$ K^0 \stackrel{i}{\rightarrow} M^{op}(\delta) \stackrel{r}{\rightarrow} K^1$$ where
$r$ denotes the natural retraction of $M^{op}(\delta)$ onto $K^1$. Consequently, $\delta^{\ast} = i^{\ast} \circ r^{\ast}$. Proposition \ref{leftkanadj} implies that the left Kan extension functor $\overline{\delta}_{!}$ is
a left adjoint to $i^{\ast}$. By Proposition \ref{compadjoint}, it will suffice to prove that $r^{\ast}$ is a right adjoint to the restriction functor $j^{\ast}: \Fun(M^{op}(\delta),\calD) \rightarrow \Fun(K^1,\calD)$. 
Using Corollary \ref{tweezegork}, we deduce that $j^{\ast}$ is a coCartesian fibration. Moreover, there is a simplicial homotopy 
$\Fun(M^{op}(\delta), \calD) \times \Delta^1 \rightarrow \Fun(M^{op}(\delta), \calD)$ from the identity to $r^{\ast} \circ j^{\ast}$, which is a fiberwise homotopy over $\Fun(K^1, \calD)$. 
It follows that for every object $F$ of $\Fun(K^1,\calD)$, $r^{\ast} F$ is a final object of the $\infty$-category $\Fun(M^{op}(\delta), \calD) \times_{ \Fun(K^1,\calD)} \{F\}$. Applying Proposition \ref{quuquu}, we deduce that $r^{\ast}$ is right adjoint to $j^{\ast}$ as desired.
\end{proof}

Let $\delta: K^{0} \rightarrow K^{1}$ be a map of simplicial sets and $\calD$ an $\infty$-category which which that left Kan extension $\delta_{!}: \Fun(K^0,\calD) \rightarrow \delta_{!} \Fun(K^1,\calD)$ is defined. In general, the terminology ``Kan extension'' is perhaps somewhat unfortunate: if
$F: K^0 \rightarrow \calD$ is a diagram, then $\delta^{\ast} \delta_{!} F$ need not coincide with $F$, even up to equivalence. If $\delta$ is fully faithful, then the unit map
$F \rightarrow \delta^{\ast} \delta_{!} F$ is an equivalence: this follows from
Proposition \ref{compkan}. We will later need the following more precise assertion:

\begin{proposition}\label{timeless}
Let $\delta: \calC^{0} \rightarrow \calC^{1}$ and $f_0: \calC^{0} \rightarrow \calD$ be functors between $\infty$-categories, and let $f_{1}: \calC^{1} \rightarrow \calD$, $\alpha: f_{0} \rightarrow \delta^{\ast} f_1 = f_1 \circ \delta$ be a left Kan extension of $f_0$ along $\delta$.
Let $C$ be an object of $\calC^{0}$ such that, for every $C' \in \calC^{0}$, the functor
$\delta$ induces an isomorphism
$$ \bHom_{\calC^{0}}(C',C) \rightarrow \bHom_{\calC^{1}}( \delta C', \delta C)$$
in the homotopy category $\calH$. Then the morphism
$\alpha(C): f_0(C) \rightarrow f_1(\delta C)$ is an equivalence in $\calD$.
\end{proposition}

\begin{proof}
Choose a diagram
$$ \xymatrix{ M^{op}(\delta) \ar[r]^{G} \ar[dr] & \calM \ar[d] \ar[r]^{F} & \calD \\
& \Delta^1 & }$$
where $\calM$ is a correspondence from $\calC^{0}$ to $\calC^{1}$ associated to $\delta$, $F$ is a left Kan extension of $f_0 = F| \calC^{0}$, and $F \circ G$ is the map
$M^{op}(\delta) \rightarrow \calD$ determined by $f_0$, $f_1$, and $\alpha$.
Let $u: C \rightarrow \delta C$ be the morphism in $\calM$ given by the image
of $\{C\} \times \Delta^1 \subseteq M^{op}(\delta)$ under $G$. Then
$\alpha(C) = F(u)$, so it will suffice to prove that $F(u)$ is an equivalence. Since
$F$ is a left Kan extension of $f_0$ at $\delta C$, the composition
$$ (\calC^{0}_{/\delta C})^{\triangleright} \rightarrow
\calM \stackrel{F}{\rightarrow} \calD$$
is a colimit diagram. Consequently, it will suffice to prove that
$u: C \rightarrow \delta C$ is a final object of
$\calC^{0}_{/ \delta C}$. Consider the diagram
$$ \calC^{0}_{/C} \leftarrow \calC^{0}_{/u} \stackrel{q}{\rightarrow} \calC^{0}_{/\delta C}.$$
The $\infty$-category on the left has a final object $\id_{C}$, and the map on the left
is a trivial fibration of simplicial sets. We deduce that $s^0 u$ is a final object of\
$\calC^{0}_{/u}$. Since $q( s^0 u) = u \in \calC^{0}_{/\delta C}$, it will suffice to show that
$q$ is an equivalence of $\infty$-categories. We observe that $q$ is a map of right fibrations over $\calC^{0}$. According to Proposition \ref{apple1}, it will suffice to show that for each object $C'$ in $\calC^{0}$, the map $q$ induces a homotopy equivalence of Kan complexes
$$ \calC^{0}_{/u} \times_{ \calC^{0}} \{C'\} \rightarrow \calC^{0}_{/ \delta C} \times_{\calC^{0}} \{C'\}.$$
This map can be identified with the map
$$ \bHom_{\calC^0}( C',C) \rightarrow \bHom_{\calM}(C', \delta C)
\simeq \bHom_{\calC^1}(\delta C', \delta C),$$
in the homotopy category $\calH$, and is therefore a homotopy equivalence by assumption.
\end{proof}

We conclude this section by proving that the construction of left Kan extensions behaves well in families.

\begin{lemma}\label{longerwait}
Suppose given a commutative diagram
$$ \xymatrix{ \calC^{0} \ar[r]^{q} \ar[dr]^{i} & \calC \ar[d]^{p} \ar[r]^{F} & \calD \\
& \calE & } $$
of $\infty$-categories, where $p$ and $q$ are coCartesian fibrations, 
$i$ is the inclusion of a full subcategory, and $i$ carries $q$-coCartesian morphisms
of $\calC^{0}$ to $p$-coCartesian morphisms of $\calC$. The following conditions are equivalent:
\begin{itemize}
\item[$(1)$] The functor $F$ is a left Kan extension of $F| \calC^{0}$. 
\item[$(2)$] For each object $E \in \calE$, the induced functor
$F_{E}: \calC_{E} \rightarrow \calD$ is a left Kan extension of $F_E | \calC^{0}_{E}$. 
\end{itemize}
\end{lemma}

\begin{proof}
Let $C$ be an object of $\calC$ and let $E = p(C)$. Consider the composition
$$ (\calC^0_{E})^{\triangleright}_{/C} \stackrel{G^{\triangleright}}{\rightarrow} (\calC^{0}_{/C})^{\triangleright} \stackrel{F_{C}}{\rightarrow} \calD.$$
We will show that $F_C$ is a colimit diagram if and only if $F_C \circ G^{\triangleright}$ is a colimit diagram. For this, it suffices to show that the inclusion $G: (\calC^{0}_{E})_{/C} \subseteq
\calC^{0}_{/C}$ is cofinal. According to Proposition \ref{verylonger}, the projection $p': \calC_{/C} \rightarrow \calE_{/E}$
is a coCartesian fibration, and a morphism
$$ \xymatrix{ C' \ar[rr]^{f} \ar[dr] & & C'' \ar[dl] \\
& C & }$$
in $\calC_{/C}$ is $p'$-coCartesian if and only if $f$ is $p$-coCartesian. It follows that
$p'$ restricts to a coCartesian fibration $\calC'_{/C} \rightarrow \calE_{/E}$. 
We have a pullback diagram of simplicial sets
$$ \xymatrix{ (\calC^{0}_{E})_{/C} \ar[r]^{G} \ar[d] & \calC^0_{/C} \ar[d] \\
\{ \id_{E} \} \ar[r]^{G_0} & \calE_{/E}. }$$
The right vertical map is smooth (Proposition \ref{strokhop}) and $G_0$ is right anodyne, so that $G$ is right anodyne as desired.
\end{proof}

\begin{proposition}\label{longwait2}\label{Kan extension!in families}
Let
$$ \xymatrix{ X \ar[dr]^{p} \ar[rr]^{\delta} & & Y \ar[dl]^{q} \\
& S & }$$
be a commutative diagram of simplicial sets, where $p$ and $q$ are coCartesian fibrations,
and $\delta$ carries $p$-coCartesian edges to $q$-coCartesian edges.
Let $f_0: X \rightarrow \calC$ be a diagram in an $\infty$-category $\calC$, and let
$f_1: Y \rightarrow \calC$, $\alpha: f_0 \rightarrow f_1 \circ \delta$ be a left extension of 
$f_0$. The following conditions are equivalent:
\begin{itemize}
\item[$(1)$] The transformation $\alpha$ exhibits $f_1$ as a left Kan extension of $f_0$ along
$\delta$.
\item[$(2)$] For each vertex $s \in S$, the restriction $\alpha_{s}: f_0|X_{s} \rightarrow
(f_1 \circ \delta)|X_s$ exhibits $f_1|Y_{s}$ as a left Kan extension of $f_0|X_{s}$ along
$\delta_{s}: X_{s} \rightarrow Y_{s}$.
\end{itemize}
\end{proposition}

\begin{proof}
Choose an equivalence of simplicial categories $\sCoNerve(S) \rightarrow \calE$, where $\calE$ is fibrant, and let $[1]$ denote the linearly ordered set $\{0,1\}$, regarded as a category. Let
$\phi'$ denote the induced map $\sCoNerve( S \times \Delta^1) \rightarrow \calE \times [1]$.
Let $M$ denote the marked simplicial set
$$((X^{op})^{\natural} \times (\Delta^1)^{\sharp}) \coprod_{ (X^{op})^{\natural} \times \{0\} }
( Y^{op})^{\natural}.$$
Let $\St_{\phi}^{+}: (\mSet)_{(S \times \Delta^1)^{op}} \rightarrow (\mSet)^{\calE \times [1]}$ denote the straightening functor defined in \S \ref{markmodel2}, and choose a fibrant replacement
$$ \St_{\phi}^{+} M \rightarrow Z$$
in $(\mSet)^{\calE \times [1]}$. 
Let $S' = \sNerve(\calE)$, so that $S' \times \Delta^1 \simeq
\sNerve(\calE \times [1])$, and let $\psi: \sCoNerve(S' \times \Delta^1) \rightarrow \calE \times [1]$
be the counit map. Then $$\Un^{+}_{\psi}(Z)$$
is a fibrant object of $(\mSet)_{/ (S' \times \Delta^1)^{op}}$, which we may identify with a coCartesian fibration of simplicial sets $\calM \rightarrow S' \times \Delta^1$. 

We may regard $\calM$ as a correspondence from $\calM^{0} = \calM \times_{\Delta^1} \{0\}$ to
$\calM^{1} = \calM \times_{\Delta^1} \{1\}$. By construction, we have a unit map
$$ u: M^{op}(\delta) \rightarrow \calM \times_{S'} S.$$
Theorem \ref{straightthm} implies that the induced maps
$u_0: X \rightarrow \calM^{0} \times_{S'} S$, $u_1: Y \rightarrow \calM^{1} \times_{S'} S$ are equivalences of coCartesian fibrations. Proposition \ref{basechangefunky} implies that the maps
$\calM^{0} \times_{S'} S \rightarrow \calM^{0}$, $\calM^{1} \times_{S'} S \rightarrow \calM^{1}$ are categorical equivalences. 

Let $u'$ denote the composition
$$ M^{op}(\delta) \stackrel{u}{\rightarrow} \calM \times_{S'} S \rightarrow \calM,$$
and let $u'_0: X \rightarrow \calM^{0}$, $u'_1: Y \rightarrow \calM^{1}$ be defined similarly. The above argument shows that $u'_0$ and $u'_1$ are categorical equivalences. 
Consequently, the map $u'$ is a quasi-equivalence of coCartesian fibrations over $\Delta^1$, and therefore a categorical equivalence (Proposition \ref{simplexplay}). Replacing $\calM$ by the product
$\calM \times K$ if necessary, where $K$ is a contractible Kan complex, we may suppose that $u'$ is a cofibration of simplicial sets. Since $\calD$ is an $\infty$-category, there exists a functor $F: \calM \rightarrow \calD$ as indicated in the diagram below:
$$ \xymatrix{ M^{op}(\delta) \ar[rr]^{ (f_0, f_1, \alpha)} \ar[d] & & \calD \\
\calM. \ar@{-->}[urr]^{F} & & }$$
Consequently, we may reformulate condition $(1)$ as follows:
\begin{itemize}
\item[$(1')$] The functor $F$ is a left Kan extension of
$F | \calM^{0}$. 
\end{itemize}

Proposition \ref{apple1} now implies that, for each
vertex $s$ of $S$, the map $X_{s} \rightarrow \calM^{0}_{s}$ is a categorical equivalence.
Similarly, for each vertex $s$ of $S$, the inclusion $Y_{s} \rightarrow \calM^{1}_{s}$ is a categorical equivalence. It follows that the inclusion $M^{op}(\delta)_{s} \rightarrow \calM_{s}$ is a
quasi-equivalence, and therefore a categorical equivalence (Proposition \ref{simplexplay}). 
Consequently, we may reformulate condition $(2)$ as follows:

\begin{itemize}
\item[$(2')$] For each vertex $s \in S$, the functor $F| \calM_{s}$ is a left Kan extension of
$F| \calM^{0}_{s}$. 
\end{itemize}

Using Lemma \ref{basekann}, it is easy to see that the collection of objects $s \in S'$ such that
$F| \calM_{s}$ is a left Kan extension of $F| \calM^{0}_{s}$ is stable under equivalence. Since
the inclusion $S \subseteq S'$ is a categorical equivalence, we conclude that $(2')$ is equivalent to the following apparently stronger condition:

\begin{itemize}
\item[$(2'')$] For every object $s \in S'$, the functor $F| \calM_{s}$ is a left Kan extension of
$F| \calM^{0}_{s}$. 
\end{itemize}

The equivalence of $(1')$ and $(2'')$ follows from Lemma \ref{longerwait}. 
\end{proof}



