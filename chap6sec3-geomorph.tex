\section{The $\infty$-Category of $\infty$-Topoi}\label{chap6sec4}
\setcounter{theorem}{0}

In this section, we will show that the collection of all $\infty$-topoi can be organized into an $\infty$-category $\RGeom$. The objects of $\RGeom$ are $\infty$-topoi, and the morphisms are called {\it geometric morphisms}; we will give a definition in \S \ref{gemor1}. In \S \ref{colimtop}, we will show that $\RGeom$ admits (small) colimits. In \S \ref{inftyfiltlim}, we will show that $\RGeom$ admits 
(small) {\em filtered} limits; we will treat the case of general limits in \S \ref{genlim}.

Let $\calX$ be an $\infty$-topos containing an object $U$. In \S \ref{gemor2}, we will show that the 
$\infty$-category $\calX_{/U}$ is an $\infty$-topos. Moreover, this $\infty$-topos is equipped with
a canonical geometric morphism $\calX_{/U} \rightarrow \calX$. Geometric morphisms which arise via this construction are said to be {\em \'{e}tale}. In \S \ref{structuretheor}, we will define a more general notion of {\em algebraic} morphism between $\infty$-topoi. We will also prove a structure theorem which implies that, under certain hypotheses, every $\infty$-topos $\calX$ admits an algebraic morphism to an $\infty$-category of sheaves on a $2$-category.

\subsection{Geometric Morphisms}\label{gemor1}

In classical topos theory, the correct notion of {\em morphism} between two topoi is an adjunction
$$ \Adjoint{f^{\ast}}{\calX}{\calY}{f_{\ast}}$$
where the functor $f^{\ast}$ is left exact. We will introduce the same ideas in the $\infty$-categorical setting.

\begin{definition}\label{geomorph}\index{gen}{geometric morphism}
Let $\calX$ and $\calY$ be $\infty$-topoi.
A {\it geometric morphism} from $\calX$ to $\calY$ is a functor
$f_{\ast}: \calX \rightarrow \calY$ which admits a left exact left adjoint (which we will typically denote by $f^{\ast}$).
\end{definition}

\begin{remark}\index{gen}{pullback functor}\index{gen}{pushforward functor}
Let $f_{\ast}: \calX \rightarrow \calY$ be a geometric morphism from an $\infty$-topos $\calX$ to another $\infty$-topos $\calY$, so that $f_{\ast}$ admits a left adjoint $f^{\ast}$. 
Either functor $f_{\ast}$ and $f^{\ast}$ determines the other up to equivalence (in fact, up to contractible ambiguity). We will often abuse terminology by referring to $f^{\ast}$ as a geometric morphism from $\calX$ to $\calY$. We will always indicate in our notation whether the left or right adjoint is being considered: a superscripted asterisk indicates a left adjoint (pullback functor), and a subscripted asterisk indicates a right adjoint (pushforward functor).
\end{remark}

\begin{remark}
Any equivalence of $\infty$-topoi is a geometric morphism. If $f_{\ast}, g_{\ast}: \calX \rightarrow \calY$ are homotopic, then $f_{\ast}$ is a geometric morphism if and only if $g_{\ast}$ is a geometric morphism (because we can identify left adjoints of $f_{\ast}$ with left adjoints of $g_{\ast}$). 
\end{remark}

\begin{remark}
Let $f_{\ast}: \calX \rightarrow \calY$ and $g_{\ast}: \calY \rightarrow \calZ$ be geometric morphisms. Then $f_{\ast}$ and $g_{\ast}$ admit left exact left adjoints, which we will denote by
$f^{\ast}$ and $g^{\ast}$, respectively. The composite functor $f^{\ast} \circ g^{\ast}$ is left
exact, and is a left adjoint to $g_{\ast} \circ f_{\ast}$ by Proposition \ref{compadjoint}. We conclude that $g_{\ast} \circ f_{\ast}$ is a geometric morphism, so the class of geometric morphisms is stable under composition.
\end{remark}

\begin{definition}\index{not}{LTop@$\LGeom$}\index{not}{RTop@$\RGeom$}
Let $\widehat{\Cat}_{\infty}$ denote the $\infty$-category of (not necessarily small) $\infty$-categories. We define subcategories $\LGeom, \RGeom \subseteq \widehat{\Cat}_{\infty}$ as follows:
\begin{itemize}
\item[$(1)$] The objects of $\LGeom$ and $\RGeom$ are the $\infty$-topoi.
\item[$(2)$] A functor $f^{\ast}: \calX \rightarrow \calY$ between $\infty$-topoi belongs to
$\LGeom$ if and only if $f^{\ast}$ preserves small colimits and finite limits.
\item[$(3)$] A functor $f_{\ast}: \calX \rightarrow \calY$ between $\infty$-topoi belongs to
$\RGeom$ if and only if $f_{\ast}$ has a left adjoint which is left exact.
\end{itemize}
\end{definition}

The $\infty$-categories $\LGeom$ and $\RGeom$ are canonically anti-equivalent. To prove this, we will use the argument of Corollary \ref{warhog}. First, we need a definition.

\begin{definition}\index{gen}{fibration!topos}\label{skuzz}
A map $p: X \rightarrow S$ of simplicial sets is a {\it topos fibration} if the following conditions are satisfied:
\begin{itemize}
\item[$(1)$] The map $p$ is both a Cartesian fibration and a coCartesian fibration.
\item[$(2)$] For every vertex $s$ of $S$, the corresponding fiber $X_{s} = X \times_{S} \{s\}$ is an $\infty$-topos.
\item[$(3)$] For every edge $e: s \rightarrow s'$ in $S$, the associated functor
$X_{s} \rightarrow X_{s'}$ is left exact.
\end{itemize}
\end{definition}

The following analogue of Proposition \ref{surtog} follows immediately from the definitions:

\begin{proposition}\label{surtog2}
\begin{itemize}
\item[$(1)$] Let $p: X \rightarrow S$ be a Cartesian fibration of simplicial sets, classified
by a map $\chi: S^{op} \rightarrow \widehat{\Cat}_{\infty}$. Then $p$ is a topos fibration if and only if $\chi$ factors through $\RGeom \subseteq \widehat{\Cat}_{\infty}$. 

\item[$(2)$] Let $p: X \rightarrow S$ be a coCartesian fibration of simplicial sets, classified by a map $\chi: S \rightarrow \widehat{\Cat}_{\infty}$. Then $p$ is a topos fibration if and only if
$\chi$ factors through $\LGeom \subseteq \widehat{\Cat}_{\infty}$.
\end{itemize}
\end{proposition}

\begin{corollary}\label{suytoy}
For every simplicial set $S$, there is a canonical bijection
$$ [ S, \LGeom ] \simeq [ S^{op}, \RGeom ]$$
where $[K, \calC]$ denotes the collection of equivalence classes of objects of $\Fun(K,\calC)$.
In particular, $\LGeom$ and $\RGeom^{op}$ are canonically isomorphic
in the homotopy category of $\infty$-categories.
\end{corollary}

\begin{proof}
According to Proposition \ref{surtog2}, both $[S, \LGeom]$ and $[S^{op}, \RGeom]$ can be identified with the collection of equivalence classes of topos fibrations $X \rightarrow S$.
\end{proof}

The following proposition is a simple reformulation of some of the results of \S \ref{truncintro}.

\begin{proposition}
Let $f_{\ast}: \calX \rightarrow \calY$ be a geometric morphism between $\infty$-topoi, having a left adjoint $f^{\ast}$. Then $f^{\ast}$ and
$f_{\ast}$ carry $k$-truncated objects to $k$-truncated objects and $k$-truncated morphisms to $k$-truncated morphisms, for any integer $k \geq -2$. Moreover, there is a $($canonical$)$ equivalence
of functors $f^{\ast} \tau^{\calY}_{\leq k} \simeq \tau^{\calX}_{\leq k} f^{\ast}$.
\end{proposition}

\begin{proof}
The first assertion follows immediately from Lemma \ref{trunc}, since $f_{\ast}$ and $f^{\ast}$ are both left-exact. The second follows from Proposition \ref{compattrunc}.
\end{proof}

\begin{definition}\index{not}{FunLast@$\Fun_{\ast}(\calX,\calY)$}\index{not}{FunUast@$\Fun^{\ast}(\calX,\calY)$}\label{defhomst}
Let $\calX$ and $\calY$ be $\infty$-topoi. We let
$\Fun_{\ast}(\calX, \calY)$ denote the full subcategory of $\Fun(\calX,\calY)$ spanned by
geometric morphisms $f_{\ast}: \calX \rightarrow \calY$, and
$\Fun^{\ast}(\calY, \calX)$ the full subcategory of $\Fun(\calY,\calX)$ spanned by
their left adjoints.
\end{definition}

\begin{remark}
According to Proposition \ref{switcheroo}, the $\infty$-categories
$\Fun_{\ast}(\calX, \calY)$ and $\Fun^{\ast}(\calY, \calX)$ are canonically anti-equivalent to one another.
\end{remark}

\begin{warning}\label{tooobig}
If $\calX$ and $\calY$ are $\infty$-topoi, then the $\infty$-category $\Fun_{\ast}(\calX, \calY)$ of
geometric morphisms from $\calX$ to $\calY$ is {\em not} necessarily small, or even equivalent to a small $\infty$-category. This phenomenon is familiar in classical topos theory. For example,
there is a classifying topos $\calA$ for abelian groups, having the property that for {\em any} topos
$\calX$, the category $\calC$ of geometric morphisms $\calX \rightarrow \calA$ is equivalent to the category of abelian group objects of $\calX$. This category is almost never small (for example, when $\calX$ is the topos of sets, $\calC$ is equivalent to the category of abelian groups).
\end{warning}

In spite of Warning \ref{tooobig}, the $\infty$-category of geometric morphisms between two $\infty$-topoi can be reasonably controlled:

\begin{proposition}\label{nottoobig}
Let $\calX$ and $\calY$ be $\infty$-topoi. Then the
$\infty$-category $\Fun^{\ast}(\calY, \calX)$ of geometric morphisms from
$\calX$ to $\calY$ is accessible.
\end{proposition}

\begin{proof}
For each regular cardinal $\kappa$, let $\calY^{\kappa}$ denote the full subcategory of
$\calY$ spanned by $\kappa$-compact objects. Choose a regular cardinal $\kappa$ such that $\calY$ is $\kappa$-accessible and $\calY^{\kappa}$ is stable under finite limits in $\calY$. We may therefore identify $\calY$ with $\Ind^{\kappa}(\calC)$, where
$\calC$ is a minimal model for $\calY^{\kappa}$. According to Proposition \ref{intprop},
composition with the Yoneda embedding $j: \calC \rightarrow \calY$ induces an equivalence
from the $\infty$-category of $\kappa$-continuous functors $\Fun_{\kappa}(\calY,\calX)$
to the $\infty$-category $\Fun(\calC,\calX)$. We now make the following observations:

\begin{itemize}
\item[$(1)$] A functor $F: \calY \rightarrow \calX$ preserves all small colimits if and only if
$F \circ j: \calC \rightarrow \calX$ preserves $\kappa$-small colimits (Proposition \ref{sumatch}).

\item[$(2)$] A colimit-preserving functor $F: \calY \rightarrow \calX$ is left exact if and only if the composition $F \circ j: \calC \rightarrow \calX$ is left exact (Proposition \ref{natash}). 

\end{itemize}

Invoking Proposition \ref{switcheroo}, we deduce that the $\infty$-category $\Fun^{\ast}(\calY, \calX)$ is equivalent to the full subcategory $\calM \subseteq \calX^{\calC}$ consisting of functors which preserve $\kappa$-small colimits and finite limits. Proposition \ref{horse1} implies that $\Fun(\calC,\calX)$ is accessible. For every $\kappa$-small (finite) diagram $p: K \rightarrow \calC$, the full subcategory of $\Fun(\calC,\calX)$ which
preserve colimits (limits) of $p$ is an accessible subcategory of $\Fun(\calC,\calX)$
(Example \ref{colexam}). Up to isomorphism, there are only a bounded number of $\kappa$-small (finite) diagrams in $\calC$. Consequently, $\calM$ is an intersection of a bounded number of accessible subcategories of $\Fun(\calC,\calX)$, and therefore accessible by (Proposition \ref{boundint}).
\end{proof}

\subsection{Colimits of $\infty$-Topoi}\label{colimtop}

Our goal in this section is to construct colimits in the $\infty$-category $\RGeom$ of $\infty$-topoi. 
According to Corollary \ref{suytoy}, it will suffice construct {\em limits} in the $\infty$-category $\LGeom$. 

\begin{proposition}\label{quathorse}\index{gen}{coproduct!of $\infty$-topoi}
Let $\{ \calX_{\alpha} \}_{\alpha \in A}$ be a collection of $\infty$-topoi, parametrized by a $($small$)$
set $A$. Then the product $\calX = \prod_{\alpha \in A} \calX_{\alpha}$ is an $\infty$-topos.
Moreover, each projection $\pi^{\ast}_{\alpha}: \calX \rightarrow \calX_{\alpha}$ is left exact and colimit preserving. The corresponding geometric morphisms exhibit $\calX$ as a product of the family $\{ \calX_{\alpha} \}_{\alpha \in A}$ in the $\infty$-category $\LGeom$.
\end{proposition}

\begin{proof}
Proposition \ref{complexhorse2} implies that $\calX$ is presentable.
It is clear that a diagram $\overline{p}: K^{\triangleright} \rightarrow \calX$ is a colimit if and only if each composition
$\pi^{\ast}_{\alpha} \circ \overline{p}: K^{\triangleright} \rightarrow \calX_{\alpha}$
is a colimit diagram in $\calX_{\alpha}$. 
Similarly, a diagram $\overline{q}: K^{\triangleleft} \rightarrow \calX$ is a limit if and only if each composition
$\pi^{\ast}_{\alpha} \circ \overline{q}: K^{\triangleleft} \rightarrow \calX_{\alpha}$
is a limit diagram in $\calX_{\alpha}$. Using criterion $(2)$ of Theorem \ref{mainchar}, we deduce that $\calX$ is an $\infty$-topos, and that each $\pi^{\ast}_{\alpha}$ preserves {\em all} limits and colimits that exist in $\calX$. Choose a right adjoint
$\pi_{\ast}^{\alpha}: \calX_{\alpha} \rightarrow \calX$ to each $\pi^{\ast}_{\alpha}$. 

According to Theorem \ref{colimcomparee}, the $\infty$-category $\calX$ is a product
of the family $\{ \calX_{\alpha} \}_{\alpha \in A}$ in the $\infty$-category $\widehat{\Cat}_{\infty}$. Since $\LGeom$ is a subcategory of $\widehat{\Cat}_{\infty}$, it will suffice to prove the following assertion:

\begin{itemize}
\item For every $\infty$-topos $\calY$ and every functor $f^{\ast}: \calY \rightarrow \calX$
such that each of the composite functors $\calY \rightarrow \calX_{\alpha}$ is left exact and colimit preserving, $f^{\ast}$ is itself left exact and colimit preserving.
\end{itemize}

This follows immediately from the fact that limits and colimits are computed pointwise.
\end{proof}

\begin{proposition}\label{horse32}\index{gen}{pushout!of $\infty$-topoi}
Let $$ \xymatrix{ \calX' \ar[r]^{{q'}^{\ast}} \ar[d]^{{p'}^{\ast}} & \calX \ar[d]^{p^{\ast}} \\
\calY' \ar[r]^{q^{\ast}} & \calY }$$
be a diagram of $\infty$-categories which is homotopy Cartesian (with respect to the Joyal model structure). Suppose further that $\calX$, $\calY$, and $\calY'$ are $\infty$-topoi, and that
$p^{\ast}$ and $q^{\ast}$ are left exact and colimit preserving. Then $\calX'$ is an $\infty$-topos. Moreover, for any $\infty$-topos $\calZ$ and any functor $f^{\ast}: \calZ \rightarrow \calX$, $f^{\ast}$ is left exact and colimit preserving if and only if the compositions ${p'}^{\ast} \circ f^{\ast}$ and ${q'}^{\ast} \circ f^{\ast}$ are left exact and colimit preserving. In particular $($taking $f^{\ast} = \id_{\calX}${}$)$, the functors
${p'}^{\ast}$ and ${q'}^{\ast}$ are left exact and colimit preserving.
\end{proposition}

\begin{proof}
The second claim follows immediately from Lemma \ref{bird3} and the dual result concerning limits.
To prove the first, we observe that $\calX'$ is presentable by Proposition \ref{horse22}. To show that
$\calX$ is an $\infty$-topos, it will suffice to show that it satisfies criterion $(2)$ of Theorem \ref{mainchar}. This follows immediately from Lemma \ref{bird3}, given that
$\calX$ and $\calY'$ satisfy criterion $(2)$ of Theorem \ref{mainchar}.
\end{proof}

\begin{proposition}\label{colimtopoi}\index{gen}{colimit!of $\infty$-topoi}
The $\infty$-category $\LGeom$ admits small limits, and the inclusion functor
$\LGeom \subseteq \widehat{\Cat}_{\infty}$ preserves small limits.
\end{proposition}

\begin{proof}
According to Proposition \ref{alllimits}, it suffices to prove this result for pullbacks and small products. In the case of products, we apply Proposition \ref{quathorse}. For pullbacks, we
use Proposition \ref{horse32} and Theorem \ref{colimcomparee}.
\end{proof}

\subsection{Filtered Limits of $\infty$-Topoi}\label{inftyfiltlim}

We now consider the problem of computing limits in the $\infty$-category $\RGeom$ of $\infty$-topoi. This is quite a bit more difficult than the analogous problem for colimits, because the inclusion functor
$i: \RGeom \subseteq \widehat{\Cat}_{\infty}$ does not commute with limits in general.
However, the inclusion $i$ does commute with {\em filtered} limits:

\begin{theorem}\label{sutcar}\index{gen}{filtered limit!of $\infty$-topoi}
The $\infty$-category $\RGeom$ admits small filtered limits $($that is, limits indexed by diagrams
$\calC^{op} \rightarrow \RGeom$ where $\calC$ is a small, filtered $\infty$-category$)$. Moreover, the inclusion $\RGeom \subseteq \widehat{\Cat}_{\infty}$ preserves small, filtered limits.
\end{theorem}

The remainder of this section is devoted to the proof of Theorem \ref{sutcar}. Our basic strategy is to mimic the proof of Theorem \ref{surbus}. Our first step is to show that the limit (in $\widehat{\Cat}_{\infty}$) of a filtered diagram of $\infty$-topoi is itself an $\infty$-topos. This is equivalent to a more concrete assertion: if $p: X \rightarrow S$ is a topos fibration, and
$S^{op}$ is a small, filtered $\infty$-category, then the $\infty$-category $\calC$ of Cartesian sections of $p$ is an $\infty$-topos. We saw in Proposition \ref{seccatdog} that
$\calC$ is an accessible localization of the $\infty$-category $\bHom_{S}(S,X)$ spanned by
{\em all} sections of $p$. Our first step will be to show that $\bHom_{S}(S,X)$ is an $\infty$-topos. For this, the hypothesis that $S^{op}$ is filtered is irrelevant.

\begin{lemma}
Let $p: X \rightarrow S$ be a topos fibration, where $S$ is a small simplicial set. The
$\infty$-category $\bHom_{S}(S,X)$ of sections of $p$ is an $\infty$-topos.
\end{lemma}

\begin{proof}
This is a special case of Proposition \ref{prestorkus}.
%We will follow the proof of Proposition \ref{storkus}. First, we note that for any diagram
%$$ \xymatrix{ T \ar[rr] \ar[dr] & & T' \ar[dl] \\
%& S, & }$$ 
%the associated pullback functor $\bHom_{S}(T', X) \rightarrow \bHom_{S}(T,X)$ preserves all small limits and colimits, by Proposition \ref{limiteval}. The simplicial set
%$\bHom_{S}(S,X)$ is isomorphic to the (homotopy) inverse limit of the tower
%$\{ \bHom_{S}( \sk^{n} S, X) \}_{n \geq 0}$. If each term in the tower is an $\infty$-topos, then
%the limit is an $\infty$-topos by Proposition \ref{colimtopoi}. In other words, we may reduce to the case where $S$ is of finite dimension $n$. 

%If $S$ is empty there is nothing to prove. Otherwise, we may assume $n \geq 0$ so that there is a pushout digram of simplicial sets
%$$ \xymatrix{ S_n \times \bd \Delta^n \ar[r] \ar[d] & S_n \times \Delta^n \ar[d] \\
%\sk^{n-1} S \ar[r] & S. }$$
%We therefore therefore a homotopy pullback diagram of $\infty$-categories
%$$ \xymatrix{ \bHom_{S}( S, X) \ar[r] \ar[d] & \bHom_{S}( \sk^{n-1} S, X) \ar[d] \\
%\bHom_{S}( S_n \times \Delta^n,X) \ar[r] & \bHom_{S}( S_n \times \bd \Delta^n, X). }$$
%Using Proposition \ref{horse32} and the inductive hypothesis, we may reduce to proving
%that $\bHom_{S}(S_n \times \Delta^n,X)$ is accessible. We may write this latter simplicial
%set as a product of $\infty$-categories $\bHom_{S}( \Delta^n, X)$. Using Lemma \ref{quathorse}, we may reduce to the case $S = \Delta^n$. If $n = 0$, then $\bHom_{S}( \Delta^n,X)$ is a fiber of $p$, which is an $\infty$-topos by assumption. If $n > 1$, then we have a trivial fibration
%$$ \bHom_{S}(S, X) \rightarrow \bHom_{S}( \Lambda^n_1, X).$$
%Since the horn $\Lambda^n_1$ is of dimension $< n$, we may conclude by applying the inductive hypothesis. We are therefore reduced to the case $S = \Delta^1$.

%According to Lemma \ref{surgem}, the $\infty$-category $\bHom_{\Delta^1}(\Delta^1, X)$ can be identified with a homotopy limit of the diagram
%$$ X_{ \{ 0\} } \stackrel{F}{\rightarrow} X_{ \{1\} } \leftarrow \Fun(\Delta^1, X_{ \{1\} }),$$
%where $F: X_{ \{0\} } \rightarrow X_{ \{1\} }$ is the associated functor. Since $p$ is a topos fibration, $F$ preserves colimits and finite limits. We now conclude by applying Proposition \ref{horse32}.
\end{proof}

\begin{proposition}\label{steak1}
Let $A$ be a $($small$)$ filtered partially ordered set, and let $p: X \rightarrow \Nerve(A)$
be a topos fibration. Let $\calC = \bHom_{ \Nerve(A) }( \Nerve(A), X)$ be the $\infty$-category of sections of $p$, and let $\calC' \subseteq \calC$ be the full subcategory of $\calC$
spanned by the Cartesian sections of $p$. Then $\calC'$ is a topological localization of $\calC$.
\end{proposition}

\begin{proof}
Let us say that a subset $A' \subseteq A$ is {\it dense} if there exists $\alpha \in A$ such that
$$ \{ \beta \in A: \beta \geq \alpha \} \subseteq A'.$$
For each morphism $f$ in $\calC$, let $A(f) \subseteq A$ be the collection of all
$\alpha \in A$ such that the image of $f$ in $X_{\alpha}$ is an equivalence. Let
$S$ be the collection of all monomorphisms $f$ in $\calC$ such that $A(f)$ is dense.
It is clear that $S$ is stable under pullbacks, so that $S^{-1} \calC$ is a topological localization of $\calC$. To complete the proof, it will suffice to show that $\calC' = S^{-1} \calC$.

We first claim that each object of $\calC'$ is $S$-local. Let $f: C \rightarrow C'$
belong to $S$, and let $D \in \calC'$. Choose $\alpha_0$ such that $A(f)$ contains
$A' = \{ \beta \in A: \beta \geq \alpha_0 \}$, and let $R^{\ast}$ denote a right adjoint to
the restriction functor 
$$ R: \bHom_{ \Nerve(A)}( \Nerve(A), X) \rightarrow \bHom_{ \Nerve(A)}( \Nerve(A'), X).$$
According to Proposition \ref{leftkanadj}, the essential image of $R^{\ast}$ consists of those functors
$E: \Nerve(A) \rightarrow X$ which are $p$-right Kan extensions of $E | \Nerve(A')$. We claim that $D$ satisfies this condition. In other words, we claim that
for each $\alpha \in A$, the map
$$ \overline{q}: \Nerve(A'')^{\triangleleft}
\rightarrow \Nerve(A) \stackrel{D}{\rightarrow} X$$
is a $p$-limit, where $A'' = \{ \beta \in A: \beta \geq \alpha, \beta \geq \alpha_0 \}$. Since $\overline{q}$ carries each edge of
$\Nerve(A'')^{\triangleleft}$
to a $p$-Cartesian edge of $X$, it suffices to verify that the simplicial set
$\Nerve(A'')$ is weakly contractible (Proposition \ref{timal}). This follows immediately from the observation that $A''$ is a filtered partially ordered set.

We may therefore suppose that $D = R^{\ast} \overline{D}$, 
where $\overline{D} = D | \Nerve(A')$ is a Cartesian section of the induced
map $p': X \times_{ \Nerve(A)} \Nerve(A') \rightarrow \Nerve(A')$. 
We wish to prove that composition with $f$ induces a homotopy equivalence
$$ \bHom_{\calC}( C', R^{\ast} \overline{D}) \rightarrow \bHom_{\calC}(C, R^{\ast} \overline{D}).$$
This follows immediately from the fact that $R$ and $R^{\ast}$ are adjoint, since $R(f)$ is an equivalence.

We now show that every $S$-local object of $\calC$ belongs to $\calC'$. Let
$C \in \calC$ be a section of $p$ which is $S$-local. Choose $\alpha \leq \beta$ in $A$, and let 
$$ \Adjoint{F}{X_{\alpha}}{X_{\beta}}{G}$$
denote the (adjoint) functors associated to the (co)Cartesian fibration $p: X \rightarrow \Nerve(A)$.
The section $C$ gives rise to a pair of objects 
$C_{\alpha} \in X_{\alpha}$, $C_{\beta} \in X_{\beta}$, and a morphism 
$\phi: C_{\alpha} \rightarrow C_{\beta}$ in the $\infty$-category $X$. The map $\phi$ induces a morphism $u: C_{\alpha} \rightarrow G C_{\beta}$ in $X_{\alpha}$, which is well-defined up to equivalence. We wish to show that $\phi$ is $p$-Cartesian, which is equivalent to the assertion that $u$ is an equivalence in $X_{\alpha}$. Equivalently, we wish to show that for each object
$P \in X_{\alpha}$, composition with $u$ induces a homotopy equivalence
$$ \bHom_{X_{\alpha}}( P, C_{\alpha}) \rightarrow \bHom_{X_{\alpha}}(P, G C_{\beta}).$$

We may identify $P$ with a diagram
$$ \xymatrix{ \{ \alpha \} \ar[r]^{P} \ar@{^{(}->}[d] & X \ar[d] \\
\Nerve(A) \ar@{=}[r] \ar@{-->}[ur]^{D} & \Nerve(A). }$$
Using Corollary \ref{kanexistleft}, choose an extension $D$ as indicated in the diagram above, so that $D$ is a left Kan extension of $D | \{ \alpha \}$ over $\Nerve(A)$.
Similarly, we have a diagram
$$ \xymatrix{ \{ \beta \} \ar[r]^{F(P)} \ar@{^{(}->}[d] & X \ar[d] \\
\Nerve(A) \ar@{=}[r] \ar@{-->}[ur]^{D'} & \Nerve(A). }$$
and we can choose $D'$ to be a $p$-left Kan extension of
$D' | \{ \beta \}$.

Proposition \ref{leftkanadj} implies that for every object $E \in \calC$, the restriction maps
$$ \bHom_{\calC}(D,E) \rightarrow \bHom_{X_{\alpha}}( P, E(\alpha) )$$
$$ \bHom_{\calC}(D',E) \rightarrow \bHom_{X_{\beta}}( F(P), E(\beta) )$$
are equivalences. In particular, the equivalence between $D(\beta)$ and $F(P)$ induces
a morphism $\theta: D' \rightarrow D$. 

We have a commutative diagram in the homotopy category $\calH$:
$$ \xymatrix{ \bHom_{\calC}(D,C) \ar[rr]^{ \circ \theta} \ar[d] & & \bHom_{\calC}(D',C) \ar[d] \\
\bHom_{X_{\alpha}}(P, C_{\alpha}) \ar[r]^{\circ u} & \bHom_{X_{\alpha}}(P, G(C_{\beta})) &
\bHom_{X_{\beta}}( F(P), C_{\beta} ). \ar[l] }$$
The vertical maps are homotopy equivalences, and the 
the horizontal map on the lower right is a homotopy equivalence because $F$ and $G$ are adjoint.
To complete the proof, it will suffice to show that the upper horizontal map is an equivalence.
Since $C$ is $S$-local, it will suffice to show that $\theta \in S$.

Let $\beta \leq \beta'$, and consider the diagram
$$ \xymatrix{ & D'(\beta) \ar[r]^{w'} \ar[d]^{\theta(\beta)} & D'(\beta') \ar[d]^{\theta(\beta')} \\
D(\alpha) \ar[r]^{v} & D(\beta) \ar[r]^{w} & D(\beta'). }$$
in the $\infty$-category $X$. Since $D'$ is a $p$-left Kan extension of $D' | \{ \beta \}$, we conclude that $w'$ is $p$-coCartesian. Similarly, since $D$ is a $p$-left Kan
extension of $D | \{ \alpha \}$, we conclude that $v$ and $w \circ v$ are $p$-coCartesian.
It follows that $w$ is $p$-coCartesian as well (Proposition \ref{protohermes}). Since $\theta(\beta)$ is an equivalence by construction, we conclude that $\theta(\beta')$ is an equivalence. Thus
$A(\theta) \subseteq A$ is dense.

It remains only to show that $\theta$ is a monomorphism. For this, it suffices to show that
$\theta(\gamma)$ is an monomorphism in $X_{\gamma}$ for each $\gamma \in A$.
If $\gamma \geq \beta$, this follows from the above argument. Suppose $\gamma \ngeq \beta$.
Since $D'$ is a $p$-left Kan extension of $D' | \{ \beta \}$ over $\Nerve(A)$, we conclude
that $D'(\gamma)$ is a $p$-colimit of the empty diagram, and therefore an initial object
of $X_{\gamma}$. It follows that any map $D'(\gamma) \rightarrow D(\gamma)$ is a monomorphism.
\end{proof}

\begin{proposition}\label{steak2}
Let $A$ be a $($small$)$ filtered partially ordered set, let $p: X \rightarrow \Nerve(A)^{op}$, and let
$\calY \subseteq \bHom_{ \Nerve(A)}(\Nerve(A), X)$ be the full subcategory spanned by the Cartesian sections of $p$. For each $\alpha \in A$, the evaluation map
$\pi_{\ast}: \calY \rightarrow X_{\alpha}$ is a geometric morphism of $\infty$-topoi.
\end{proposition}

\begin{proof}
Let $A' = \{ \beta \in A: \alpha \leq \beta \}$. Using Theorem \ref{hollowtt}, we conclude that
the inclusion $\Nerve(A') \subseteq \Nerve(A)$ is cofinal. Corollary \ref{blurt} implies that
the restriction map
$$ \bHom_{ \Nerve(A)}( \Nerve(A), X) \rightarrow \bHom_{\Nerve(A)}(\Nerve(A'), X)$$
induces an equivalence on the full subcategories spanned by Cartesian sections. Consequently, we are free to replace $A$ by $A'$ and thereby assume that $\alpha$ is a least element of $A$.

The functor $\pi_{\ast}$ factors as a composition
$$ \calY \stackrel{\phi_{\ast}}{\rightarrow} \bHom_{ \Nerve(A)}( \Nerve(A), X) \stackrel{\psi_{\ast}}{\rightarrow}
X_{\alpha}$$
where $\phi_{\ast}$ denotes the inclusion functor and $\psi_{\ast}$ the evaluation functor.
Proposition \ref{steak1} implies that $\phi_{\ast}$ is a geometric morphism; it therefore suffices to show that $\psi_{\ast}$ is a geometric morphism as well. 

Let $\psi^{\ast}$ be a left adjoint to $\psi_{\ast}$ (the existence of $\psi^{\ast}$ follows from Proposition \ref{leftkanadj}, as indicated below). We wish to show that $\psi^{\ast}$ is left exact. According to Proposition \ref{limiteval}, it will suffice to show that the composition
$$ \theta: X_{\alpha} \stackrel{\psi^{\ast}}{\rightarrow} \bHom_{ \Nerve(A)}( \Nerve(A), X) 
\stackrel{e_{\beta}}{\rightarrow} X_{\beta},$$
is left exact, where $e_{\beta}$ denotes the functor given by evaluation at $\beta$.

Let $f: \Delta^1 \rightarrow \Nerve(A)$ be the edge joining $\alpha$ to $\beta$, let
$\calC$ be the $\infty$-category of coCartesian sections of $p$, and let
$\calC'$ be the $\infty$-category of coCartesian sections of the induced map
$p': X \times_{ \Nerve(A)} \Delta^1 \rightarrow \Delta^1$. We observe that
$\calC$ consists precisely of those sections $s: \Nerve(A) \rightarrow X$ of $p$ which
are $p$-left Kan extensions of $s | \{ \alpha \}$. Applying Proposition \ref{lklk},
we conclude that the evaluation map $e_{\alpha}: \calC \rightarrow X_{\alpha}$ is a trivial fibration, and that (by Proposition \ref{leftkanadj}) we may identify $\psi^{\ast}$ with the composition
$$ X_{\alpha} \stackrel{q}{\rightarrow} \calC \subseteq \bHom_{\Nerve(A)}(\Nerve(A),X),$$
where $q$ is a section of $e_{\alpha}|\calC$. Let $q': X_{\alpha} \rightarrow \calC'$
be the composition of $q$ with the restriction map $\calC \rightarrow \calC'$. Then
$\theta$ can be identified with the composition
$$ X_{\alpha} \stackrel{q'}{\rightarrow} \calC' \stackrel{e_{\beta}}{\rightarrow} X_{\beta},$$
which is the functor $X_{\alpha} \rightarrow X_{\beta}$ associated to $f: \alpha \rightarrow \beta$ by the coCartesian fibration $p$. Since $p$ is a topos fibration, $\theta$ is left exact as desired.
\end{proof}

Let $G$ be a profinite group and let $X$ be a set with a continuous action of $G$.
Then we can recover $X$ as the direct limit of the fixed point sets $X^{U}$, where $U$ ranges over the collection of open subgroups of $G$. Our next result is an $\infty$-categorical analogue of this observation.

\begin{lemma}\label{suruti}
Let $p: X \rightarrow S^{\triangleright}$ be a Cartesian fibration of simplicial sets, which
is classified by a colimit diagram $S^{\triangleright} \rightarrow \Cat_{\infty}^{op}$, and let
$\overline{s}: S^{\triangleright} \rightarrow X$ be a Cartesian section of $p$. Then $\overline{s}$ is a $p$-colimit diagram.
\end{lemma}

\begin{proof}
In virtue of Corollary \ref{tttroke}, we may suppose that $S$ is an $\infty$-category. Unwinding the definitions, we must show that the map $X_{\overline{s}/} \rightarrow X_{s/}$ induces an equivalence of $\infty$-categories when restricted to the inverse image of the cone point of
$S^{\triangleright}$. Fix an object $x \in X$ lying over the cone point of $S^{\triangleright}$.
Let $\overline{f}: S^{\triangleright} \rightarrow X$ be the constant map with value $x$, and let
$f = \overline{f} | S$. To complete the proof, it will suffice to show that the restriction map
$$ \theta: \bHom_{ \Fun(S^{\triangleright}, X)}( \overline{s}, \overline{f} )
\rightarrow \bHom_{ \Fun(S,X) }( s,f)$$ is a homotopy equivalence. To prove this, choose a
$p$-Cartesian transformation $\overline{\alpha}: \overline{g} \rightarrow \overline{f}$, where
$\overline{g}: S^{\triangleright} \rightarrow X$ is a section of $p$ (automatically Cartesian).
Let $g = \overline{g} | S$ and let $\alpha: g \rightarrow f$ be the associated transformation. Let
$\overline{\calC}$ be the full subcategory of $\bHom_{S^{\triangleright}}(S^{\triangleright},X)$
spanned by the Cartesian sections of $p$, and let $\calC \subseteq \bHom_{S^{\triangleright}}(S,X)$ be defined similarly.
We have a commutative diagram in the homotopy category $\calH$
$$ \xymatrix{ \bHom_{\overline{\calC} }( \overline{s}, \overline{g} )
\ar[r]^{\theta'} \ar[d]^{\overline{\alpha}} & \bHom_{\calC}( s, g) \ar[d]^{\alpha} \\
\bHom_{ \Fun(S^{\triangleright}, X)}( \overline{s}, \overline{f} ) \ar[r]^{\theta} &
\bHom_{ \Fun(S, X)}(s,f). }$$
Proposition \ref{compspaces} implies that the vertical maps are homotopy equivalences, and
Proposition \ref{charcatlimit} implies that $\theta'$ is a homotopy equivalence (since the restriction map $\overline{\calC} \rightarrow \calC$ is an equivalence of $\infty$-categories). It follows that
$\theta$ is a homotopy equivalence as well.
\end{proof}

\begin{lemma}\label{steakknife}
Let $p: X \rightarrow S$ be a presentable fibration, let $\calC$ 
be the full subcategory of $\bHom_{S}(S,X)$ spanned by the Cartesian sections of $p$. For
each vertex $s$ of $S$, let $\psi(s)_{\ast}: \calC \rightarrow X_{s}$ be the functor given by evaluation at $s$, and let $\psi(s)^{\ast}$ be a left adjoint to $\psi(s)_{\ast}$. There
exists a diagram $\theta: S \rightarrow \Fun(\calC, \calC)$ with the following properties:
\begin{itemize}
\item[$(1)$] For each vertex $s$ of $S$, $\theta(s)$ is equivalent to the composition
$\psi(s)^{\ast} \circ \psi(s)_{\ast}$.
\item[$(2)$] The identity functor $\id_{\calC}$ is a colimit of $\theta$ in the $\infty$-category of functors $\Fun(\calC, \calC)$. 
\end{itemize}
\end{lemma}

\begin{proof}
Without loss of generality, we may suppose that $p$ extends to a presentable fibration
$\overline{p}: \overline{X} \rightarrow S^{\triangleright}$, which is classified by colimit
diagram $S^{\triangleright} \rightarrow \LPres$ (and therefore by a colimit diagram
$S^{\triangleright} \rightarrow \Cat^{op}_{\infty}$, in virtue of Theorem \ref{surbus}). Let
$\overline{\calC}$ be the $\infty$-category of Cartesian sections of $\overline{p}$, so that
we have trivial fibrations
$$ \calC \leftarrow \overline{\calC} \rightarrow X_{\infty},$$
where $X_{\infty} = \overline{X} \times_{S^{\triangleright}} \{\infty\}$, and $\infty$
denotes the cone point of $S^{\triangleright}$. For each vertex $s$ of
$S^{\triangleright}$, we let $\overline{\psi}(s)_{\ast}: \overline{\calC} \rightarrow \overline{X}_s$
be the functor given by evaluation at $s$, and $\overline{\psi}(s)^{\ast}$ a left adjoint to
$\overline{\psi}(s)_{\ast}$. To complete the proof, it will suffice to construct a map 
$\theta': S \rightarrow \Fun( \overline{\calC}, X_{\infty})$ with the following properties:
\begin{itemize}
\item[$(1')$] For each vertex $s$ of $S$, $\theta'(s)$ is equivalent to the composition
$\overline{\psi}(\infty)_{\ast} \circ \overline{\psi}(s)^{\ast} \circ \overline{\psi}(s)_{\ast}$. 
\item[$(2')$] The functor $\overline{\psi}(\infty)_{\ast}$ is a colimit of $\theta'$.
\end{itemize}

Let $e: \overline{\calC} \times S^{\triangleright} \rightarrow \overline{X}$ be the evaluation map. 
Choose an $\overline{p}$-coCartesian natural transformation $e \rightarrow e'$, where
$e'$ is a map from $\overline{\calC} \times S^{\triangleright}$ to $X_{\infty}$. Lemma \ref{suruti} implies that for each object $X \in \overline{\calC}$, the restriction 
$e| \{X\} \times S^{\triangleright}$ is an $\overline{p}$-colimit diagram in $\overline{X}$. 
Applying Proposition \ref{chocolatelast}, we deduce that 
$e' | \{X\} \times S^{\triangleright}$ is a colimit diagram in $X_{\infty}$. According to Proposition
\ref{limiteval}, $e'$ determines a colimit diagram $S^{\triangleright} \rightarrow \Fun(\overline{\calC}, X_{\infty})$. Let $\theta'$ be the restriction of this diagram to $S$. Then the colimit of $\theta'$
can be identified with $e' | \overline{\calC} \times \{ \infty \}$, which is equivalent
to $e | \overline{\calC} \times \{\infty \} = \overline{\psi}(\infty)_{\ast}$. This proves $(2')$. 
To verify $(1')$, we observe that $e' | \overline{\calC} \times \{s\}$ can be identified with
the composition of $\overline{\psi}(s)_{\ast} = e | \overline{\calC} \times \{s\}$ with the
functor $X_{s} \rightarrow X_{\infty}$ associated to the coCartesian fibration
$\overline{p}$, which can in turn be identified with $\overline{\psi}(\infty)_{\ast} \circ \overline{\psi}(s)^{\ast}$ (both are left adjoints to the pullback functor
$X_{\infty} \rightarrow X_{s}$ associated to $\overline{p}$). 
\end{proof}


\begin{proposition}\label{steak3}
Let $A$ be a $($small$)$ filtered partially ordered set, let $p: X \rightarrow \Nerve(A)$, and let
$\calY \subseteq \bHom_{ \Nerve(A)}(\Nerve(A), X)$ be the full subcategory spanned by the Cartesian sections of $p$. Let $\calZ$ be an $\infty$-topos, and $\pi_{\ast}: \calZ \rightarrow \calY$ an arbitrary functor. Suppose that, for each $\alpha \in A$, the composition
$$ \calZ \stackrel{\pi_{\ast}}{\rightarrow} \calY \rightarrow X_{\alpha}$$
is a geometric morphism of $\infty$-topoi. Then $\pi_{\ast}$ is a geometric morphism of $\infty$-topoi.
\end{proposition}

\begin{proof}
Let $\pi^{\ast}$ denote a left adjoint to $\pi_{\ast}$. Since $\pi^{\ast}$ commutes with colimits,
Lemma \ref{steakknife} implies that $\pi^{\ast}$ can be written as the colimit of a diagram
$q: \Nerve(A) \rightarrow \calZ^{\calY}$ having the property that for each
$\alpha \in A$, $q(\alpha)$ is equivalent to $\pi^{\ast} \psi(\alpha)^{\ast} \psi(\alpha)_{\ast}$,
where $\psi(\alpha)_{\ast}$ denotes the evaluation functor at $\alpha$ and $\psi(\alpha)^{\ast}$ its left adjoint. Each composition $\pi^{\ast} \psi(\alpha)^{\ast}$ is left adjoint to the geometric morphism
$\psi(\alpha)_{\ast} \pi_{\ast}$, and therefore left exact. It follows that $q(\alpha)$ is left exact.
Since filtered colimits in $\calZ$ are left exact (Example \ref{tucka}), we conclude that
the functor $\pi^{\ast}$ is left exact, as desired.
\end{proof}

\begin{proof}[Proof of Theorem \ref{sutcar}]
Let $\calC$ be a small, filtered $\infty$-category, and let $q: \calC^{op} \rightarrow \RGeom$
be an arbitrary diagram. Choose a limit $\overline{q}: (\calC^{\triangleright})^{op} \rightarrow \widehat{\Cat}_{\infty}$ of $q$ in the $\infty$-category $\widehat{\Cat}_{\infty}$. We must show that
$\overline{q}$ factors through $\RGeom$, and is a limit diagram in $\RGeom$.

Using Proposition \ref{rot}, we may assume without loss of generality that $\calC$ is the nerve of a filtered partially ordered set $A$. Let $p: X \rightarrow \Nerve(A)^{op}$ be the topos fibration classified by $q$ (Proposition \ref{surtog2}). Then the image of the cone point of $( \calC^{\triangleright})^{op}$ under $\overline{q}$ is equivalent to the $\infty$-category $\calX$ of Cartesian sections of $p$ (Corollary \ref{blurt}). It follows from Proposition \ref{steak1} that
$\calX$ is an $\infty$-topos. Moreover, Proposition \ref{steak2} ensures that for each $\alpha \in A$, the evaluation map $\calX \rightarrow X_{\alpha}$ is a geometric morphism. This proves that 
$\overline{q}$ factors through $\RGeom$. To complete the proof, we must show that $\overline{q}$ is a limit diagram in $\RGeom$. Since $\RGeom$ is a subcategory of $\widehat{\Cat}_{\infty}$, and $\overline{q}$ is a limit diagram in $\widehat{\Cat}_{\infty}$, this reduces immediately to the statement of Proposition \ref{steak3}.
\end{proof}

\subsection{General Limits of $\infty$-Topoi}\label{genlim}

Our goal in this section is to construct general limits in the $\infty$-category $\RGeom$. Our strategy is necessarily rather different from that of \S \ref{inftyfiltlim}, because the inclusion
$i: \RGeom \rightarrow \widehat{\Cat}$ does not preserve limits in general.
In fact, $i$ does not even preserve the final object:

\begin{proposition}\label{spacefinall}\index{gen}{final object!of the $\infty$-category of $\infty$-topoi}
Let $\calX$ be an $\infty$-topos. Then $\Fun^{\ast}(\SSet, \calX)$ is a contractible Kan complex.
In particular, $\SSet$ is a final object in the $\infty$-category $\RGeom$ of $\infty$-topoi.
\end{proposition}

\begin{proof}
We observe that $\SSet \simeq \Shv(\Delta^0)$ where the $\infty$-category $\Delta^0$ is endowed with the ``discrete'' topology (so that the empty sieve does not constitute a cover
of the unique object). According to Proposition \ref{igrute}, the $\infty$-category
$\Fun^{\ast}(\SSet, \calX)$ is equivalent to the full subcategory of $\calX \simeq \Fun(\Delta^0, \calX)$ spanned
by those objects $X \in \calX$ which correspond to left exact functors $\Delta^0 \rightarrow \calX$.
It is clear that these are precisely the final objects of $\calX$, which form a contractible Kan complex
(Proposition \ref{initunique}). 
\end{proof}

To construct limits in general, we first develop some tools for describing $\infty$-topoi via
``generators and relations''. This will allow us to reduce the construction of limits in $\RGeom$ to the problem of constructing colimits in $\Cat_{\infty}$.

\begin{lemma}\label{sleepyswine}
Let $\calC$ be a small $\infty$-category, $\kappa$ a regular cardinal, and suppose given
a $($small$)$ collection of $\kappa$-small diagrams $\{ \overline{f}_{\alpha}: K_{\alpha}^{\triangleright} \rightarrow \calC \}_{\alpha \in A}$. Then there exists a functor $F: \calC \rightarrow \calD$ with the following properties:
\begin{itemize}
\item[$(1)$] The $\infty$-category $\calD$ is small and admits $\kappa$-small colimits.
\item[$(2)$] For each $\alpha \in A$, the induced map $F \circ \overline{f}_{\alpha}: K_{\alpha}^{\triangleright} \rightarrow \calC$ is a colimit diagram.
\item[$(3)$] Let $\calE$ be an arbitrary $\infty$-category which admits $\kappa$-small colimits. Let
$\Fun'(\calD, \calE)$ denote the full subcategory of $\Fun(\calD, \calE)$ spanned by those functors which preserve $\kappa$-small colimits. Then composition with $F$ induces a fully faithful embedding
$$ \theta: \Fun'(\calD, \calE) \rightarrow \Fun( \calC, \calE).$$
The essential image of $\theta$ consists those functors $F': \calC \rightarrow \calE$ such that
each $F' \circ \overline{f}_{\alpha}$ is a colimit diagram in $\calE$.
\end{itemize}
\end{lemma}

\begin{proof}
Let $j: \calC \rightarrow \calP(\calC)$ denote the Yoneda embedding. For each $\alpha \in A$, let
$f_{\alpha} = \overline{f}_{\alpha} | K_{\alpha}$, and let $C_{\alpha} \in \calC$ denote the image of the cone point under $\overline{f}_{\alpha}$. Let $D_{\alpha} \in \calP(\calC)$ denote a colimit of the induced diagram $j \circ f_{\alpha}$, so that $j \circ \overline{f}_{\alpha}$ induces a map
$s_{\alpha} = D_{\alpha} \rightarrow j( C_{\alpha} )$. Let $S = \{ s_{\alpha} \}_{\alpha \in A}$, let $\calX$ denote the localization $S^{-1} \calP(\calC)$, and let $L: \calP(\calC) \rightarrow \calX$ denote a left adjoint to the inclusion. Let $\calD'$ be the smallest full subcategory of $\calX$ that contains the essential image of the functor $L \circ j$ and is stable under $\kappa$-small colimits, let
$\calD$ be a minimal model for $\calD'$, and let $F: \calC \rightarrow \calD$ denote the composition of
$L \circ j$ with a retraction of $\calD'$ onto $\calD$. It follows immediately from the construction that $\calD$ satisfies conditions $(1)$ and $(2)$.

We observe that for each $\alpha \in A$, the domain and codomain of
$s_{\alpha}$ are both $\kappa$-compact objects of $\calP(\calC)$. It follows that
$\calX$ is stable under $\kappa$-filtered colimits in $\calP(\calC)$. Corollary \ref{starmin} implies that $L$ carries $\kappa$-compact objects of $\calP(\calC)$ to $\kappa$-compact objects of $\calX$. 
Since the collection of $\kappa$-compact objects of $\calX$ is stable under $\kappa$-small colimits, we conclude that $\calD'$ consists of $\kappa$-compact objects of $\calX$. Invoking Proposition \ref{uterr}, we deduce that the inclusion $\calD \subseteq \calX$ determines an equivalence
$\Ind_{\kappa}(\calD) \simeq \calX$. 

We now prove $(3)$. We observe that there exists a fully faithful embedding $i: \calE \rightarrow \calE'$ which preserves $\kappa$-small colimits, where $\calE'$ admits arbitrary small colimits (for example, we can take $\calE' = \Fun( \calE, \widehat{\SSet})^{op}$ and $i$ to be the Yoneda embedding).
Replacing $\calE$ by $\calE'$ if necessary, we may assume that $\calE$ itself admits arbitrary small colimits. We have a homotopy commutative diagram
$$ \xymatrix{ \Fun^{L}( \calX, \calE) \ar[r]^{\theta'} \ar[d] & \Fun^{L}( \calP(\calC), \calE) \ar[d] \\
\Fun'(\calD, \calE) \ar[r]^{\theta} & \Fun( \calC, \calE), }$$
where $\Fun^{L}( \calY, \calE)$ denotes the full subcategory of $\Fun(\calY, \calE)$ spanned by those functors which preserve small colimits. Propositions \ref{intprop} and \ref{sumatch} imply that the left vertical arrow is an equivalence, while Theorem \ref{charpresheaf} implies that the right vertical arrow is an equivalence. It will therefore suffice to show that $\theta'$ is fully faithful, and that the essential image of $\theta'$ consists of those colimit-preserving functors $F'$ from $\calP(\calC)$ to $\calE$ such that $F' \circ j \circ \overline{f}_{\alpha}$ is a colimit diagram, for each $\alpha \in A$. This follows immediately from Proposition \ref{unichar}.
\end{proof}

\begin{definition}\index{not}{Catinftylex@$\Cat_{\infty}^{\lex}$}
Let $\Cat_{\infty}^{\lex}$ denote the subcategory of $\Cat_{\infty}$ defined as follows:
\begin{itemize}
\item[$(1)$] A small $\infty$-category $\calC$ belongs to $\Cat_{\infty}^{\lex}$ if and only if
$\calC$ admits finite limits.
\item[$(2)$] Let $f: \calC \rightarrow \calD$ be a functor between small $\infty$-categories which admit finite limits. Then $f$ is a morphism in $\Cat_{\infty}^{\lex}$ if and only if $f$ preserves finite limits.
\end{itemize}
\end{definition}

\begin{lemma}\label{pugswell}
The $\infty$-category $\Cat_{\infty}^{\lex}$ admits small colimits. 
\end{lemma}

\begin{proof}
Let $p: \calJ \rightarrow \Cat_{\infty}^{\lex}$ be a small diagram, which carries each
vertex $j \in \calJ$ to an $\infty$-category $\calC_{j}$.
Let $\calC$ be a colimit of the diagram $p$ in $\Cat_{\infty}$, and for each
$j \in J$, let $\phi_{j}: \calC_{j} \rightarrow \calC$ be the associated functor.
Consider the collection of all isomorphism classes of diagrams $\{ f: K^{\triangleleft} \rightarrow \calC \}$, where $K$ is a finite simplicial set and the map $f$ admits a factorization
$$ K^{\triangleleft} \stackrel{f_0}{\rightarrow} \calC_{j} \stackrel{\phi_{j}}{\rightarrow} \calC,$$
where $f_0$ is a limit diagram in $\calC_{j}$. Invoking the dual of Lemma \ref{sleepyswine}, we deduce the existence of a functor $F: \calC \rightarrow \calD$ with the following properties:
\begin{itemize}
\item[$(1)$] The $\infty$-category $\calD$ is small and admits finite limits.
\item[$(2)$] Each of the compositions $F \circ \phi_{j}$ is left exact.
\item[$(3)$] For every $\infty$-category $\calE$ which admits finite limits, composition with
$F$ induces an equivalence from the full subcategory of $\Fun(\calD, \calE)$ spanned by the left exact functors to the full subcategory of $\Fun( \calC, \calE)$ spanned by those functors $F': \calC \rightarrow \calE$ such that each $F' \circ \phi_{j}$ is left exact.
\end{itemize}
It follows that that $\calD$ can be identified with a colimit of the diagram $p$ in
the $\infty$-category $\Cat_{\infty}^{\lex}$.
\end{proof}

\begin{lemma}\label{squarepeg}
Let $\calC$ be a small $\infty$-category which admits finite limits, and let
$f_{\ast}: \calX \rightarrow \calP(\calC)$ be a geometric morphism of $\infty$-topoi.
Then there exists a small $\infty$-category $\calD$ which admits finite limits, a left
exact functor $f'': \calC \rightarrow \calD$ such that $f_{\ast}$ is equivalent to the composition,
$$ \calX \stackrel{f'_{\ast}}{\rightarrow} \calP( \calD) \stackrel{ f''_{\ast}}{\rightarrow} \calP(\calC)$$
where $f'_{\ast}$ is a fully faithful geometric morphism and $f''_{\ast}$ is given by composition with $f''$.
\end{lemma}

\begin{proof}
Without loss of generality we may assume that $\calX$ is minimal. Let $f^{\ast}$ be a left adjoint to $f_{\ast}$. Choose a regular cardinal $\kappa$ large enough that the composition
$$ \calC \stackrel{j_{\calC}}{\rightarrow} \calP(\calC) \stackrel{f^{\ast}}{\rightarrow} \calX$$
carries each object $C \in \calC$ to a $\kappa$-compact object of $\calX$. Enlarging
$\kappa$ if necessary, we may assume that $\calX$ is $\kappa$-accessible and that the collection
of $\kappa$-compact objects is stable under finite limits (Proposition \ref{tcoherent}). Let $\calD$ be the collection of $\kappa$-compact objects of $\calX$. The proof of Proposition \ref{precisechar} shows that the inclusion $\calD \subseteq \calX$ can be extended to a left exact localization functor ${f'}^{\ast}: \calP(\calD) \rightarrow \calX$. 

Using Theorem \ref{charpresheaf}, we conclude that the composition
$j_{\calD} \circ f^{\ast} \circ j_{\calC}: \calC \rightarrow \calP(\calD)$ can be extended
to a colimit-preserving functor ${f''}^{\ast}: \calP(\calC) \rightarrow \calP(\calD)$, and that
${f'}^{\ast} \circ {f''}^{\ast}$ is homotopic to $f^{\ast}$. Proposition \ref{natash} implies that ${f''}^{\ast}$ is left exact. It follows that ${f'}^{\ast}$ and ${f''}^{\ast}$ admit right adjoints $f'_{\ast}$ and
$f''_{\ast}$ with the desired properties.
\end{proof}

\begin{proposition}\label{swunder}
The $\infty$-category $\RGeom$ of $\infty$-topoi admits pullbacks.
\end{proposition}

\begin{proof}
Suppose first that we are given a pullback square
$$ \xymatrix{ \calW \ar[r]^{f'_{\ast}} \ar[d]^{g'_{\ast}} & \calX \ar[d]^{g_{\ast}} \\
\calY \ar[r]^{f_{\ast}} & \calZ }$$
in the $\infty$-category of $\RGeom$. We make the following observations:

\begin{itemize}
\item[$(a)$] Suppose that $\calZ$ is a left exact localization of another $\infty$-topos $\calZ'$. Then the induced diagram
$$ \xymatrix{ \calW \ar[r] \ar[d] & \calX \ar[d] \\
\calY \ar[r] & \calZ }$$
is also a pullback square.

\item[$(b)$] Let $S^{-1} \calX$ and $T^{-1} \calY$ be left exact localizations of $\calX$ and $\calY$, respectively. Let $U$ be the smallest strongly saturated collection of morphisms in
$\calW$ which contains ${f'}^{\ast} S$ and ${g'}^{\ast} T$, and is closed under pullbacks. Using Corollary \ref{sweetums}, we deduce that $U$ is generated by a (small) set of morphisms in $\calW$. It follows that the diagram
$$ \xymatrix{ U^{-1} \calW \ar[r] \ar[d] & S^{-1} \calX \ar[d] \\
T^{-1} \calY \ar[r] & \calZ }$$
is again a pullback in $\RGeom$.
\end{itemize}

Now suppose given an arbitrary diagram
$$ \calX \stackrel{g_{\ast} }{\rightarrow} \calZ \stackrel{f_{\ast}}{\leftarrow} \calY$$
in $\RGeom$. We wish to prove that there exists a fiber product $\calX \times_{ \calZ} \calY$
in $\RGeom$. The proof of Proposition \ref{precisechar} implies that there exists
a small $\infty$-category $\calC$ which admits finite limits, such that
$\calZ$ is a left exact localization of $\calP(\calC)$. Using $(a)$, we can reduce to the case where $\calZ = \calP(\calC)$. Using $(b)$ and Lemma \ref{squarepeg}, we can reduce to the case where
$\calX = \calP(\calD)$ for some small $\infty$-category $\calD$ which admits finite limits, and $g_{\ast}$ is induced by composition with a left exact functor $g: \calC \rightarrow \calD$. Similarly, we can assume that $f_{\ast}$ is determined by a left exact functor $f: \calC \rightarrow \calD'$.
Using Lemma \ref{pugswell}, we can form a pushout diagram
$$ \xymatrix{ \calE & \calD \ar[l] \\
\calD' \ar[u] & \calC \ar[u]^{g} \ar[l]^{f} }$$
in the $\infty$-category $\Cat_{\infty}^{\lex}$. Using Proposition \ref{natash} and Theorem \ref{charpresheaf}, it is not difficult to see that the induced diagram
$$ \xymatrix{ \calP(\calE) \ar[r] \ar[d] & \calP(\calD) \ar[d]^{g_{\ast}} \\
\calP( \calD') \ar[r]^{f_{\ast}} & \calP( \calC) }$$
is the desired pullback square in $\RGeom$.
\end{proof}

\begin{corollary}\label{geolit}
The $\infty$-category $\RGeom$ admits small limits.
\end{corollary}

\begin{proof}
Using Corollaries \ref{uterrr} and \ref{allfin}, it suffices to show that $\RGeom$ admits filtered limits, a final object, and pullbacks. The existence of filtered limits follows from Theorem \ref{sutcar}, the existence of a final object follows from Proposition \ref{spacefinall}, and the existence of pullbacks follows from Proposition \ref{swunder}.
\end{proof}

\begin{remark}
Our construction of fiber products in $\RGeom$ is somewhat inexplicit. We will later give a more concrete construction in the case of ordinary products; see \S \ref{products}.
\end{remark}

We conclude this section by proving a companion result to Corollary \ref{geolit}. First, a few general remarks. The $\infty$-category $\RGeom$ is most naturally viewed as an {\it $\infty$-bicategory}, since we can consider also noninvertible natural transformations between geometric morphisms. Correspondingly, we can consider a more general theory of {\it $\infty$-bicategorical} limits in $\RGeom$. While we do not want to give any precise definitions, we would like to point out Corollary \ref{geolit} can be generalized to show that $\RGeom$ admits all (small) $\infty$-bicategorical limits. In more concrete terms, this just means that $\RGeom$ is {\it cotensored} over $\Cat_{\infty}$ in the following sense:

\begin{proposition}\label{cotens}
Let $\calX$ be an $\infty$-topos, and let $\calD$ be a small $\infty$-category. Then there exists an $\infty$-topos $\Mor(\calC, \calX)$ and a functor $e: \calC \rightarrow \Fun_{\ast}( \Mor(\calC, \calX), \calX)$
with the following universal property:
\begin{itemize}
\item[$(\ast)$] For every $\infty$-topos $\calY$, composition with $e$ induces an equivalence of $\infty$-categories
$$ \Fun_{\ast}( \calY, \Mor(\calC, \calX) ) \rightarrow \Fun( \calC, \Fun_{\ast}( \calY, \calX) ).$$
\end{itemize}
\end{proposition}

\begin{proof}
We first treat the case where $\calX = \calP(\calD)$, where $\calD$ is a small $\infty$-category which admits finite limits. Using Lemma \ref{sleepyswine}, we conclude that there exists a functor
$e_0: \calC^{op} \times \calD \rightarrow \calD'$ with the following properties:
\begin{itemize}
\item[$(1)$] The $\infty$-category $\calD'$ is small and admits finite limits.
\item[$(2)$] For each object $C \in \calC$, the induced functor
$$ \calD \simeq \{ C \} \times \calD \subseteq \calC^{op} \times \calD \stackrel{e_0}{\rightarrow} \calD'$$
is left exact.
\item[$(3)$] Let $\calE$ be an arbitrary $\infty$-category which admits finite limits. Then
composition with $e_0$ induces an equivalence from full subcategory of $\Fun( \calD', \calE)$ spanned by the left exact functors to the full subcategory of $\Fun( \calC^{op} \times \calD, \calE)$ spanned by those functors which restrict to left-exact functors $\{ C \} \times \calD \rightarrow \calE$, for each $C \in \calC$.
\end{itemize}
In this case, we can define $\Mor(\calC, \calX)$ to be $\calP( \calD')$, and $e: \calC \rightarrow
\Fun_{\ast}( \calP( \calD'), \calP(\calD) )$ to be given by composition with $e_0$; the universal property
$(\ast)$ follows immediately from Theorem \ref{charpresheaf} and Proposition \ref{natash}.

In the general case, we invoke Proposition \ref{precisechar} to reduce to the case where
$\calX = S^{-1} \calX'$ is an accessible left-exact localizaton of an $\infty$-topos
$\calX'$, where $\calX' \simeq \calP(\calD)$ is as above so that we can
construct an $\infty$-topos $\Mor( \calC, \calX')$ and a map 
$e': \calC \rightarrow \Fun_{\ast}( \Mor(\calC, \calX'), \calX')$ satisfying the condition $(\ast)$.
For each $C \in \calC$, let $e'(C)_{\ast}$ denote the corresponding geometric morphism
from $\Mor(\calC, \calX')$ to $\calX'$, let $e'(C)^{\ast}$ denote a left adjoint to
$e'(C)_{\ast}$, and let $S(C) = e'(C)^{\ast} S$ be the image of $S$ in the collection of morphisms
of $\Mor(\calC, \calX')$. Since each $e'(C)^{\ast}$ is a colimit-preserving functor, each of the sets 
$S(C)$ is generated under colimits by a small collection of morphisms in $\Mor(\calC, \calX')$. Let $T$ be the smallest collection of morphisms in
$\Mor(\calC, \calX')$ which is strongly saturated, stable under pullbacks, and contains each of the sets $S_{C}$. Using Corollary \ref{sweetums}, we conclude that $T$ is generated (as a strongly saturated class of morphisms) by a small collection of morphisms in $\Mor(\calC, \calX')$. It follows that
$\Mor(\calC, \calX) = T^{-1} \Mor( \calC, \calX')$ is an $\infty$-topos. By construction,
the map $e'$ restricts to give a functor
$e: \calC \rightarrow \Fun_{\ast}( \Mor( \calC, \calX), \calX)$. Unwinding the definitions, we see that
$e$ has the desired properties.
\end{proof}

\begin{remark}\label{sablewise}
Let $\calX$ be an $\infty$-topos, and let $\RGeom^{\Delta}$ denote the simplicial
subcategory of $\widehat{\Cat}_{\infty}^{\Delta}$ corresponding to the subcategory
$\RGeom \subseteq \widehat{\Cat}_{\infty}$, so that
$\RGeom \simeq \Nerve( \RGeom^{\Delta})$. The construction 
$\calY \mapsto \Fun_{\ast}( \calX, \calY)$ determines a simplicial functor from
$\RGeom^{\Delta}$ to $\widehat{\Cat}_{\infty}^{\Delta}$, which in turn induces a functor
$$\theta_{\calX}: \RGeom \rightarrow \widehat{\Cat}_{\infty}.$$
We claim that $\theta_{\calX}$ preserves small limits (this translates
into the condition that limits in $\RGeom$ really give {\em $\infty$-bicategorical limits} in the
$\infty$-bicategory of $\infty$-topoi).

To prove this, fix an arbitrary $\infty$-category $\calC$, and let $e_{\calC}: \widehat{\Cat}_{\infty} \rightarrow \widehat{\SSet}$ be the functor corepresented by $\calC$. It will suffice to show
that $e_{\calC} \circ \theta_{\calX}$ preserves small limits. The collection of all $\infty$-categories $\calC$ which satisfy this condition is stable under all colimits, so we may assume without loss of generality that $\calC$ is small. It now suffices to observe that $e_{\calC} \circ \theta_{\calX}$ is
equivalent to the functor corepresented by the $\infty$-topos $\Fun(\calC, \calX)$.
\end{remark}

\subsection{\'{E}tale Morphisms of $\infty$-Topoi}\label{gemor2}

Let $f: X \rightarrow Y$ be a continuous map of topological spaces.
We say that $f$ is {\it \'{e}tale} (or a {\it local homeomorphism}) if, for
every point $x \in X$, there exist open sets $U \subseteq X$
containing $x$ and $V \subseteq Y$ containing $f(x)$ such that
$f$ induces a homeomorphism $U \rightarrow V$.  
Let $\calF$ denote the sheaf of sections of $f$: that is, $\calF$ is a sheaf of
sets on $Y$ such that for every open set $V \subseteq Y$,
$\calF(V)$ is the set of all continuous maps $s: V \rightarrow X$ such that
$f \circ s = \id: V \rightarrow Y$. The construction
$(f: X \rightarrow Y) \mapsto \calF$ determines an equivalence of
categories, from the category of topological spaces which
are \'{e}tale over $Y$ to the category of sheaves (of sets) on $Y$.
In particular, we can recover the topological space $X$ (up to homeomorphism) from the sheaf of sets $\calF$ on $Y$. For example, we
can reconstruct the category $\Shv_{\Set}(X)$ of sheaves on $X$
as the overcategory $\Shv_{\Set}(Y)_{/ \calF}$.

Our goal in this section is to develop an analogous theory of \'{e}tale morphisms in the setting of $\infty$-topoi. Suppose given a geometric morphism
$f_{\ast}: \calX \rightarrow \calY$. Under what circumstances should we say that
$f_{\ast}$ is \'{e}tale? By analogy with the case of topological spaces, we should expect that an \'{e}tale morphism determines a ``sheaf'' on $\calY$: that is, an object $U$ of the $\infty$-category $\calY$. Moreover, we should then be able to recover the $\infty$-category $\calX$ as an overcategory $\calY_{/U}$. 
The following result guarantees that this expectation is somewhat reasonable:

\begin{proposition}\label{generalslice}\index{gen}{overcategory!of an $\infty$-topos}
Let $\calX$ be an $\infty$-topos, and let $U$ be an object of $\calX$.
\begin{itemize}
\item[$(1)$] The $\infty$-category $\calX_{/U}$ is an $\infty$-topos.
\item[$(2)$] The projection $\pi_{!}: \calX_{/U} \rightarrow \calX$ has a right adjoint $\pi^{\ast}$ which
commutes with colimits. Consequently, $\pi^{\ast}$ itself has a right adjoint $\pi_{\ast}: \calX_{/U} \rightarrow \calX$, which is a geometric morphism of $\infty$-topoi.
\end{itemize}
\end{proposition}

\begin{proof}
The existence of a right adjoint $\pi^{\ast}$ to the projection $\pi_{!}: \calX_{/U} \rightarrow \calX$ follows from the assumption that $\calX$ admits finite limits. Moreover, the assertion that $\pi^{\ast}$ preserves colimits is a special case of the assumption that colimits in $\calX$ are universal. This proves $(2)$.

To prove $(1)$, we will show that $\calX_{/U}$ satisfies criterion $(2)$ of Theorem \ref{mainchar}. 
We first observe that $\calX_{/U}$ is presentable (Proposition \ref{slicstab}).
Let $K$ be a small simplicial set, and let $\overline{\alpha}: \overline{p} \rightarrow \overline{q}$
be a natural transformation of diagrams $\overline{p}, \overline{q}: K^{\triangleright}
\rightarrow \calX_{/U}$. Suppose that $\overline{q}$ is a colimit diagram, and that
$\alpha = \overline{\alpha} | K$ is a Cartesian transformation. The projection
$\pi_{!}$ preserves all colimits (since it is a left adjoint), so that $\pi_{!} \circ \overline{q}$ is a colimit
diagram in $\calX$. Since $\pi_{!}$ preserves pullback squares (Proposition \ref{goeselse}), $\pi_{!} \circ \alpha$
is a Cartesian transformation. By assumption, $\calX$ is an $\infty$-topos, so that
Theorem \ref{mainchar} implies that $\pi_{!} \circ \overline{p}$ is a colimit diagram
if and only if $\pi_{!} \circ \overline{\alpha}$ is a Cartesian transformation. Using
Propositions \ref{goeselse} and \ref{needed17}, we conclude
that $\overline{p}$ is a colimit diagram if and only if $\overline{\alpha}$ is a Cartesian transformation, as desired.
\end{proof}

A geometric morphism $f_{\ast}: \calX \rightarrow \calY$ of $\infty$-topoi is said to be {\it \'{e}tale}\index{gen}{\'{e}tale morphism} if it arises via the construction of Proposition \ref{generalslice}; that is, if $f$ admits a factorization
$$ \calX \stackrel{f'_{\ast}}{\rightarrow} \calY_{/U} \stackrel{f''_{\ast}}{\rightarrow} \calY$$ where
$U$ is an object of $\calY$, $f'_{\ast}$ is a categorical equivalence, and $f''_{\ast}$ is a right adjoint to the pullback functor ${f''}^{\ast}: \calY \rightarrow \calY_{/U}$. We note that in this case, $f^{\ast}$ has a {\em left adjoint} $f_{!} = f''_{!} \circ f'_{\ast}$. Consequently, $f^{\ast}$ preserves {\em all} limits, not just finite limits. 

\begin{remark}\label{mark}
Given an \'{e}tale geometric morphism $f: \calX_{/U} \rightarrow \calX$ of $\infty$-topoi, the
description of the pushforward functor $f_{\ast}$ is slightly more complicated than that of $f_{!}$ (which is merely the forgetful functor) or $f^{\ast}$ (which is given by taking products with $U$).
Given an object $p: X \rightarrow U$ of $\calX_{/U}$, the pushforward $f_{\ast} X$ is an object of
$\calX$ which represents the functor ``sections of $p$''.
\end{remark}

The collection of \'{e}tale geometric morphisms contains all equivalences and is stable under composition. Consequently, we can consider the subcategory $\RGeom_{\mathet} \subseteq \RGeom$\index{not}{RGeomet@$\RGeom_{\mathet}$} containing all objects of $\RGeom$, whose morphisms are precisely the \'{e}tale geometric morphisms. Our goal in this section is to study the $\infty$-category $\RGeom_{\mathet}$. Our main results are the following:

\begin{itemize}
\item[$(a)$] If $\calX$ is an $\infty$-topos containing an object $U$, then the associated
\'{e}tale geometric morphism $\pi_{\ast}: \calX_{/U} \rightarrow \calX$ can be described by a
universal property. Namely, $\calX_{/U}$ is universal among $\infty$-topoi $\calY$
with a geometric morphism $\phi_{\ast}: \calY \rightarrow \calX$ such that
$\phi^{\ast} U$ admits a global section (Proposition \ref{goodking}). 

\item[$(b)$] There is a simple criterion for testing whether a geometric morphism
$\pi_{\ast}: \calX \rightarrow \calY$ is \'{e}tale. Namely, $\pi_{\ast}$ is \'{e}tale if and only if
the functor $\pi^{\ast}$ admits a left adjoint $\pi_{!}$, the functor $\pi_{!}$ is conservative, and an appropriate push-pull formula holds in the the $\infty$-category $\calY$ (Proposition \ref{pushpo}).

\item[$(c)$] Given a pair of topological spaces $X_0$ and $X_1$ and a homeomorphism
$\phi: U_0 \simeq U_1$ between open subsets $U_0 \subseteq X_0$ and $U_1 \subseteq X_1$,
we can ``glue'' $X_0$ to $X_1$ along $\phi$ to obtain a new topological space
$X = X_0 \coprod_{U_0} X_1$. Moreover, the topological space $X$ contains
open subsets homeomorphic to $X_0$ and $X_1$. In the setting of $\infty$-topoi, it is possible to make much more general ``gluing'' constructions of the same type. We can formulate this idea more precisely as follows: given any diagram $\{ \calX_{\alpha} \}$ in the $\infty$-category $\RGeom_{\mathet}$ having a colimit $\calX$ in $\RGeom$, each of the associated geometric morphisms $\calX_{\alpha} \rightarrow \calX$ is \'{e}tale (Theorem \ref{prescan}). Using this fact, we will show that the $\infty$-category $\RGeom_{\mathet}$ admits small colimits.
\end{itemize}

\begin{remark}
We will say that a geometric morphism of $\infty$-topoi $f^{\ast}: \calY \rightarrow \calX$ is
\'{e}tale if and only if its right adjoint $f_{\ast}: \calX \rightarrow \calY$ is \'{e}tale. We let
$\LGeom_{\mathet}$ denote the subcategory of $\LGeom$ spanned by the \'{e}tale geometric morphisms, so that there is a canonical equivalence $\RGeom_{\mathet} \simeq \LGeom_{\mathet}^{op}$.\index{not}{Ltopet@$\LGeom_{\mathet}$}
\end{remark}

Our first step is to obtain a more precise formulation of the universal property described in $(a)$:

\begin{definition}
Let $f^{\ast}: \calX \rightarrow \calY$ be a geometric morphism of $\infty$-topoi. Let
$U$ be an object of $\calX$ and $\alpha: 1_{\calY} \rightarrow f^{\ast} U$ a morphism in
$\calY$, where $1_{\calY}$ denotes a final object of $\calY$. We will say that
$\alpha$ {\it exhibits $\calY$ as a classifying $\infty$-topos for sections of $U$}
if, for every $\infty$-topos $\calZ$, the diagram
$$ \xymatrix{ \Fun^{\ast}( \calY, \calZ) \ar[r]^{\circ f^{\ast}} \ar[d]^{\phi} & \Fun^{\ast}(\calX, \calZ) \ar[d]^{\phi_0} \\
\calZ_{\ast} \ar[r] & \calZ }$$
is a homotopy pullback square of $\infty$-categories. Here $\calZ_{\ast}$ denotes the
$\infty$-category of pointed objects of $\calZ$ (that is, the full subcategory of
$\Fun(\Delta^1, \calZ)$ spanned by morphisms $f: Z \rightarrow Z'$ where $Z$ is a final object
of $\calZ$), and the morphisms $\phi$ and $\phi_0$ are given by evaluation on $\alpha$ and $U$, respectively.
\end{definition}

Let $\calX$ be an $\infty$-topos containing an object $U$. It follows immediately from the definition that
a classifying $\infty$-topos for sections of $U$ is uniquely determined up to equivalence, provided that it exists. For the existence, we have the following result:

\begin{proposition}\label{goodking}
Let $\calX$ be an $\infty$-topos containing an object $U$, let
$\pi_{!}: \calX_{/U} \rightarrow \calX$ be the projection map, and let
$\pi^{\ast}: \calX \rightarrow \calX_{/U}$ be a right adjoint to $\pi_{!}$. 
Let $1_{U}$ denote the identity map from $U$ to itself, regarded as a
$($final$)$ object of $\calX_{/U}$, and let $\alpha: 1_{U} \rightarrow \pi^{\ast} U$ be
adjoint to the identity map $\pi_{!} 1_{U} \simeq U$. Then $\alpha$ exhibits
$\calX_{/U}$ as a classifying $\infty$-topos for sections of $U$.
\end{proposition}

Before giving the proof of Proposition \ref{goodking}, we summarize some of the pleasant consequences.

\begin{corollary}\label{goodelk}
Let $\calX$ be an $\infty$-topos containing an object $U$, and let
$\pi^{\ast}: \calX \rightarrow \calX_{/U}$ be the corresponding \'{e}tale geometric morphism.
For every $\infty$-topos $\calZ$, composition with $\pi^{\ast}$ induces a left fibration
$$ \Fun^{\ast}( \calX_{/U}, \calZ ) \rightarrow \Fun^{\ast}( \calX, \calZ).$$
Moreover, the fiber over a geometric morphism $\phi^{\ast}: \calX \rightarrow \calZ$
is homotopy equivalent to the mapping space $\bHom_{\calZ}( 1_{\calZ}, \phi^{\ast} U )$. 
\end{corollary}

\begin{remark}\label{goodilk}
Corollary \ref{goodelk} implies in particular the existence of homotopy fiber sequences
$$ \bHom_{\calZ}( 1_{\calZ}, \phi^{\ast} U ) \rightarrow \bHom_{ \LGeom }( \calX_{/U}, \calZ)
\rightarrow \bHom_{\LGeom}(\calX, \calZ)$$
(where the fiber is taken over a geometric morphism $\phi^{\ast} \in \bHom_{\LGeom}(\calX, \calZ)$).

Suppose that $\calZ = \calX_{/V}$, and that $\phi^{\ast}$ is a right adjoint to the projection
$\calX_{/V} \rightarrow \calZ$. We then deduce the existence of a canonical homotopy equivalence
$$ \bHom_{\calX}(V,U) \simeq \bHom_{\calZ}( 1_{\calZ}, \phi^{\ast} U ) \simeq
\bHom_{ \LGeom_{\calX/}}( \calX_{/U}, \calX_{/V} ).$$
\end{remark}

\begin{remark}\label{pusha}
It follows from Remark \ref{goodilk} that if $f^{\ast}: \calX \rightarrow \calY$ is a geometric morphism
of $\infty$-topoi and $U \in \calX$ is an object, then the induced diagram
$$ \xymatrix{ \calX \ar[r] \ar[d] & \calY \ar[d] \\
\calX_{/U} \ar[r] & \calY_{/ f^{\ast} U } }$$
is a pushout square in $\LGeom$.
\end{remark}


\begin{corollary}\label{toadscan}
Suppose given a commutative diagram
$$ \xymatrix{ & \calY \ar[dr]^{g_{\ast}} & \\
\calX \ar[ur]^{f_{\ast}} \ar[rr]^{h_{\ast}} & & \calZ }$$
in $\LGeom^{op}$, where $g_{\ast}$ is \'{e}tale. Then $f_{\ast}$ is \'{e}tale if and only if $h_{\ast}$ is \'{e}tale.
\end{corollary}

\begin{proof}
The ``only if'' direction is obvious. To prove the converse, let us suppose
that $g_{\ast}$ and $h_{\ast}$ are both \'{e}tale, so that we have
equivalences $\calX \simeq \calZ_{/U}$ and $\calU \simeq \calZ_{/V}$ for some pair
of objects $U, V \in \calZ$. 
Using Remark \ref{goodilk}, we deduce that the morphism $f_{\ast}$ is determined by
a map $U \rightarrow V$ in $\calZ$, which we can identify with an object $\overline{V} \in \calY$
such that $\calX \simeq \calY_{/ \overline{V} }$. 
\end{proof}

\begin{remark}\label{postit}
Let $\calX$ be an $\infty$-topos. The projection map
$$p: \Fun( \Delta^1, \calX) \rightarrow \Fun( \{1\}, \calX) \simeq \calX$$
is a Cartesian fibration. Moreover, for every morphism $\alpha: U \rightarrow V$ in
$\calX$, the associated functor $\alpha^{\ast}: \calX^{/V} \rightarrow \calX^{/U}$ is an \'{e}tale 
geometric morphism of $\infty$-topoi, so that $p$ is classified by a functor
$\chi_0: \calX^{op} \rightarrow \LGeom_{\mathet}$. The functor $\chi_0$ carries the final object
of $\calX$ to an $\infty$-topos equivalent to $\calX$, and therefore factors as a composition
$$ \calX^{op} \stackrel{\chi}{\rightarrow} (\LGeom_{\mathet})_{\calX/} \rightarrow \LGeom_{\mathet}.$$
The argument of Remark \ref{goodilk} shows that $\chi$ is fully faithful, and
it follows immediately from the definitions that $\chi$ is essentially surjective.
Corollary \ref{toadscan} allows us to identify $(\LGeom_{\mathet})_{\calX/}$ with
the full subcategory of $\LGeom_{\calX/}$ spanned by the \'{e}tale geometric morphisms
$f^{\ast}: \calX \rightarrow \calY$. Consequently, we can regard $\chi$ as a fully faithful embedding
of $\calX$ into the $\infty$-category $(\LGeom^{op})_{/\calX}$ of $\infty$-topoi over $\calX$, whose essential image consists of those $\infty$-topoi which are \'{e}tale over $\calX$.
\end{remark}

%\begin{remark}
%Let $\calX$ be an $\infty$-topos. Then Remark \ref{postit} implies that we can identify
%\'{e}tale geometric morphisms $\pi_{\ast}: \calY \rightarrow \calX$ with objects $U \in \calX$, so that
%$\calY \simeq \calX_{/U}$. Suppose that $\calX$ is $n$-localic, for some $n \geq 0$. Then
%$\calY$ is $n$-localic if and only if the object $U \in \calX$ is $n$-truncated (Lemma \ref{tub}). 

%In the special case $n=0$, we see that $0$-localic $\infty$-topoi which are \'{e}tale over $\calX$ can be identified with discrete objects of $\calX$. If $\calX$ has enough points, then
%$\calX \simeq \Shv(X)$ for some topological space $X$. We then recover the identification
%between sheaves of sets on $X$ (that is, discrete objects of $\calX$) and topological spaces $Y$ equipped with a local homeomorphism $Y \rightarrow X$.
%\end{remark}

\begin{proof}[Proof of Proposition \ref{goodking}]
Let $p: \calM \rightarrow \Delta^1$ be a correspondence from $\calX_{/U} \simeq \calM \times_{ \Delta^1} \{0\}$ to $\calX \simeq \calM \times_{ \Delta^1} \{1\}$ associated to the adjoint functors
$$ \Adjoint{ \pi_{!} }{ \calX_{/U} }{\calX}{\pi^{\ast}.}$$
Let $\alpha_0$ be a morphism from $1_{U} \in \calX_{/U}$ to $1_{\calX} \in \calX$
in $\calM$ (so that $\alpha_0$ is determined uniquely up to homotopy). We observe
that there is a retraction $r: \calM \rightarrow \calX_{/U}$ which restricts to $\pi^{\ast}$
on $\calX \subseteq \calM$, and we can identify $\alpha$ with $r( \alpha_0 )$. 

Let $\calZ$ be an arbitrary $\infty$-topos. Let $\calC$ be the full subcategory of
$\Fun( \calM, \calZ)$ spanned by those functors $F: \calM \rightarrow \calZ$ with the following properties:
\begin{itemize}
\item[$(a)$] The restriction $F | \calX_{/U}: \calX_{/U} \rightarrow \calZ$ preserves small colimits and finite limits.
\item[$(b)$] The functor $F$ is a left Kan extension of $F| \calX_{/U}$. In other words, 
$F$ carries $p$-Cartesian morphisms in $\calM$ to equivalences in $\calZ$.
\end{itemize}
Proposition \ref{lklk} implies that the restriction map
$\calC \rightarrow \Fun^{\ast}( \calX_{/U}, \calZ)$ is a trivial Kan fibration. Moreover,
this trivial Kan fibration has a section given by composition with $r$. It will therefore suffice
to show that the diagram
$$ \xymatrix{ \calC \ar[r] \ar[d] & \Fun^{\ast}( \calX, \calZ) \ar[d] \\
\calZ_{\ast} \ar[r] & \calZ }$$
is a homotopy pullback square. In other words, we wish to show that restriction along
$\alpha_0$ and the inclusion $\calX \subseteq \calM$ induce a categorical equivalence
$\calC \rightarrow \calZ_{\ast} \times_{ \calZ} \Fun^{\ast}( \calX, \calZ)$.

We define simplicial subsets $\calM'' \subseteq \calM' \subseteq \calM$ as follows:
\begin{itemize}
\item[$(i)$] Let $\calM'' \simeq \calX \coprod_{ \{1\} } \Delta^1$ be the union of
$\calX$ with the $1$-simplex of $\calM$ corresponding to the morphism $\alpha_0$.
\item[$(ii)$] Let $\calM'$ be the full subcategory of $\calM$ spanned by $\calX$
together with the object $1_U$.
\end{itemize}

We can identify $\calZ_{\ast} \times_{ \calZ} \Fun^{\ast}(\calX, \calZ)$ with the full subcategory
$\calC'' \subseteq \Fun( \calM'', \calZ)$ spanned by those functors $F$ satisfying the following conditions:
\begin{itemize}
\item[$(a')$] The restriction $F| \calX$ preserves small colimits and finite limits.
\item[$(b')$] The object $F( 1_{U} )$ is final in $\calZ$.  
\end{itemize}
Let $\calC'$ be the full subcategory of $\Fun( \calM', \calZ)$ spanned by those functors which
satisfy $(a')$ and $(b')$. To complete the proof, it will suffice to show that the restriction maps
$$ \calC \stackrel{u}{\rightarrow} \calC' \stackrel{v}{\rightarrow} \calC''$$
are trivial Kan fibrations.

We first show that $u$ is a trivial Kan fibration. In view of Proposition \ref{lklk}, it will suffice to prove the following:
\begin{itemize}
\item[$(\ast)$] A functor $F: \calM \rightarrow \calZ$ satisfies $(a)$ and $(b)$ if and only if
it satisfies $(a')$ and $(b')$, and $F$ is a right Kan extension of $F | \calM'$.
\end{itemize}
To prove the ``only if'' direction, let us suppose that $F$ satisfies $(a)$ and $(b)$. Without loss
of generality, we may suppose $F = F_0 \circ r$, where $F_0 = F | \calX_{/U}$. Then 
$F | \calX = F_0 \circ \pi^{\ast}$. Since $F_0$ and $\pi^{\ast}$ both 
preserve small colimits and finite limits, we deduce $(a')$. Condition $(b')$ is an immediate
consequence of $(a)$. We must show that $F$ is a right Kan extension of $F | \calX$.
Unwinding the definitions (and applying Corollary \ref{hollowtt}), we are reduced to showing that for every object $\overline{V} \in \calX_{/U}$ corresponding to a morphism $V \rightarrow U$ in $\calX$, the diagram
$$ \xymatrix{ F(\overline{V}) \ar[r] \ar[d] & F( V) \ar[d] \\
F(1_U) \ar[r] & F(U) }$$
is a pullback square in $\calZ$. Since $F = F_0 \circ r$ and $F_0$ preserves finite limits,
it suffices to show that the square
$$ \xymatrix{ \overline{V} \ar[r] \ar[d] & \pi^{\ast} V \ar[d] \\
1_{U} \ar[r] & \pi^{\ast} U }$$
is a pullback square in $\calX_{/U}$. In view of Proposition \ref{needed17}, it suffices
to observe that the square
$$ \xymatrix{ V \ar[r] \ar[d] & V \times U \ar[d] \\
U \ar[r] & U \times U }$$
is a pullback in $\calX$.

Now let us suppose that $F$ is a right Kan extension of $F_1 = F | \calM'$, and that
$F_1$ satisfies conditions $(a')$ and $(b')$. We first claim that $F$ satisfies $(b)$.
In other words, we claim that for every object $V \in \calX$, the canonical map
$F( \pi^{\ast} V) \rightarrow F(V)$ is an equivalence. Consider the diagram
$$ \xymatrix{ F( \pi^{\ast} V) \ar[r] \ar[d] & F(V \times U) \ar[d] \ar[r] & F(V) \ar[d] \\
F(1_U) \ar[r] & F(U) \ar[r] & F(1_{\calX}). }$$
Since $F$ is a right Kan extension of $F_1$, the left square is a pullback.
Since $F_1$ satisfies $(a)$, the right square is a pullback. Therefore the outer square is a pullback. Condition $(b')$ implies that the lower horizontal composition is an equivalence, so the upper horizontal composition is an equivalence as well.

We now prove that $F$ satisfies $(a)$. Condition $(b')$ implies that the functor
$F_0 = F | \calX_{/U}$ preserves final objects. It will therefore suffice to show that
$F_0$ preserves small colimits and pullback squares.
Since $F$ is a right Kan extension of $F_1$, the
functor $F_0$ can be described by the formula
$$ V \mapsto F( \pi_{!} V) \times_{ F(U) } F(1_U).$$
It therefore suffices to show that the functors $\pi_{!}$, $F|\calX$, and
$\bigdot \times_{ F(U)} F(1_U)$ preserve small colimits and pullback squares.
For $\pi_{!}$, this follows from Propositions \ref{needed17} and \ref{goeselse}.
For $F|\calX$, we invoke assumption $(a')$. For the functor $\bigdot \times_{ F(U)} F(1_{U})$, we
invoke our assumption that $\calZ$ is an $\infty$-topos (so that colimits in $\calZ$ are universal).
This completes the verification that $u$ is a trivial Kan fibration.

To complete the proof, we must show that the functor $v$ is a trivial Kan fibration.
We note that $v$ fits into a pullback diagram
$$ \xymatrix{ \calC' \ar[r]^{v} \ar[d] & \calC'' \ar[d] \\
\Fun( \calM', \calZ) \ar[r]^{v'} & \Fun( \calM'', \calZ ). }$$
It will therefore suffice to show that $v'$ is a trivial Kan fibration. Since
$\calZ$ is an $\infty$-category, we need only show that the inclusion
$\calM'' \subseteq \calM'$ is a categorical equivalence of simplicial sets. This is a special case of
Proposition \ref{simplexplay}.
\end{proof}

%We say that a geometric morphism of $\infty$-topoi $f^{\ast}: \calX \rightarrow \calY$ is
%{\it \'{e}tale} if there exists an object $U \in \calX$ and a morphism $\alpha: 1_{\calX} \rightarrow f^{\ast} U$ which exhibits $\calY$ as a classifying $\infty$-topos for sections of $U$. In this case, Proposition
%\ref{goodking} implies that $U$ is canonically determined by $f^{\ast}$.
%Namely, if $f^{\ast}$ is \'{e}tale, then it admits a left adjoint $f_{!}$ and $\alpha$ is adjoint to an equivalence $f_{!} 1_{\calY} \simeq U$.

We next establish the recognition principle promised in $(b)$:

\begin{proposition}\label{pushpo}
Let $f^{\ast}: \calX \rightarrow \calY$ be a geometric morphism of $\infty$-topoi. Then
$f^{\ast}$ is \'{e}tale if and only if the following conditions are satisfied:
\begin{itemize}
\item[$(1)$] The functor $f^{\ast}$ admits a left adjoint $f_{!}$ $($in view of Corollary \ref{adjointfunctor}, this is equivalent to the assumption that $f^{\ast}$ preserves small limits$)$.
\item[$(2)$] The functor $f_{!}$ is conservative. That is, if $\alpha$ is a morphism in $\calY$ such
that $f_{!} \alpha$ is an equivalence in $\calX$, then $\alpha$ is an equivalence in $\calY$.
\item[$(3)$] For every morphism $X \rightarrow Y$ in $\calX$, every object $Z \in \calY$, and
every morphism $f_{!} Z \rightarrow Y$, the induced diagram
$$ \xymatrix{ f_{!} ( f^{\ast} X \times_{ f^{\ast} Y } Z) \ar[r] \ar[d] & f_{!} Z \ar[d] \\
X \ar[r] & Y }$$
is a pullback square in $\calX$.
\end{itemize}
\end{proposition}

\begin{remark}
Condition $(3)$ of Proposition \ref{pushpo} can be regarded as a push-pull formula: it provides a canonical equivalence
$$ f_{!}( f^{\ast} X \times_{ f^{\ast} Y} Z) \simeq X \times_{Y} f_{!} Z.$$
In particular, when $Y$ is final in $\calX$, we have an equivalence
$f_{!}( f^{\ast} X \times Z) \simeq X \times f_{!} Z$: in other words, the functor
$f_{!}$ is ``linear'' with respect to the action of $\calX$ on $\calY$.
\end{remark}

\begin{proof}[Proof of Proposition \ref{pushpo}]
Suppose first that $f^{\ast}$ is an \'{e}tale geometric morphism. Without loss of generality, we may suppose that $\calY = \calX_{/U}$, and that $f^{\ast}$ is right adjoint to the forgetful functor
$f_{!}: \calX_{/U} \rightarrow \calX$. Assertions $(1)$ and $(2)$ are obvious, and assertion
$(3)$ follows from the observation that, for every diagram
$$ X \rightarrow Y \leftarrow Z \rightarrow U,$$
the induced map $(X \times U) \times_{ Y \times U } Z \rightarrow X \times_{Y} Z$
is an equivalence in $\calX$.

For the converse, let us suppose that $(1)$, $(2)$ and $(3)$ are satisfied. We wish to show that
$f^{\ast}$ is \'{e}tale. Let $U = f_{!} 1_{\calY}$. Let 
$F$ denote the composition $\calY \simeq \calY_{/1_{\calY}} \stackrel{f_{!}}{\rightarrow} \calX_{/U}$. 
To complete the proof, it will suffice to show that $F$ is an equivalence of $\infty$-categories. Proposition \ref{curpse} implies that $F$ admits a right adjoint $G$, given by the formula
$$(X \rightarrow U) \mapsto f^{\ast} X \times_{ f^{\ast} U} 1_{\calY}.$$
Assumption $(3)$ guarantees that the counit map $v: FG \rightarrow \id_{\calX_{/U}}$ is
an equivalence. To complete the proof, it suffices to show that for each $Y \in \calY$, the
unit map $u_Y: Y \rightarrow GFY$ is an equivalence. The map $Fu_Y: FY \rightarrow FGFY$
has a left homotopy inverse (given by $v_{FY}$) which is an equivalence, so that $F u_{Y}$ is
an equivalence. It follows that $f_{!} u_Y$ is an equivalence, so that $u_Y$ is an equivalence
by virtue of assumption $(2)$. Thus $G$ is a homotopy inverse to $F$, so that $F$ is an equivalence of $\infty$-categories as desired.
\end{proof}

Our final goal in this section is to prove the following result:

\begin{theorem}\label{prescan}
The $\infty$-category $\RGeom_{\mathet}$ admits small colimits, and the inclusion
$\RGeom_{\mathet} \subseteq \RGeom$ preserves small colimits.
\end{theorem}

The proof of Theorem \ref{prescan} is rather technical and will occupy our attention for the remainder of this section. However, the analogous result
is elementary if we work with $\infty$-topoi which are assume to be \'{e}tale over a fixed
base $\calX$. In this case, Theorem \ref{prescan} can be reduced (with the aid of Remark \ref{postit}) to the following assertion:

\begin{proposition}\label{toadsteal}
Let $\calX$ be an $\infty$-topos, and let $\chi: \calX \rightarrow \LGeom^{op}_{/ \calX}$ be
the functor of Remark \ref{postit}. Then $\chi$ preserves small colimits.
\end{proposition}

\begin{proof}
Combine Propositions \ref{needed17}, \ref{colimtopoi}, and Theorem \ref{charleschar}.
\end{proof}

\begin{proof}[Proof of Theorem \ref{prescan}]
As a first step, we establish the following:
\begin{itemize}
\item[$(\ast)$] Suppose given a small diagram $p: K \rightarrow \LGeom^{op}$ 
and a geometric morphism of $\infty$-topoi $\phi_{\ast}: \colim(p) \rightarrow \calY$. Suppose
further that for each vertex $v$ in $K$, the induced geometric morphism
$\phi(v)_{\ast}: p(v) \rightarrow \calY$ is \'{e}tale. Then $\phi_{\ast}$ is \'{e}tale.
\end{itemize}
To prove $(\ast)$, we note that $\phi_{\ast}$ determines a functor
$\overline{p}: K \rightarrow \LGeom^{op}_{/\calY}$ lifting $p$. Since
each $\phi(v)_{\ast}$ is \'{e}tale, Remark \ref{postit} implies that 
$\overline{p}$ factors as a composition
$$ K \stackrel{q}{\rightarrow} \calY \stackrel{\chi}{\rightarrow} \LGeom^{op}_{/\calY}.$$
Let $U \in \calY$ be a colimit of the diagram $q$. Then Corollary \ref{toadscan} implies that
$\colim(p) \simeq \calY_{/U}$, so that $\phi_{\ast}$ is \'{e}tale as desired.

We now return to the proof of Theorem \ref{prescan}. Using Proposition \ref{appendixdiagram} and its proof, we can reduce the proof to the following special cases:
\begin{itemize}
\item[$(a)$] The $\infty$-category $\LGeom^{op}_{\mathet}$ admits small coproducts, and
the inclusion $\LGeom^{op}_{\mathet} \subseteq \LGeom^{op}$ preserves small coproducts.
\item[$(b)$] The $\infty$-category $\LGeom^{op}_{\mathet}$ admits coequalizers, and the inclusion
$\LGeom^{op}_{\mathet} \subseteq \LGeom^{op}$ preserves coequalizer diagrams.
\end{itemize}

We first prove $(a)$. In view of $(\ast)$, it will suffice to prove the following:
\begin{itemize}
\item[$(a')$] Let $\{ \calX_{\alpha} \}$ be a small collection of $\infty$-topoi, and let
$\calX$ be their coproduct in $\LGeom^{op}$ (so that we have an equivalence of $\infty$-categories $\calX \simeq \prod_{\alpha} \calX_{\alpha}$). Then each of the associated geometric morphisms
$\phi_{\alpha}^{\ast}: \calX \rightarrow \calX_{\alpha}$ is \'{e}tale.
\end{itemize}
To prove $(a')$, we may assume without loss of generality that $\calX = \prod_{\alpha} \calX_{\alpha}$
and that $\phi_{\alpha}^{\ast}$ is given by projection onto the corresponding factor. The desired
result then follows from the criterion of Proposition \ref{pushpo} (more concretely: let
let $U \in \calX$ be an object whose image in $\calX_{\alpha}$ is
a final object $U_{\alpha} \in \calX_{\alpha}$, and whose image in $\calX_{\beta}$ is an initial object 
$U_{\beta} \in \calX_{\beta}$ for $\beta \neq \alpha$. Then $\calX_{/U} \simeq \prod_{\beta} (\calX_{\beta})_{/U_{\beta}} \simeq \calX_{\alpha}$.) 

To prove $(b)$, we can again invoke $(\ast)$ to reduce to the following assertion:
\begin{itemize}
\item[$(b')$] Suppose given a diagram
$$\xymatrix{ \calY \ar@<.4ex>[r] \ar@<-.4ex>[r] & \calX_0}.$$
in $\LGeom_{\mathet}^{op}$, having colimit $\calX$ in $\LGeom^{op}$. Then
the induced geometric morphism $\phi_{\ast}: \calX_0 \rightarrow \calX$ is \'{e}tale.
\end{itemize}

To prove $(b')$, we identify the diagram in question with a functor
$p: \calI \rightarrow \LGeom^{op}$, and $\calI$ with the subcategory of
$\Nerve(\cDelta)^{op}$ spanned by the objects $\{ [0], [1] \}$ and injective maps of linearly ordered sets.
Let $\calX_{\bigdot}$ be the simplicial object of $\LGeom^{op}$ given by left Kan extension along
the inclusion $\calI \subseteq \Nerve(\cDelta)^{op}$, so that each $\calX_{n}$ is equivalent to a coproduct (in $\LGeom^{op}$) of $\calX_0$ with $n$ copies of $\calY$. Using $(\ast)$ and Corollary \ref{toadscan}, we deduce that $\calX_{\bigdot}$ is a simplicial object in $\LGeom^{op}_{\mathet}$.
Consequently, assertion $(b')$ is an immediate consequence of Lemma \ref{kan0} and
the following:
\begin{itemize}
\item[$(b'')$] Let $\calX_{\bigdot}$ be a simplicial object of $\LGeom^{op}_{\mathet}$, and let
$\calX$ be its geometric realization in $\LGeom^{op}$. Then the induced geometric morphism
$\phi_{\ast}: \calX_0 \rightarrow \calX$ is \'{e}tale.
\end{itemize}

The proof of $(b'')$ is based on the following Lemma, whose proof we defer until the end of this section:

\begin{lemma}\label{santan}
Suppose given a simplicial object $\calX_{\bigdot}$ in $\LGeom^{op}_{\mathet}$. Then
there exists a morphism of simplicial objects $\calX_{\bigdot} \rightarrow \calX'_{\bigdot}$
of $\LGeom^{op}_{\mathet}$ with the following properties:
\begin{itemize}
\item[$(1)$] The induced map $\calX_0 \rightarrow \calX'_0$ is an equivalence of $\infty$-topoi.
\item[$(2)$] The simplicial object $\calX'_{\bigdot}$ is a groupoid object in $\LGeom^{op}$.
\item[$(3)$] The induced map of geometric realizations $($in $\LGeom^{op}${}$)$ is an equivalence
$| \calX_{\bigdot} | \rightarrow | \calX'_{\bigdot} |$.
\end{itemize}
\end{lemma}

Using Lemma \ref{santan}, we can reduce the proof of $(b'')$ to the special case where
$\calX_{\bigdot}$ is a groupoid object of $\LGeom^{op}$. The diagram
$$ \Nerve( \cDelta )^{op} \stackrel{\calX_{\bigdot}}{\rightarrow} \LGeom^{op}_{\mathet} 
\subseteq \widehat{\Cat}_{\infty}^{op}$$
is classified by a Cartesian fibration $q: \calZ \rightarrow \Nerve( \cDelta)^{op}$.
Here we can identify $\calX_{n}$ with the fiber $\calZ_{[n]} = \calZ \times_{ \Nerve(\cDelta)^{op} } \{ [n] \}$, and every map of linearly ordered sets $\alpha: [m] \rightarrow [n]$ induces a geometric morphism $\alpha^{\ast}: \calZ_{[m]} \rightarrow \calZ_{[n]}$. Since the geometric morphism $\alpha^{\ast}$ is \'{e}tale, it admits a left adjoint $\alpha_{!}$, so that $q$ is also a coCartesian fibration (Corollary \ref{grutt1}).

It follows from Propositions \ref{colimtopoi} and \ref{charcatlimit} that we can identify
$\calX$ with the full subcategory of $\Fun_{ \Nerve(\cDelta)^{op} } ( \Nerve(\cDelta)^{op}, \calZ)$ spanned by the Cartesian sections of $q$; under this identification, the pullback functor
$\phi^{\ast}$ corresponds to the functor $\calX \rightarrow \calZ_{[0]} \simeq \calX_0$ given by evaluation at $[0]$. 

Let $1_{\calX}$ denote a final object of $\calX$, which we regard as a section of $q$. 
Let $T: \Nerve(\cDelta)^{op} \rightarrow \Nerve(\cDelta)^{op}$ denote the shift functor
$[n] \mapsto [n] \star [0]$, and let $\beta_0: T \rightarrow \id_{ \Nerve(\cDelta)^{op} }$ denote the evident natural transformation. Let $\beta: (1_{\calX} \circ T) \rightarrow U_{\bigdot}$ be a
natural transformation in $\Fun( \Nerve(\cDelta)^{op}, \calZ)$ lifting $\beta$ which is
$q$-coCartesian. Since $\calX_{\bigdot}$ is a groupoid object of $\LGeom^{op}_{\mathet}$, we deduce that $U_{\bigdot}$ is a Cartesian section of $q$, which we can identify with an object of
$\calX$.

Let $S = \Nerve(\cDelta)^{op} \times \Delta^1$, so that $\beta_0$ defines a map
$S \rightarrow \Nerve(\cDelta)^{op}$. Let $\calZ' = \calZ \times_{ \Nerve(\cDelta)^{op} } S$
and let $\beta_{S} = \beta$, regarded as a section of of the projection $q': \calZ' \rightarrow S$.
Let $\calZ'' = {\calZ'}^{/\beta_S}$ (see \S \ref{consweet} for an explanation of this notation). 
Let $q'': \calZ'' \rightarrow S$. The fibers of $q''$ can be described as follows:
\begin{itemize}
\item The fiber of $q''$ over $([n], 0)$ can be identified with
$\calZ_{[n+1]}^{/ 1_{\calZ_{[n+1]}}} \simeq \calZ_{[n+1]}$.
\item The fiber of $q''$ over $([n], 1)$ can be identified with
$\calZ_{[n]}^{/U_{n}} \simeq \calZ_{[n+1]}$. 
\end{itemize}
Proposition \ref{colimfam} implies that the projection $q'': \calZ'' \rightarrow S$ is a coCartesian fibration, classified by a map $\chi: S \rightarrow \widehat{\Cat}_{\infty}$. The above description
shows that $\chi$ can be regarded as an equivalence from $\chi^0 = \chi | \Nerve(\cDelta)^{op} \times \{0\}$ to $\chi^1 = \chi | \Nerve(\cDelta)^{op} \times \{1\}$ in the $\infty$-category of simplicial objects
of $\widehat{\Cat}_{\infty}$. Moreover, the functor $\chi^0$ classifies the pullback of
the coCartesian fibration $q$ by the translation map $T: \Nerve(\cDelta)^{op} \rightarrow
\Nerve(\cDelta^{op})$, so that $\chi^0$ and $\chi^1$ factor through $\LGeom^{op}_{\mathet}$. 
Lemma \ref{bclock} implies that the colimit of $\chi^0$ (hence also of $\chi^1$) in
$\LGeom^{op}$ is canonically equivalent to $\calX_0$. On the other hand, Propositions \ref{charcatlimit} and \ref{colimtopoi} allow us to identify $\colim( \chi^1)$ with the $\infty$-category
of Cartesian sections of the projection $\calZ'' \times_{ S} ( \Nerve(\cDelta)^{op} \times \{1\} )
\rightarrow \Nerve(\cDelta)^{op}$, which is isomorphic to $\calX^{/U_{\bigdot}}$ as a simplicial set.
We now complete the proof by observing that the resulting identification
$\calX_0 \simeq \calX^{/U_{\bigdot}}$ is compatible with the projection $\phi_{\ast}: \calX_0 \rightarrow \calX$.
\end{proof}

The remainder of this section is devoted to the proof of Lemma \ref{santan}. We first need to introduce a bit of notation. We begin with a few remarks about the behavior of $\infty$-topoi under change of universe.

\begin{notation}
Let $\calX$ be an $\infty$-topos and $\calC$ an arbitrary $\infty$-category. We let
$\Shv_{\calC}(\calX)$ denote the full subcategory of $\Fun( \calX^{op}, \calC)$ spanned by those functors which preserve small limits.\index{not}{ShvCX@$\Shv_{\calC}(\calX)$} We will refer to
$\Shv_{\calC}(\calX)$ as the {\it $\infty$-category of $\calC$-valued sheaves on $\calX$}.
\end{notation}

\begin{remark}\label{quest}
Let $\calX$ be an $\infty$-topos. Proposition \ref{representable} implies that the Yoneda
embedding $\calX \rightarrow \Shv_{\SSet}(\calX)$ is an equivalence; in other words, we can identify $\calX$ with the $\infty$-category of sheaves of (small) spaces on itself. Let $\widehat{\SSet}$ denote the $\infty$-category of spaces which belong to some larger universe $\calU$. We claim the following:
\begin{itemize}
\item[$(a)$] The $\infty$-category $\Shv_{ \widehat{\SSet} }( \calX)$ can be regarded
as an $\infty$-topos in $\calU$.
\item[$(b)$] The inclusion $\Shv_{\SSet}(\calX) \subseteq \Shv_{\widehat{\SSet}}(\calX)$ preserves small colimits.
\end{itemize}
To prove $(a)$, let us suppose that $\calX = S^{-1} \calP(\calC)$, where
$\calC$ is a small $\infty$-category and $S$ is a strongly saturated class of morphisms
in $\calP(\calC)$, which is stable under pullbacks and of small generation.
Theorem \ref{charpresheaf} and Proposition \ref{unichar} allow us to identify
$\Shv_{\widehat{\SSet}}(\calX)$ with $S^{-1} \widehat{\calP}(\calC)$, where
$\widehat{\calP}(\calC)$ denotes the presheaf $\infty$-category
$\Fun( \calC^{op}, \widehat{\SSet})$. Let $\widehat{S}$ denote the strongly saturated class of morphisms of $\widehat{\calP}(\calC)$ generated by $S$. Then $\widehat{S}$ is of small generation (and therefore of $\calU$-small generation); to complete the proof of $(a)$ it will suffice to show that $\widehat{S}$ is stable under pullbacks. 

Let $\widehat{\calP}(\calC)^{0}$ denote the full subcategory of $\widehat{\calP}(\calC)$ spanned by those objects $X$ with the following property: 
\begin{itemize}
\item[$(\ast)$] Let
$$ \xymatrix{ Y \ar[d]^{f} \ar[r] & Y' \ar[d]^{f'} \\
X \ar[r] & X' }$$
be a pullback diagram in $\widehat{\calP}(\calC)$. If $f' \in S$, then $f \in \widehat{S'}$.
\end{itemize}
Since colimits in $\widehat{\calP}(\calC)$ are universal, the subcategory
$\widehat{\calP}(\calC)^{0}$ is stable under $\calU$-small colimits in
$\widehat{\calP}(\calC)$. Moreover, since $S$ is stable under pullbacks
in $\calP(\calC)$ (and since the inclusion $\calP(\calC) \subseteq \widehat{\calP}(\calC)$ is fully faithful), the $\infty$-category $\widehat{\calP}(\calC)^{0}$ contains $\calP(\calC)$.
Since $\widehat{\calP}(\calC)$ is generated (under $\calU$-small colimits) by the essential image of the Yoneda embedding $\calC \rightarrow \calP(\calC)$, we conclude that
$\widehat{\calP}(\calC)^{0} = \widehat{\calP}(\calC)$. 

We now let $S'$ denote the collection of all morphisms in $\widehat{\calP}(\calC)$ such that, for every pullback diagram
$$ \xymatrix{ Y \ar[d]^{f} \ar[r] & Y' \ar[d]^{f'} \\
X \ar[r] & X' }$$
in $\calP(\calC)$, if $f' \in S'$ then $f \in \widehat{S}$. The above argument shows that
$S \subseteq S'$. Since $S'$ is strongly saturated, we conclude that $\widehat{S} \subseteq S'$, so that $\widehat{S}$ is stable under pullbacks as desired. This completes the proof of $(a)$.

To prove $(b)$, it will suffice to show that the composite map
$$ \calP(\calC) \rightarrow S^{-1} \calP(\calC) \rightarrow \widehat{S}^{-1} \widehat{\calP}(\calC)$$
preserves small colimits. We can rewrite this as the composition of a pair of functors
$$ \calP(\calC) \stackrel{i}{\rightarrow} \widehat{\calP}(\calC)
\stackrel{L}{\rightarrow} \widehat{S}^{-1} \widehat{\calP}(\calC).$$
The functor $L$ is left adjoint to the inclusion of
$\widehat{S}^{-1} \widehat{\calP}(\calC)$ into $\widehat{\calP}(\calC)$, and therefore
preserves all $\calU$-small colimits. It therefore suffices to show that the inclusion
$i: \Fun( \calC^{op}, \SSet) \rightarrow \Fun( \calC^{op}, \widehat{\SSet})$ preserves small colimits.
In view of Proposition \ref{limiteval}, it will suffice to prove the inclusion
$i_0: \SSet \rightarrow \widehat{\SSet}$ preserves small colimits. We note that
$i_0$ is an equivalence from $\SSet$ to the full subcategory
$\widehat{\SSet}^0 \subseteq \widehat{\SSet}$ spanned by the essentially small spaces.
It now suffices to observe that the collection of essentially small spaces is stable under small colimits
(this follows from Corollaries \ref{apegrape} and \ref{tyrmyrr}).
\end{remark}

\begin{remark}\label{postquest}
Let $\calU$ be a universe as in Example \ref{quest}, let $f^{\ast}: \calX \rightarrow \calY$ be a geometric morphism of $\infty$-topoi, and let
$\widehat{f}_{\ast}: \Shv_{ \widehat{\SSet}}(\calX) \rightarrow \Shv_{ \widehat{\SSet}}(\calY)$ be
given by composition with $f^{\ast}$. Then $\widehat{f}_{\ast}$ can be identified with a geometric morphism in the universe $\calU$. To prove this, let $\kappa$ denote the regular cardinal
in the universe $\calU$ such that small sets (in our original universe) can be identified with
$\kappa$-small sets in $\calU$. It follows from Corollary \ref{indpr} that we can identify
$\Shv_{\widehat{\SSet}}(\calX)$ and $\Shv_{\widehat{\SSet}}(\calX)$ with
$\widehat{\Ind}_{\kappa}(\calX)$ and $\widehat{\Ind}_{\kappa}(\calY)$, respectively.
Proposition \ref{adjobs} implies that $\widehat{f}_{\ast}$ admits a left adjoint
$\widehat{f}^{\ast}$ which fits into a commutative diagram
$$ \xymatrix{ \calX \ar[r]^{f^{\ast}} \ar[r] \ar[d] & \calY \ar[d] \\
\Shv_{\widehat{\SSet}}(\calX) \ar[r]^{ \widehat{f}^{\ast} } & \Shv_{ \widehat{\SSet}}(\calY). }$$
To complete the proof, it will suffice to show that $\widehat{f}^{\ast}$ is left exact.
Since $f^{\ast}$ preserves final objects, the functor $\widehat{f}^{\ast}$ preserves final objects as well.
It therefore suffices to show that $\widehat{f}^{\ast}$ preserves pullback diagrams. Using Proposition \ref{urgh1} and Example \ref{tucka}, we conclude that every pullback diagram
in $\Shv_{ \widehat{\SSet}}(\calX)$ can be obtained as a $\calU$-small, $\kappa$-filtered colimit of pullback diagrams in $\calX$. The desired result now follows from the assumption that $f^{\ast}$ is left exact, and the observation that the class of pullback diagrams in $\Shv_{ \widehat{\SSet}}(\calY)$ is stable under $\calU$-small filtered colimits (Example \ref{tucka}).
\end{remark}

For the remainder of this section, we fix a larger universe $\calU$. Let
$\widehat{\SSet}$ denote the $\infty$-category of $\calU$-small spaces.

\begin{notation}
Let $F: \LGeom \rightarrow \widehat{\SSet}$ be a functor. For every $\infty$-topos
$\calX$, we let $F_{\calX}: \calX^{op} \rightarrow \widehat{\SSet}$ denote the composition
$$ \calX^{op} \simeq \LGeom_{\mathet}^{\calX/} \rightarrow \LGeom \stackrel{F}{\rightarrow} \widehat{\SSet}.$$
We will say that $F_{\calX}$ is a {\it sheaf} if, for every $\infty$-topos $\calX$, the functor
$F_{\calX}$ preserves small limits. We let $\widehat{\Shv}( \LGeom^{op} )$ denote the full subcategory of $\Fun( \LGeom, \widehat{\SSet} )$ spanned by the sheaves.
\end{notation}

\begin{example}\label{stup}
Let $\calX$ be an $\infty$-topos, and let $e_{\calX}: \LGeom \rightarrow \widehat{\SSet}$ be the functor
represented by $\calX$. Proposition \ref{toadsteal} implies that $e_{\calX}$ belongs to
$\widehat{\Shv}( \LGeom^{op} )$. We will say that a sheaf $F \in \widehat{\Shv}( \LGeom^{op} )$ is
{\it representable} if $F \simeq e_{\calX}$ for some $\infty$-topos $\calX$.
\end{example}

\begin{lemma}\label{kumba}
The $\infty$-category $\widehat{\Shv}( \LGeom^{op} )$ is an $\infty$-topos in the universe $\calU$.
Moreover, for every $\infty$-topos $\calX$, the restriction functor
$F \mapsto F_{\calX}$ determines a functor $\widehat{\Shv}(\LGeom^{op}) \rightarrow
\Shv_{ \widehat{\SSet} }(\calX)$ which preserves $\calU$-small colimits and finite limits.
\end{lemma}

\begin{proof}
Let $\Fun^{\mathet}( \Delta^1, \LGeom)$ denote the full subcategory $\Fun( \Delta^1, \LGeom)$ spanned by the \'{e}tale morphisms, and let
$e: \Fun^{\mathet}( \Delta^1, \LGeom) \rightarrow \LGeom$ be given by evaluation at the vertex
$\{0\} \in \Delta^1$. Since the collection of \'{e}tale morphisms in $\LGeom$ is stable under pushouts (Remark \ref{pusha}), the map $e$ is a coCartesian fibration.

We define a simplicial set $\calK$ equipped with a projection
$p: \calK \rightarrow \LGeom$ so that the following universal property is satisfied: for every
simplicial set $K$, we have a natural bijection
$$ \Hom_{\Ind(\calG^{op})}(K, \calK) = \Hom_{\sSet}(
K \times_{ \LGeom } \Fun^{\mathet}(\Delta^1, \LGeom), \widehat{\SSet} ).$$
Then $\calK$ is an $\infty$-category, whose objects can be identified with pairs
$(\calX, F_{\calX})$, where $\calX$ is an $\infty$-topos
and $F_{\calX}: \LGeom^{\calX/}_{\mathet} \rightarrow \widehat{\SSet}$ is a functor.
It follows from Corollary \ref{skinnysalad} that the projection $p$ is a Cartesian fibration, and that
a morphism $(\calX,F_{\calX}) \rightarrow (\calY, F_{\calY})$ is $p$-Cartesian if and only if, for every object $U \in \calX$, the canonical map
$F_{\calX}( \calX_{/U}) \rightarrow F_{\calY}( \calY_{/f^{\ast} U})$ is a homotopy equivalence, where $f^{\ast}$ denotes the underlying geometric morphism from $\calX$ to $\calY$.

Let $\calK_0$ denote the full subcategory of $\calK$ spanned by pairs $(\calX, F_{\calX})$ where
the functor $F_{\calX}$ preserves small limits. It follows from the above that the Cartesian
fibration $p$ restricts to a Cartesian fibration $p_0: \calK_0 \rightarrow \LGeom$ (with the same class of Cartesian morphisms). The fiber of $\calK_0$ over an object $\calX \in \LGeom$ can be identified with
$\Shv_{\widehat{\SSet}}( \calX)$, which is an $\infty$-topos in the universe $\calU$ (Remark \ref{quest}). Moreover, to every geometric morphism $f^{\ast}: \calX \rightarrow \calY$ in
$\LGeom$, the Cartesian fibration $p_0$ associates the pushforward functor $\widehat{f}_{\ast}: \Shv_{ \widehat{\SSet} }(\calY) \rightarrow \Shv_{ \widehat{\SSet} }(\calX)$ given by composition with $f^{\ast}$. It follows from Remark \ref{postquest} that $\widehat{f}_{\ast}$ admits a left adjoint
$\widehat{f}^{\ast}$, and that $\widehat{f}^{\ast}$ is left exact. We may summarize the situation
by saying that $p_0$ is a topos fibration (see Definition \ref{skuzz}); in particular, $p_0$ is a coCartesian fibration.

Let $\calY = \Fun_{ \LGeom}( \LGeom, \calK_0)$ denote the $\infty$-category of sections of $p_0$. Unwinding the definition, we can identify $\calY$ with the $\infty$-category
$\Fun( \Fun^{\mathet}( \Delta^1, \LGeom),\widehat{\SSet})$.
Let $\LGeom'$ denote the essential image of the (fully faithful) diagonal embedding
$\LGeom \rightarrow \Fun( \Delta^1, \LGeom)$. Consider the following conditions on a section $s: \LGeom \rightarrow \calK_0$ of $p_0$:
\begin{itemize}
\item[$(a)$] The functor $s$ carries \'{e}tale morphisms in $\LGeom$ to $p$-coCartesian morphisms in $\calX$.
\item[$(b)$] Let $S: \Fun^{\mathet}( \Delta^1, \LGeom) \rightarrow \widehat{\SSet}$ be the functor
corresponding to $s$. Then, for every commutative diagram
$$ \xymatrix{ & \calY \ar[dr] & \\
\calX \ar[ur] \ar[rr] & & \calZ }$$
of \'{e}tale morphisms in $\LGeom$, the induced map
$S(\calX \rightarrow \calZ) \rightarrow S(\calY \rightarrow \calZ)$ is an equivalence in $\widehat{\SSet}$.
\item[$(c)$] For every \'{e}tale morphism $f^{\ast}: \calX \rightarrow \calY$ in $\LGeom$, the canonical
map $S( \id_{\calX} ) \rightarrow S(f^{\ast})$ is an equivalence in $\widehat{\SSet}$.
\item[$(d)$] The functor $S$ is a left Kan extension of $S | \LGeom'$.
\end{itemize}
Unwinding the definitions, we see that $(a) \Leftrightarrow (b) \Rightarrow (c) \Leftrightarrow (d)$.
Moreover, the implication $(c) \Rightarrow (b)$ follows by a two-out-of-three argument.
Let $\calY'$ denote the full subcategory of $\calY$ spanned by those sections which
satisfy the equivalent conditions $(a)$ through $(d)$; it follows from Proposition \ref{lklk}
that composition with the diagonal embedding $\LGeom \rightarrow \Fun^{\mathet}( \Delta^1, \LGeom)$ induces an equivalence $\theta: \calY' \rightarrow \widehat{\Shv}(\LGeom^{op})$. The desired result now follows from Proposition \ref{prestorkus}.
\end{proof}

To prove Lemma \ref{santan}, we need a criterion which will allow us to detect \'{e}tale geometric morphisms of $\infty$-topoi in terms of the functors that they represent. To formulate this criterion, we introduce a bit of temporary terminology.

\begin{definition}\label{squee}
Let $\alpha: F \rightarrow G$ be a morphism in $\widehat{\Shv}(\LGeom^{op})$. We will say that
$\alpha$ is {\it universal} if, for every geometric morphism of $\infty$-topoi
$f^{\ast}: \calX \rightarrow \calY$, the induced diagram
$$ \xymatrix{ \widehat{f}^{\ast} F_{\calX} \ar[r] \ar[d] & \widehat{f}^{\ast} G_{\calX} \ar[d] \\
F_{\calY} \ar[r] & G_{\calY} }$$
is a pullback square in $\Shv_{ \widehat{\SSet}}( \calY)$. (Here $\widehat{f}^{\ast}$ denotes
the geometric morphism described in Remark \ref{postquest}).
\end{definition}

\begin{remark}
The collection of universal morphisms in $\widehat{\Shv}(\LGeom^{op})$ is stable under pullbacks
and composition, and contains every equivalence in $\widehat{\Shv}(\LGeom^{op})$.
\end{remark}

\begin{remark}\label{sapper}
Let $p: K \rightarrow \widehat{\Shv}(\LGeom^{op})$ be a small diagram having a colimit $F$.
Assume that for every edge $v \rightarrow v'$ of $K$, the induced map $p(v) \rightarrow p(v')$
is universal. Then each of the induced maps $p(v) \rightarrow F$ is universal. This follows immediately from Theorem \ref{charleschar}.
\end{remark}

\begin{lemma}\label{sabus}
Let $\alpha: F \rightarrow G$ be a morphism in $\widehat{\Shv}(\LGeom^{op})$, and assume
that $G$ is representable by an $\infty$-topos $\calX$. Then $F$ is representable by an
$\infty$-topos \'{e}tale over $\calX$ if the following conditions are satisfied:
\begin{itemize}
\item[$(1)$] The morphism $\alpha$ is universal $($in the sense of Definition \ref{squee}$)$.
\item[$(2)$] For every $\infty$-topos $\calY$, the homotopy fibers of the induced map
$F(\calY) \rightarrow G(\calY)$ are essentially small.
\end{itemize}
\end{lemma}

\begin{remark}
In fact, the converse to Lemma \ref{sabus} is true as well, but we will not need this fact.
\end{remark}

\begin{proof}
Choose a point $\eta \in G(\calX)$ which induces an equivalence $e_{\calX} \rightarrow G$.
The map $\eta$ induces a global section $\overline{\eta}$ of $G_{\calX}$ in the $\infty$-topos
$\Shv_{ \widehat{\SSet}}(\calX)$. Let $F_0$ denote the fiber product $F_{\calX} \times_{ G_{\calX} } 1_{ \Shv_{ \widehat{\SSet}}(\calX)}.$
Assumption $(2)$ implies that the functor $F_0: \calX^{op} \rightarrow \widehat{\SSet}$
takes values which are essentially small. It follows from Proposition \ref{representable} that
$F_0$ is representable by an object $U \in \calX$. In particular, we have a tautological point
$\eta' \in F_0(U)$, which determines a commutative diagram
$$ \xymatrix{ e_{ \calX_{/U}} \ar[r] \ar[d] & F \ar[d]^{\alpha} \\
e_{\calX} \ar[r] & G }$$
in $\widehat{\Shv}(\LGeom^{op})$. To complete the proof, it will suffice to show that the upper horizontal map is an equivalence. 

Fix an $\infty$-topos $\calY$, and let $R: \widehat{\Shv}( \LGeom^{op} )
\rightarrow \Shv_{ \widehat{\SSet}}(\calY)$ be the restriction map; we will show that the induced map
$R( e_{\calX_{/U}}) \rightarrow R(F)$ is an equivalence in $\Shv_{ \widehat{\SSet}}(\calY)$. It
will suffice to show that for every $V \in \calY$, the map
$R(e_{\calX_{/U}}(V) \rightarrow R(F)(V)$ induces a homotopy equivalence after passing
to the homotopy fibers over any point $\eta' \in R(G)(V)$. Replacing $\calY$ by
$\calY_{/V}$, we may assume that $\eta'$ is induced by a geometric morphism
$f^{\ast}: \calX \rightarrow \calY$, which determines a map
$1 \rightarrow R(G)$, where $1$ denotes the final object of $\Shv_{\widehat{\SSet}}(\calY)$.
Let $F' = R( e_{\calX_{/U}}) \times_{ R(G)} 1$ and $F'' = R(F) \times_{ R(G) } 1$; to complete the proof it will suffice to show that the induced map $F' \rightarrow F''$ is an equivalence.

We have a commutative diagram
$$ \xymatrix{ F'' \ar[r] \ar[d] & \widehat{f}^{\ast} F_{\calX} \ar[r] \ar[d] & R(F) \ar[d] \\
1 \ar[r]^{ \widehat{f}^{\ast} \overline{\eta} }& \widehat{f}^{\ast} G_{\calX} \ar[r] & R(G) }$$
in the $\infty$-category $\Shv_{ \widehat{\SSet}}(\calY)$. Here the right square is a pullback
since $\alpha$ is universal, and the outer square is a pullback by construction.
It follows that the left square is also a pullback, so that 
$$F'' \simeq \widehat{f}^{\ast} (1_{\calX} \times_{ G_{\calX} } F_{\calX})
\simeq \widehat{f}^{\ast} F_0.$$
We note that $\widehat{f}^{\ast} F_0$ can be identified with the functor represented by
the object $f^{\ast} U \in \calY$, which (by virtue of Remark \ref{goodilk}) is equivalent
to $F'$ as desired. 
\end{proof}

\begin{lemma}\label{guesstimate}
Let $\kappa$ be an uncountable regular cardinal, and let $X_{\bigdot}$ be a simplicial
object of $\SSet$ with the following properties:
\begin{itemize}
\item[$(a)$] For each $n \geq 0$, the connected components of $X_{\bigdot}$ are essentially $\kappa$-small.
\item[$(b)$] For every morphism $[m] \rightarrow [n]$ in $\cDelta$, the induced map
$X_{n} \rightarrow X_{m}$ has essentially $\kappa$-small homotopy fibers.
\end{itemize}
Let $X$ be the geometric realization of $X_{\bigdot}$. Then the induced map
$X_0 \rightarrow X$ has essentially $\kappa$-small homotopy fibers.
\end{lemma}

\begin{proof}
Replacing $X$ by one of its connected components $X'$ (and each $X_{n}$ by the
inverse image $X' \times_{X} X_n$), we may suppose that $X$ is connected.

Let $R \subseteq \pi_0 X_0 \times \pi_0 X_0$ denote the image of $\pi_0 X_1$, and let
$\sim$ denote the equivalence relation on $\pi_0 X_0$ generated by $R$. It follows from
assumption $(b)$ that for every $\kappa$-small subset $A \subseteq \pi_0 X_0$, the
intersections $R \cap (A \times \pi_0 X_0)$ and $R \cap (\pi_0 X_0 \times A)$ are
again $\kappa$-small. Since $\kappa$ is uncountable, it follows that the every $\sim$-equivalence class is $\kappa$-small. Since $(\pi_0 X_0)/ \sim$ is isomorphic to $\pi_0 X \simeq \ast$, 
we conclude that $\pi_0 X$ is itself $\kappa$-small. Combining this with $(a)$, we conclude that $X_0$ is essentially $\kappa$-small. Invoking $(b)$, we deduce that
each $X_{n}$ is essentially $\kappa$-small, so that $X$ is essentially $\kappa$-small. The desired conclusion now follows from the long exact sequences associated to the fibration sequences
$X_0 \times_{X} \{\ast\} \rightarrow X_0 \rightarrow X.$
\end{proof}

\begin{lemma}\label{carbcount}
Let $\calX$ be an $\infty$-topos. Then:
\begin{itemize}
\item[$(1)$] The inclusion $\Shv_{ \widehat{\SSet}}(\calX) \subseteq \Fun( \calX^{op}, \widehat{\SSet})$ admits a left exact left adjoint $L$.
\item[$(2)$] Let $F \in \Fun( \calX^{op}, \widehat{\SSet})$ be a functor such that each of the spaces $F(X)$ is essentially small. Then each of the spaces $LF(X)$ is essentially small. 
\item[$(3)$] Let $\alpha: F \rightarrow G$ be a morphism in $\Fun( \calX^{op}, \widehat{\SSet})$ such that, for each $X \in \calX$, the homotopy fibers of the induced map $F(X) \rightarrow G(X)$ are essentially small. Then for each $X \in \calX$, the homotopy fibers of the map $LF(X) \rightarrow LG(X)$ are also essentially small.
\end{itemize}
\end{lemma}

\begin{proof}
The existence of the left adjoint $L$ follows from Lemma \ref{stur1}. Since
$\Shv_{ \widehat{\SSet}}(\calX)$ contains the essential image of the Yoneda embedding
$j: \calX \rightarrow \Fun( \calX^{op}, \widehat{\SSet})$, we can identify $L \circ j$ with $j$.
Since $j$ is left exact, Proposition \ref{natash} implies that $L$ is also left exact. This proves $(1)$.

We now prove $(2)$. Choose a (small) regular cardinal $\kappa$ such that $\calX$ is $\kappa$-accessible, and let $\calX^{\kappa}$ denote the full subcategory of $\calX$ spanned by the $\kappa$-compact objects. Let $T$ denote the composition
$$ \Fun( \calX^{op}, \widehat{\SSet}) \stackrel{T'}{\rightarrow}
\Fun( (\calX^{\kappa})^{op}, \widehat{\SSet}) \stackrel{T''}{\rightarrow} \Fun( \calX^{op}, \widehat{\SSet}),$$ where $T'$ is the restriction functor and $T''$ is given by the right Kan extension. We have an evident natural transformation $\id \rightarrow T$, which exhibits $T$ as a localization functor
on $\Fun( \calX^{op}, \widehat{\SSet})$. Proposition \ref{sumoto} implies that
every $\widehat{\SSet}$-valued sheaf on $\calX$ is $T$-local. It follows that the canonical map
$L \rightarrow LT$ is an equivalence of functors. In particular, to prove that $LF(X)$ is locally small, 
we may assume without loss of generality that $F$ is $T$-local.

Let $\Fun'( \calX^{op}, \widehat{\SSet})$ denote the full subcategory of $\Fun(\calX^{op}, \widehat{\SSet})$ spanned by the $T$-local functors (in other words, those functors which are right Kan extensions of their restriction to $(\calX^{\kappa})^{op}$; by Proposition \ref{lklk} this
$\infty$-category is equivalent to $\Fun( (\calX^{\kappa})^{op}, \widehat{\SSet} )$). We can identify
$\Fun'( \calX^{op}, \widehat{\SSet} )$ with the $\infty$-category of $\widehat{\SSet}$-valued sheaves
$\Shv_{ \widehat{\SSet}}( \calP( \calX^{\kappa}) )$ on the $\infty$-topos $\calP( \calX^{\kappa})$. Let $F'$
be the image of $F$ under this identification; we observe that the functor
$F': \calP( \calX^{\kappa})^{op} \rightarrow \widehat{\SSet}$ takes essentially small values.
In Remark \ref{quest}, we saw that this $\infty$-category contains $\Shv_{\widehat{\SSet}}( \calX)$ as a left exact localization, and that the localization functor $L': \Shv_{\widehat{\SSet}}(\calP(\calX^{\kappa})) \rightarrow \Shv_{\widehat{\SSet}}( \calX)$. Since $F'$ belongs to the essential image of
the inclusion $\Shv_{ \SSet}( \calP(\calX^{\kappa})) \subseteq \Shv_{ \widehat{\SSet}}( \calP( \calX^{\kappa}))$, the argument given there proves that $L'F'$ belongs to the essential image of
the inclusion $\Shv_{\SSet}(\calX) \subseteq \Shv_{ \widehat{\SSet}}(\calX)$, so that
$LF(X)$ is essentially small as desired.

To prove $(3)$, let us fix a point $\eta \in LG(X)$. We wish to prove the following stronger version of
$(3)$:
\begin{itemize}
\item[$(3')$] For every map $U \rightarrow X$ in $\calX$, the homotopy fiber of the induced map
$LF \rightarrow LG$ is essentially small (here the homotopy fiber is taken over the point determined by $\eta$).
\end{itemize}

Let $\calX^{0}_{/X}$ denote the full subcategory of $\calX_{/X}$ spanned by those morphisms
$U \rightarrow X$ for which condition $(2')$ is satisfied. Since $LF$ and $LG$ belong to
$\Shv_{\widehat{\SSet}}(\calX)$ (and since the collection of essentially small spaces is stable under small limits), we conclude that $\calX^{0}_{/X}$ is stable under small colimits in $\calX_{/X}$. 

Let $\calX^{1}_{/X}$ be the largest sieve contained in $\calX^{0}_{/X}$ (in other words, a morphism $U \rightarrow X$ belongs to $\calX^{1}_{/X}$ if and only if, for every morphism $V \rightarrow U$ in $\calX$, the composite map $V \rightarrow X$ beongs to $\calX^{0}_{/X}$). Since colimits in $\calX$ are universal, we conclude that $\calX^{1}_{/X}$ is stable under small colimits in $\calX_{/X}$. It
follows that $\calX^{1}_{/X} \simeq \calX_{/X_0}$ for some monomorphism
$i: X_0 \rightarrow X$ in $\calX$. We wish to show that $i$ is an equivalence.

Since $L$ is left exact, we have $L( G \times_{LG} j(X) ) \simeq L j(X) \simeq j(X)$.
In particular, the map $G \times_{LG} j(X) \rightarrow j(X_0)$ cannot factor through $j(X_0)$ 
unless $i$ is an equivalence. It will therefore suffice to show that $G \times_{ j(X) } j(X_0) \simeq G$.
In other words, it will suffice to show that if $U \in \calX_{/X}$ and $\eta' \in G(U)$ is a point such that
the images of $\eta$ and $\eta'$ lie in the same connected component of $LG(U)$, then
$U \in \calX^{1}_{/X}$. Since the existence of $\eta'$ is stable under the process of
replacing $U$ by some further refinement $V \rightarrow U$, it will suffice to show that
$U \in \calX^{0}_{/X}$. Replacing $X$ by $U$, we obtain the following reformulation of $(3')$:

\begin{itemize}
\item[$(3'')$] Let $\eta' \in G(X)$. Then the homotopy fiber $Z$ of the induced map
$LF(X) \rightarrow LG(X)$ (over the point determined by $\eta$) is essentially small.
\end{itemize}

Since $L$ is left exact, we can identify $Z$ with $LF_0(X)$, where
$F_0 = F \times_{G} j(X)$. Since the homotopy fibers of the maps $F(Y) \rightarrow G(Y)$ are
essentially small, we may assume without loss of generality that $F_0 \in \Fun( \calX^{op}, \SSet)$. 
Invoking $(2)$, we deduce that the values of $LF_0$ are essentially small as desired.
\end{proof}

\begin{proof}[Proof of Lemma \ref{santan}]
Let $\calX_{\bigdot}$ be a simplicial object of $\LGeom^{op}_{\mathet}$, and let
$F_{\bigdot}$ be its image under the Yoneda embedding 
$j: \LGeom^{op} \rightarrow \widehat{\Shv}( \LGeom^{op} )$. Let $F$ be a geometric realization
of $|F_{\bigdot}|$. We will prove the following:
\begin{itemize}
\item[$(\ast)$] The map $\beta: F_0 \rightarrow F$ satisfies conditions $(1)$ and $(2)$ of Lemma \ref{sabus}.
\end{itemize}
Assuming $(\ast)$ for the moment, we will complete the proof of Lemma \ref{santan}. Let
$F'_{\bigdot}$ be a \Cech nerve of the induced map
$F_0 \rightarrow F$ (so that $F'_{n} \simeq F_0 \times_{F} F_0 \times \ldots \times_{F} F_0$; in particular $F'_0 \simeq F_0$). We first claim that each $F'_{n}$ is representable by an $\infty$-topos $\calX'_{n}$, and that each inclusion $[0] \hookrightarrow [n]$ induces an \'{e}tale map of $\infty$-topos $\calX'_{n} \rightarrow \calX'_0 \simeq \calX_0$. Since $F'_{\bigdot}$ is a groupoid object of $\widehat{\Shv}( \LGeom^{op} )$ (and $F'_0 \simeq F_0$ is representable by the $\infty$-topos $\calX_0$), it will suffice to prove this result when $n=1$. Consider the pullback diagram
$$ \xymatrix{ F'_1 \ar[r]^{\beta'} \ar[d] & F'_0 \ar[d] \\
F_0 \ar[r]^{\beta} & F. }$$
It follows from condition $(\ast)$ that $\beta'$ satisfies conditions $(1)$ and $(2)$ of Lemma \ref{sabus},
so that $F'_1$ is representable by an $\infty$-topos \'{e}tale over $\calX_0$ as desired.

Since the Yoneda embedding $j$ is fully faithful, we may assume without loss of generality
that $F'_{\bigdot}$ is the image under $j$ of a groupoid object $\calX'_{\bigdot}$ of
$\LGeom^{op}$. Using Corollary \ref{toadscan}, we deduce that $\calX'_{\bigdot}$ 
defines a simplicial object of the subcategory $\LGeom^{op}_{\mathet}$. The evident
natural transformation $F_{\bigdot} \rightarrow F'_{\bigdot}$ induces a map of
simplicial objects $\alpha: \calX_{\bigdot} \rightarrow \calX'_{\bigdot}$; we claim that $\alpha$ has the desired properties. The only nontrivial point is to verify that the induced map of geometric realizations
$| \calX_{\bigdot} | \rightarrow | \calX'_{\bigdot} |$ is an equivalence of $\infty$-topoi. For this, it suffices to show that for every $\infty$-topos $\calY$, the upper horizontal map in the diagram
$$\xymatrix{ \bHom_{ \LGeom^{op} }( | \calX'_{\bigdot} |, \calY) \ar[r] \ar[d] & \bHom_{ \LGeom^{op} }( | \calX_{\bigdot } |, \calY) \ar[d] \\
\varprojlim \bHom_{ \LGeom^{op} }( \calX'_{n}, \calY) \ar[r] & \varprojlim \bHom_{ \LGeom^{op} }( \calX_{n}, \calY ) }$$
is a homotopy equivalence. Since the vertical maps are homotopy equivalences, it suffices to show that the lower horizontal map is a homotopy equivalence. Since $j$ is fully faithful, it suffices to show that the lower horizontal map in the analogous diagram
$$\xymatrix{ \bHom_{ \widehat{\Shv}(\LGeom^{op}) }( | F'_{\bigdot} |, e_{\calY}) \ar[r] \ar[d] & \bHom_{ \widehat{\Shv}(\LGeom^{op})}( | F_{\bigdot } |, e_{\calY}) \ar[d] \\
\varprojlim \bHom_{ \widehat{\Shv}(\LGeom^{op})}( F'_{n}, e_{\calY}) \ar[r] & \varprojlim \bHom_{ \widehat{\Shv}(\LGeom^{op}) }( F_{n}, e_{\calY} ) }$$
is a homotopy equivalence. Again, the vertical maps are homotopy equivalences, so we are reduced to showing that the upper horizontal map is a homotopy equivalence. This follows from the fact that we have an equivalence $|F_{\bigdot}| \simeq F \simeq |F'_{\bigdot}|$ in $\widehat{\Shv}( \LGeom^{op} )$, since groupoid objects in $\widehat{\Shv}(\LGeom^{op})$ are effective (Lemma \ref{kumba}).

It remains to prove $(\ast)$. Remark \ref{sapper} implies that $\beta: F_0 \rightarrow F$ is universal.
To complete the proof, we must show that for every $\infty$-topos $\calY$, the homotopy fibers
of the induced map $F_0( \calY) \rightarrow F(\calY)$ are essentially small. Let
$R: \widehat{\Shv}( \LGeom^{op}) \rightarrow \Shv_{ \widehat{\SSet}}( \calY)$ denote the restriction map,
let $G_{\bigdot}$ denote the image of $F_{\bigdot}$ under $R$, and let $G = | G_{\bigdot} | \simeq R(F)$. Let $G'$ denote the geometric realization of $G_{\bigdot}$ in the larger $\infty$-category
$\Fun( \calY^{op}, \widehat{\SSet})$, and let $L: \Fun( \calY^{op}, \widehat{\SSet} ) \rightarrow \Shv_{ \widehat{\SSet} }(\calY)$ be a left adjoint to the inclusion (see Lemma \ref{carbcount}). Then we can identify the map $G_0 \rightarrow G$ with the image under $L$ of the map $u: G_0 \rightarrow G'$. In view of Lemma \ref{carbcount}, it will suffice to show that for every object $U \in \calY$, the induced map $G_0(U) \rightarrow G'(U)$ has essentially small homotopy fibers.

For each $n \geq 0$ and each object $U \in \calY$, we can identify $G_{n}(U)$ with the
maximal Kan complex contained in $\Fun^{\ast}( \calX_{n}, \calY_{/U})$. Since the $\infty$-category $\Fun^{\ast}( \calX_{n}, \calY_{/U})$ is locally small (Proposition \ref{nottoobig}), we conclude that each connected component of $G_{n}(U)$ is essentially small.
Moreover, for every morphism $[m] \rightarrow [n]$ in $\cDelta$, the induced map
$\beta: G_{n}(U) \rightarrow G_{m}(U)$ is induced by composition with an \'{e}tale geometric morphism
$g^{\ast}: \calX_{m} \rightarrow \calX_{n}$, so that the homotopy fibers of $\beta$ are essentially
small by Remark \ref{goodilk}. The desired result now follows from Lemma \ref{guesstimate}.
\end{proof}


\subsection{Structure Theory for $\infty$-Topoi}\label{structuretheor}

In this section we will analyze the following question: given a geometric morphism
$f: \calX \rightarrow \calY$ of $\infty$-topoi, when is $f$ an equivalence? Clearly, this is true if and only if the pullback functor $f^{\ast}$ is both fully faithful and essentially surjective. It is useful to isolate and study these conditions individually.

\begin{definition}\label{nutro}\index{gen}{image of a geometric morphism}
Let $f: \calX \rightarrow \calY$ be a geometric morphism of $\infty$-topoi. The
{\it image} of $f$ is defined to be the smallest full subcategory of $\calX$ which contains
$f^{\ast} \calY$ and is stable under small colimits and finite limits. 
We will say that $f$ is {\it algebraic} if the image of $f$ coincides with $\calX$.\index{gen}{algebraic morphism}
\end{definition}

Our first goal is to prove that the image of a geometric morphism is itself an $\infty$-topos.

\begin{proposition}\label{proet}
Let $f: \calX \rightarrow \calZ$ be a geometric morphism of $\infty$-topoi, and let $\calY$
be the image of $f$. Then $\calY$ is an $\infty$-topos. Moreover, the inclusion
$\calY \subseteq \calX$ is left exact and colimit-preserving, so we have obtain a factorization of $f$ as a composition of geometric morphisms
$$ \calX \stackrel{g}{\rightarrow} \calY \stackrel{h}{\rightarrow} \calZ$$
where $h$ is algebraic and $g^{\ast}$ is fully faithful. 
\end{proposition}

\begin{proof}
We will show that $\calY$ satisfies the $\infty$-categorical versions of Giraud's axioms (see
Theorem \ref{mainchar}). Axioms $(ii)$, $(iii)$, and $(iv)$ are concerned with the interaction between colimits and finite limits. Since $\calX$ satisfies these axioms, and $\calY \subseteq \calX$ is stable under the relevant constructions, $\calY$ automatically satisfies these axioms as well. The only nontrivial point is to verify $(i)$, which asserts that $\calY$ is presentable. 

Choose a small collection of objects $\{ Z_{\alpha} \}$ which generate $\calZ$ under colimits.
Now choose an uncountable regular cardinal $\tau$ with the following properties:

\begin{itemize}
\item[$(1)$] Each $f^{\ast}(Z_{\alpha})$ is a $\tau$-compact object of $\calX$.
\item[$(2)$] The final object $1_{\calX}$ is $\tau$-compact.
\item[$(3)$] The limits functor $\Fun(\Lambda^2_2, \calX) \rightarrow \calX$
(a right adjoint to the diagonal functor) is $\tau$-continuous and preserves $\tau$-compact objects.
\end{itemize}

Let $\calY'$ be the collection of all objects of $\calY$ which are $\tau$-compact when considered as objects of $\calX$. Clearly, each object of $\calY'$ is also $\tau$-compact when regarded as an object of $\calY$. Moreover, because $\calX$ is accessible, $\calY'$ is essentially small. It will therefore suffice to prove that $\calY'$ generates $\calY$ under colimits.

Choose a minimal model $\calY'_0$ for $\calY$. Since $\calX$ is accessible, the full subcategory $\calX^{\kappa}$ spanned by the $\kappa$-compact objects is essentially small, so that $\calY'_0$ is small. According to Proposition \ref{intprop}, there exists a $\tau$-continuous functor
$F: \Ind_{\tau}( \calY'_{0} ) \rightarrow \calX$ whose composition with the Yoneda embedding
is equivalent to the inclusion $\calY'_0 \subseteq \calX$. Since $\calY'_0$ admits $\tau$-small colimits, $\Ind_{\tau}( \calY'_{0} )$ is presentable.
Proposition \ref{uterr} implies that $F$ is fully faithful; let $\calY''$ be its essential image. To complete the proof, it will suffice to show that $\calY'' = \calY$. 

Since $\calY$ is stable under colimits in $\calX$, we have $\calY'' \subseteq \calY$. 
According to Proposition \ref{sumatch}, $F$ preserves small colimits, so that $\calY''$ is stable under small colimits in $\calX$. By construction, $\calY''$ contains each $f^{\ast}(Z_{\alpha})$. Since $f^{\ast}$ preserves colimits, we conclude that $\calY''$ contains $f^{\ast} \calZ$. By definition $\calY$ is the smallest full subcategory of $\calX$ which contains $f^{\ast} \calZ$ and is stable under small colimits and finite limits. It remains only to show that $\calY''$ is stable under finite limits. Assumption $(2)$ guarantees that $\calY''$ contains the final object of $\calX$, so we
need only show that $\calY''$ is stable under pullbacks. Consider a diagram
$p: \Lambda^2_2 \rightarrow \calY''$. The proof of Proposition \ref{horse1} (applied
with $K = \Lambda^2_2$ and $\kappa = \omega$) shows that $p$ can be written
as a $\tau$-filtered colimit of diagrams $p_{\alpha}: \Lambda^2_2 \rightarrow \calY''$. 
Since filtered colimits in $\calX$ are left exact (Example \ref{tucka}), we conclude that the limit of
$p$ can be obtained as a $\tau$-filtered colimit of limits of the diagrams $p_{\beta}$. In view of assumption $(3)$, each of these limits lies in $\calY'$, so that the limit of $p$ lies in $\calY''$ as desired.
\end{proof}

\begin{remark}
The factorization of Proposition \ref{proet} is unique up to (canonical) equivalence.
\end{remark}

The terminology of Definition \ref{nutro} is partially justified by the following observations: 

\begin{proposition}\label{charproet}
\begin{itemize}
\item[$(1)$] Every \'{e}tale geometric morphism between $\infty$-topoi is algebraic.
\item[$(2)$] The collection of algebraic geometric morphisms of $\infty$-topoi is stable
under filtered limits $($in $\RGeom${}$)$.
\end{itemize}
\end{proposition}

\begin{proof}
We first prove $(1)$. Let $\calX$ be an $\infty$-topos, let $U$ be an object of $\calX$,
let $\pi_{!}: \calX^{/U} \rightarrow \calX$ be the projection functor, and let $\pi^{\ast}$ be a 
left adjoint to $\pi_{!}$. Let $f: X \rightarrow U$ be an object of $\calX^{/U}$, and let
$F: f \rightarrow \id_{U}$ be a morphism in $\calX^{/U}$ (uniquely determined up to equivalence; for example, we can take $F$ to be the composition of $f$ with a retraction $\Delta^1 \times \Delta^1 \rightarrow \Delta^1$). Let $g: F \rightarrow \pi^{\ast} \pi_{!} F$ be the unit map for the adjunction
between $\pi^{\ast}$ and $\pi_{!}$. We claim that $g$ is a pullback square in $\calX_{/U}$. According to Proposition \ref{goeselse}, it will suffice to verify that the image of $g$ under $\pi_{!}$ is a pullback square in $\calX$. But this square can be identified with
$$ \xymatrix{ X \ar[r] \ar[d] & X \times U \ar[d] \\
U \ar[r]^-{\delta} & U \times U, }$$
which is easily shown to be Cartesian. It follows that, in $\calX_{/U}$, $f$ can be obtained
as a fiber product of the final object with objects that lie in the essential image of $\pi^{\ast}$.
It follows that $\pi^{\ast} \calX$ generates $\calX_{/U}$ under finite limits, so that
$\pi$ is algebraic.

To prove $(2)$, we consider a geometric morphism $f: \calX \rightarrow \calY$ which
is a filtered limit of algebraic geometric morphisms $\{ f_{\alpha}: \calX_{\alpha} \rightarrow \calY_{\alpha} \}$ in the $\infty$-category $\Fun(\Delta^1, \RGeom)$. Let $\calX' \subseteq \calX$
be a full subcategory which is stable under finite limits, small colimits, and contains
$f^{\ast} \calY$. We wish to prove that $\calX' = \calX$. For each $\alpha$, we have
a diagram of $\infty$-topoi
$$ \xymatrix{ \calX \ar[r]^{f} \ar[d]^{\psi(\alpha)} & \calY \ar[d] \\
\calX_{\alpha} \ar[r]^{f_{\alpha}} & \calY_{\alpha}. }$$
Let $\calX'_{\alpha}$ be the preimage of $\calX'$ under $\psi(\alpha)^{\ast}$. Then
$\calX'_{\alpha} \subseteq \calX_{\alpha}$ is stable under finite limits, small colimits, and
contains the essential image of $f_{\alpha}^{\ast}$. Since $f_{\alpha}$ is algebraic, we conclude
that $\calX'_{\alpha} = \calX_{\alpha}$. In other words, $\calX'$ contains the essential image
of each $\psi(\alpha)^{\ast}$. Lemma \ref{steakknife} implies that every object of $\calX$ can be realized as a filtered colimit of objects, each of which belongs to the essential image of
$f_{\alpha}^{\ast}$ for $\alpha$ appropriately chosen. Since $\calX'$ is stable under small colimits, we conclude that $\calX' = \calX$. It follows that $f$ is algebraic, as desired.
\end{proof}

\begin{remark}\label{weakcon}
It is possible to formulate a converse to Proposition \ref{charproet}. Namely, one can characterize the class of algebraic morphisms as the smallest class of geometric morphisms which contains all \'{e}tale morphisms and is stable under certain kinds of filtered limits. However, it is necessarily to allow limits which are parametrized not just by filtered $\infty$-categories, but filtered {\em stacks} over $\infty$-topoi. The precise statement requires ideas which lie outside the scope of this book.
\end{remark}

Having achieved a rudimentary understanding of the class of algebraic geometric morphisms, we now turn our attention to the opposite extreme: namely, geometric morphisms $f: \calX \rightarrow \calY$ where $f^{\ast}$ is fully faithful. 

\begin{proposition}\label{unterware}
Let $f: \calX \rightarrow \calY$ be a geometric morphism of $\infty$-topoi. Suppose that
$f^{\ast}$ is fully faithful and essentially surjective on $1$-truncated objects. Then $f^{\ast}$ is essentially surjective on $n$-truncated objects for all $n$.
\end{proposition}

The proof uses ideas which will be introduced in \S \ref{homotopysheaves} and \S \ref{chmdim}.

\begin{proof}
Without loss of generality, we may identify $\calY$ with the essential image of $f^{\ast}$.
We use induction on $n$. The result is obvious for $n = 1$. Assume that $n > 1$, and let $X$ be an $n$-truncated object of $\calX$. By the inductive hypothesis, $U= \tau_{\leq n-1} X$ belongs to $\calY$.
Replacing $\calX$ and $\calY$ by $\calX_{/ U}$ and $\calY_{ / U}$, we may suppose that $X$ is $n$-connective.

We observe that $\pi_{n} X$ is an abelian group object of the ordinary topos
$\Disc(\calX_{/X})$. Since $X$ is $2$-connective, Proposition \ref{nicelemma}
implies that the pullback functor $\Disc(\calX) \rightarrow \Disc( \calX_{/X})$ is an equivalence of categories. We may therefore identify $\pi_n X$ with an abelian group object
$A \in \Disc(\calX)$. Since $A$ is discrete, it belongs to $\calY$. It follows that
the Eilenberg-MacLane object $K(A,n+1)$ belongs to $\calY$. Since $X$ is an $n$-gerb banded by $A$, Theorem \ref{starthm} implies the existence of a pullback diagram
$$ \xymatrix{ X \ar[r] \ar[d] & 1_{\calX} \ar[d] \\
1_{\calX} \ar[r] & K(A,n+1). }$$
Since $\calY$ is stable under pullbacks in $\calX$, we conclude that $X \in \calY$ as desired.
\end{proof}

\begin{corollary}\label{unwhere}
Let $f: \calX \rightarrow \calY$ be a geometric morphism of $\infty$-topoi. Suppose that
$f^{\ast}$ is fully faithful and essentially surjective on $1$-truncated objects, and that
$\calX$ is $n$-localic $($see \S \ref{nlocalic}$)$. Then $f$ is an equivalence of $\infty$-topoi.
\end{corollary}

\begin{remark}
In the situation of Corollary \ref{unwhere}, one can eliminate the hypothesis that $\calX$ is $n$-localic in the presence of suitable finite-dimensionality assumptions on $\calX$ and $\calY$; see \S \ref{homdim}.
\end{remark}

\begin{remark}
Let $\calX$ be an $n$-localic $\infty$-topos, and let $\calY$ be the $2$-localic $\infty$-topos
associated to the $2$-topos $\tau_{\leq 1} \calX$, so that we have a geometric morphism
$f: \calX \rightarrow \calY$. It follows from Corollary \ref{unwhere} that $f$ is algebraic.
Roughly speaking, this tells us that there is only a very superficial interaction between the theory of $k$-categories and ``topology'', for $k > 2$. On the other hand, this statement fails dramatically if
$k=1$: the relationship between an ordinary topos and its underlying locale is typically very complicated, and not algebraic in any reasonable sense. It is natural to ask what happens when $k=2$. In other words, does Proposition \ref{unterware} remain valid if $f^{\ast}$ is only assumed to be essentially surjective on discrete objects? An affirmative answer would indicate that our theory of $\infty$-topoi is a relatively modest extension of classical topos theory. A counterexample could be equally interesting, if it were to illustrate a nontrivial interaction between higher category theory and geometry.
\end{remark}