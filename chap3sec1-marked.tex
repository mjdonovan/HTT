\section{Marked Simplicial Sets}\label{twuf}

The Joyal model structure on $\sSet$ is a powerful tool in the study of  $\infty$-categories. However, in {\em relative} situations it is somewhat inconvenient. Roughly speaking, a categorical fibration $p: X \rightarrow S$ determines a family of $\infty$-categories $X_{s}$, parametrized by the vertices $s$ of $S$. However, we are generally more interested in those cases where $X_{s}$ can be regarded as a functor of $s$. As we explained in \S \ref{funkymid}, this naturally translates into the assumption that $p$ is a Cartesian (or coCartesian) fibration. According to Proposition \ref{funkyfibcatfib}, every Cartesian fibration is a categorical fibration, but the converse is false. Consequently, it is natural to to try to endow $(\sSet)_{/S}$ with some {\em other} model structure, in which the fibrant objects are precisely the Cartesian fibrations over $S$. 

Unfortunately, this turns out to be an unreasonable demand. In order to have a model category, we need to be able to form fibrant replacements: in other words, we need the ability to enlarge an arbitrary map $p: X \rightarrow S$ into a commutative diagram
$$ \xymatrix{ X \ar[dr]^{p} \ar[rr]^{\phi} & & Y \ar[dl]^{q} \\
& S & }$$
where $q$ is a Cartesian fibration {\em generated by $p$}. A question arises: for which edges $f$ of $X$ should $\phi(f)$ be $q$-Cartesian edge of $Y$? This sort of information is needed for the construction of $Y$; consequently, we need a formalism in which certain edges of $X$ have been distinguished, or {\it marked}.

\setcounter{theorem}{0}

\begin{definition}\index{gen}{marked!simplicial set}\index{gen}{simplicial set!marked}
A  {\it marked simplicial set} is a pair $(X,\calE)$ where $X$ is a simplicial set and
$\calE$ is a set of edges of $X$ which contains every degenerate edge. We will say that an edge of $X$ will be called {\it marked} if it belongs to $\calE$.\index{gen}{marked!edge}

A morphism $f: (X,\calE) \rightarrow (X', \calE')$ of marked simplicial sets is a map $f: X \rightarrow X'$ having the property that $f(\calE) \subseteq \calE'$. The category of marked simplicial sets will be denoted by $\mSet$.\index{not}{sSetm@$\mSet$}
\end{definition}

Every simplicial set $S$ may be regarded as a marked simplicial set, usually in many different ways. The two extreme cases deserve special mention: if $S$ is a simplicial set, we let $S^{\sharp} = (S,S_1)$ denote the marked simplicial set in which {\em every} edge of $S$ has been marked, and $S^{\flat} = (S, s_0(S_0))$ the marked simplicial set in which only the degenerate edges of $S$ have been marked.\index{not}{Xsharp@$X^{\sharp}$}\index{not}{Xflat@$X^{\flat}$}

\begin{notation}
Let $S$ be a simplicial set. We let $(\mSet)_{/S}$ denote the category of marked simplicial
sets equipped with a map to $S$ (which might otherwise be denoted as
$(\mSet)_{ / S^{\sharp} }$).\index{not}{sSetM/S@$(\mSet)_{/S}$}
\end{notation}

Our goal in this section is to study the theory of marked simplicial sets, and in particular to endow 
each $(\mSet)_{/S}$ with the structure of a model category. We will begin in \S \ref{bicat1} by introducing the notion of a {\it marked anodyne} morphism in $\mSet$. In \S \ref{bicat11}, we will establish a basic stability property of the class of marked anodyne maps, which implies the stability of Cartesian fibrations under exponentiation (Proposition \ref{doog}). In \S \ref{markmodel} we will introduce the {\it Cartesian model structure} on $(\mSet)_{/S}$, for every simplicial set $S$. In \S \ref{markprop}, we will study these model categories; in particular, we will see that each $(\mSet)_{/S}$ is a {\it simplicial} model category, whose fibrant objects are precisely the 
Cartesian fibrations $X \rightarrow S$ (with Cartesian edges of $X$ marked). Finally, we will conclude with \S \ref{compmodel}, where we compare the Cartesian model structure on $(\mSet)_{/S}$ with other model structures considered in this book (such as the Joyal and contravariant model structures).

\subsection{Marked Anodyne Morphisms}\label{bicat1}

In this section, we will introduce the class of {\em marked anodyne} morphisms in
$\mSet$. The definition is chosen so that the condition that a map
$\overline{X} \rightarrow \overline{S}$ have the right lifting property with
respect to all marked anodyne morphisms is closely related to the condition
that the underlying map of simplicial sets $X \rightarrow S$ be a Cartesian fibration (we refer the reader to Proposition \ref{dubudu} for a more precise statement). The theory of marked anodyne maps is a technical device which will prove useful when we discuss the Cartesian model structure in
\S \ref{markmodel}: every marked anodyne morphism is a trivial cofibration with respect to the Cartesian model structure, but not conversely. In this respect, the class of marked anodyne morphisms of $\mSet$ is analogous to the class of inner anodyne morphisms of $\sSet$. 

\begin{definition}\label{markanod}\index{gen}{anodyne!marked}\index{gen}{marked!anodyne}
The class of {\em marked anodyne} morphisms in $\mSet$ is the smallest weakly saturated (see \S \ref{liftingprobs}) class of morphisms such that:
\begin{itemize}
\item[$(1)$] For each $0 < i < n$, the inclusion $(\Lambda^n_i)^{\flat} \subseteq (\Delta^n)^{\flat}$ is marked anodyne.

\item[$(2)$] For every $n >0$, the inclusion
$$ ( \Lambda^n_n, \calE \cap (\Lambda^n_n)_{1} ) \subseteq ( \Delta^n, \calE )$$
is marked anodyne, where $\calE$ denotes the set of all degenerate edges of $\Delta^n$, together with the final edge $\Delta^{ \{n-1,n\} }$.

\item[$(3)$] The inclusion
$$ (\Lambda^2_1)^{\sharp} \coprod_{ (\Lambda^2_1)^{\flat} } (\Delta^2)^{\flat} \rightarrow (\Delta^2)^{\sharp}$$ is marked anodyne.

\item[$(4)$] For every Kan complex $K$, the map $K^{\flat} \rightarrow K^{\sharp}$ is marked anodyne.
\end{itemize}
\end{definition}

\begin{remark}
The definition of a marked simplicial set is self-dual. However, Definition
\ref{markanod} is not self-dual: if $A \rightarrow B$ is
marked anodyne, then the opposite morphism $A^{op} \rightarrow B^{op}$ need not be marked-anodyne. This reflects the fact that the theory of Cartesian fibrations is not self-dual.
\end{remark}

\begin{remark}
In part $(4)$ of Definition \ref{markanod}, it suffices to allow $K$ to range over a set of representatives for all isomorphism classes of Kan complexes with only countably many simplices. Consequently, we deduce that the class of marked anodyne morphisms in $\mSet$ is of small generation, so that the small object argument applies (see \S \ref{liftingprobs}). We will refine this observation further: see Corollary \ref{techycor}, below.
\end{remark}

\begin{remark}\label{sillin}
In Definition \ref{markanod}, we are free to replace $(1)$ by
\begin{itemize}
\item[$(1')$] For every inner anodyne map $A \rightarrow B$ of simplicial sets, the
induced map $A^{\flat} \rightarrow B^{\flat}$ is marked anodyne.
\end{itemize}
\end{remark}

\begin{proposition}\label{goldegg}
Consider the following classes of morphisms in $\mSet$:

\begin{itemize}
\item[$(2)$] All inclusions
$$ ( \Lambda^n_n, \calE \cap (\Lambda^n_n)_{1} ) \subseteq ( \Delta^n, \calE ),$$
where $n > 0$ and $\calE$ denotes the set of all degenerate edges of $\Delta^n$, together with the final edge $\Delta^{ \{n-1,n\} }$.

\item[$(2')$] All inclusions
$$ ( (\bd \Delta^n)^{\flat} \times (\Delta^1)^{\sharp})  \coprod_{ (\bd \Delta^n)^{\flat} \times
\{1\}^{\sharp} } ((\Delta^n)^{\flat} \times \{1\}^{\sharp}) \subseteq (\Delta^n)^{\flat} \times (\Delta^1)^{\sharp}.$$ 

\item[$(2'')$] All inclusions
$$ ( A^{\flat} \times (\Delta^1)^{\sharp})  \coprod_{ A^{\flat} \times
\{1\}^{\sharp} } (B^{\flat} \times \{1\}^{\sharp}) \subseteq B^{\flat} \times (\Delta^1)^{\sharp},$$
where $A \subseteq B$ is an inclusion of simplicial sets. 
\end{itemize}

The classes $(2')$ and $(2'')$ generate the same weakly saturated class of morphisms of $\mSet$, which contains the weakly saturated class generated by $(2)$. Conversely, the weakly saturated class of morphisms generated by $(1)$ and $(2)$ from Definition \ref{markanod} contains $(2')$ and $(2'')$.
\end{proposition}

\begin{proof}
To see that each of the morphisms specified in $(2'')$ is contained in the weakly saturated class generated by $(2')$, it suffices to work cell-by-cell with the inclusion $A \subseteq B$. The converse is obvious, since the class of morphisms of type $(2')$ is contained in the class of morphisms of type $(2'')$. To see that the weakly saturated class generated by $(2'')$ contains $(2)$, it suffices to show every morphism in $(2)$ is a retract of a morphism in $(2'')$. For this, we consider
maps
$$\Delta^{n} \stackrel{j}{\rightarrow} \Delta^n \times \Delta^1 \stackrel{r}{\rightarrow} \Delta^{n}.$$ 
Here $j$ is the composition of the identification $\Delta^n \simeq \Delta^n \times \{0\}$ with the inclusion $\Delta^n \times \{0\} \subseteq \Delta^n \times \Delta^1$, and $r$ may be identified with the map of partially ordered sets
$$r(m,i) =
\begin{cases} n & \text{if } m = n-1, i=1 \\
m & \text{otherwise.}  \end{cases}$$

Now we simply observe that $j$ and $r$ exhibit the inclusion
$$ ( \Lambda^n_n, \calE \cap (\Lambda^n_n)_{0} ) \subseteq ( \Delta^n, \calE ),$$
as a retract of
$$ ( (\Lambda^n_n)^{\flat} \times (\Delta^1)^{\sharp})  \coprod_{ (\Lambda_n^n)^{\flat} \times
\{1\}^{\sharp} } ((\Delta^n)^{\flat} \times \{1\}^{\sharp}) \subseteq (\Delta^n)^{\flat} \times (\Delta^1)^{\sharp}.$$ 

To complete the proof, we must show that each of the inclusions
$$ ( (\bd \Delta^n)^{\flat} \times (\Delta^1)^{\sharp})  \coprod_{ (\bd \Delta^n)^{\flat} \times
\{1\}^{\sharp} } ((\Delta^n)^{\flat} \times \{1\}^{\sharp}) \subseteq (\Delta^n)^{\flat} \times (\Delta^1)^{\sharp}$$ 
of type $(2')$ belongs to the weakly saturated class generated by $(1)$ and $(2)$. To see this, we
consider the filtration
$$ Y_{n+1} \subseteq \ldots \subseteq Y_0 = \Delta^n \times \Delta^1$$ which is the
{\em opposite} of the filtration defined in the proof of Proposition \ref{usejoyal}.
We let $\calE_i$ denote the class of all edges of $Y_i$ which are marked in
$(\Delta^n)^{\flat} \times (\Delta^1)^{\sharp}$. 
It will suffice to show that each inclusion $f_i: (Y_{i+1},\calE_{i+1}) \subseteq (Y_{i}, \calE_{i})$ lies in the weakly saturated class generated by $(1)$ and $(2)$. For $i \neq 0$, the map $f_i$ is a pushout of
$(\Lambda^{n+1}_{n+1-i})^{\flat} \subseteq (\Delta^{n+1})^{\flat}$. For $i=0$, $f_i$
is a pushout of 
$$ ( \Lambda^{n+1}_{n+1}, \calE \cap (\Lambda^{n+1}_{n+1})_{1} ) \subseteq ( \Delta^{n+1}, \calE ),$$
where and $\calE$ denotes the set of all degenerate edges of $\Delta^{n+1}$, together with $\Delta^{ \{n,n+1\}}$.
\end{proof}

We now characterize the class of marked-anodyne maps:

\begin{proposition}\label{dubudu}
A map $p: X \rightarrow S$ in $\mSet$ has the right lifting property with respect to all marked anodyne maps if and only if the following conditions are satisfied:
\begin{itemize}
\item[$(A)$] The map $p$ is an inner fibration of simplicial sets.
\item[$(B)$] An edge $e$ of $X$ is marked if and only if $p(e)$ is marked and $e$ is $p$-Cartesian.
\item[$(C)$] For every object $y$ of $X$ and every marked edge $\overline{e}: \overline{x} \rightarrow p(y)$ in $S$, there exists a marked edge $e: x \rightarrow y$ of $X$ with $p(e) = \overline{e}$. 
\end{itemize}
\end{proposition}

\begin{proof}
We first prove the ``only if'' direction. Suppose that $p$ has the right lifting property with respect to all marked anodyne maps. By considering maps of the form $(1)$ from Definition \ref{markanod}, we deduce that $(A)$ holds. Considering $(2)$ in the case $n=0$, we deduce that $(C)$ holds. Considering $(2)$ for $n > 0$, we deduce that every marked edge of $X$ is $p$-Cartesian.
For the converse, let us suppose that $e: x \rightarrow y$ is a $p$-Cartesian edge of $X$ and that $p(e)$ is marked in $S$. Invoking $(C)$, we deduce that there exists a marked edge $e': x' \rightarrow y$ with $p(e)=p(e')$. Since $e'$ is Cartesian,
we can find a $2$-simplex $\sigma$ of $X$ with $d_0(\sigma) = e'$, $d_1(\sigma)=e$, and 
$p(\sigma)= s_1 p(e)$. Then
$d_2(\sigma)$ an equivalence between $x$ and $x'$ in the $\infty$-category $X_{p(x)}$. Let $K$ denote the largest Kan complex contained in $X_{p(x)}$. Since $p$ has the right lifting property with respect to
$K^{\flat} \rightarrow K^{\sharp}$, we deduce that every edge of $K$ is marked; in particular, $d_2(\sigma)$ is marked. Since $p$ has the right lifting property with respect to the morphism described in $(3)$ of Definition \ref{markanod}, we deduce that $d_1(\sigma)=e$ is marked.

Now suppose that $p$ satisfied the hypotheses of the proposition. We must show that $p$ has the right lifting property with respect to the classes of morphisms $(1)$, $(2)$, $(3)$, and $(4)$ of Definition \ref{markanod}. For $(1)$, this follows from the assumption that $p$ is an inner fibration. 
For $(2)$, this follows from $(C)$ and from the assumption that every marked edge is $p$-Cartesian. For $(3)$, we are free to replace $S$ by $(\Delta^2)^{\sharp}$; then $p$ is a Cartesian fibration over an $\infty$-category $S$ and we may apply Proposition \ref{protohermes} to deduce that the class of $p$-Cartesian edges is stable under composition.

Finally, for $(4)$, we may replace $S$ by $K^{\sharp}$; 
then $S$ is a Kan complex and $p$ is a Cartesian fibration, so the $p$-Cartesian edges of $X$
are precisely the equivalences in $X$. Since $K$ is a Kan complex, any
map $K \rightarrow X$ carries the edges of $K$ to equivalences in $X$.
\end{proof}

By Quillen's small object argument, we 
deduce that a map $j: A \rightarrow B$ in $\mSet$ is marked anodyne if and only if it has the left lifting property with respect to all morphisms $p: X \rightarrow S$ satisfying the hypotheses of Proposition \ref{dubudu}. From this, we deduce:

\begin{corollary}\label{hermes}
The inclusion
$$ i: (\Lambda^2_2)^{\sharp} \coprod_{ (\Lambda^2_2)^{\flat} } (\Delta^2)^{\flat} \hookrightarrow (\Delta^2)^{\sharp}$$ is marked anodyne.
\end{corollary}

\begin{proof}
It will suffice to show that $i$ has the left lifting property with respect to any of the morphisms $p: X \rightarrow S$ described in Proposition \ref{dubudu}. Without loss of generality, we may replace $S$ by $(\Delta^2)^{\sharp}$; we now apply Proposition \ref{protohermes}.
\end{proof}

The following somewhat technical corollary will be needed in \S \ref{markmodel}:

\begin{corollary}\label{techycor}
In Definition \ref{markanod}, we can replace the class of morphisms $(4)$ by
\begin{itemize}
\item[$(4')$] the map $j: A^{\flat} \rightarrow (A, s_0 A_0 \bigcup \{f\} )$, where
$A$ is the quotient of $\Delta^3$ which co-represents the functor
$$ \Hom_{\sSet}(A,X) = \{ \sigma \in X_3, e \in X_1: d_1 \sigma = s_0 e, 
d_2 \sigma = s_1 e \}$$ and $f \in A_1$ is the image of
$\Delta^{ \{0,1\}} \subseteq \Delta^3$ in $A$.
\end{itemize}
\end{corollary}

\begin{proof}
We first show that for every Kan complex $K$, the map $i: K^{\flat} \rightarrow K^{\sharp}$
lies in the weakly saturated class of morphisms generated by $(4')$. We note that $i$ can be obtained 
as an iterated pushout of morphisms having the form
$K^{\flat} \rightarrow (K, s_0 K_0 \bigcup \{e\})$, where $e$ is an edge of $K$. It therefore suffices to show that there exists a map $p: A \rightarrow K$ such that $p(f) = e$. In other words, we must prove that there exists a $3$-simplex $\sigma: \Delta^3 \rightarrow K$ with $d_1 \sigma = s_0 e$ and $d_2 \sigma = s_1 e$. This follows immediately from the Kan extension condition.

To complete the proof, it will suffice to show that the map $j$ is marked anodyne.
To do so, it suffices to prove that for any diagram
$$ \xymatrix{ A^{\flat} \ar@{^{(}->}[d] \ar[r] & X \ar[d]^{p} \\
(A, s_0 A_0 \cup \{f\}) \ar[r] \ar@{-->}[ur] & S }$$
for which $p$ satisfies the conditions of Proposition \ref{dubudu}, there
exists a dotted arrow as indicated, rendering the diagram commutative. This is
simply a reformulation of Proposition \ref{sworkk}.
\end{proof}

\begin{definition}\label{conf1}\index{not}{Xnatural@$X^{\natural}$}
Let $p: X \rightarrow S$ be a Cartesian fibration of simplicial sets. We let
$X^{\natural}$ denote the marked simplicial set $(X, \calE)$, where $\calE$ is the set of
$p$-Cartesian edges of $X$.
\end{definition}

\begin{remark}\label{abus}
Our notation is slightly abusive, since $X^{\natural}$ depends not only on $X$ but also
on the map $X \rightarrow S$. 
\end{remark}

\begin{remark}\label{abuss}
According to Proposition \ref{dubudu}, a map
$(Y, \calE) \rightarrow S^{\sharp}$ has the right lifting property with respect to all marked anodyne maps if and only if the underlying map $Y \rightarrow S$ is a Cartesian fibration and
$(Y, \calE) = Y^{\natural}$. 
\end{remark}

We conclude this section with the following easy result, which will be needed later:

\begin{proposition}\label{eggwhite}
Let $p: X \rightarrow S$ be an inner fibration of simplicial sets, and let $f: A \rightarrow B$
be a marked anodyne morphism in $\mSet$, let
$q: B \rightarrow X$ be map of simplicial sets which carries each marked edge of $B$
to a $p$-Cartesian edge of $X$, and $q_0 = q \circ f$. 
Then the induced map
$$ X_{/q} \rightarrow X_{/q_0} \times_{S_{/pq_0}} S_{/pq}$$
is a trivial fibration of simplicial sets. 
\end{proposition}

\begin{proof}
It is easy to see that the class of all morphisms $f$ of $\mSet$ which satisfy the desired conclusion is weakly saturated. It therefore suffices to prove that this class contains collection of generators for the weakly saturated class of marked anodyne morphisms.
If $f$ induces a left anodyne map on the underlying simplicial sets, then the desired result is automatic. It therefore suffices to consider the case where $f$ is the inclusion
$$ ( \Lambda^n_n, \calE \cap (\Lambda^n_n)_{1} ) \subseteq ( \Delta^n, \calE )$$
as described in $(2)$ of Definition \ref{markanod}. In this case, a lifting problem
$$ \xymatrix{ \bd \Delta^m \ar[r] \ar@{^{(}->}[d] & X_{/q} \ar[d] \\
\Delta^m \ar[r] \ar@{-->}[ur] & X_{/q_0} \times_{S_{/pq_0}} S_{/pq} }$$
can be reformulated as an equivalent lifting problem
$$ \xymatrix{ \Lambda^{n+m+1}_{n+m+1} \ar[r]^{\sigma_0} \ar@{^{(}->}[d] & X \ar[d]^{p} \\
\Delta^{n+m+1} \ar[r] \ar@{-->}[ur] & S. }$$
This lifting problem admits a solution, since the hypothsis on $q$ guarantees
that $\sigma_0$ carries $\Delta^{ \{n+m, n+m+1\} }$ to a $p$-Cartesian edge of $X$.
\end{proof}

\subsection{Stability Properties of Marked Anodyne Morphisms}\label{bicat11}

Our main goal in this section is to prove the following stability result:

\begin{proposition}\label{doog}\index{gen}{Cartesian fibration!and functor categories}
Let $p: X \rightarrow S$ be a Cartesian fibration of
simplicial sets, and let $K$ be an arbitrary simplicial set. Then:
\begin{itemize}
\item[$(1)$] The induced map $p^K: X^{K} \rightarrow S^{K}$ is a
Cartesian fibration.

\item[$(2)$] An edge $\Delta^1 \rightarrow X^K$ is $p^K$-Cartesian if
and only if, for every vertex $k$ of $K$, the induced edge $\Delta^1
\rightarrow X$ is $p$-Cartesian.
\end{itemize}
\end{proposition}

We could easily have given an ad-hoc proof of this result in \S \ref{slib}. However, we have opted instead to give a proof using the language of marked simplicial sets. 

\begin{definition}
A morphism $(X, \calE) \rightarrow (X', \calE')$ in $\mSet$ is a {\it cofibration} if the underlying map $X \rightarrow X'$ of simplicial sets is a cofibration.
\end{definition}

The main ingredient we will need to prove Proposition \ref{doog} is the following:

\begin{proposition}\label{markanodprod}
The class of marked anodyne maps in $\mSet$ is stable under smash products with arbitrary cofibrations. In other words, if $f: X \rightarrow X'$ is marked anodyne, and 
$g: Y \rightarrow Y'$ is a cofibration, then the induced map
$$ (X \times Y') \coprod_{ X \times Y} (X' \times Y) \rightarrow X' \times Y'$$ is marked anodyne.
\end{proposition}

\begin{proof}
The argument is tedious, but straightforward. 
Without loss of generality, we may suppose that $f$ belongs either to the class $(2')$ of
Proposition \ref{goldegg}, or one of the classes specified in $(1)$, $(3)$, or $(4)$ of Definition \ref{markanod}.
The class of cofibrations is generated by the inclusions $(\bd \Delta^n)^{\flat} \subseteq (\Delta^n)^{\flat}$ and $(\Delta^1)^{\flat} \subseteq (\Delta^1)^{\sharp}$; thus we may suppose that $g: Y \rightarrow Y'$ is one of these maps. There are eight cases to consider:

\begin{itemize}
\item[(A1)] Let $f$ be the inclusion $(\Lambda^n_i)^{\flat} \subseteq (\Delta^n)^{\flat}$ and $g$ the inclusion $(\bd \Delta^n)^{\flat} \rightarrow (\Delta^n)^{\flat}$, where $0 < i < n$. Since the class of inner anodyne maps between simplicial sets is stable under smash products with inclusions, the smash product of $f$ and $g$ is marked-anodyne (see Remark \ref{sillin}). 

\item[(A2)] Let $f$ denote the inclusion $(\Lambda^n_i)^{\flat} \rightarrow (\Delta^n)^{\flat}$, and $g$ the map $(\Delta^1)^{\flat} \rightarrow (\Delta^1)^{\sharp}$, where $0 < i < n$. Then the smash product of $f$ and $g$ is an isomorphism (since $\Lambda^n_i$ contains all vertices of $\Delta^n$).

\item[(B1)] Let $f$ be the inclusion $$(\{ 1\}^{\sharp} \times (\Delta^n)^{\flat} ) \coprod_{ \{1\}^{\sharp} \times (\bd \Delta^n)^{\flat} } ((\Delta^1)^{\sharp} \times (\bd \Delta^n)^{\flat} ) \subseteq (\Delta^1)^{\sharp} \times (\Delta^n)^{\flat},$$ and let $g$ be the inclusion $(\bd \Delta^n)^{\flat} \rightarrow (\Delta^n)^{\flat}$. Then the smash product of $f$ and $g$ belongs to the class $(2'')$ of Proposition \ref{goldegg}.

\item[(B2)] Let $f$ be the inclusion $$(\{ 1\}^{\sharp} \times (\Delta^n)^{\flat} ) \coprod_{ \{1\}^{\sharp} \times (\bd \Delta^n)^{\flat} } ((\Delta^1)^{\sharp} \times (\bd \Delta^n)^{\flat} ) \subseteq (\Delta^1)^{\sharp} \times (\Delta^n)^{\flat},$$ and let $g$ denote the map
$(\Delta^1)^{\flat} \rightarrow (\Delta^1)^{\sharp}$. If $n > 0$, then the smash product of $f$ and $g$ is an isomorphism. If $n = 0$, then the smash product may be identified with
the map $ (\Delta^1 \times \Delta^1, \calE) \rightarrow (\Delta^1 \times \Delta^1)^{\sharp}$, where $\calE$ consists of all degenerate edges together with $\{0\} \times \Delta^1$, $\{1\} \times \Delta^1$, and $\Delta^1 \times \{1\}$. This map may be obtained as a composition of
two marked anodyne maps: the first is of type $(3)$ in Definition \ref{markanod} (adjoining the ``diagonal'' edge to $\calE$) and the second is the map described in Corollary \ref{hermes}
(adjoining the edge $\Delta^1 \times \{0\}$ to $\calE$).

\item[(C1)] Let $f$ be the inclusion $$(\Lambda^2_1)^{\sharp} \coprod_{ (\Lambda^2_1)^{\flat} } (\Delta^2)^{\flat} \rightarrow (\Delta^2)^{\sharp},$$ and let $g$ the inclusion
$(\bd \Delta^n)^{\flat} \subseteq (\Delta^n)^{\flat}$. Then the smash product of $f$ and $g$ is an isomorphism for $n > 0$, and isomorphic to $f$ for $n=0$.

\item[(C2)] Let $f$ be the inclusion $$(\Lambda^2_1)^{\sharp} \coprod_{ (\Lambda^2_1)^{\flat} } (\Delta^2)^{\flat} \rightarrow (\Delta^2)^{\sharp},$$ and let $g$ be the canonical map $(\Delta^1)^{\flat} \rightarrow (\Delta^1)^{\sharp}.$ Then the smash product of $f$ and $g$ is a pushout of the map $f$.

\item[(D1)] Let $f$ be the map $K^{\flat} \rightarrow K^{\sharp}$, where $K$ is a Kan complex, and let $g$ the inclusion $(\bd \Delta^n)^{\flat} \subseteq (\Delta^n)^{\flat}$. Then the smash product of $f$ and $g$ is an isomorphism for $n > 0$, and isomorphic to $f$ for $n=0$.

\item[(D2)] Let $f$ be the map $K^{\flat} \rightarrow K^{\sharp}$, where $K$ is a Kan complex, and let $g$ be the map $(\Delta^1)^{\flat} \rightarrow (\Delta^1)^{\sharp}$. The smash product of $f$ and $g$ can be identified with the inclusion
$$(K \times \Delta^1, \calE) \subseteq (K \times \Delta^1)^{\sharp},$$ where $\calE$
denotes the class of all edges $e=(e',e'')$ of $K \times \Delta^1$ for which either $e': \Delta^1 \rightarrow K$ or $e'': \Delta^1 \rightarrow \Delta^1$ is degenerate. This inclusion can be obtained as a transfinite composition of pushouts of the map $$(\Lambda^2_1)^{\sharp} \coprod_{ (\Lambda^2_1)^{\flat} } (\Delta^2)^{\flat} \rightarrow (\Delta^2)^{\sharp}.$$
\end{itemize}
\end{proof}

We now return to our main objective:

\begin{proof}[Proof of Proposition \ref{doog}]
Since $p$ is a Cartesian fibration, it induces a map
$X^{\natural} \rightarrow S^{\sharp}$ which has the right lifting property with respect to all marked anodyne maps. By Proposition \ref{markanodprod}, the induced map
$$ (X^{\natural})^{K^{\flat}} \rightarrow (S^{\sharp})^{K^{\flat}}=(S^K)^{\sharp}$$
has the right lifting property with respect to all marked anodyne morphisms. The desired result now follows from Remark \ref{abuss}.
\end{proof}

\subsection{The Cartesian Model Structure}\label{markmodel}

Let $S$ be a simplicial set. Our goal in this section is to introduce
the {\it Cartesian model structure} on the category $(\mSet)_{/S}$ of marked simplicial sets over $S$. We will eventually show that the fibrant objects of
$(\mSet)_{/S}$ correspond precisely to Cartesian fibrations $X \rightarrow S$, and that they encode (contravariant) functors from $S$ into the $\infty$-category
$\Cat_{\infty}$.

The category $\mSet$ is {\it Cartesian-closed}; that is, for any two objects
$X,Y \in \mSet$, there exists an internal mapping object $Y^X$ equipped with an
``evaluation map'' $Y^X \times X \rightarrow Y$ which induces bijections
$$\Hom_{\mSet}(Z,Y^X) \rightarrow \Hom_{\mSet}(Z \times X, Y)$$
for every $Z \in \mSet$. We let $\bHom^{\flat}(X,Y)$ denote the underlying simplicial
set of $Y^X$, and $\bHom^{\sharp}(X,Y) \subseteq \bHom^{\flat}(X,Y)$ the simplicial subset
consisting of all simplices $\sigma \subseteq \bHom^{\flat}(X,Y)$ such that every edge of
$\sigma$ is a marked edge of $Y^X$. Equivalently, we may describe these simplicial sets by the mapping properties
$$ \Hom_{\sSet}(K, \bHom^{\flat}(X,Y)) \simeq \Hom_{\mSet}( K^{\flat} \times X,Y) $$
$$ \Hom_{\sSet}(K, \bHom^{\sharp}(X,Y)) \simeq \Hom_{\mSet}(K^{\sharp} \times X,Y).$$\index{not}{Homflat@$\bHom^{\flat}(X,Y)$}\index{not}{Homsharp@$\bHom^{\sharp}(X,Y)$}

If $X$ and $Y$ are objects of $(\mSet)_{/S}$, then we let
$\bHom_{S}^{\sharp}(X,Y)$ and $\bHom_{S}^{\flat}(X,Y)$ denote the simplicial subsets
of $\bHom^{\sharp}(X,Y)$ and $\bHom^{\flat}(X,Y)$ classifying those maps which are compatible with the projections to $S$.\index{not}{MapSflat@$\bHom_{S}^{\flat}(X,Y)$}\index{not}{MapSSharp@$\bHom_{S}^{\sharp}(X,Y)$}

\begin{remark}
If $X \in (\mSet)_{/S}$ and $p: Y \rightarrow S$ is a Cartesian fibration, then
$\bHom^{\flat}_{S}(X,Y^{\natural})$ is an $\infty$-category, and $\bHom^{\sharp}_{S}(X,Y^{\natural})$ is the largest Kan complex contained in $\bHom^{\flat}_S(X,Y^{\natural})$.
\end{remark}

\begin{lemma}\label{hoopsp}
Let $f: \calC \rightarrow \calD$ be a functor between $\infty$-categories.
The following are equivalent:
\begin{itemize}
\item[$(1)$] The functor $f$ is a categorical equivalence.
\item[$(2)$] For every simplicial set $K$, the induced map
$\Fun(K,\calC) \rightarrow \Fun(K,\calD)$ is a categorical equivalence.
\item[$(3)$] For every simplicial set $K$, the functor $\Fun(K,\calC) \rightarrow \Fun(K,\calD)$ induces
a homotopy equivalence between the largest Kan complex contained in $\Fun(K,\calC)$ and the largest Kan complex contained in $\Fun(K,\calD)$.
\end{itemize}
\end{lemma}

\begin{proof}
The implications $(1) \Rightarrow (2) \Rightarrow (3)$ are obvious. Suppose that $(3)$ is satisfied.
Let $K = \calD$. According to $(3)$, there exists an object $x$ of $\Fun(K,\calC)$ whose image
in $\Fun(K,\calD)$ is equivalent to the identity map $K \rightarrow \calD$. We may identify
$x$ with a functor $g: \calD \rightarrow \calC$ having the property that $f \circ g$ is homotopic
to the identity $\id_{\calD}$. It follows that $g$ also has the property asserted by $(3)$, so the same argument shows that there is a functor $f': \calC \rightarrow \calD$ such that
$g \circ f'$ is homotopic to $\id_{\calC}$. It follows that $f \circ g \circ f'$ is homotopic to both
$f$ and $f'$, so that $f$ is homotopic to $f'$. Thus $g$ is a homotopy inverse to $f$, which proves that $f$ is an equivalence.
\end{proof}

\begin{proposition}\label{markdefeq}
Let $S$ be a simplicial set, and let $p: X \rightarrow Y$ be a morphism in $(\mSet)_{/S}$. The following are equivalent:
\begin{itemize}
\item[$(1)$] For every Cartesian fibration $Z \rightarrow S$, the induced map 
$$ \bHom^{\flat}_{S}(Y, Z^{\natural}) \rightarrow \bHom^{\flat}_{S}(X, Z^{\natural})$$ is an
equivalence of $\infty$-categories.
\item[$(2)$] For every Cartesian fibration $Z \rightarrow S$, the induced map
$$ \bHom^{\sharp}_{S}(Y, Z^{\natural}) \rightarrow \bHom^{\sharp}_{S}(X, Z^{\natural})$$ is a homotopy equivalence of Kan complexes.
\end{itemize}
\end{proposition}

\begin{proof}
Since $\bHom^{\sharp}_{S}(M,Z^{\natural})$ is the largest Kan complex contained in 
$\bHom^{\flat}_{S}(M, Z^{\natural})$, it is clear that $(1)$ implies $(2)$. Suppose that $(2)$ is satisfied, and let $Z \rightarrow S$ be a Cartesian fibration. We wish to show that
$$ \bHom^{\flat}_{S}(Y, Z^{\natural}) \rightarrow \bHom^{\flat}_{S}(X, Z^{\natural})$$
is an equivalence of $\infty$-categories. According to Lemma \ref{hoopsp}, it suffices to show that
$$ \bHom^{\flat}_{S}(Y, Z^{\natural})^K \rightarrow \bHom^{\flat}_{S}(X, Z^{\natural})^K$$ 
induces a homotopy equivalence on the maximal Kan complexes contained in each side.
Let $Z(K) = Z^K \times_{S^K} S$. Proposition \ref{doog} implies that $Z(K) \rightarrow S$
is a Cartesian fibration, and that there is a natural identification
$$ \bHom^{\flat}_{S}(M, Z(K)^{\natural}) \simeq \bHom^{\flat}_{S}(M, Z(K)^{\natural}).$$
The largest Kan complex contained in the right hand side is $\bHom^{\sharp}_{S}(M,Z(K)^{\natural})$. On the other hand, the natural map
$$ \bHom^{\sharp}_{S}(Y,Z(K)^{\natural}) \rightarrow \bHom^{\sharp}_{S}(X, Z(K)^{\natural})$$ is homotopy equivalence by assumption $(2)$.
\end{proof}

We will say that a map $X \rightarrow Y$ in $(\mSet)_{/S}$ is a {\it Cartesian equivalence} if it satisfies the equivalent conditions of Proposition \ref{markdefeq}.\index{gen}{equivalence!Cartesian}\index{gen}{Cartesian!equivalence}

\begin{remark}
Let $f: X \rightarrow Y$ be a morphism in $(\mSet)_{/S}$ which is {\em marked anodyne} when regarded as a map of marked simplicial sets. Since the smash product of $f$ with any inclusion $A^{\flat} \subseteq B^{\flat}$ is also marked-anodyne, we deduce that the map
$$ \phi: \bHom^{\flat}_{S}(Y, Z^{\natural}) \rightarrow \bHom^{\flat}_{S}(X, Z^{\natural})$$ is a trivial fibration for {\em every} Cartesian fibration $Z \rightarrow S$. Consequently, $f$ is a Cartesian equivalence.
\end{remark}

Let $S$ be a simplicial set and let $X,Y \in (\mSet)_{/S}$. We will say a pair of morphisms
$f,g: X \rightarrow Y$ are {\it strongly homotopic} if there exists a contractible Kan complex
$K$ and a map $K \rightarrow \bHom^{\flat}_{S}(X,Y)$, whose image contains both of the vertices $f$ and $g$. If $Y = Z^{\natural}$, where $Z \rightarrow S$ is a Cartesian fibration, then this simply means that $f$ and $g$ are equivalent when viewed as objects of the $\infty$-category $\bHom^{\flat}_{ S}(X,Y)$.

\begin{proposition}\label{crispy}
Let $X \stackrel{p}{\rightarrow} Y \stackrel{q}{\rightarrow} S$ be a diagram of simplicial sets, where both $q$ and $q \circ p$ are Cartesian fibrations. The following assertions are equivalent:

\begin{itemize}
\item[$(1)$] The map $p$ induces a Cartesian equivalence $X^{\natural} \rightarrow Y^{\natural}$ in
$(\mSet)_{/S}$.

\item[$(2)$] There exists a map $r: Y \rightarrow X$ which is a strong homotopy inverse to $p$, in the sense that $p \circ r$ and $r \circ p$ are both strongly homotopic to the identity.

\item[$(3)$] The map $p$ induces a categorical equivalence $X_{s} \rightarrow Y_{s}$, for each vertex $s$ of $S$.
\end{itemize}
\end{proposition}

\begin{proof}
The equivalence between $(1)$ and $(2)$ is easy, as is the assertion that $(2)$ implies $(3)$. It therefore suffices to show that $(3)$ implies $(2)$. We will construct $r$ and a homotopy from $r \circ p$ to the identity. It then follows that the map $r$ satisfies $(3)$, so the same argument will show that $r$ has a right homotopy inverse; by general nonsense this right homotopy inverse is automatically homotopic to $p$ and the proof will be complete.

Choose a transfinite sequence of simplicial subsets $S(\alpha) \subseteq S$, where each
$S(\alpha)$ is obtained from $\bigcup_{ \beta < \alpha} S(\beta)$ by adjoining a single
nondegenerate simplex (if such a simplex exists).
We construct
$r_{\alpha}: Y \times_{S} S(\alpha) \rightarrow X$ and an equivalence $h_{\alpha}: (X \times_{S} S(\alpha) ) \times \Delta^1\rightarrow X \times_{S} S(\alpha)$ from $r_{\alpha} \circ p$ to the identity, by induction on $\alpha$. By this device we may reduce to the case where $S = \Delta^n$, and the maps
$$r^0: Y' \rightarrow X$$
$$h^0: X' \times \Delta^1 \rightarrow X$$ 
are already specified, where $Y' = Y \times_{\Delta^n} \bd \Delta^n \subseteq Y$ and
$X' = X \times_{\Delta^n} \bd \Delta^n \subseteq X$. We may regard $r'$ and $h'$ together as defining a map
$\psi_0: Z'  \rightarrow X$, where
$$Z' = Y' \coprod_{ X' \times \{0\} } (X' \times \Delta^1) \coprod_{ X' \times \{1\} } X.$$
Let $Z = Y \coprod_{ X \times \{0\} } X \times \Delta^1$; then our goal is to solve the lifting problem depicted in the following diagram:
$$ \xymatrix{ Z' \ar[r]^{\psi_0} \ar@{^{(}->}[d] & X \ar[d] \\
Z \ar[r] \ar@{-->}[ur]^{\psi} & \Delta^n }$$
in such a way that $\psi$ carries $\{x\} \times \Delta^1$ to an equivalence in $X$, for
every object $x$ of $X$. We note that this last condition is vacuous for $n > 0$.

If $n=0$, the problem amounts to constructing a map $Y \rightarrow X$ which is homotopy inverse to $p$: this is possible in view of the assumption that $p$ is a categorical equivalence.
For $n > 0$, we note that any map $\phi: Z \rightarrow X$ extending $\phi_0$ is automatically compatible with the projection to $S$ (since $S$ is a simplex and $Z'$ contains all vertices of $Z$). 
Since the inclusion $Z' \subseteq Z$ is a cofibration between cofibrant objects in the model category $\sSet$ (with the Joyal model structure), and $X$ is a $\infty$-category (since $q$ is an inner fibration and $\Delta^n$ is a $\infty$-category), Proposition \ref{princex} asserts that it is sufficient to show that the extension $\phi$ exists up to homotopy. Since Corollary \ref{usefir} implies that $p$ is an equivalence, we are free to replace the inclusion $Z' \subseteq Z$ with the weakly equivalent inclusion
$$ (X \times \{1\}) \coprod_{ X \times_{ \Delta^n} \bd\Delta^n \times \Delta^1 }
(X \times_{\Delta^n} \bd \Delta^n \times \{1\}) \subseteq X \times \Delta^1.$$
Since $\phi_0$ carries $\{x\} \times \Delta^1$ to a $(q \circ p)$-Cartesian edge of $X$, for
every vertex $x$ of $X$, the existence of $\phi$ follows from Proposition \ref{goldegg}.
\end{proof}

\begin{lemma}\label{insdod}
Let $S$ be a simplicial set, let $i: X \rightarrow Y$ be a cofibration in $(\mSet)_{/S}$, and let $Z \rightarrow S$ be a Cartesian fibration. Then the associated map $p: \bHom_{S}^{\sharp}(Y,Z^{\natural}) \rightarrow \bHom_{S}^{\sharp}(X, Z^{\natural})$ is a Kan fibration.
\end{lemma}

\begin{proof}
Let $A \subseteq B$ be an anodyne inclusion of simplicial sets. We must show that $p$ has the right lifting property with respect to $p$. Equivalently, we must show that $Z^{\natural} \rightarrow S$ has the right lifting property with respect to the inclusion
$$ (B^{\sharp} \times X) \coprod_{A^{\sharp} \times X} (A^{\sharp} \times Y) \subseteq
B^{\sharp} \times Y.$$
This follows from Proposition \ref{markanodprod}, since the inclusion $A^{\sharp} \subseteq B^{\sharp}$ is marked anodyne.
\end{proof}

\begin{proposition}\label{markmodell}\index{gen}{model category!Cartesian}\index{gen}{Cartesian model structure}
Let $S$ be a simplicial set. There exists a left proper, combinatorial model structure on $(\mSet)_{/S}$, which may be described as follows:

\begin{itemize}
\item[$(C)$] The cofibrations in $(\mSet)_{/S}$ are those morphisms $p: X \rightarrow Y$ in $(\mSet)_{/S}$ which are cofibrations when regarded as morphisms of simplicial sets.

\item[$(W)$] The weak equivalences in $(\mSet)_{/S}$ are the Cartesian equivalences.

\item[$(F)$] The fibrations in $(\mSet)_{/S}$ are those maps which have the right lifting property with respect to every map which is simultaneously a cofibration and a Cartesian equivalence.
\end{itemize}
\end{proposition}

\begin{proof}
It suffices to show that the hypotheses of Proposition \ref{goot} are satisfied by the class $(C)$ of cofibrations and the class $(W)$.
\begin{itemize}
\item[(1)] The class $(W)$ of Cartesian equivalences is perfect, in the sense of Definition \ref{perfequiv}. To prove this, we first observe that the class of marked anodyne maps
is generated by the classes $(1)$, $(2)$, $(3)$ of Definition \ref{markanod} and
$(4')$ of Corollary \ref{techycor}. By Proposition \ref{quillobj}, there exists a functor
$T$ from $(\mSet)_{/S}$ to itself and a (functorial) factorization
$$X \stackrel{i_X}{\rightarrow} T(X) \stackrel{j_X}{\rightarrow} S^{\sharp}$$
where $i_X$ is marked anodyne (and therefore a Cartesian equivalence) and
$j_{X}$ has the right lifting property with respect to all marked anodyne maps, and therefore
corresponds to a Cartesian fibration over $S$. Moreover, the functor $T$ commutes
with filtered colimits. According to Proposition \ref{crispy}, a map $X \rightarrow Y$
in $(\mSet)_{/S}$ is a Cartesian equivalence if and only if, for each vertex $s \in S$, 
the induced map $T(X)_{s} \rightarrow T(Y)_{s}$ is a categorical equivalence.
It follows from Corollary \ref{perfpull} that $(W)$ is a perfect class of morphisms.

\item[(2)] The class of weak equivalences is stable under pushouts by cofibrations.
Suppose given a pushout diagram
$$ \xymatrix{ X \ar[r]^{p} \ar[d]^{i} & Y \ar[d] \\
X' \ar[r]^{p'} & Y' }$$ where $i$ is a cofibration and $p$ is a Cartesian equivalence. We wish to show that $p'$ is also a Cartesian equivalence. In other words, we must show that for any Cartesian fibration $Z \rightarrow S$, the associated map $\bHom^{\sharp}_{S}(Y', Z^{\natural}) \rightarrow
\bHom^{\sharp}_{S}(X', Z^{\natural})$ is a homotopy equivalence. Consider the pullback diagram
$$ \xymatrix{ \bHom^{\sharp}_{S}(Y', Z^{\natural}) \ar[r] \ar[d] & \bHom^{\sharp}_{S}(X',Z^{\natural}) \ar[d] \\
\bHom^{\sharp}_{S}(Y, Z^{\natural}) \ar[r] & \bHom^{\sharp}_{S}(X,Z^{\natural}). }$$
Since $p$ is a Cartesian equivalence, the bottom horizontal arrow is a homotopy equivalence.
According to Lemma \ref{insdod}, the right vertical arrow is a Kan fibration; it follows that the
diagram is homotopy Cartesian and so the top horizontal arrow is an equivalence as well.

\item[(3)] A map $p: X \rightarrow Y$ in $(\mSet)_{/S}$ which has the right lifting property with
respect to every map in $(C)$ belongs to $(W)$. Unwinding the definition, we see that
$p$ is a trivial fibration of simplicial sets, and that an edge $e$ of $X$ is marked if and only if
$p(e)$ is a marked edge of $Y$. It follows that $p$ has a section $s$, with $s \circ p$ fiberwise homotopic to $\id_{X}$. From this, we deduce easily that $p$ is a Cartesian equivalence.

%To prove this, we first note that by Qullen's small object argument, every
%map $X \rightarrow S^{\sharp}$ in $\mSet$ admits a factorization
%$$ X \stackrel{\alpha_X}{\rightarrow} TX \stackrel{\beta_{X}}{\rightarrow} S^{\sharp}$$
%where $\alpha_X$ is marked anodyne, and $\beta_{X}$ has the right lifting property with
%respect to all marked anodyne morphisms (and is therefore of the form $Z_{X}^{\natural} \rightarrow S^{\sharp}$, for some Cartesian fibration $Z_{X} \rightarrow S$). Moreover, we may construct the above diagram functorially in $X$, and in such a way that the functor $X \mapsto TX$ commutes with filtered colimits. 
\end{itemize}

\end{proof}

\begin{warning}
Let $S$ be a simplicial set. We must be careful to distinguish between
{\it Cartesian fibrations} of simplicial sets (in the sense of Definition \ref{defcart}) and
fibrations with respect to the Cartesian model structure on $(\mSet)_{/S}$ (in the sense of
Proposition \ref{markmodell}). Though distinct, these notions are closely related: for example, the fibrant objects of $(\mSet)_{/S}$ are precisely those objects of the form $X^{\natural}$, where
$X \rightarrow S$ is a Cartesian fibration (Proposition \ref{markedfibrant}).
\end{warning}

\begin{remark}\label{twuff}
The definition of the Cartesian model structure on $(\mSet)_{/S}$ is not self-opposite. Consequently, we can define another model structure on $(\mSet)_{/S}$ as follows:
\begin{itemize}
\item[$(C)$] The cofibrations in $(\mSet)_{/S}$ are precisely the monomorphisms.
\item[$(W)$] The weak equivalences in $(\mSet)_{/S}$ are precisely the
{\it coCartesian equivalences}\index{gen}{equivalence!coCartesian}\index{gen}{coCartesian equivalence}:
that is, those morphisms $f: \overline{X} \rightarrow \overline{Y}$ such that the induced map
$f^{op}: \overline{X}^{op} \rightarrow \overline{Y}^{op}$ is a Cartesian equivalence in
$(\mSet)_{/S^{op}}$. 
\item[$(F)$] The fibrations in $(\mSet)_{/S}$ are those morphisms which have the right lifting property with respect to every morphism satisfying both $(C)$ and $(W)$.
\end{itemize}
We will refer to this model structure on $(\mSet)_{/S}$ as the {\it coCartesian model structure}.\index{gen}{coCartesian model structure}\index{gen}{model category!coCartesian}
\end{remark}

\subsection{Properties of the Cartesian Model Structure}\label{markprop}

In this section, we will establish some of the basic properties of Cartesian model structures
on $(\mSet)_{/S}$ which was introduced in \S \ref{markmodel}. In particular, we will show that each $(\mSet)_{/S}$ is a {\it simplicial} model category, and characterize its fibrant objects.

\begin{proposition}\label{markedfibrant}
An object $X \in (\mSet)_{/S}$ is fibrant $($with respect to the Cartesian model structure$)$ if and only if $X \simeq Y^{\natural}$, where $Y \rightarrow S$ is a Cartesian fibration.
\end{proposition}

\begin{proof}
Suppose first that $X$ is fibrant. The small object argument implies that there exists a marked anodyne map $j: X \rightarrow Z^{\natural}$ for some Cartesian fibration $Z \rightarrow S$. Since $j$ is marked anodyne, it is a Cartesian equivalence. Since $X$ is fibrant, it has the extension property with respect to the trivial cofibration $j$; thus $X$ is a retract of $Z^{\natural}$. It follows that $X$ is isomorphic to $Y^{\natural}$, where $Y$ is a retract of $Z$.

Now suppose that $Y \rightarrow S$ is a Cartesian fibration; we claim that $Y^{\natural}$ has
the right lifting property with respect to any trivial cofibration $j: A \rightarrow B$ in $(\mSet)_{/S}$.
Since $j$ is a Cartesian equivalence, the map $\eta: \bHom_{S}^{\sharp}(B,Y^{\natural}) \rightarrow
\bHom_{S}^{\sharp}(A, Y^{\natural})$ is a homotopy equivalence of Kan complexes.
Hence, for any map $f: A \rightarrow Z^{\natural}$, there is a map $g: B \rightarrow Z^{\natural}$
such that $g|A$ and $f$ are joined by an edge $e$ of $\bHom_{S}^{\sharp}(A,Z^{\natural})$.
Let $M = (A \times (\Delta^1)^{\sharp}) \coprod_{ A \times \{1\}^{\sharp} } (B \times \{1\}^{\sharp})
\subseteq B \times (\Delta^1)^{\sharp}$. We observe that $e$ and $g$ together determine a map $M \rightarrow Z^{\natural}$. Consider the diagram
$$ \xymatrix{ M \ar[r] \ar[d] & Z^{\natural} \ar[d] \\
B \times (\Delta^1)^{\sharp} \ar[r] \ar@{-->}[ur]^{F} & S^{\sharp}. }$$
The left vertical arrow is marked anodyne, by Proposition \ref{markanodprod}. Consequently, there exists a dotted arrow $F$ as indicated. We note that $F|B \times \{0\}$ is an extension
of $f$ to $B$, as desired.
\end{proof}

%\begin{proposition}
%Let $S$ be a simplicial set. Then the Cartesian model structure on $(\mSet)_{/S}$
%is left proper.
%\end{proposition}

%\begin{proof}
%For any map $f: X \rightarrow Y$ in $(\mSet)_{/S}$, we let
%$M(f)$ denote the ``mapping cylinder''
%$$ ( X \times (\Delta^1)^{\sharp} ) \coprod_{ X \times \{1\}^{\sharp} } Y.$$
%Then there exists a factorization
%$$ X = X \times \{0\}^{\sharp} \subseteq M(f) \rightarrow Y.$$
%The map $M(f) \rightarrow Y$ is a Cartesian equivalence (since the inclusion
%$Y \subseteq M(f)$ is marked-anodyne). Consequently, $f$ is a Cartesian equivalence if %and only if the cofibration $X \subseteq M(f)$ is a trivial cofibration.

%Suppose that $f$ is a Cartesian equivalence, and let $X \rightarrow X'$ be any cofibration. %We wish to show that the induced map $f': X' \rightarrow Y' = X' \coprod_{X} Y$ is a %Cartesian equivalence. For this, it suffices to show that $X' \subseteq M(f')$ is a trivial %cofibration. But this inclusion factors as
%$$ X' \subseteq M(f) \coprod_{X} X' \rightarrow M(f) \coprod_{X \times %(\Delta^1)^{\sharp}} (X' \times (\Delta^1)^{\sharp}) \simeq M(f').$$
%The map on the left is a pushout of the inclusion $X \subseteq M(f)$, and therefore a %trivial cofibration. The map on the right is a pushout of $g: X' \coprod_{ X} (X \times %(\Delta^1)^{\sharp}) \subseteq X' \times (\Delta^1)^{\sharp}$. Let 
%$h: X' \rightarrow X' \coprod_{X} (X \times (\Delta^1)^{\sharp} )$ be the natural map; to %show that $g$ is a Cartesian equivalence, it suffices to show that $h$ and $g \circ h$
%are Cartesian equivalences. But both of these maps are marked anodyne.
%\end{proof}

We now study the behavior of the Cartesian model structures with respect to products.

\begin{proposition}\label{urlt}
Let $S$ and $T$ simplicial sets, and let $Z$ be an object of $(\mSet)_{/T}$. Then the functor $$
(\mSet)_{/S} \rightarrow (\mSet)_{/S \times T}$$
$$ X \mapsto X \times Z$$
preserves Cartesian equivalences.
\end{proposition}

\begin{proof}
Let $f: X \rightarrow Y$ be a Cartesian equivalence in $(\mSet)_{/S}$. We wish to show that $f \times \id_{Z}$ is a Cartesian equivalence in $(\mSet)_{/S \times T}$. Let $X \rightarrow X'$ be a marked anodyne map where $X' \in (\mSet)_{/S}$ is fibrant. Now choose a marked-anodyne map $X' \coprod_{X} Y \rightarrow Y'$, where $Y' \in (\mSet)_{/S}$ is fibrant. Since the product maps $X \times Z \rightarrow X' \times Z$ and $Y \times Z \rightarrow Y' \times Z$ are also marked anodyne
(by Proposition \ref{markanodprod}), it suffices to show that $X' \times Z \rightarrow Y' \times Z$ is a Cartesian equivalence. In other words, we may reduce to the situation where $X$ and $Y$ are fibrant. By Proposition \ref{crispy}, $f$ has a homotopy inverse $g$; then
$g \times \id_Y$ is a homotopy inverse to $f \times \id_Y$.
\end{proof}

\begin{corollary}\label{compatprod}
Let $f: A \rightarrow B$ be a cofibration in $(\mSet)_{/S}$ and $f': A' \rightarrow B'$ a cofibration in $(\mSet)_{/T}$. Then the smash product map
$$(A \times B' ) \coprod_{ A \times B } (A' \times B) \rightarrow A' \times B'$$ is a cofibration in
$(\mSet)_{/S \times T}$, which is trivial if either $f$ or $g$ is trivial.
\end{corollary}

\begin{corollary}
Let $S$ be a simplicial set, and regard $(\mSet)_{/S}$ as a simplicial category with mapping objects given by $\bHom^{\sharp}_{S}(X,Y)$. Then $(\mSet)_{/S}$ is a {\it simplicial} model category.
\end{corollary}

\begin{proof}
Unwinding the definitions, we are reduced to proving the following: given a cofibration
$i: X \rightarrow X'$ in $(\mSet)_{/S}$ and a cofibration $j: Y \rightarrow Y'$ in $\sSet$, the induced cofibration
$$ (X' \times Y^{\sharp}) \coprod_{ X \times Y^{\sharp} } (X \times {Y'}^{\sharp})
\subseteq X' \times {Y'}^{\sharp}$$
in $(\mSet)_{/S}$ is trivial if either $i$ is a Cartesian equivalence of $j$ is a weak homotopy equivalence. If $i$ is trivial, this follows immediately
from Corollary \ref{compatprod}. If $j$ is trivial, the same argument applies, provided that
we can verify that $Y^{\sharp} \rightarrow {Y'}^{\sharp}$ is a Cartesian equivalence in $\mSet$.
Unwinding the definitions, we must show that for every $\infty$-category $Z$, the restriction map
$$ \theta: \bHom^{\sharp}( {Y'}^{\sharp}, Z^{\natural} ) \rightarrow \bHom^{\sharp}( Y^{\sharp}, Z^{\natural}) $$
is a homotopy equivalence of Kan complexes. Let $K$ be the largest Kan complex contained in $Z$, so that $\theta$ can be identified with the restriction map
$$ \bHom_{\sSet}(Y', K) \rightarrow \bHom_{\sSet}(Y,K).$$
Since $j$ is a weak homotopy equivalence, this map is a trivial fibration.
\end{proof}

\begin{remark}
There is a second simplicial structure on $(\mSet)_{/S}$, where the simplicial mapping spaces are given by $\bHom_{S}^{\flat}(X,Y)$. This simplicial structure is {\em not} compatible with
the Cartesian model structure: for fixed $X \in (\mSet)_{/S}$, the functor
$$A \mapsto A^{\flat} \times X$$ does not carry weak homotopy equivalences (in the $A$-variable) to Cartesian equivalences. It does, however, carry {\em categorical} equivalences (in $A$) to Cartesian equivalences, and consequently $(\mSet)_{/S}$ is endowed with the structure of a $\sSet$-enriched model category, where we regard $\sSet$ as equipped with the Joyal model structure. This second simplicial structure reflects the fact that $(\mSet)_{/S}$ is really a model for an $\infty$-bicategory.
\end{remark}

\begin{remark}\label{duality}
Suppose $S$ is a Kan complex. A map $p: X \rightarrow S$ is a Cartesian fibration if and only if it is a coCartesian fibration (this follows in general from Proposition \ref{groob}; if $S = \Delta^0$, the main case of interest for us, it is obvious). Moreover, the class $p$-coCartesian edges of $X$ coincides with the class of $p$-Cartesian edges of $X$: both may be described as the class of equivalences in $X$. Consequently, if $A \in (\mSet)_{/S}$, then
$$ \bHom^{\flat}_{S}(A, X^{\natural}) \simeq \bHom^{\flat}_{S^{op}}(A^{op}, (X^{op})^{\natural})^{op},$$
where $A^{op}$ is regarded as a marked simplicial set in the obvious way. It follows that a map
$A \rightarrow B$ is a Cartesian equivalence in $(\mSet)_{/S}$ if and only if $A^{op} \rightarrow B^{op}$ is a Cartesian equivalence in $(\mSet)_{/S^{op}}$. In other words, the Cartesian model structure on $(\mSet)_{/S}$ is self-dual {\em when $S$ is a Kan complex}. In particular, if $S = \Delta^0$, we deduce that the functor $$ A \mapsto A^{op}$$ determines an autoequivalence of the model category $\mSet \simeq (\mSet)_{/\Delta^0}$.
\end{remark}

\subsection{Comparison of Model Categories}\label{compmodel}

Let $S$ be a simplicial set. We now have a plethora of model structures on categories of simplicial sets over $S$:

\begin{itemize}
\item[(0)] Let $\calC_0$ denote the category $(\sSet)_{/S}$ of simplicial sets over $S$ endowed with the {\em Joyal} model structure defined in \S \ref{compp3}: the cofibrations are monomorphisms of simplicial sets, and the weak equivalences are categorical equivalences.
\item[(1)] Let $\calC_1$ denote the category $(\mSet)_{/S}$ of marked simplicial sets over $S$,
endowed with the {\em marked} model structure of Proposition \ref{markmodell}: the cofibrations are
maps $(X,\calE_X) \rightarrow (Y, \calE_Y)$ which induce monomorphisms $X \rightarrow Y$, and the weak equivalences are the Cartesian equivalences.
\item[(2)] Let $\calC_2$ denote the category $(\mSet)_{/S}$ of marked simplicial sets over $S$,
endowed with the following {\em localization} of the Cartesian model structure: a map
$f: (X, \calE_X) \rightarrow (Y, \calE_Y)$ is a cofibration if the underlying map $X \rightarrow Y$ is a monomorphism, and a weak equivalence if $f: X^{\sharp} \rightarrow Y^{\sharp}$ is a marked
equivalence in $(\mSet)_{/S}$.
\item[(3)] Let $\calC_3$ denote the category $(\sSet)_{/S}$ of simplicial sets over $S$, which
is endowed with the {\em contravariant} model structure described in \S \ref{contrasec}: the cofibrations are the monomorphisms, and the weak equivalences are the contravariant equivalences.
\item[(4)] Let $\calC_4$ denote the category $(\sSet)_{/S}$ of simplicial sets over $S$, endowed with the usual homotopy-theoretic model structure: the cofibrations are the monomorphisms of simplicial sets, and the weak equivalences are the weak homotopy equivalences of simplicial sets.
\end{itemize}

The goal of this section is to study the relationship between these five model categories.
We may summarize the situation as follows:

\begin{theorem}\label{bigdiag}
There exists a sequence of Quillen adjunctions
$$ \calC_0 \stackrel{F_0}{\rightarrow} \calC_1 \stackrel{F_1}{\rightarrow} \calC_2
\stackrel{F_2}{\rightarrow} \calC_3 \stackrel{F_3}{\rightarrow} \calC_4$$
$$ \calC_0 \stackrel{G_0}{\leftarrow} \calC_1 \stackrel{G_1}{\leftarrow} \calC_2
\stackrel{G_2}{\leftarrow} \calC_3 \stackrel{G_3}{\leftarrow} \calC_4$$
which may be described as follows:
\begin{itemize}
\item[$(A0)$] The functor $G_0$ is the forgetful functor from $(\mSet)_{/S}$ to $(\sSet)_{/S}$, which
ignores the collection of marked edges. The functor $F_0$ is the left adjoint to $G_0$, which is given by
$X \mapsto X^{\flat}$. The Quillen adjunction $(F_0,G_0)$ is a Quillen equivalence if $S$ is a Kan complex.

\item[$(A1)$] The functors $F_1$ and $G_1$ are the identity functors on $(\mSet)_{/S}$.

\item[$(A2)$] The functor $F_2$ is the forgetful functor from $(\mSet)_{/S}$ to $(\sSet)_{/S}$, which ignores
the collection of marked edges. The functor $G_2$ is the right adjoint to $F_2$, which is given by
$X \mapsto X^{\sharp}$. The Quillen adjunction $(F_2,G_2)$ is a Quillen equivalence for
{\em every} simplicial set $S$.

\item[$(A3)$] The functors $F_3$ and $G_3$ are the identity functors on $(\mSet)_{/S}$. The Quillen
adjunction $(F_3, G_3)$ is a Quillen equivalence whenever $S$ is a Kan complex.
\end{itemize}
\end{theorem}

The rest of this section is devoted to giving a proof of Theorem \ref{bigdiag}. We will organize our efforts as follows. First, we verify that the model category $\calC_{2}$ is well-defined (the analogous results for the other model structures have already been established). We then consider each of the adjunctions $(F_i,G_i)$ in turn, and show that it has the desired properties.

\begin{proposition}\label{lmark}
Let $S$ be a simplicial set. There exists a left proper, combinatorial model structure on the category
$(\mSet)_{/S}$ which may be described as follows:
\begin{itemize}
\item[$(C)$] A map $f: (X, \calE_{X}) \rightarrow (Y, \calE_Y)$ is a cofibration if and only if the underlying map $X \rightarrow Y$ is a monomorphism of simplicial sets.
\item[$(W)$] A map $f: (X, \calE_{X}) \rightarrow (Y, \calE_{Y})$ is a weak equivalence
if and only if the induced map $X^{\sharp} \rightarrow Y^{\sharp}$ is a Cartesian equivalence
in $(\mSet)_{/S}$.
\item[$(F)$] A map $f: (X, \calE_{X}) \rightarrow (Y, \calE_Y)$ is a fibration if and only if
it has the right lifting property with respect to all trivial cofibrations.
\end{itemize}
\end{proposition}

\begin{proof}
It suffices to show that the conditions of Proposition \ref{goot} are satisfied. We check them in turn:

\begin{itemize}
\item[(1)]  The class $(W)$ of Cartesian equivalences is perfect, in the sense of Definition \ref{perfequiv}. This follows from Corollary \ref{perfpull}, since the class of Cartesian equivalences is perfect, and the functor $(X, \calE_X) \rightarrow X^{\sharp}$ commutes with filtered colimits.

\item[(2)] The class of weak equivalences is stable under pushouts by cofibrations. This
follows from the analogous property of the Cartesian model structure, since the functor
$(X, \calE_X) \mapsto X^{\sharp}$ preserves pushouts.

\item[(3)] A map $p: (X,\calE_X) \rightarrow (Y,\calE_Y)$ which has the right lifting property with respect to
every cofibration is a weak equivalence. In this case, the underlying map of simplicial
sets is a trivial fibration, so the induced map $X^{\sharp} \rightarrow Y^{\sharp}$ has the right lifting property with respect to all trivial cofibrations, and is a Cartesian equivalence as observed in the proof of Proposition \ref{markmodell}.

\end{itemize}
\end{proof} 

\begin{proposition}\label{markedjoyal}
Let $S$ be simplicial set. Consider the adjoint functors
$$ \Adjoint{F_0}{(\sSet)_{/S}}{(\mSet)_{/S}}{G_0}$$
described by the formulas
$$ F_0(X) = X^{\flat}$$
$$ G_0(X,\calE) = X.$$
The adjoint functors $(F_0,G_0)$ determine a Quillen adjunction between $(\sSet)_{/S}$ (with the Joyal model structure) and $(\mSet)_{/S}$ (with the Cartesian model structure). If $S$ is a Kan complex, then $(F_0,G_0)$ is a Quillen equivalence.
\end{proposition}

\begin{proof}
To prove that $(F_0,G_0)$ is a Quillen adjunction, it will suffice to show that $F_1$ preserves cofibrations and trivial cofibrations. The first claim is obvious. For the second, we must show that if
$X \subseteq Y$ is a categorical equivalence of simplicial sets over $S$, then the induced map
$X^{\flat} \rightarrow Y^{\flat}$ is a Cartesian equivalence in $(\mSet)_{/S}$. For this, it suffices to
show that for any Cartesian fibration $p: Z \rightarrow S$, the restriction map
$$ \bHom_{S}^{\flat}(Y^{\flat}, Z^{\natural}) \rightarrow \bHom_{S}^{\flat}(X^{\flat},Z^{\natural})$$ is
a trivial fibration of simplicial sets. In other words, we must show that for every inclusion
$A \subseteq B$ of simplicial sets, it is possible to solve any lifting problem of the form
$$ \xymatrix{ A \ar[r] \ar@{^{(}->}[d] & \bHom_{S}^{\flat}(Y^{\flat}, Z^{\natural}) \ar[d] \\
B \ar[r] \ar@{-->}[ur] & \bHom_{S}^{\flat}(X^{\flat}, Z^{\natural}). }$$
Replacing $Y$ by $Y \times B$ and $X$ by $(X \times B) \coprod_{ X \times A} (Y \times A)$, 
we may suppose that $A = \emptyset$ and $B = \ast$. Moreover, we may rephrase the lifting problem as the problem of constructing the dotted arrow indicated in the following diagram:
$$ \xymatrix{ X \ar@{^{(}->}[d] \ar[r] & Z \ar[d]^{p} \\
Y \ar[r] \ar@{-->}[ur] \ar[r] & S }$$
By Proposition \ref{funkyfibcatfib}, $p$ is a categorical fibration, and the lifting problem has a solution in virtue of the assumption that $X \subseteq Y$ is a categorical equivalence.

Now suppose that $S$ is a Kan complex. We want to prove that $(F_0,G_0)$ is a Quillen equivalence. In other words, we must show that for any fibrant object of $(\mSet)_{/S}$ corresponding to a Cartesian fibration $Z \rightarrow S$, a map $X \rightarrow Z$ in $(\sSet)_{/S}$ is a categorical equivalence if and only if the associated map $X^{\flat} \rightarrow Z^{\natural}$ is a Cartesian equivalence.

Suppose first that $X \rightarrow Z$ is a categorical equivalence. Then the induced map
$X^{\flat} \rightarrow Z^{\flat}$ is a Cartesian equivalence, by the argument given above. It therefore suffices to show that $Z^{\flat} \rightarrow Z^{\natural}$ is a Cartesian equivalence.
Since $S$ is a Kan complex, $Z$ is an $\infty$-category; let $K$ denote the largest Kan complex contained in $Z$. The marked edges of $Z^{\natural}$ are precisely the edges which belong to $K$, so we have a pushout diagram
$$ \xymatrix{ K^{\flat} \ar[r] \ar[d] & K^{\sharp} \ar[d] \\
Z^{\flat} \ar[r] & Z^{\natural}. }$$
It follows that $Z^{\flat} \rightarrow Z^{\natural}$ is marked anodyne, and therefore a Cartesian equivalence.

Now suppose that $X^{\flat} \rightarrow Z^{\natural}$ is a Cartesian equivalence. Choose a factorization $X \stackrel{f}{\rightarrow} Y \stackrel{g}{\rightarrow} Z$, where $f$ is a categorical equivalence and $g$ is a categorical fibration. We wish to show that $g$ is a categorical equivalence. 
Proposition \ref{groob} implies that $Z \rightarrow S$ is a categorical fibration, so that $X' \rightarrow S$ is a categorical fibration. Applying Proposition \ref{groob} again, we deduce that
$Y \rightarrow S$ is a Cartesian fibration. Thus we have a factorization
$$ X^{\flat} \rightarrow Y^{\flat} \rightarrow Y^{\natural} \rightarrow Z^{\natural}$$
where the first two maps are Cartesian equivalences by the arguments given above, and the 
composite map is a Cartesian equivalence. Thus $Y^{\natural} \rightarrow Z^{\natural}$ is an equivalence between fibrant objects of $(\mSet)_{/S}$, and therefore admits a homotopy inverse. The existence of this homotopy inverse proves that $g$ is a categorical equivalence, as desired.
\end{proof}

\begin{proposition}\label{marklocal}
Let $S$ be a simplicial set, and let $F_1$ and $G_1$ denote the identity functor from
$(\mSet)_{/S}$ to itself. Then $(F_1,G_1)$ determines a Quillen adjunction between
$\calC_1$ and $\calC_2$.
\end{proposition}

\begin{proof}
We must show that $F_1$ preserves cofibrations and trivial cofibrations. The first claim is obvious. For the second,
let $B: (\mSet)_{/S} \rightarrow (\mSet)_{/S}$ be the functor defined by
$$B(M, \calE_M) = M^{\sharp}.$$
We wish to show that if $X \rightarrow Y$ is a Cartesian equivalence in $(\mSet)_{/S}$, then
$B(X) \rightarrow B(Y)$ is a Cartesian equivalence.

We first observe that if $X \rightarrow Y$ is marked anodyne, then the induced map $B(X) \rightarrow B(Y)$ is also marked anodyne: by general nonsense, it suffices to check this for the generators described in Definition \ref{markanod}, for which it is obvious. Now return to the case of a general Cartesian equivalence $p: X \rightarrow Y$, and choose a diagram
$$ \xymatrix{ X \ar[r]^{i} \ar[d]^{p} & X' \ar[d] \ar[dr]^{q}\\
Y \ar[r] & X' \coprod_{X} Y \ar[r]^{j} & Y' }$$
in which $X'$ and $Y'$ are (marked) fibrant and $i$ and $j$ are marked anodyne. It follows that
$B(i)$ and $B(j)$ are marked anodyne, and therefore Cartesian equivalences. Thus, to prove that
$B(p)$ is a Cartesian equivalence, it suffices to show that $B(q)$ is a Cartesian equivalence.
But $q$ is a Cartesian equivalence between fibrant objects of $(\mSet)_{/S}$, and therefore has a homotopy inverse. It follows that $B(q)$ also has a homotopy inverse, and is therefore a Cartesian equivalence as desired.
\end{proof}

\begin{remark}
In the language of model categories, we may summarize Proposition \ref{marklocal} by saying that the model structure of Proposition \ref{lmark} is a {\em localization} of the Cartesian model structure on $(\mSet)_{/S}$.
\end{remark}

\begin{proposition}\label{romb}
Let $S$ be a simplicial set, and consider the adjunction
$$ \Adjoint{F_2}{ (\mSet)_{/S}}{ (\sSet)_{/S}}{G_2}$$
determined by the formulas
$$ F_2( X, \calE) = X$$
$$G_2(X) = X^{\sharp}.$$
The adjoint functors $(F_2,G_2)$ determines a Quillen equivalence between $\calC_2$ and
$\calC_3$.
\end{proposition}

\begin{proof}
We first claim that $F_2$ is conservative: that is, a map $f: (X, \calE_X) \rightarrow
(Y, \calE_Y)$ is a weak equivalence in $\calC_2$ if and only if the induced map
$X \rightarrow Y$ is a weak equivalence in $\calC_3$. Unwinding the definition, $f$ is a weak equivalence if and only if $X^{\sharp} \rightarrow Y^{\sharp}$ is a Cartesian equivalence. This holds if and only if, for every Cartesian fibration $Z \rightarrow S$, the induced map
$$\phi: \bHom^{\sharp}_S(Y^{\sharp}, Z^{\natural}) \rightarrow \bHom^{\sharp}_S(X^{\sharp}, Z^{\natural})$$ is a homotopy equivalence. Let $Z^0 \rightarrow S$ be the right fibration
associated to $Z \rightarrow S$ (see Corollary \ref{relativeKan}). There are natural identifications $\bHom^{\sharp}_S(Y^{\sharp}, Z^{\natural}) \simeq \bHom_{S}(Y, Z^0)$, $\bHom^{\sharp}_S(X^{\sharp}, Z^{\natural}) \simeq \bHom_{S}(X,Z^0)$. Consequently, $f$ is a weak equivalence if and only if, for every right fibration $Z^0 \rightarrow S$, the associated map
$$ \bHom_{S}(Y, Z^0) \rightarrow \bHom_{S}(X, Z^0)$$ is a homotopy equivalence. Since $\calC_3$ is a simplicial model category for which the fibrant objects are precisely the right fibrations $Z^0 \rightarrow S$ (Corollary \ref{usewhere1}), this is equivalent to the assertion that $X \rightarrow Y$ is a weak equivalence in $\calC_3$. 

To prove that $(F_2, G_2)$ is a Quillen adjunction, it suffices to show that $F_2$ preserves cofibrations and trivial cofibrations. The first claim is obvious, and the second follows because $F_2$ preserves all weak equivalences, by the above argument.

To show that $(F_2, G_2)$ is a Quillen equivalence, we must show that the unit and counit
$$ LF_2 \circ RG_2 \rightarrow \id$$
$$ \id \rightarrow RG_2 \circ LF_2$$
are weak equivalences. In view of the fact that $F_2 = LF_2$ is conservative, the second assertion follows from the first. As to the first, it suffices to show that if $X$ is a fibrant object of
$\calC_3$, then the counit map $(F_2 \circ G_2)(X) \rightarrow X$ is a weak equivalence.
But this map is an isomorphism.
\end{proof}

\begin{proposition}\label{strstr}
Let $S$ be a simplicial set, and let $F_3$ and $G_3$ denote the identity functor
from $(\sSet)_{/S}$ to itself. Then $(F_3, G_3)$ gives a Quillen adjunction between
$\calC_3$ and $\calC_4$. If $S$ is a Kan complex, then $(F_3, G_3)$ is a Quillen equivalence 
$($in other words, the model structures on $\calC_3$ and $\calC_4$ coincide$)$.
\end{proposition}

\begin{proof}
To prove that $(F_3,G_3)$ is a Quillen adjunction, it suffices to prove that $F_3$ preserves cofibrations and weak equivalences. The first claim is obvious (the cofibrations in $\calC_3$ and $\calC_4$ are the same). For the second, we note that both $\calC_3$ and $\calC_4$ are simplicial model categories in which every object is cofibrant. Consequently, a map $f: X \rightarrow Y$
is a weak equivalence if and only if, for every fibrant object $Z$, the associated map
$\bHom(Y,Z) \rightarrow \bHom(X,Z)$ is a homotopy equivalence of Kan complexes.
Thus, to show that $F_3$ preserves weak equivalences, it suffices to show that
$G_3$ preserves fibrant objects. A map $p: Z \rightarrow S$ is fibrant as an object of 
$\calC_4$ if and only if $p$ is a Kan fibration, and fibrant as an object of $\calC_3$ if and only if $p$ is a right fibration (Corollary \ref{usewhere1}). Since every Kan fibration is a right fibration, it follows that $F_3$ preserves weak equivalences. If $S$ is a Kan complex, then the converse holds: according to Lemma \ref{toothie}, every right fibration $p: Z \rightarrow S$ is a Kan fibration. It follows that $G_3$ preserves weak equivalences as well, so that the two model structures under consideration coincide.
\end{proof}

